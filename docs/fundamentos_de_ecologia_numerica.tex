% Options for packages loaded elsewhere
\PassOptionsToPackage{unicode}{hyperref}
\PassOptionsToPackage{hyphens}{url}
%
\documentclass[
]{book}
\usepackage{amsmath,amssymb}
\usepackage{lmodern}
\usepackage{iftex}
\ifPDFTeX
  \usepackage[T1]{fontenc}
  \usepackage[utf8]{inputenc}
  \usepackage{textcomp} % provide euro and other symbols
\else % if luatex or xetex
  \usepackage{unicode-math}
  \defaultfontfeatures{Scale=MatchLowercase}
  \defaultfontfeatures[\rmfamily]{Ligatures=TeX,Scale=1}
\fi
% Use upquote if available, for straight quotes in verbatim environments
\IfFileExists{upquote.sty}{\usepackage{upquote}}{}
\IfFileExists{microtype.sty}{% use microtype if available
  \usepackage[]{microtype}
  \UseMicrotypeSet[protrusion]{basicmath} % disable protrusion for tt fonts
}{}
\makeatletter
\@ifundefined{KOMAClassName}{% if non-KOMA class
  \IfFileExists{parskip.sty}{%
    \usepackage{parskip}
  }{% else
    \setlength{\parindent}{0pt}
    \setlength{\parskip}{6pt plus 2pt minus 1pt}}
}{% if KOMA class
  \KOMAoptions{parskip=half}}
\makeatother
\usepackage{xcolor}
\usepackage{color}
\usepackage{fancyvrb}
\newcommand{\VerbBar}{|}
\newcommand{\VERB}{\Verb[commandchars=\\\{\}]}
\DefineVerbatimEnvironment{Highlighting}{Verbatim}{commandchars=\\\{\}}
% Add ',fontsize=\small' for more characters per line
\usepackage{framed}
\definecolor{shadecolor}{RGB}{248,248,248}
\newenvironment{Shaded}{\begin{snugshade}}{\end{snugshade}}
\newcommand{\AlertTok}[1]{\textcolor[rgb]{0.94,0.16,0.16}{#1}}
\newcommand{\AnnotationTok}[1]{\textcolor[rgb]{0.56,0.35,0.01}{\textbf{\textit{#1}}}}
\newcommand{\AttributeTok}[1]{\textcolor[rgb]{0.77,0.63,0.00}{#1}}
\newcommand{\BaseNTok}[1]{\textcolor[rgb]{0.00,0.00,0.81}{#1}}
\newcommand{\BuiltInTok}[1]{#1}
\newcommand{\CharTok}[1]{\textcolor[rgb]{0.31,0.60,0.02}{#1}}
\newcommand{\CommentTok}[1]{\textcolor[rgb]{0.56,0.35,0.01}{\textit{#1}}}
\newcommand{\CommentVarTok}[1]{\textcolor[rgb]{0.56,0.35,0.01}{\textbf{\textit{#1}}}}
\newcommand{\ConstantTok}[1]{\textcolor[rgb]{0.00,0.00,0.00}{#1}}
\newcommand{\ControlFlowTok}[1]{\textcolor[rgb]{0.13,0.29,0.53}{\textbf{#1}}}
\newcommand{\DataTypeTok}[1]{\textcolor[rgb]{0.13,0.29,0.53}{#1}}
\newcommand{\DecValTok}[1]{\textcolor[rgb]{0.00,0.00,0.81}{#1}}
\newcommand{\DocumentationTok}[1]{\textcolor[rgb]{0.56,0.35,0.01}{\textbf{\textit{#1}}}}
\newcommand{\ErrorTok}[1]{\textcolor[rgb]{0.64,0.00,0.00}{\textbf{#1}}}
\newcommand{\ExtensionTok}[1]{#1}
\newcommand{\FloatTok}[1]{\textcolor[rgb]{0.00,0.00,0.81}{#1}}
\newcommand{\FunctionTok}[1]{\textcolor[rgb]{0.00,0.00,0.00}{#1}}
\newcommand{\ImportTok}[1]{#1}
\newcommand{\InformationTok}[1]{\textcolor[rgb]{0.56,0.35,0.01}{\textbf{\textit{#1}}}}
\newcommand{\KeywordTok}[1]{\textcolor[rgb]{0.13,0.29,0.53}{\textbf{#1}}}
\newcommand{\NormalTok}[1]{#1}
\newcommand{\OperatorTok}[1]{\textcolor[rgb]{0.81,0.36,0.00}{\textbf{#1}}}
\newcommand{\OtherTok}[1]{\textcolor[rgb]{0.56,0.35,0.01}{#1}}
\newcommand{\PreprocessorTok}[1]{\textcolor[rgb]{0.56,0.35,0.01}{\textit{#1}}}
\newcommand{\RegionMarkerTok}[1]{#1}
\newcommand{\SpecialCharTok}[1]{\textcolor[rgb]{0.00,0.00,0.00}{#1}}
\newcommand{\SpecialStringTok}[1]{\textcolor[rgb]{0.31,0.60,0.02}{#1}}
\newcommand{\StringTok}[1]{\textcolor[rgb]{0.31,0.60,0.02}{#1}}
\newcommand{\VariableTok}[1]{\textcolor[rgb]{0.00,0.00,0.00}{#1}}
\newcommand{\VerbatimStringTok}[1]{\textcolor[rgb]{0.31,0.60,0.02}{#1}}
\newcommand{\WarningTok}[1]{\textcolor[rgb]{0.56,0.35,0.01}{\textbf{\textit{#1}}}}
\usepackage{longtable,booktabs,array}
\usepackage{calc} % for calculating minipage widths
% Correct order of tables after \paragraph or \subparagraph
\usepackage{etoolbox}
\makeatletter
\patchcmd\longtable{\par}{\if@noskipsec\mbox{}\fi\par}{}{}
\makeatother
% Allow footnotes in longtable head/foot
\IfFileExists{footnotehyper.sty}{\usepackage{footnotehyper}}{\usepackage{footnote}}
\makesavenoteenv{longtable}
\usepackage{graphicx}
\makeatletter
\def\maxwidth{\ifdim\Gin@nat@width>\linewidth\linewidth\else\Gin@nat@width\fi}
\def\maxheight{\ifdim\Gin@nat@height>\textheight\textheight\else\Gin@nat@height\fi}
\makeatother
% Scale images if necessary, so that they will not overflow the page
% margins by default, and it is still possible to overwrite the defaults
% using explicit options in \includegraphics[width, height, ...]{}
\setkeys{Gin}{width=\maxwidth,height=\maxheight,keepaspectratio}
% Set default figure placement to htbp
\makeatletter
\def\fps@figure{htbp}
\makeatother
\setlength{\emergencystretch}{3em} % prevent overfull lines
\providecommand{\tightlist}{%
  \setlength{\itemsep}{0pt}\setlength{\parskip}{0pt}}
\setcounter{secnumdepth}{5}
\newlength{\cslhangindent}
\setlength{\cslhangindent}{1.5em}
\newlength{\csllabelwidth}
\setlength{\csllabelwidth}{3em}
\newlength{\cslentryspacingunit} % times entry-spacing
\setlength{\cslentryspacingunit}{\parskip}
\newenvironment{CSLReferences}[2] % #1 hanging-ident, #2 entry spacing
 {% don't indent paragraphs
  \setlength{\parindent}{0pt}
  % turn on hanging indent if param 1 is 1
  \ifodd #1
  \let\oldpar\par
  \def\par{\hangindent=\cslhangindent\oldpar}
  \fi
  % set entry spacing
  \setlength{\parskip}{#2\cslentryspacingunit}
 }%
 {}
\usepackage{calc}
\newcommand{\CSLBlock}[1]{#1\hfill\break}
\newcommand{\CSLLeftMargin}[1]{\parbox[t]{\csllabelwidth}{#1}}
\newcommand{\CSLRightInline}[1]{\parbox[t]{\linewidth - \csllabelwidth}{#1}\break}
\newcommand{\CSLIndent}[1]{\hspace{\cslhangindent}#1}
\usepackage{booktabs}
\usepackage{amsthm}
\makeatletter
\def\thm@space@setup{%
  \thm@preskip=8pt plus 2pt minus 4pt
  \thm@postskip=\thm@preskip
}
\makeatother
\usepackage{booktabs}
\usepackage{longtable}
\usepackage{array}
\usepackage{multirow}
\usepackage{wrapfig}
\usepackage{float}
\usepackage{colortbl}
\usepackage{pdflscape}
\usepackage{tabu}
\usepackage{threeparttable}
\usepackage{threeparttablex}
\usepackage[normalem]{ulem}
\usepackage{makecell}
\usepackage{xcolor}
\usepackage{caption}
\ifLuaTeX
  \usepackage{selnolig}  % disable illegal ligatures
\fi
\IfFileExists{bookmark.sty}{\usepackage{bookmark}}{\usepackage{hyperref}}
\IfFileExists{xurl.sty}{\usepackage{xurl}}{} % add URL line breaks if available
\urlstyle{same} % disable monospaced font for URLs
\hypersetup{
  pdftitle={Fundamentos de Ecologia Numérica},
  pdfauthor={Prof.~Elvio S. F. Medeiros; Laboratório de Ecologia; Universidade Estadual da Paraíbal; Campus V, João Pessoa, PB},
  hidelinks,
  pdfcreator={LaTeX via pandoc}}

\title{Fundamentos de Ecologia Numérica}
\usepackage{etoolbox}
\makeatletter
\providecommand{\subtitle}[1]{% add subtitle to \maketitle
  \apptocmd{\@title}{\par {\large #1 \par}}{}{}
}
\makeatother
\subtitle{Disciplina de Ecologia Numérica do Curso de Ciências Biológicas do Campus V da UEPB}
\author{Prof.~Elvio S. F. Medeiros \and Laboratório de Ecologia \and Universidade Estadual da Paraíbal \and Campus V, João Pessoa, PB}
\date{Last updated: 16/01/2026 14:12:53}

\begin{document}
\maketitle

{
\setcounter{tocdepth}{1}
\tableofcontents
}
\hypertarget{apresentauxe7uxe3o}{%
\chapter{Apresentação}\label{apresentauxe7uxe3o}}

\hypertarget{motivauxe7uxe3o}{%
\section{Motivação}\label{motivauxe7uxe3o}}

Este é um site experimental da disciplina de graduação de Ecologia Numérica do Curso de Ciências Biológicas da Universidade Estadual da Paraíba.

O site experimental da disciplina de Ecologia Numérica da UEPB é uma plataforma criada com o objetivo de apresentar exemplos didáticos aos alunos, explorando diferentes conceitos e técnicas utilizadas na análise de dados em Ecologia. É importante destacar que os exemplos apresentados no site podem conter imprecisões ou códigos demasiado extensos (aparentemente desnecessários para um ecólogo e programador de R mais experiente), mas isso é proposital, pois a finalidade é oferecer um material didático para os alunos, que possa facilitar a compreensão dos conceitos abordados durante as aulas. Dessa forma, o site é uma ferramenta complementar às aulas presenciais, que busca fornecer exemplos práticos e didáticos para os estudantes de Ecologia Numérica da UEPB.

Como mencionado, a finalidade é didatica e de apresentar o tema seguindo passo-a-passo e contextualizando com dados ecológicos reais. Por isso, além das informações sobre o conteúdo do site experimental da disciplina de ecologia numérica da UEPB, é importante destacar que qualquer erro ou problema relacionado ao site pode ser reportado ao \href{elviomedeiros@servidor.uepb.edu.br}{responsável pelo site}.

É fundamental que os usuários do site compreendam que plágio e outras formas de má conduta acadêmica são inaceitáveis e podem ter sérias consequências. O \href{elviomedeiros@servidor.uepb.edu.br}{responsável pelo site} se compromete a remover qualquer conteúdo ofensivo ou sem citar a devida fonte consultada, que possa ter sido publicado erroneamente ou por engano.

\hypertarget{organizauxe7uxe3o-do-livro}{%
\section{Organização do livro}\label{organizauxe7uxe3o-do-livro}}

Este livro é organizado em \texttt{CINCO} partes:

Na \texttt{PARTE\ I}, são apresentadas as bases teóricas fundamentais sobre ecologia numérica e análise multivariada.

Na \texttt{PARTE\ II}, é apresentada a fundamentação teórica para o entedimento das das BASES DE DADOS que são trabalhados ao longo das \texttt{PARTES\ III\ e\ IV} e execução das estatísticas descritas.

Na \texttt{PARTE\ III}, são apresentados os primeiros Módulos do RStudio sobre ESTATÍSTICAS DESCRITIVAS EM ECOLOGIA DE COMUNIDADES, na forma de tutoriais discutidos, que podem ser executados passo-a-passo sobre .

Na \texttt{PARTE\ IV}, são apresentados os Módulos do RStudio mais avançados, sobre TÉCNICAS DE ANÁLISE MULTIVARIADA EM ECOLOGIA DE COMUNIDADES.

Por fim, na \texttt{PARTE\ V}, são apresentadas informações gerais de apoio para o melhor entendimento do conteúdo, como uma lista geral das tabelas usadas na \texttt{PARTE\ II}, um glossário de termos, bibliografia geral, entre outras fontes de informação.

\hypertarget{como-estudar-por-esse-livro}{%
\section{Como estudar por esse livro?}\label{como-estudar-por-esse-livro}}

Simples! Siga a sequência que é apresentada.

\begin{itemize}
\tightlist
\item
  Estude a teoria (\texttt{PARTE\ I}), certifique-se de que entende os conceitos e é capaz de tomar decisões fundamentadas na teoria. Revise/relembre do Ensino Médio a matemática necessária (Conjuntos, Matriz, Álgebra, Teorema de Pitágoras, etc.).
\item
  Leia a fundamentação das bases de dados que usará subsequentemente (\texttt{PARTE\ II}). Mas lembre, você não precisa ser especialista nos grupos taxonômicos apresentados aqui como exemplo. O ecólogo trabalha com dados, e eles falam por si só.
\item
  Parta para a ação (\texttt{PARTES\ III\ e\ IV}), começe com os primeiros módulos do R, siga as instruções passo-a-passo, execute linha por linha de código, atente para os resultados.
\item
  Consulte os detalhes e referências apresentados na \texttt{PARTE\ V}.
\end{itemize}

\hypertarget{para-um-estudo-aprofundado}{%
\subsection{Para um estudo aprofundado}\label{para-um-estudo-aprofundado}}

\hypertarget{intro}{%
\chapter{Introdução ao R/RStudio}\label{intro}}

\hypertarget{resumo}{%
\subsection*{RESUMO}\label{resumo}}
\addcontentsline{toc}{subsection}{RESUMO}

Bem-vindo à disciplina de Ecologia Numérica. A base desta disciplina está no uso de técnicas multivariadas de classificação e ordenação, como são usadas em ecologia e gestão de recursos. Todos os textos e aulas foram elaborados para serem complementados pela Bibliografia Básica e pelo Material de Apoio (ver subsequentemente). Usaremos o software RStudio!, desenvolvido por RStudio Team (\protect\hyperlink{ref-RN358}{2022}) (\url{http://www.rstudio.com/}).

Embora seja necessário algum conhecimento matemático e uma base em bioestatística, estes assuntos não têm uma abordagem basicamente matemática e o material de apoio será usado para revisar temas de outras disciplinas. Os procedimentos são explicados em termos de fórmulas e algoritmos e o aluno obterá pelo menos uma compreensão intuitiva do que os métodos envolvem. Para aqueles com mais interesse nos temas tratados, a bibliografia básica oferece maior profundidade. A Ecologia Numérica é a ferramenta básica do biólogo no tocante à análise quantitativa de quaisquer tipos de dados. Ela cobre, além dos assuntos abordados
nessa disciplina, áreas de rápido desenvolvimento de análise temporal e espacial. Esperamos que, ao concluir esta disciplina, o texto seja de valor duradouro em seu trabalho profissional.

Os softwares disponíveis para análise multivariada são diversos. A maioria dos pacotes maiores tem algumas das rotinas padrão, mas um software especializado, como o R/RStudio, também é necessário para fazer o melhor uso dos avanços recentes em aplicações ecológicas. O R é um ambiente e uma linguagem de programação gratuita e de ``código aberto''. Veremos mais sobre isso subsequentemente.

Legendre e Legendre (\protect\hyperlink{ref-RN1278}{1998}) alertam o ecologista numérico sobre as armadilhas do uso de computadores - que \textbf{a computabilidade não garante (i) que os dados satisfaçam o método e (ii) que os resultados sejam interpretados corretamente em termos ecológicos.} A isso, infelizmente, deve ser adicionada a advertência adicional de que os computadores não negam a necessidade de verificar a exatidão dos cálculos.

A conclusão desta disciplina ajudará a evitar essas armadilhas, fornecendo a você as habilidades para aplicar essas técnicas com sucesso para o avanço da ciência e do gerenciamento de nossos recursos naturais. Espero que com a conclusão dessa disciplina o aluno possa evitar essas armadilhas e possua as habilidades para aplicar essas técnicas com sucesso para avançar na sua área de atuação.

\hypertarget{plano-de-estudo}{%
\section{Plano de estudo}\label{plano-de-estudo}}

O objetivo principal é familiarizar-se com o uso do RStudio para que você possa executar as operações básicas com confiança. Você aprenderá como instalar o software, importar e exportar dados, exibi-los em forma de planilha, modificá-los das principais formas e exibir informações básicas. Além disso, você também aprenderá a exportar resultados para permitir a apresentação e análises posteriores em outros pacotes estatísticos. Ao ler os artigos de revisão em Leitura Suplementar, você será apresentado a algumas das diversidades do trabalho multivariado e as críticas que têm recebido na ecologia, mas também um excelente exemplo de uma análise multivariada cuidadosamente descrita. Ele leva você através da configuração objetiva, seleção de caracteres e codificação de dados, em seguida, em várias análises e, em seguida, o desenho de conclusões apropriadas.

Portanto, para obter o máximo benefício, recomenda-se que você faça o seguinte:

● leia a o conteúdo Teórico inteiro sobre conceitos-chave minuciosamente;\\
● leia e estude todo o Material de Apoio disponível do Google Sala de Aula;\\
● complete os exemplos e tutorias passo-a-passo (quando houver);\\
● leia os artigos introdutórios na Leitura Suplementar (esse passo não é obrigatório nem necessário para completar o curso;\\
● participe e tire dúvidas no Chat da disciplina no Google Sala de Aula;\\
● complete os Exemplos aplicados;\\
● avalie sua compreensão dos materiais, usando os quizes, que são perguntas de múltipla escolha baseadas no Google Sala de Aula.

Conforme você avança ao longo da disciplina, concentre-se mais no fluxo de procedimentos para entrada (input) e modificação de dados, em vez de nos detalhes das opções e saídas (outputs). Eles sempre podem ser consultados quando necessário, aqui ou no sistema de ajuda do R/RStudo.

Os scripts do R/RStudio documentam o histórico de sua análise. Também é aconselhável, ao fazer uma análise, registrar notas adicionais sobre a interpretação da análise (usando {\#}). Isso também pode ser feito diretamente nos arquivos de resultados.

Ao completar cada Módulo o aluno deve ser capaz de:

● instalar o R e RStudio\\
● decidir a melhor forma de preparar seus dados em vários formatos (.xlsx, .csv, etc)\\
● configurar diretórios de projetos\\
● importar, modificar e exportar seus dados\\
● preparar um arquivo de nome de espécies\\
● particionar arquivos de dados em subconjuntos\\
● considerar o tratamento de valores faltantes\\
● exportar resultados para outros softwares para análise e apresentação\\
● salvar e documentar um arquivo de resultados

Ao estudar cada módulo, concentre-se mais no fluxo dos procedimentos para entrada e modificação de dados, em vez dos detalhes das opções. Estes podem sempre ser revisados quando necessário, aqui ou nos materiais de apoio. A janela do Console ou (``Work Area'') documenta o histórico de sua análise. Também é aconselhável ao fazer uma análise, registrar e fazer anotações adicionais sobre sua interpretação da análise etc. Isso pode ser feito diretamente na janela de \texttt{Scripts.}

\hypertarget{pruxe9-requisitos}{%
\section{Pré-requisitos}\label{pruxe9-requisitos}}

Familiaridade básica com computadores, tipos de arquivos (.xlsx, .docx, etc), processadores de texto e planilhas é necessária aqui. Espera-se que você seja capaz de compreender estatísticas univariadas básicas.

\hypertarget{principais-conceitos}{%
\section{Principais conceitos}\label{principais-conceitos}}

\hypertarget{o-ambiente-de-programauxe7uxe3o-do-r}{%
\subsection{O ambiente de programação do R}\label{o-ambiente-de-programauxe7uxe3o-do-r}}

O programa R oferece uma poderosa ferramenta para manipulação, análise e visualização de dados, e nos últimos anos tornou-se muito popular entre os ecologistas (e não apenas eles). Ele oferece grande liberdade e permite que você faça o que quiser com seus dados, o que não é o caso de softwares clicáveis como SAS, SPSS ou STATISTICA, entre outros. No entanto, ele também surge com uma curva de aprendizado íngreme, tendo em vista sua linguagem baseada em S/S-Plus, e um certo nível de frustração por conta de mensagens de erro frequentes. A menor imprecisão na digitação do código impede que ele seja executado, e gasto de tempo na busca do erro. Isso, em parte, estimula a ``copia-cola'' de scripts prontos, levando o usuário a executar analises sem saber o que está fazendo. Exceto pelo livro de Jean Valentin (\protect\hyperlink{ref-RN2759}{VALENTIN, 2000}), a grande maioria dos textos impressos em ecologia numérica são em inglês. Assim como inúmeros tutoriais em inglês pela internet (veja \href{https://www.davidzeleny.net/wiki/doku.php/recol:start}{Introduction to R for Ecologists de David Zelený}. Em português temos o excelente livro on-line \href{https://analises-ecologicas.com/}{Análises Ecológicas no R} de SILVA et al. (\protect\hyperlink{ref-RN2940}{2022}).

O R não é um software do tipo aplicativo, é um ambiente de programação com um conjunto integrado de ferramentas de software para manipulação de dados, cálculo e apresentação gráfica. A linguagem R foi elaborada inicialmente com o objetivo específico de análises de dados, na qual muitas técnicas estatísticas, clássicas e modernas, podem ser implementadas, entre outras coisas não tão estatísticas. Algumas dessas técnicas estão implementadas no ambiente básico do R (``R base''), mas muitas estão implementadas em pacotes adicionas (``packages'').

O ambiente R foi desenvolvido baseado na linguagem S, no final da década de 90 início dos anos 2000. A sua estrutura de código aberto (que vem da linguagem S) e de software público e gratuito atraiu um grande número de desenvolvedores, sendo que há hoje inúmeros pacotes para o R (veja \href{https://melina-leite.github.io/consultR/}{Introdução ao R para Análise de Dados em Consultoria Ambiental}).

Para um excelente minicurso introdutório sobre o R \href{https://www.youtube.com/watch?v=2mS-SzVWPZQ\&list=PLzWDDw1w8cTS4i_B49WOWtjngjcMqTruG\&index=1}{clique aqui}

\begin{quote}
``Uma das coisas mais importantes que você pode fazer é dedicar um tempo para aprender uma linguagem de programação de verdade {[}e aprender a base teórica e estatística{]}. Aprender a programar é como aprender outro idioma: exige tempo e treinamento, e não há resultados práticos imediatos. Mas se você supera essa primeira subida íngreme da curva de aprendizado, os ganhos como cientista são enormes. Programar não vai apenas livrar você da camisa de força dos pacotes estatísticos, mas também irá aguçar suas habilidades analíticas e ampliar os horizontes de modelagem ecológica e estatística.''

--- (Nicholas J. Gotelli em Gotelli \& Ellison 2011)
\end{quote}

Os softwares disponíveis para análise multivariada são diversificados. A maioria dos pacotes tem algumas das análises padrão, mas o R é necessário para fazer o melhor uso dos recentes avanços em aplicações ecológicas. A literatura, por outro lado, alerta o ecologista numérico para as armadilhas do uso de computadores. A computação não garante (i) que os dados satisfaçam o método, e (ii) que os resultados sejam interpretados corretamente em termos ecológicos. Para isso, é necessária uma base teórica forte além de que os computadores não negam a necessidade de verificar a correção dos cálculos. Uma das ferramentas mais populares para análise de dados no R é o RStudio, uma interface de desenvolvimento integrada (IDE) que oferece recursos avançados para desenvolvimento, depuração e gerenciamento de projetos em R. Embora o RStudio seja uma ferramenta poderosa para análise de dados, é importante lembrar que a análise multivariada não depende apenas do software utilizado, mas também de uma base teórica forte e de verificação da correção dos cálculos. Além disso, a literatura alerta que o uso de computadores não garante a adequação dos dados ao método e a interpretação correta dos resultados em termos ecológicos. Portanto, é importante combinar o uso de ferramentas como o RStudio com uma sólida formação teórica e verificação rigorosa dos resultados.

\hypertarget{o-que-uxe9-o-r-studio}{%
\section{O que é o R Studio?}\label{o-que-uxe9-o-r-studio}}

O RStudio é um ambiente integrado de desenvolvimento (IDE) para a linguagem de programação R, projetado para ajudar os usuários a trabalhar com dados de forma mais eficiente e produtiva (Figura \ref{fig:rm1fig1}). Ele oferece uma ampla variedade de recursos e ferramentas para análise de dados, incluindo edição de código, visualização de gráficos, depuração, gerenciamento de projetos e colaboração. O RStudio é gratuito e de código aberto, disponível para Windows, Mac e Linux, e é usado por uma comunidade crescente de cientistas de dados, pesquisadores, analistas e desenvolvedores em todo o mundo. Ele oferece suporte para várias bibliotecas e pacotes do R, incluindo o \texttt{tidyverse}, que é uma coleção popular de pacotes para manipulação de dados e visualização. Com o RStudio, os usuários podem facilmente importar, manipular e visualizar dados, executar análises estatísticas e criar visualizações gráficas interativas. Além disso, o RStudio oferece a possibilidade de publicar relatórios e apresentações dinâmicas em diversos formatos, como HTML, PDF e Word, facilitando a comunicação e colaboração entre os usuários.

\begin{figure}
\includegraphics[width=0.8\textwidth]{imagens/rm1.fig1} \caption{Interface do usuário do ambiente integrado de desenvolvimento (IDE) RStudio para a linguagem de programação R.}\label{fig:rm1fig1}
\end{figure}

O RStudio oferece diversas vantagens em relação ao uso do R base. Algumas dessas vantagens incluem:

\begin{enumerate}
\def\labelenumi{\arabic{enumi}.}
\tightlist
\item
  Interface amigável: o RStudio possui uma interface gráfica amigável e intuitiva, com recursos como realce de sintaxe, sugestões de código e facilidade na navegação entre arquivos, o que torna o processo de desenvolvimento mais fácil e produtivo.
\item
  Facilidade de gerenciamento de projetos: o RStudio oferece ferramentas para gerenciamento de projetos, que permitem aos usuários organizar e controlar arquivos e scripts em um ambiente centralizado.
\item
  Depuração eficiente: o RStudio oferece recursos de depuração integrados, como pontos de interrupção e rastreamento de variáveis, que facilitam a identificação e correção de erros no código.
\item
  Suporte para colaboração: o RStudio permite que múltiplos usuários colaborem em um projeto, com recursos como controle de versão integrado e ferramentas para compartilhamento de código.
\item
  Integração com outras ferramentas: o RStudio integra-se com diversas ferramentas populares de análise de dados, como o Git e o LaTeX, permitindo uma maior eficiência e produtividade no processo de análise e geração de relatórios.
\end{enumerate}

Essas são apenas algumas das vantagens que o RStudio oferece em relação ao R. É importante notar que, embora o RStudio seja uma ferramenta poderosa para análise de dados, ele não substitui a necessidade de conhecimento em R e estatística para obter resultados precisos e significativos.

\hypertarget{o-essencial}{%
\section{O essencial}\label{o-essencial}}

\hypertarget{licenuxe7a-de-software}{%
\subsection{Licença de software}\label{licenuxe7a-de-software}}

A licença de uso do RStudio depende da versão do software que está sendo utilizada. O RStudio oferece duas versões principais: a versão gratuita e a versão paga, conhecida como RStudio Pro. A versão gratuita do RStudio é licenciada sob a licença AGPLv3, o que significa que é um software de código aberto e livremente distribuível. Isso permite que os usuários baixem, modifiquem e distribuam o código-fonte do RStudio gratuitamente. No entanto, o uso comercial do RStudio sob a licença AGPLv3 pode ter algumas limitações, como a obrigatoriedade de disponibilizar o código-fonte modificado para os usuários. Já a versão paga do RStudio, o RStudio Pro, requer uma licença comercial para uso em ambientes empresariais ou acadêmicos. A licença comercial do RStudio Pro oferece recursos adicionais, como suporte técnico e atualizações regulares do software. Além disso, a licença comercial permite que os usuários utilizem o RStudio Pro em um ambiente sem a obrigação de divulgar o código-fonte modificado. É importante lembrar que as licenças de uso do RStudio podem mudar com o tempo, portanto, é recomendado que os usuários verifiquem as informações de licenciamento diretamente no site do RStudio antes de utilizar o software.

\hypertarget{interface-do-usuuxe1rio}{%
\subsection{Interface do usuário}\label{interface-do-usuuxe1rio}}

Uma interface de programa estatístico é uma ferramenta que permite que os usuários interajam com uma linguagem de programação de maneira mais amigável e visual. Essas interfaces geralmente apresentam uma interface gráfica do usuário (GUI) que oferece um ambiente integrado para a análise de dados. Os usuários podem importar dados, criar visualizações, executar análises estatísticas e gerar relatórios. As interfaces de programa estatístico podem variar desde aplicativos de desktop autônomos até plug-ins para aplicativos de planilha, e muitos são desenvolvidos para atender a necessidades específicas de usuários em áreas como finanças, pesquisa médica, ciências sociais e outras. Essas interfaces tornam a análise de dados mais acessível a uma gama mais ampla de usuários, independentemente de sua experiência em programação ou estatística.

Linguagens de programação estatística de código aberto tem sido amplamente usadas em análises de dados e pesquisa em todo o mundo. Suas interfaces ou GUIs, embora poderosas, podem ser um pouco intimidadoras para novos usuários, uma vez que é necessário habilidades de programação básicas.

\hypertarget{interface-do-usuuxe1rio-do-rstudio}{%
\subsection{Interface do usuário do RStudio}\label{interface-do-usuuxe1rio-do-rstudio}}

A interface de usuário do RStudio é composta por várias janelas e painéis que permitem a interação com o R de forma organizada e intuitiva. A janela principal é dividida em quatro painéis (Figura \ref{fig:R-eng}):

\begin{figure}
\includegraphics[width=0.8\textwidth]{imagens/interfaceR-eng} \caption{Interface do usuário do ambiente integrado de desenvolvimento (IDE) RStudio para a linguagem de programação R com os termos em inglês e em português. O operador de atribuição ( <- ), um dos operadores mais utilizados no R pode ser chamado pela combinação alt + - (alt e a tecla menos). De: R – software for data analysis | Ciência Prática (wordpress.com) https://cienciapratica.wordpress.com/2015/07/20/r-software-for-data-analysis/}\label{fig:R-eng}
\end{figure}

\textbf{Editor de código}: é onde o usuário pode criar, editar e salvar scripts R. O editor fornece recursos como destaque de sintaxe, indentação automática, autocompletar e outras ferramentas úteis para a programação.

\textbf{Console}: é onde o usuário pode digitar comandos diretamente no R e ver a saída correspondente. É aqui que a maior parte da interação com o R ocorre.

\textbf{Arquivos/Plots/Pacotes/Ajuda}: este painel pode ser dividido em quatro subpainéis, cada um dos quais mostra uma guia diferente. O painel de arquivos permite a navegação pelos arquivos e pastas do computador. O painel de plots mostra gráficos gerados no R. O painel de pacotes permite a instalação, atualização e carregamento de pacotes do R. O painel de ajuda exibe informações sobre funções e pacotes do R.

\textbf{Ambiente/Histórico/Git/Conexões}: este painel também pode ser dividido em quatro subpainéis. O painel de ambiente exibe informações sobre variáveis e funções criadas no R. O painel de histórico mostra os comandos recentes digitados no console. O painel de Git permite que o usuário integre o RStudio com um repositório Git para controle de versão. O painel de conexões permite a conexão com bancos de dados e outros recursos externos.

A interface de usuário do RStudio também inclui uma barra de ferramentas na parte superior da janela, que fornece acesso rápido a comandos e recursos comuns. Além disso, há uma barra de menus padrão que permite o acesso a todas as opções e recursos do RStudio. No geral, a interface de usuário do RStudio é altamente personalizável e pode ser configurada para atender às necessidades específicas do usuário.

\hypertarget{projetos}{%
\subsection{Projetos}\label{projetos}}

Um `Projeto' no RStudio é um conjunto de dados e arquivos de resultados associados entre si, junto com as opções e configurações no momento em que o Projeto foi salvo pela última vez. Quando você abre um projeto, esses arquivos são abertos e suas opções e configurações são restauradas da última vez. Você só pode ter um projeto aberto por vez, mas pode alternar facilmente entre eles abrindo um projeto salvo. Um projeto pode ter mais de um arquivo de dados associado. Se você tiver um projeto aberto e abrir ou criar qualquer arquivo nele, o projeto será considerado modificado. Ao sair do RStudio ou abrir um novo projeto, será perguntado se deseja salvar o projeto modificado. Para melhor gerenciamento de dados, você pode manter seus arquivos de projeto em seus próprios diretórios, separados do software RStudio. Ele lembra o nome do caminho completo para cada projeto. Os detalhes de como o RStudio gerencia diretórios e definições de gerenciamento de arquivos são fornecidos no sistema de Ajuda. Você pode redefinir o diretório padrão (working directory) para que o RStudio navegue até o diretório que você escolheu e abra os arquivos ou projetos que você encontrará no diretório escolhido.

\hypertarget{nomes-de-arquivos}{%
\subsection{Nomes de arquivos}\label{nomes-de-arquivos}}

Alguns dos tipos de extensão de arquivos que podem ser criados ou manipulados no RStudio incluem:

● Arquivos de script: São arquivos de texto contendo código R, com extensões de arquivo como .R, .Rmd, .Rnw, .Rhistory, entre outros.\\
● Arquivos de dados: São arquivos contendo dados que podem ser lidos e manipulados com o R, como .csv, .txt, .xls, .xlsx, .rda, .rdata, entre outros.\\
● Arquivos de pacotes: São arquivos compactados que contêm código, documentação e dados necessários para um pacote do R, com extensão .tar.gz.\\
● Arquivos de imagem: São arquivos de imagem gerados a partir de gráficos criados no R, com extensões como .pdf, .png, .jpeg, .bmp, entre outros.\\
● Arquivos de projeto: São arquivos que contêm todas as informações necessárias para reproduzir um projeto R, incluindo arquivos de script, dados e outros recursos, com extensão .Rproj.

Esses são apenas alguns dos tipos de arquivos que podem ser usados ou gerados no RStudio. O RStudio também oferece suporte a muitos outros tipos de arquivos, como arquivos de ajuda, arquivos de instalação de pacotes, entre outros.

\hypertarget{preparauxe7uxe3o-dos-arquivos}{%
\section{Preparação dos arquivos}\label{preparauxe7uxe3o-dos-arquivos}}

Arquivos de dados multivariádos são geralmente maiores e mais complexos do que outros, portanto, é necessário cuidado extra. Eles podem ser preparados de duas maneiras principais. A primeira forma é como a matriz completa de valores (.xlsx), onde cada linha mostra os valores de cada coluna. Isso é bom quando há poucos zeros. É assim que uma planilha funciona. A outra maneira, preferida pelos ecologistas para seus dados de espécies, é registrar apenas os valores diferentes de zero para cada espécie. Cada linha, representando uma unidade de amostragem, é então lida como uma lista de verificação ou registro quadrat. Este formato compacto (.csv ou tab delimited) é muito mais eficiente quando o tamanho da matriz de dados é muito grande. O RStudio considera as linhas no arquivo de dados como as unidades de amostragem e as colunas como as variáveis. Dependendo de quantos existem, pode ser mais conveniente inserir seus dados ao contrário e, em seguida, transpô-los para o RStudio. Veremos como fazer isso mais tarde.

\hypertarget{formatos-de-arquivo-de-dados}{%
\subsection{Formatos de arquivo de dados}\label{formatos-de-arquivo-de-dados}}

Você pode usar qualquer editor de texto, processador de texto ou planilha para construir sua planilha de dados, embora seja recomendado o uso do Excel. No entanto, a entrada de dados no RStudio deve ser feita dentro do ambiente R por linha de comando. Os dados podem ser preparados no formato de planilha (spreadsheet), mas você deve seguir o formato cuidadosamente para saber onde e como colocar as informações nas células da planilha.

A seguir é apresentada a estrutura básica de uma matriz comunitária e como importar um arquivo XLS/XLSX para o RStudio usando a programação R (Figura \ref{fig:ppbio2}). Como tudo no R, existem várias formas para se ler os arquivos Excel. Usaremos primeiro pacote readxl (que já vem instalado com o R base) (veja \url{https://www.tutorialkart.com/r-tutorial/r-read-excel-xls-xlsx/})

\begin{figure}
\includegraphics[width=0.8\textwidth]{imagens/rm1.fig3} \caption{Usaremos uma matriz multivariada (sítios x espécies, matriz comunitária) do PPBio. Identifique na figura abaixo os elementos que compõem uma matriz multivariada}\label{fig:ppbio2}
\end{figure}

\hypertarget{o-arquivo-de-espuxe9cies}{%
\subsection{O arquivo de espécies}\label{o-arquivo-de-espuxe9cies}}

Ao usar o formato de dados compacto, você viu que as variáveis (colunas) são numeradas, não nomeadas. Portanto, para aplicar um nome a eles, você precisa criar um arquivo de texto separado que contenha os códigos e nomes. Isso é mais útil quando as variáveis são espécies, uma vez que você pode ter um único arquivo de espécie que se aplica a todos os seus arquivos de dados e aos projetos.

O arquivo de espécies também inclui o nome completo da espécie, que é usado em alguns arquivos de resultados. Esta facilidade é obviamente muito útil para registro de campo quando você pode identificar a maioria das espécies. Leve a lista de espécies com você e apenas escreva o número do código em sua folha de dados. Observe também os seguintes guias de formato e regras para o arquivo de espécies.

● Nomes de código curtos são recomendados para reduzir a superimposição de nomes em quaisquer resultados gráficos.\\
● Os registros podem estar em qualquer ordem neste arquivo. (No entanto, você achará útil manter as espécies em ordem alfabética porque a maior parte dos outputs é ordenada pela sequência de espécies em seu arquivo de espécies.).

\hypertarget{formato-de-valores-separados-por-vuxedrgula-.csv}{%
\subsection{Formato de valores separados por vírgula (.csv)}\label{formato-de-valores-separados-por-vuxedrgula-.csv}}

Um arquivo ASCII de valores separados por vírgula (formato *.csv) também pode ser usado para importar / exportar dados do RStudio. Este formato de arquivo requer uma formatação muito cuidadosa.

\hypertarget{editando-a-matriz-de-dados}{%
\subsection{Editando a matriz de dados}\label{editando-a-matriz-de-dados}}

O RStudio fornece uma pletora de formas para edição de matrizes. Isso é mais útil para pequenas correções. Essas edições são salvas no formato de planilha para que você possa alternar entre a edição do arquivo e o software de planilha. Use a planilha para uma edição mais extensa de suas matrizes. Você pode fazer operações do tipo matriz facilmente na janela principal. A exclusão de linhas ou colunas, transformações e transposição e multiplicação de matrizes estão disponíveis com linhas de comando.

\hypertarget{arquivos-de-resultados}{%
\subsection{Arquivos de resultados}\label{arquivos-de-resultados}}

Para manter um registro completo de suas análises, é melhor anexar arquivos de resultados sucessivos a um arquivo de scripts nomeado. Faça isso salvando o arquivo de resultado e, então, após cada operação sucessiva, selecione Salvar script. Você também deve adquirir o hábito de inserir um título descritivo significativo (usando {\#}) para cada análise. Isso fornece documentação adicional de sua análise e aparece no script. As tabelas no arquivo de resultado são construídas com espaços, não tabulações. Portanto, o alinhamento das colunas no RStudio e nos processadores de texto só é obtido selecionando uma fonte de pitch-fixo, como Courier. Sendo arquivos de texto simples, você também pode pesquisar ou editar arquivos de resultados e scripts usando o R ou qualquer editor de texto.

Você pode exportar seus arquivos de resultado para sua planilha.

\hypertarget{gruxe1ficos}{%
\subsection{Gráficos}\label{gruxe1ficos}}

Você pode salvar imagens gráficas do RStudio como metarquivos do Windows (.wmf) ou metarquivos aprimorados (*.emf), usando linhas de comando ou a interface gráfica. Um gráfico salvo dessa forma é uma imagem vetorial que pode ser aberta em um pacote gráfico ou colada em processadores de texto e redimensionada ou editada sem perda de detalhes. No entanto, certifique-se de não alterar a proporção de aspecto ao redimensionar. Se o formato *.emf não for lido corretamente por seu outro software, tente salvar o gráfico no formato .wmf.

\hypertarget{conclusuxe3o}{%
\section{Conclusão}\label{conclusuxe3o}}

Foram apresentadas as três primeiras etapas da sequência abaixo, incluindo a preparação da importantíssima matriz de dados e como importá-la e exportá-la. Resumir os dados e decidir sobre o tratamento dos valores ausentes é um primeiro passo para a familiarização com conjuntos de dados multivariados.

\hypertarget{sequuxeancia-tuxedpica-na-preparauxe7uxe3o-e-anuxe1lise-de-dados}{%
\section{Sequência típica na preparação e análise de dados}\label{sequuxeancia-tuxedpica-na-preparauxe7uxe3o-e-anuxe1lise-de-dados}}

Para realizar uma análise multivariada, precisamos seguir uma sequência de etapas analíticas, como a seguir:

\begin{enumerate}
\def\labelenumi{\arabic{enumi}.}
\item
  Entrada de dados e verificação de erros\\
  Crie os arquivos de dados e espécies usando sua planilha ou editor de texto ou banco de dados.
  Verifique e arquive uma cópia do seu arquivo de dados limpo.
\item
  Crie o arquivo de planilha .xlsx\\
  Faça isso salvando a planilha ou importando outros formatos para o RStudio. Verifique se há erros de importação. Faça resumos para reproduzir seus registros de unidade de amostragem para dupla verificação.
\item
  Examine a estrutura básica dos dados\\
  Use os comandos de sumários para examinar os valores de linha e coluna
\item
  Faça uma Análise de Outliers\\
  Procure outliers. Examine as relações entre curvas de espécies e espécie-área.
\item
  Decida se/e como modificar os dados\\
  Mesclar, particionar, transformar, podar, transpor matriz etc.
\item
  Fazer análises, ordenar e classificar (grupos)\\
  Cada análise pode exigir o carregamento de várias matrizes principais e secundárias
\item
  Inspecionar os resultados (gráficos e arquivos de resultados)
\end{enumerate}

Salve e imprima os resultados

\hypertarget{material-de-apoio}{%
\section{Material de apoio}\label{material-de-apoio}}

Recursos na Web Apostilas
RStudio Desktop - Posit
The Comprehensive R Archive Network (r-project.org)
Landeiro-Introducao.pdf (r-project.org)

\hypertarget{apuxeandices}{%
\section*{Apêndices}\label{apuxeandices}}
\addcontentsline{toc}{section}{Apêndices}

\hypertarget{sites-consultados}{%
\section*{Sites consultados}\label{sites-consultados}}
\addcontentsline{toc}{section}{Sites consultados}

Como importar dados do Excel para o R: {[}\url{https://youtu.be/U6ksXvvY6Q0}{]}\\
Como exportar dados do R para o Excel: {[}\url{https://youtu.be/a7EJE_2mtGk}{]}

\hypertarget{leitura-suplementar}{%
\section*{Leitura suplementar}\label{leitura-suplementar}}
\addcontentsline{toc}{section}{Leitura suplementar}

A multivariate analysis on spatiotemporal evolution of Covid-19 in Brazil (\url{https://doi.org/10.1016/j.idm.2020.08.012})\\
Genetic markers in the playground of multivariate analysis (\url{https://www.nature.com/articles/hdy2008130})

\hypertarget{referuxeancias}{%
\section*{Referências}\label{referuxeancias}}
\addcontentsline{toc}{section}{Referências}

\hypertarget{part-bases-de-dados}{%
\part{BASES DE DADOS}\label{part-bases-de-dados}}

\hypertarget{bases}{%
\chapter{Bases de Dados. Dados do PPBio Semiárido}\label{bases}}

\hypertarget{exemplos-de-conjunto-de-dados-ecoluxf3gicos-do-programa-de-pesquisa-em-biodiversidade-ppbio-do-semiuxe1rido}{%
\subsection*{Exemplos de conjunto de dados ecológicos do Programa de Pesquisa em Biodiversidade (PPBio) do Semiárido}\label{exemplos-de-conjunto-de-dados-ecoluxf3gicos-do-programa-de-pesquisa-em-biodiversidade-ppbio-do-semiuxe1rido}}
\addcontentsline{toc}{subsection}{Exemplos de conjunto de dados ecológicos do Programa de Pesquisa em Biodiversidade (PPBio) do Semiárido}

\hypertarget{resumo-1}{%
\subsection*{RESUMO}\label{resumo-1}}
\addcontentsline{toc}{subsection}{RESUMO}

Para entender a distribuição das espécies de peixes e seu uso de habitat, uma série de variáveis ambientais foram avaliadas como preditores da composição e riqueza da assembleia de peixes em sistemas aquáticos tropicais semiáridos. Nós pesquisamos a composição de espécies de assembleias de peixes em sistemas aquáticos semiáridos e estabelecemos seu grau de associação com a estrutura do habitat aquático. Os locais consistiam em trechos de riachos com fluxo de água superficial, poças temporárias isoladas e reservatórios artificiais (açudes). A amostragem de peixes foi realizada em quatro ocasiões durante as estações chuvosa (abril e junho de 2006) e seca (setembro e dezembro de 2006).

\textbf{Palavras-chave}: rios intermitentes, reservatórios, conservação, composição de substratos.

\hypertarget{sobre-os-dados-do-ppbio}{%
\section{Sobre os dados do PPBio}\label{sobre-os-dados-do-ppbio}}

Usaremos ao longo desse livro dados que fazem parte de um estudo mais amplo sobre ecologia de rios do semiárido, coletados no Programa de Pesquisa em Biodiversidade - PPBio (Veja \href{https://ppbio.inpa.gov.br/Sobre}{Programa de Pesquisa em Biodiversidade -- PPBio}). Parte desses dados está armazenada em planilhas de Excel \texttt{ppbio**.xlsx} (Figura \ref{fig:200ppbio}). Essas matrizes de dados são descritas na Tabela \ref{tab:200arqs}. As planilhas \texttt{ppbio**.xlsx} contém vários tipos de dados arranjados em matrizes \texttt{n\ x\ m} que incluem dados de abundância de espécies em diferentes unidades amostrais (UA's), dados da estrutura do habitat físico, e variáveis em escala de bacia hidrográfica, dados de contagem de indivíduos ajustados para Captura Por Unidade de Esforço (CPUE), etc (Figura \ref{fig:200ppbio}).

\begin{figure}
\includegraphics[width=0.8\textwidth]{imagens/ppbio} \caption{Parte da planilha de dados brutos do PPBio.}\label{fig:200ppbio}
\end{figure}

Por exemplo, essa é a \textbf{matriz bruta de dados}, porque os valores ainda não foram ajustados para os valores de Captura Por Unidade de Esforço (CPUE), nem foram relativizados ou transformados. Outros tipos de arquivos existem sobre esses dados (Tabela \ref{tab:200arqs}).

Várias das espécies nessa matriz tem grande importância ecológica, como é o caso de \emph{Astyanax bimaculatus}\footnote{A etimologia do gênero \emph{Astyanax} vem da mitologia Grega. Heitor personagem da ``Ilíada'', tinha um filho chamado Astíanax.} (Figura \ref{fig:200asbim}), que é muito comum em rios intermitentes e serve de alimento para predadores maiores (\protect\hyperlink{ref-RN2908}{SILVA; DUARTE; MEDEIROS, 2018}) como a espécie \emph{Hoplias malabaricus}\footnote{Do Grego, \emph{hoplon}, arma ou armadura, em referência aos dentes caniniformes muito desenvolvidos, e forte estrutura óssea na cabeça.} (Figura \ref{fig:200hmala}) (\protect\hyperlink{ref-RN2724}{SILVA et al., 2010}).

\begin{figure}
\includegraphics[width=0.8\textwidth]{imagens/spp/Asbim_u4} \caption{*Astyanax bimaculatus*, a espécie mais comum da matriz de dados ppbio. Peru, by Eakins, R. Fonte: <https://www.fishbase.se/summary/Astianax-bimaculatus.html>}\label{fig:200asbim}
\end{figure}

\begin{figure}
\includegraphics[width=0.8\textwidth]{imagens/spp/Homal_u6} \caption{*Hoplias malabaricus*, espécie que cresce para se tornar um importante predador. Brazil, by Roselet, F.F.G. Fonte: <https://www.fishbase.se/summary/Hoplias-malabaricus.html>}\label{fig:200hmala}
\end{figure}

Outras espécies como \emph{Apareiodon hasemani}\footnote{A etimologia do nome \emph{Apareiodon} vem do Grego, \emph{a}, sem, \emph{pareia}, lateral ou bochecha, e \emph{odous} dentição, em referência a ausência de dentes laterais no aparato bucal dessa espécie.''} (Figura \ref{fig:200apdavis} tem importância trófica por estar na base da cadeia alimentar, enquanto espécies da família Loricariidae, como \emph{Pseudancistrus genisetiger}\footnote{A etimologia do nome \emph{Pseudancistrus} vem do Grego, \emph{pseudes}, falso, e \emph{agkistron}, gancho, em referência a falsos ganchos presentes na cabeça em algumas espécies do gênero.} (Figura \ref{fig:200pgen}), tem importância para taxonomia.

\begin{figure}
\includegraphics[width=0.8\textwidth]{imagens/spp/Apdav_u0} \caption{*Apareiodon sp.*, importante espécie bentopelágica das bacias dos rios Jaguaribe e Paraíba. Brazil, by Ramos, T.P.A. Fonte: <https://www.fishbase.se/summary/Apareiodon-davisi.html>}\label{fig:200apdavis}
\end{figure}

\begin{figure}
\includegraphics[width=0.8\textwidth]{imagens/spp/genis} \caption{*Pseudancistrus genisetiger*, uma espécie endêmica das bacias hidrográfcas do nordeste. By Medeiros, E.S.F. Fonte: Arquivo pessoal}\label{fig:200pgen}
\end{figure}

As planilhas \texttt{ppbio**.xlsx} contém o delineamento amostral de um dos estudos do Projeto PPBio (Figura \ref{fig:200del}. Nas linhas são apresentadas as abreviações dos nomes das unidades amostrais (UA's) e nas colunas são apresentados os nomes abreviados das espécies - temos portando uma matriz comunitária (Tabela \ref{tab:200arqs}). No corpo da planilha temos os valores para o tipo de dados amostrado. Quantitativo, semi-quatitativo ou qualitativo.\\
Qual desses tipos de dados você acha que é apresentado na planilha?

\begin{figure}
\includegraphics[width=0.8\textwidth]{imagens/rm1.fig3} \caption{Associação entre a planilha de dados brutos do PPBio e o delineamento amostral do estudo.}\label{fig:200del}
\end{figure}

\hypertarget{arqs}{%
\section{Arquivos disponíveis}\label{arqs}}

A seguir, apresento uma tabela com as informações essenciais sobre as matrizes de dados que serão utilizadas ao longo deste livro (Tabela \ref{tab:200arqs}). Nela, estão descritos os diferentes tipos de matrizes, suas finalidades analíticas e o tipo de dado para cada uma delas. Essas informações servirão de referência para compreender a origem, a estrutura e o tratamento recomendado para os dados empregados nas análises subsequentes. Você pode baixar essas matrizes clicando no link para o arquivo em \texttt{.xlsx} na coluna \texttt{Arquivo}. Ao vizualizar a matriz que deseja baixar, clique em \texttt{Arquivo} \textgreater{} \texttt{Baixar} \textgreater{} \texttt{Microsoft\ Excel\ (.xlsx)}. Atente para a pasta onde seu computador baixa o arquivo desejado, você vai precisar dessa informação depois.

\begin{verbatim}
## Warning in table_info$align_vector[column] <-
## unlist(lapply(table_info$align_vector_origin[column], : número de itens para
## para substituir não é um múltiplo do comprimento do substituto
\end{verbatim}

\begin{table}

\caption{\label{tab:200arqs}Matrizes disponíveis para análises, com suas descrições e tipos de dados recomendados.}
\centering
\begin{tabular}[t]{>{\raggedright\arraybackslash}p{2cm}>{\raggedright\arraybackslash}p{2cm}>{\raggedright\arraybackslash}p{7cm}>{\raggedright\arraybackslash}p{5cm}}
\toprule
Arquivo (.xlsx) & Tipo de matriz & Descrição & Tipo de dados\\
\midrule
{}[ppbio06c-peixes](https://docs.google.com/spreadsheets/d/1Olyew6L3SCFeSXWJC54Ak3MKmgHD3sTK/edit?usp=drive\_link\&ouid=101766125969589673518\&rtpof=true\&sd=true) & Matriz comunitária & O arquivo ppbio06 traz os dados brutos que serão usados nas análises. A matriz de dados brutos contendo 26 localidades em estações do ano diferentes (objetos) x 35 espécies (atributos), antes de qualquer modificação. & Contagens de indivíduos com alta amplitude de variação, sugerido uso de matriz relativizada.\\
{}[ppbio06p-amb](https://docs.google.com/spreadsheets/d/1V\_QxcorksaSOa40Uzj1z2jy02WgKeFTx/edit?usp=drive\_link\&ouid=101766125969589673518\&rtpof=true\&sd=true) & Matriz ambiental & O arquivo ppbio06h traz os dados brutos que serão usados nas análises. A matriz de dados brutos contendo 26 localidades em estações diferentes (objetos) x 35 variáveis ambienteis (atributos) medidas em diferentes escalas espaciais, antes de qualquer modificação. & Unidades de medição diferentes (cm, m, °C, mg/L, etc.), com uma alta amplitude de variação, sugerido uso de matriz transformada e/ou reescalada.\\
{}[ppbio06-grupos](https://docs.google.com/spreadsheets/d/1H52eBBxZ1k-6Xgrse87etq6mq3mMUTZi/edit?usp=drive\_link\&ouid=101766125969589673518\&rtpof=true\&sd=true) & Matriz de grupos & O arquivo ppbio06 traz os dados brutos que serão usados nessa análise. A matriz de dados brutos contendo 26 locais/ocasiões (objetos) x 35 espécies (atributos), antes de qualquer modificação. & Contagens de indivíduos com alta amplitude de variação, sugerido uso de matriz relativizada.\\
ppbio06cpue & Matriz comunitária & O arquivo ppbio06cpue traz os valores depois de terem sidos ajustados pela Captura Por Unidade de Esforço (CPUE), onde o número de indivíduos de cada espécie em uma determinada UA é dividido pelo esforço de captura daquela UA. Isso significa que os dados foram relativizados pela CPUE. A matriz de dados brutos contendo 26 localidades em estações do ano diferentes (objetos) x 35 espécies (atributos), antes de qualquer modificação. & Densidades de indivíduos (no. de indivíduos por Unidade de Esforço de Captura) com alta amplitude de variação, sugerido uso de matriz relativizada.\\
\bottomrule
\end{tabular}
\end{table}

\hypertarget{codificauxe7uxe3o-das-variuxe1veis}{%
\subsection{Codificação das variáveis}\label{codificauxe7uxe3o-das-variuxe1veis}}

Os arquivos da base de dados do projeto são fornecidos em formato Excel (.xlsx). Por exemplo, o arquivo \texttt{ppbio06-*.xlsx}, traz os dados brutos que serão usados nas análises. A matriz de dados brutos contem mais de 20 localidades (m=linhas ou objetos) em estações do ano diferentes, e cerca de 35 espécies (n=colunas ou atributos), antes de qualquer modificação. Portando é uma matriz bruta. Os valores são contagens de indivíduos, e apresentam uma alta amplitude de variação, portanto, o uso de uma matriz relativizada é sugerido (Figura \ref{tab:200arqs}). Nos nomes dos objetos e dos atributos são codificados de acordo com a tabela mostrada na Figura \ref{fig:200codes}.

\begin{figure}
\includegraphics[width=0.8\textwidth]{imagens/bases/com-codes} \caption{Codificação para as variáveis, espécies de peixes, sítios de amostragem e período de amostragem}\label{fig:200codes}
\end{figure}

\hypertarget{abrev}{%
\subsection{Abreviações}\label{abrev}}

No interesse de sistematizar o código R das várias matrizes que são comumente usadas em uma AMD, a tabela \ref{tab:200abr}, a seguir, resume seus tipos e abreviações.

\begin{verbatim}
## Warning in table_info$align_vector[column] <-
## unlist(lapply(table_info$align_vector_origin[column], : número de itens para
## para substituir não é um múltiplo do comprimento do substituto
\end{verbatim}

\begin{table}

\caption{\label{tab:200abr}Nomenclatura das matrizes em AMD em relação aos atributos das colunas e suas abreviações nos códigos do R.}
\centering
\begin{tabular}[t]{>{\raggedright\arraybackslash}p{2cm}>{\raggedright\arraybackslash}p{5cm}>{\raggedright\arraybackslash}p{2cm}}
\toprule
Nome & Atributos (colunas) & Abreviação no R\\
\midrule
Matriz comunitaria & Os atributos são táxons ou OTU's (Unidades Taxonômicas Operacionais) (ex. espécies, gêneros, morfotipos) & m\_com\\
Matriz ambiental & Os atributos são dados ambientais e variáveis  físicas e químicas (ex. pH, condutividade, temperatura) & m\_amb\\
Matriz de habitat & Os atributos são elementos da estrutura do habitat (ex. macróficas, algas, pedras, lama, etc) & m\_hab\\
Matriz bruta & Os atributos ainda não receberam nenhum tipo de tratamento estatísco (valores brutos, como coletados) & m\_bruta\\
Matriz transposta & Os atributos foram transpostos para as linhas & m\_t\\
\addlinespace
Matriz relativizada & Os atributos foram relativizados por um critério de tamanho ou de variação (ex. dividir os valores de cada coluna pela soma) & m\_rel, m\_relcol, m\_rellin\\
Matriz transformada & Foi aplicado um operador matemático a todos os atributos (ex. raiz quadrada, log) & m\_trns, m\_log10, m\_asrq\\
Matriz de distâncias & Matriz de m x m similaridades ou de distâncias (ex. Euclidiana, Manhattan, Bray-Curtis, etc) & m\_dists, m\_euclid, m\_bray\\
Matriz de trabalho & Qualquer matriz que seja o foco da análise atual (ex. comunitária, relativizada, etc) & m\_trab\\
Matriz particionada & Foram removidas linhas ou colunas (ex. linhas que são outliers e espécies zeradas) & m\_part\\
\addlinespace
Base de dados & Arquivo do Excel planilhado a partir de dados de campo ou de laboratório. Será manejada e particionada no R, para criar a Matriz bruta & ppbio06.xlsx, zoorebio.xlsx, bentos06.xlsx\\
\bottomrule
\end{tabular}
\end{table}

\hypertarget{referuxeancias-1}{%
\section*{Referências}\label{referuxeancias-1}}
\addcontentsline{toc}{section}{Referências}

\hypertarget{basepeixes}{%
\chapter{Base de Dados I - Peixes}\label{basepeixes}}

\hypertarget{estrutura-da-comunidade-peixes-no-semiuxe1rido}{%
\subsection*{Estrutura da Comunidade: Peixes no semiárido}\label{estrutura-da-comunidade-peixes-no-semiuxe1rido}}
\addcontentsline{toc}{subsection}{Estrutura da Comunidade: Peixes no semiárido}

\hypertarget{apresentauxe7uxe3o-1}{%
\subsection*{Apresentação}\label{apresentauxe7uxe3o-1}}
\addcontentsline{toc}{subsection}{Apresentação}

Exemplo de conjunto de dados ecológicos do Programa de Pesquisa em Biodiversidade (PPBio) do Semiárido. Ano de 2006. Peixes\footnote{Projeto financiado pelo CNPq/UEPB/DCR Proc.350082/2006-5, UEPB/FAPESQ Proc.68.0006/2006.0 e Projeto de Pesquisa em Biodiversidade do Semiárido (PPBio Semi-Árido).}

Autor:\\
- Prof.~Elvio S. F. Medeiros\\
- Laboratório de Ecologia\\
- Universidade Estadual da Paraíba\\
- Campus V, João Pessoa, PB

Para entender a distribuição das espécies de peixes e seu uso de habitat, uma série de variáveis ambientais foram avaliadas como preditores da composição e riqueza da assembleia de peixes em sistemas aquáticos tropicais semiáridos. Nós pesquisamos a composição de espécies de assembleias de peixes em sistemas aquáticos semiáridos e estabelecemos seu grau de associação com a estrutura do habitat aquático. Os locais consistiam em trechos de riachos com fluxo de água superficial, poças temporárias isoladas e reservatórios artificiais (açudes). A amostragem de peixes foi realizada em quatro ocasiões durante as estações chuvosa (abril e junho de 2006) e seca (setembro e dezembro de 2006).

\textbf{Texto modificado de MEDEIROS et al. (\protect\hyperlink{ref-RN1965}{2024})}: Environmental variables as predictors of fish community composition in semiarid aquatic systems. Acta Limnologica Brasiliensia, 2024, vol.~36, e4. \url{https://doi.org/10.1590/S2179-975X3023}. ISSN 2179-975X on-line version. Tradução: Google Translator.

Elvio Sergio Figueredo Medeiros\textsuperscript{1}, Marcio Joaquim da Silva\textsuperscript{2}, Telton Pedro Anselmo Ramos\textsuperscript{2} e Robson Tamar Costa Ramos\textsuperscript{1,3}

\textsuperscript{1} Grupo Ecologia de Rios do Semiárido, Universidade Estadual da Paraíba, Depto. de Biologia. Campus V. CEP 58070-450 João Pessoa - PB. Brazil. \textsuperscript{2} Universidade Federal do Rio Grande do Norte, Depto. de Biologia, Centro de Biociências, CEP 59078-900 -- Natal - RN. \textsuperscript{3} Laboratório de Sistemática e Morfologia de Peixes, Universidade Federal da Paraíba, Depto. de Sistemática e Ecologia, CCEN, CEP 58059-900, João Pessoa - PB.

\begin{itemize}
\tightlist
\item
  Autor correspondente: Dr.~Elvio Medeiros \url{https://orcid.org/0000-0002-7472-8147}. Universidade Estadual da Paraíba, João Pessoa - Paraíba -- Brasil - CEP 58.070-450. E-mail: elviomedeiros@servidor.uepb.edu.br
\end{itemize}

\hypertarget{resumo-2}{%
\subsection*{RESUMO}\label{resumo-2}}
\addcontentsline{toc}{subsection}{RESUMO}

Para compreender a distribuição das espécies de peixes e o uso do habitat, uma série de variáveis ambientais foram avaliadas como preditores da composição de montagem de peixes e riqueza em sistemas aquáticos semiáridos tropicais. Pesquisamos a composição das espécies de conjuntos de peixes em sistemas aquáticos semiáridos e estabelecemos seu grau de associação com a estrutura do habitat aquático. Os locais consistiam em córregos com fluxo de água superficial, piscinas temporárias isoladas e reservatórios feitos pelo homem. A amostragem de peixes foi realizada em quatro ocasiões durante as estações úmidas (abril e junho de 2006) e secas (setembro e dezembro de 2006). A regressão múltipla de tepwise mostrou que as variáveis morfométricas, largura de alcance do córrego, comprimento do córrego e elevação explicaram 75,6\% da variação da riqueza dos peixes. Cobertura de macrófitas e vegetação pendente somado ao poder preditivo da equação do modelo, onde o modelo final explicou 86,9\% da variação da riqueza dos peixes. A Análise de Correspondência Canônica mostrou uma relação significativa entre os dados de composição do peixe e a morfologia do local (altitude, inclinação do banco e profundidade litorânea). Entre a qualidade da água, a composição do habitat e as variáveis substrato, temperatura, areia e cascalho apresentaram maior correlação com os eixos CCA. Esses resultados indicam que as comunidades de peixes assumem diferentes estruturas e composições em diferentes tipos de habitat, de acordo com a heterogeneidade ambiental nos sistemas aquáticos de terra seca.

\textbf{Palavras-chave}: fluxos intermitentes, reservatórios, conservação, composição de substratos.

\hypertarget{introduuxe7uxe3o}{%
\section{Introdução}\label{introduuxe7uxe3o}}

Padrões de distribuição de espécies de peixes dentro dos ecossistemas são frequentemente explicados em termos da estrutura do habitat disponível (Junqueira et al., 2016). O habitat é frequentemente relatado como estruturas subaquáticas físicas, como rochas, madeira submersa, macrófitas, algas, etc, e como cobertura (ou proteção aérea) fornecida por características litorâneas, que incluem vegetação costeira ou troncos pendurados nas margens do córrego (Stewart-Koster et al., 2007). Indiscutivelmente, a qualidade e a quantidade do habitat afetam a estrutura e a composição das comunidades de peixes (Vono e Barbosa, 2001), por mudanças nas profundidades da água, velocidade e, consequentemente, tipo substrato, características litorâneas (como vegetação suspensa) e estruturas subaquáticas (Stewart-Koster et al., 2007, Medeiros e Arthington, 2011). Comumente, esses fatores são integrados e os habitats tendem a variar e segregar-se em manchas hierárquicas discretas (Frissell et al., 1986). Apesar disso, o habitat físico dos sistemas aquáticos de água doce em todo o mundo foi degradado ou modificado pelas atividades humanas (Hall et al., 2002). Tal interferência muitas vezes resulta da conversão de sistemas lóticos em lênticos, que mudam a dinâmica do habitat e seu funcionamento (Bunn e
Arthington, 2002).

Uma série de estudos relacionam a estrutura da comunidade biótica (número de espécies e distribuição de abundância) e o habitat marginal físico. Tais estudos mostram que tipos de habitat mais complexos fornecem condições favoráveis para os peixes, como substrato de crescimento, locais de desova, bem como alimentos e proteção contra predação (Pusey e Arthington, 2003, Cucherousset et al., 2007, Jeffres et al., 2008) e, portanto, uma maior diversidade de microhabitats associados às características litorâneas tem sido ligada a uma maior diversidade de espécies (Casatti et al., 2012). Como resultado, o fato de diferentes espécies de peixes apresentarem preferências particulares para diferentes tipos de habitat, cria padrões locais de composição de conjuntos de espécies (Junqueira et al., 2016). No entanto, o que leva a presença de espécies a tipos específicos de habitat muitas vezes não é claro.

Nos sistemas de riacho em regiões secas, espera-se que suas amplas variações espaciais e temporais levem a habitats físicos variáveis e, portanto, padrões específicos de estrutura de habitat (Hodges e Magoulick, 2011). Espera-se que a riqueza e a composição dos peixes sigam tais mudanças. Nesses sistemas, o habitat físico é espacial e temporalmente dinâmico devido à interação da morfologia do canal de córrego (por exemplo, largura, profundidade e substrato) e da variabilidade hidrológica (Medeiros et al., 2008, Farias et al., 2012). O estado do habitat físico a um determinado alcance será, portanto, influenciado por fatores que operam em diferentes escalas (Meninos e Thoms, 2006). No nível de captação, geomorfologia e clima afetam hidrologia, deposição de sedimentos, insumos de nutrientes e morfologia do canal (Davies et al., 2000, Mugodo et al., 2006). A nível local, características físicas e químicas da água, o uso da terra e o manejo da terra influenciarão o habitat de alcance do córrego (Hodges e Magoulick, 2011).

É importante ressaltar que a relação entre variáveis de habitat físico e riqueza e distribuição de espécies de peixes tem sido investigada em sistemas lúdicos e loticos tropicais (por exemplo, Vono e Barbosa, 2001, Casatti et al., 2012), mas essa relação para as comunidades de peixes semiáridos permanece amplamente desconhecida. Este trabalho mede a riqueza de peixes e a composição de espécies em diversos habitats aquáticos em córregos intermitentes do Semiárido Brasil, a fim de avaliar a composição das espécies de conjuntos de peixes e estabelecer seu grau de associação com a estrutura do habitat aquático. Temos a hipótese de que a fauna de peixes é irregular com as montagens representando características locais do habitat e que as variáveis ambientais serão elementos importantes explicando a distribuição de peixes.

\hypertarget{desenho-amostral}{%
\subsection{Desenho Amostral}\label{desenho-amostral}}

Este estudo foi realizado na região semiárida brasileira, em bacias hidrográficas que fluem através de uma floresta aberta arbustiva seca decidual (a ``Caatinga''). A amplitude térmica é baixa na área de estudo, com médias variando de aproximadamente 25 a 30 °C, e a precipitação média anual varia entre 600 e 1100 mm. As altitudes variam entre 100 e 1000 m (Figura \ref{fig:201mapa1}).

\begin{figure}
\includegraphics[width=19.14in]{imagens/bases/com-mapa} \caption{Área de estudo mostrando os estados do Rio Grande do Norte (RN), Paraíba (PB), Pernambuco (PE) e Alagoas (AL), principais sistemas fluviais e pontos de amostragem na região semiárida do Brasil. TA, açude Recanto; CP, riacho Cipó; CT, riacho Catureré; PC, riacho Poço da Cruz; MU, açude Mulungu e GU, açude Gurjão.}\label{fig:201mapa1}
\end{figure}

\textbf{Duas regiões} de importância biológica (\textbf{Seridó, S} e \textbf{Buíque, B}) foram escolhidas e nelas, \textbf{seis locais} ou sítios de coleta (\textbf{TA}, \textbf{CP}, \textbf{CT}, \textbf{PC}, \textbf{MU}, e \textbf{GU}) foram selecionados para representar ambientes Unidades Amostrais de ambientes temporários naturais e artificiais típicos (Figura \ref{fig:201mapa1} e Figura \ref{fig:201comcodes}).

Os locais consistiam em trechos de \textbf{riachos (-R)} com fluxo de água superficial (durante a estação chuvosa) ou poças temporárias isoladas (durante a estação seca) e \textbf{reservatórios ou açudes (-A)} artificiais criados a partir do represamento de riachos. A amostragem foi realizada durante o ano de 2006 em \textbf{quatro ocasiões} durante as estações chuvosa (\textbf{abril, 1} e \textbf{junho, 2}) e seca (\textbf{setembro, 3} e \textbf{dezembro, 4}) (Figura \ref{fig:201mapa1} e Figura \ref{fig:201comcodes}).

Várias das espécies desse estudo tem grande importância ecológica, como é o caso de \emph{Astyanax bimaculatus}\footnote{A etimologia do gênero \emph{Astyanax} vem da mitologia Grega. Heitor personagem da ``Ilíada'', tinha um filho chamado Astíanax.} (Figura \ref{fig:201asbim}), que é muito comum em rios intermitentes e serve de alimento para predadores maiores como a espécie \emph{Hoplias malabaricus}\footnote{Do Grego, \emph{hoplon}, arma ou armadura, em referência aos dentes caniniformes muito desenvolvidos, e forte estrutura óssea na cabeça.} (Figura \ref{fig:201hmala}).

\begin{figure}
\includegraphics[width=8.89in]{imagens/spp/Asbim_u4} \caption{*Astyanax bimaculatus*, a espécie mais comum da matriz de dados ppbio. Peru, by Eakins, R. Fonte: <https://www.fishbase.se/summary/Astianax-bimaculatus.html>}\label{fig:201asbim}
\end{figure}

\begin{figure}
\includegraphics[width=8.89in]{imagens/spp/Homal_u6} \caption{*Hoplias malabaricus*, espécie que cresce para se tornar um importante predador. Brazil, by Roselet, F.F.G. Fonte: <https://www.fishbase.se/summary/Hoplias-malabaricus.html>}\label{fig:201hmala}
\end{figure}

Outras espécies como \emph{Apareiodon hasemani}\footnote{A etimologia do nome \emph{Apareiodon} vem do Grego, \emph{a}, sem, \emph{pareia}, lateral ou bochecha, e \emph{odous} dentição, em referência a ausência de dentes laterais no aparato bucal dessa espécie.} (Figura \ref{fig:201apdavis} tem importância trófica por estar na base da cadeia alimentar, enquanto espécies da família Loricariidae, como \emph{Pseudancistrus genisetiger}\footnote{A etimologia do nome \emph{Pseudancistrus} vem do Grego, \emph{pseudes}, falso, e \emph{agkistron}, gancho, em referência a falsos ganchos presentes na cabeça em algumas espécies do gênero.} (Figura \ref{fig:201pgen}), tem importância para taxonomia.

\begin{figure}
\includegraphics[width=8.89in]{imagens/spp/Apdav_u0} \caption{*Apareiodon sp.*, importante espécie bentopelágica das bacias dos rios Jaguaribe e Paraíba. Brazil, by Ramos, T.P.A. Fonte: <https://www.fishbase.se/summary/Apareiodon-davisi.html>}\label{fig:201apdavis}
\end{figure}

\begin{figure}
\includegraphics[width=0.8\textwidth]{imagens/spp/genis} \caption{*Pseudancistrus genisetiger*, uma espécie endêmica das bacias hidrográfcas do nordeste. By Medeiros, E.S.F. Fonte: Arquivo pessoal}\label{fig:201pgen}
\end{figure}

\hypertarget{codificauxe7uxe3o-das-variuxe1veis-1}{%
\section{Codificação das variáveis}\label{codificauxe7uxe3o-das-variuxe1veis-1}}

Os arquivos da base de dados do projeto são fornecidos em formato Excel (.xlsx). Por exemplo, o arquivo \texttt{ppbio06*-peixes.xlsx}, traz os dados brutos que serão usados nas análises. A matriz de dados brutos contem mais de 20 localidades (n=linhas ou objetos) em estações do ano diferentes, e cerca de 35 espécies (m=colunas ou atributos), antes de qualquer modificação. Portando é uma matriz bruta. Os valores são contagens de indivíduos, e apresentam uma alta amplitude de variação, portanto, o uso de uma matriz relativizada é sugerido. Nos nomes dos objetos e dos atributos são codificados de acordo com a tabela mostrada na Figura \ref{fig:201comcodes}.

\begin{figure}
\includegraphics[width=14.82in]{imagens/bases/com-codes} \caption{Codificação para as variáveis, espécies de peixes, sítios de amostragem e período de amostragem.}\label{fig:201comcodes}
\end{figure}

\begin{figure}
\includegraphics[width=22.97in]{imagens/bases/lista} \caption{Tabela 1. Abundância, percentual e frequência de ocorrência (F.O.) de espécies coletadas nos locais de estudo (codificados conforme a Figura \\ref{fig:201comcodes}) no semiárido do Brasil. \* = Espécies não nativas.}\label{fig:unnamed-chunk-1}
\end{figure}

\hypertarget{amostragem-dos-dados}{%
\section{Amostragem dos dados}\label{amostragem-dos-dados}}

As coletas de peixes foram realizadas utilizando-se quatro tipos diferentes de equipamentos de amostragem, durante o dia, e de acordo com Medeiros et al (2010):

\begin{itemize}
\item
  uma rede de arrasto curta (4 m de comprimento, 1,5 m de malha de altura e 5 mm),
\item
  uma rede de arrasto longa (20 m de comprimento, 2 m de altura e 12 mm de malha),
\item
  um conjunto de redes de espera (30 m de comprimento e 1,5 m de altura divididas em três painéis de 10 m de 35, 45 e 55 mm de malha), e
\item
  uma rede de tarrafa (2,4 m de altura e 12 mm de malha).
\end{itemize}

O esforço de captura foi semelhante em ocasiões e locais de amostragem, e medido em horas (para o método passivo), representando o número de horas das redes de espera na água, ou réplicas (para métodos ativos), representando o número de arrastos (para redes de arrasto) ou o número de lançamentos (para a tarrafa) (Medeiros et al., 2010). Os arrastos foram semelhantes em todas as ocasiões e locais de amostragem, sendo aproximadamente 10 m de comprimento para a grande rede de arrasto e 3 a 5 m de comprimento para a rede de arrasto curta.

Os peixes capturados foram eutanasiados em solução de eugenol (\href{https://antigo.mctic.gov.br/mctic/opencms/legislacao/outros_atos/resolucoes/Resolucao_CONCEA_n_37_de_15022018.html}{Resolução Normativa CONCEA nº 37, de 15.02.2018}) fixados em formalia a 10\% neutralizada com tetraborato de sódio e posteriormente transferidos para solução de 75\% de etanol. Os espécimes foram tratados de acordo com as normas brasileiras de curadoria científica (Malabarba e Reis, 1987) e os peixes foram coletados sob a licença nº 032DIFAP/IBAMA do Instituto Brasileiro do Meio Ambiente e dos Recursos Naturais Renováveis (032DIFAP/IBAMA) de 23 de março de 2006. A triagem e identificação dos espécimes foram realizadas no Laboratório de Morfologia e Taxonomia de Peixes da Universidade Federal da Paraíba. Os espécimes-testemunho foram depositados após identificação na Coleta Ictiológica da mesma instituição.

Além da estrutura da comunidade, a estrutura física do habitat foi medida como,

\begin{itemize}
\item
  \begin{enumerate}
  \def\labelenumi{(\arabic{enumi})}
  \tightlist
  \item
    variáveis físicas e químicas,
  \end{enumerate}
\item
  \begin{enumerate}
  \def\labelenumi{(\arabic{enumi})}
  \setcounter{enumi}{1}
  \tightlist
  \item
    a morfologia do fluxo,
  \end{enumerate}
\item
  \begin{enumerate}
  \def\labelenumi{(\arabic{enumi})}
  \setcounter{enumi}{2}
  \tightlist
  \item
    composição do substrato, e
  \end{enumerate}
\item
  \begin{enumerate}
  \def\labelenumi{(\arabic{enumi})}
  \setcounter{enumi}{3}
  \tightlist
  \item
    estrutura de habitat.
  \end{enumerate}
\end{itemize}

As variáveis físicas e químicas foram medidas utilizando-se equipamentos portáteis para pH (TECNOPON MPA-210), condutividade (μS/cm) (TECNOPON MCS-150), oxigênio dissolvido (mg/L) e temperatura (°C) (Lutron DO-5510). A transparência (cm) foi medida utilizando-se um disco secchi, e a velocidade da água (m/s) foi estimada usando o método flutuante (Maitland, 1990). O alcance do córrego ou morfologia do local foi avaliado pela largura média (cm) e profundidade (cm) retiradas de três transectos colocados ao longo do alcance do córrego, piscina ou reservatório. A composição do substrato e a estrutura do habitat foram estimadas em 9 a 12 pontos de pesquisa de 1 m2 medidos nas margens (ver Medeiros et al., 2008). Em cada ponto de pesquisa, foram estimadas as proporções (\%) da composição de sedimentos (classificados como lama, areia, cascalho e seixos) e estruturas litorâneas e subaquáticas (por exemplo, macrófitas, capim, vegetação submersa, vegetação suspensa, folhiço, algas e galhos) (Pusey et al., 2004, Medeiros et al., 2008).

\hypertarget{anuxe1lises-estatuxedsticas}{%
\section{Análises estatísticas}\label{anuxe1lises-estatuxedsticas}}

Foram realizadas análises estatísticas univariadas sobre riqueza e abundância. A abundância foi padronizada por Unidade de Esforço de Captura (CPUE), onde o número de peixes capturados foi dividido pelo número de horas ou réplicas de cada técnica amostral em cada ocasião amostral e local (a partir de agora denominado abundância CPUE) (Medeiros et al., 2010). A riqueza dos peixes foi corrigida utilizando-se análises rarefação para o número médio de indivíduos de todas as ocasiões de amostragem utilizando PRIMER-e 5.0 (Clarke e Gorley, 2001). A abundância CPUE e a riqueza rarefacionada foram comparadas entre os locais de estudo usando a ANOVA one-way seguida de comparações múltiplas pós-hoc usando o teste HSD de Tukey (α=0,05) (Zar, 1999). A abundância CPUE foi transformada pela raiz quadrada, e as variáveis ambientais foram log10 (x+1) transformadas para aumentar a normalidade e a homogeneidade das variâncias (Sokal e Rohlf, 1969, Maltchik et al., 2010). A correlação entre a riqueza rarefacionada e a abundância CPUE (variáveis dependentes) com a estrutura de habitat (variáveis independentes) foi avaliada utilizando-se a regressão múltipla (RM) com ``forward selection'', onde as variáveis são inseridas no modelo com base na significância (probabilidade) do valor F (Sheridan e Lyndall, 2001, Maltchik et al., 2010).

Os padrões de variação na composição da comunidade de peixes entre os locais foram avaliados utilizando-se de Escalonamento Multidimensional Não-Métrico (NMDS), com base na distância relativizada de Bray-Curtis da matriz de dados transformada pelo arco-seno da raiz quadrada. O Procedimento de Permutação Multiresposta (MRPP) (Biondini et al., 1985, McCune e Grace, 2002) foi utilizado para testar a significância das diferenças na composição dos peixes entre locais/ocasiões de amostragem. O valor de ``A'' é apresentado como medida do grau de homogeneidade entre os grupos em comparação com a expectativa aleatória. Utilizou-se a Análise de Espécies Indicadoras (ISA) para determinar quais espécies foram indicadores significativos dos locais. O Valor Indicativo (IV) para cada espécie foi calculado utilizando-se o método de Dufrene e Legendre (1997). Este valor é testado para significância usando o teste de Monte Carlo (999 permutações). Para estabelecer correlações multivariadas entre a composição dos peixes e as variáveis ambientais, foi realizada uma Análise de Correspondência Canônica (CCA) (McCune e Grace, 2002). A matriz de dados foi centrada e normalizada, e as correlações testadas pelo teste de Monte Carlo com 999 permutações. As variáveis ambientais foram transformadas pelo log10(x+1) (Maltchik et al., 2010). As análises estatísticas (α=0,05) foram realizadas no PC-ORD 7.0 (McCune e Mefford, 1999).

\hypertarget{discussuxe3o}{%
\section{Discussão}\label{discussuxe3o}}

A riqueza das espécies e a composição comunitária dos peixes observados no presente estudo estão de acordo com outros estudos realizados no semiárido do Brasil (Medeiros e Maltchik, 2001a, Medeiros et al., 2010) com a família Characidae sendo a mais representativa em termos de riqueza e abundância. Somado ao fato de que a ordem Characiformes inclui um grande número de espécies de peixes do semiárido brasileiro (Rosa et al., 2003), a maioria das espécies de Characidae observadas no presente estudo são de pequeno porte com taxas relativamente altas de fecundidade e crescimento (Medeiros e Maltchik, 2000, 2001b) e generalistas em hábitos alimentares e no uso de habitat ( Silva et al., 2010, Silva, 2012). Em comparação com outros sistemas de rios secos, os locais de estudo mostraram águas bem oxigenadas, transparentes e relativamente quentes ao longo do período de estudo (ver, por exemplo, Medeiros e Arthington, 2011). A morfologia do alcance do rio variou principalmente de acordo com a natureza do local de estudo, fluxo, piscina isolada ou reservatório, e o habitat aquático era diversificado com uma gama de elementos de habitat disponíveis para colonização pela biota aquática. Os córregos e piscinas de fluxo apresentaram uma maior variedade de elementos de habitat marginal e composição de substratos quando comparados aos reservatórios. Estudos no Semiárido Brasil têm demonstrado que o habitat aquático tende a ser mais rico em córregos, quando comparado aos reservatórios e defende que essa riqueza em estruturas subaquáticas e a natureza mais variável dos córregos intermitentes melhorem a diversidade de peixes (Medeiros et al., 2006, Medeiros et al., 2008).

A regressão múltipla mostrou que o rio atinge morfologia e a estrutura do habitat teve a maioria das variáveis explicando a riqueza dos peixes. O fato de que a morfologia do local é um fator importante que explica a riqueza dos peixes (principalmente largura, comprimento e elevação) indica que esses fatores espaciais estão associados a diferentes faixas de variáveis ambientais que suportam um número de espécies de peixes. Isso é apoiado pelas análises multivariadas que mostraram forte segregação espacial da comunidade de peixes em padrões de montagem espacial. Foi demonstrado que diferentes conjuntos de espécies de peixes exibem preferências por tipos específicos de habitat (Martin-Smith, 1998). No presente estudo, a ausência de fluxo de água na maioria dos locais e a falta de conectividade longitudinal levaram a condições ambientais específicas, segregando assim a comunidade de peixes aos padrões locais de composição das espécies. Portanto, na ausência de fluxo de água, espera-se que a comunidade de peixes assuma diferentes composições entre os tipos de habitat (ou seja, fluxo de vazão de córrego, piscina e reservatório) de acordo com a heterogeneidade ambiental. Isso é ainda mais apoiado pela riqueza significativamente diferente entre os locais, principalmente no que diz respeito aos córregos Seridó (SE) e Escama-Peixe (EP), que apresentaram maior riqueza média. Os resultados de abundância são menos conclusivos, uma vez que a ANOVA não mostrou diferença significativa entre os locais e a regressão múltipla relatou apenas oxigênio dissolvido e vegetação pendente como preditores importantes da abundância de peixes. As abundâncias naturais de peixes nos sistemas aquáticos semiáridos brasileiros variam muito, tanto espacialmente quanto temporalmente (Medeiros et al., 2010). Isso provavelmente está associado ao movimento de indivíduos dentro de pequenos trechos do rio e entre áreas mais rasas e mais profundas em piscinas ou reservatórios maiores durante inundações (Medeiros e Arthington, 2008), bem como com padrões de recrutamento e desova de peixes (Medeiros e Maltchik, 2000).

A importância da vegetação ribeirinha e sua influência no habitat aquático é bem reconhecida (Gregory et al., 1991), pois contribui com sombra e estruturas subaquáticas para refúgio de peixes, bem como locais de desova (Pusey e Arthington, 2003). No presente estudo, variáveis associadas à presença de vegetação ribeirinha foram importantes preditores da riqueza dos peixes (vegetação pendente) e associadas à composição da montagem (detritos lenhosos e temperatura da água). Essas variáveis provavelmente estão contribuindo com condições ambientais adequadas aos peixes, bem como criando embalagens locais de condições ambientais que contribuem com a segregação espacial observada na comunidade de peixes. A presença de macrófitos aquáticos em locais de reservatórios e córregos e sua importância como preditores para a riqueza e composição de espécies também é uma indicação do papel desempenhado pelas estruturas de habitat subaquático na criação de habitats locais espacialmente segregados disponíveis para uso de peixes (Xie et al., 2001). O habitat pode ser visto como um quadro onde variáveis ambientais afetam comunidades biológicas (Southwood, 1977). No entanto, em sistemas hidrologicamente variáveis, a variabilidade do fluxo tem sido sugerida como um grande eixo influenciando no modelo de habitat (Minshall, 1988, Poff e Ward, 1989). Os sistemas aquáticos de terra seca no Brasil são altamente variáveis no que diz respeito à presença e magnitude do fluxo de água (Maltchik e Florin, 2002) e, embora os resultados indiquem que a estrutura do habitat e da morfologia do local são importantes preditores da fauna de peixes, o modelo de habitat é multidimensional (Southwood, 1977) e as diversas variáveis ambientais medidas interagem entre si. Nos sistemas semiáridos do Brasil, a estrutura do habitat está sendo conduzida por muitos componentes em diferentes escalas (Medeiros et al., 2008), onde o papel das características de captação e morfologia local aumentaria na predominância em suas respectivas escalas. Isso destaca a importância da ligação entre a estrutura do habitat e a diversidade biótica no nível do habitat local.

Alterações nos padrões naturais de vazão de água com base na construção de reservatórios e açudes, decorrentes das políticas de gestão atuais dos sistemas semiáridos aquáticos no Brasil, modificam as características hidrológicas dos córregos intermitentes altamente variáveis (Leal et al., 2005, Maltchik e Medeiros, 2006). Com as mudanças na dinâmica natural do fluxo de água vêm também mudanças na estrutura de montagem de macofículas, dinâmica de nutrientes e conectividade longitudinal, todas associadas à conversão de sistemas loticos para lenticos (Bunn e Arthington, 2002). Os dados do presente estudo mostram que essas modificações afetarão as comunidades de peixes (riqueza e composição), uma vez que variáveis altamente associadas ao fluxo de água, como largura de alcance do córrego, cobertura de macrofito, vegetação suspensa e concentração de oxigênio dissolvida foram importantes preditores da montagem de peixes. Portanto, a modificação dos padrões naturais de vazão da água e promoção de condições lenéticas tem o potencial de interferir na fauna de peixes, favorecendo espécies mais oportunistas mais adaptadas a condições de fluxo. Isso é corroborado pelo fato de que os locais dos reservatórios apresentaram menos espécies indicadoras que foram introduzidas principalmente, como Poecilia reticulata e Parachromis managuensis, ou tipicamente lenticas como \emph{Geophagus brasiliensis}.

O presente estudo mostra que as comunidades de peixes assumem diferentes estruturas e composições em diferentes tipos de habitat, de acordo com a heterogeneidade ambiental associada à natureza do habitat (reservatórios, córregos e piscinas). A riqueza e a composição dos peixes foram influenciadas principalmente por dois conjuntos de variáveis ambientais, sendo morfologia do local e características do habitat. O primeiro conjunto de variáveis relaciona-se com os níveis hierárquicos espaciais de captação, uma vez que o comprimento e a elevação do rio estão tipicamente associados à hierarquia do rio. O segundo conjunto de variáveis é tipicamente local, com estruturas subaquáticas sendo resultado de processos locais como o fluxo de água de alcance do rio e o uso da terra/vegetação ribeirinha.

\hypertarget{apuxeandices-1}{%
\section*{Apêndices}\label{apuxeandices-1}}
\addcontentsline{toc}{section}{Apêndices}

\hypertarget{tabelas-e-figuras}{%
\subsection{Tabelas e Figuras}\label{tabelas-e-figuras}}

Tabela 1. Abundância, percentual e frequência de ocorrência (F.O.) de espécies coletadas nos locais de estudo (codificados conforme a Figura \ref{fig:201comcodes}) no semiárido do Brasil. * = Espécies não nativas.''

\begin{figure}
\includegraphics[width=22.97in]{imagens/bases/lista} \end{figure}

Tabela 2. Variáveis ambientais (average ± SD) coletadas nos locais de estudo no semiárido do Brasil.

\begin{figure}
\includegraphics[width=0.8\textwidth]{imagens/bases/com_1} \end{figure}

Tabela 3. São apresentadas as variáveis ambientais e a composição dos peixes no presente estudo (apenas variáveis correlacionadas com um determinado eixo são mostradas).

\begin{figure}
\includegraphics[width=14.94in]{imagens/bases/com_2} \end{figure}

Figura 1. Área de estudo mostrando os estados do Rio Grande do Norte (RN), Paraíba (PB), Pernambuco (PE) e Alagoas (AL), principais sistemas fluviais e locais de amostragem no semiárido do Brasil. TA, açude Recanto; CP, riacho Cipó; CT, riacho Catureré; PC, riacho Poço da Cruz; MU, açude Mulungu e GU, açude Gurjão.

\begin{figure}
\includegraphics[width=19.14in]{imagens/bases/com-mapa} \end{figure}

Figura 2. Os resultados da NMDS para a composição dos peixes em todo o estudo chegam ao semiárido do Brasil. Vetores (caixa de entrada) mostram taxa correlacionada (r2 \textgreater{} 0,2) com amostras em espaço de ordenação. As letras após os códigos do site indicam ocasião de amostragem (A=Abril, J=Junho, S=Setembro e D=Dezembro).

\begin{figure}
\includegraphics[width=14.74in]{imagens/bases/com_3} \end{figure}

Figura 3. Biplot de CCA mostrando a composição de conjuntos de peixes nos locais e ocasiões de amostragem () e as variáveis ambientais explicativas definidas pela CCA.

\begin{figure}
\includegraphics[width=14.56in]{imagens/bases/com_4} \end{figure}

Arquivos fornecidos
ppbio06.xlsx - O arquivo ppbio06.xlsx traz os dados brutos que serão usados nas análises. A
matriz de dados brutos contem mais de 20 localidades em estações do ano diferentes (objetos) x
cerca de 35 espécies (atributos), antes de qualquer modificação. Os valores são contagens de
indivíduos com uma alta amplitude de variação, portanto, o uso de uma matriz relativizada é
sugerido.

\hypertarget{referuxeancias-2}{%
\section*{Referências}\label{referuxeancias-2}}
\addcontentsline{toc}{section}{Referências}

\hypertarget{basehab}{%
\chapter{Base de Dados II - Habitat}\label{basehab}}

\hypertarget{estrutura-do-habitat-variuxe1veis-ambientais-no-semiuxe1rido}{%
\subsection*{Estrutura do Habitat: Variáveis ambientais no semiárido}\label{estrutura-do-habitat-variuxe1veis-ambientais-no-semiuxe1rido}}
\addcontentsline{toc}{subsection}{Estrutura do Habitat: Variáveis ambientais no semiárido}

\hypertarget{apresentauxe7uxe3o-2}{%
\subsection*{Apresentação}\label{apresentauxe7uxe3o-2}}
\addcontentsline{toc}{subsection}{Apresentação}

Exemplo de conjunto de dados ecológicos do Programa de Pesquisa em Biodiversidade (PPBio) do Semiárido. Ano de 2006. Estrutura do habitat\footnote{Projeto financiado pelo CNPq/UEPB/DCR Proc.350082/2006-5, UEPB/FAPESQ Proc.68.0006/2006.0 e Projeto de Pesquisa em Biodiversidade do Semiárido (PPBio Semi-Árido).}

Autor:\\
- Prof.~Elvio S. F. Medeiros\\
- Laboratório de Ecologia\\
- Universidade Estadual da Paraíba\\
- Campus V, João Pessoa, PB

Para entender a distribuição das espécies de peixes e seu uso de habitat, uma série de variáveis ambientais foram avaliadas como preditores da composição e riqueza da assembleia de peixes em sistemas aquáticos tropicais semiáridos. Nós pesquisamos a composição de elementos físicos, químicos e da estrutura do habitat associados a assembleias de peixes em sistemas aquáticos semiáridos e estabelecemos seu grau de associação. Os locais consistiam em trechos de riachos com fluxo de água superficial, poças temporárias isoladas e reservatórios artificiais (açudes). A amostragem de elementos do habitat foi feita associada a amostragem da comunidade de peixes e zooplâncton em quatro ocasiões durante as estações chuvosa (abril e junho de 2006) e seca (setembro e dezembro de 2006).

\textbf{Texto modificado de MEDEIROS; SILVA; RAMOS (\protect\hyperlink{ref-RN2491}{2008})}: Application of catchment- and local-scale variables for aquatic habitat characterization and assessment in the Brazilian semi-arid region. Neotropical Biology and Conservation, 2008, vol.~3, n.~1, p.~13-20. \url{https://revistas.unisinos.br/index.php/neotropical/article/view/5440}. Tradução: Google Translator.

Elvio Sergio Figueredo Medeiros\textsuperscript{1,4}, Marcio Joaquim da Silva\textsuperscript{2}, e Robson Tamar Costa Ramos\textsuperscript{1,3}

\textsuperscript{1} Grupo Ecologia de Rios do Semiárido, Universidade Estadual da Paraíba, Depto. de Biologia. Campus V. CEP 58070-450 João Pessoa - PB. Brazil. \textsuperscript{2} Universidade Federal do Rio Grande do Norte, Depto. de Biologia, Centro de Biociências, CEP 59078-900 -- Natal - RN. \textsuperscript{3} Laboratório de Sistemática e Morfologia de Peixes, Universidade Federal da Paraíba, Depto. de Sistemática e Ecologia, CCEN, CEP 58059-900, João Pessoa - PB.

\begin{itemize}
\tightlist
\item
  Autor correspondente: Dr.~Elvio Medeiros
  \url{https://orcid.org/0000-0002-7472-8147}. Universidade Estadual da
  Paraíba, João Pessoa - Paraíba -- Brasil - CEP 58.070-450. E-mail:
  elviomedeiros@servidor.uepb.edu.br
\end{itemize}

\textsuperscript{4}Os autores agradecem a Telton Ramos e Virginia Diniz (UFPB) pela assistência em campo e ao CNPq/UEPB/DCR pela bolsa concedida a ESFM (350082/2006-5). Esta pesquisa contou com o apoio da UEPB/FAPESQ (68.0006/2006.0) e do Projeto de Pesquisa em Biodiversidade do Semiárido (PPBio Semi-Árido).

\hypertarget{resumo-3}{%
\section*{RESUMO}\label{resumo-3}}
\addcontentsline{toc}{section}{RESUMO}

O estado do habitat físico é influenciado por fatores que atuam em diversas escalas, como a geomorfologia, o clima, o regime hidrológico, o uso da terra e a qualidade da água. Este trabalho quantifica a diversidade e a disponibilidade do habitat físico em sistemas aquáticos naturais e artificiais no semiárido do Brasil, com o objetivo de levantar variáveis físicas em escala local e de bacia de drenagem, e analisar sua importância na composição de uma estrutura básica para a caracterização e avaliação desses ambientes. Este estudo foi desenvolvido nas regiões de Seridó/Borborema e Buíque/Vale do Ipojuca. Estas áreas são consideradas de extrema importância biológica e são identificadas como prioritárias para a conservação da Caatinga. Os resultados mostram que o hábitat aquático no semiárido brasileiro é diverso e dinâmico, compreendendo um conjunto amplo de elementos disponíveis para a colonização pelos organismos. Os rios mostraram uma gama de elementos do hábitat marginal e da composição do substrato igual ou superior àquela dos reservatórios. Além disso, a composição do hábitat variou entre os dois tipos de ambiente estudados (rios/reservatórios) e com as estações do ano (seca/chuvosa). Os resultados apresentados têm implicações importantes para a conservação dos ambientes aquáticos do semiárido. Tendo em vista que o hábitat físico é a estrutura básica para a colonização por parte dos organismos aquáticos, os mecanismos potenciais que mantêm a biodiversidade aquática ocorrem nos vários níveis da bacia de drenagem. Portanto, é fundamental identificar os componentes, ao longo dos sistemas ripários, que são vitais na manutenção da sua integridade biológica.

\textbf{Palavras-chave:} rios intermitentes, reservatórios, conservação, composição de substratos

\hypertarget{introduuxe7uxe3o-1}{%
\section{Introdução}\label{introduuxe7uxe3o-1}}

Córregos intermitentes no semiárido brasileiro são características paisagísticas distintivas, existindo como cursos d'água secos durante a maior parte do ano. Maltchik e Medeiros (2006) reconheceram que os extremos de inundação e seca são impulsionadores de importantes processos que mantêm a diversidade dentro desses sistemas. Portanto, as medidas de conservação no semiárido brasileiro devem incluir a manutenção do regime de fluxo natural dos sistemas aquáticos, para garantir a sobrevivência das espécies em longo prazo (Maltchik e Medeiros, 2006). No entanto, as medidas de conservação também devem levar em conta a composição e diversidade do habitat aquático disponível, pois o habitat físico é o quadro de colonização pela fauna aquática (eg Martin-Smith, 1998). Além disso, o estado deste espaço vital influenciará a estrutura biótica e a organização dentro dos sistemas aquáticos (Mugodo et al., 2006). O estado do habitat físico, nomeadamente composição e diversidade, é influenciado por factores que operam a várias escalas espaciais e temporais (Boys e Thoms, 2006). Ao nível da bacia, a geomorfologia e o clima influenciarão o habitat à escala de alcance, afectando a hidrologia, sedimentação, aportes de nutrientes e morfologia do canal (Davies et al., 2000; Mugodo et al., 2006). No nível local, o uso e manejo da terra também influenciarão o habitat em escala de alcance do córrego (Richards et al., 1996).

Os métodos de avaliação do habitat físico disponíveis para os organismos aquáticos fornecem ferramentas importantes para vários aspectos da gestão e conservação do rio e, portanto, os procedimentos usados para avaliar o habitat devem ser ecologicamente e geomorfologicamente significativos (Thomson et al., 2001). Apesar disso, métodos de avaliação de habitats precisam ser desenvolvidos e testados em sistemas aquáticos semiáridos do Brasil, pois não são tão bem desenvolvidos quanto os métodos usados para avaliar outros aspectos da condição dos rios, como a qualidade da água. A avaliação do habitat é de grande utilidade para prever a distribuição potencial de espécies-chave, como peixes (Boys e Thoms, 2006) e invertebrados (Richards et al., 1997), com base em suas necessidades de habitat. Além disso, o habitat observado e potencial (ou previsto) pode ser comparado com os requisitos de habitat de uma espécie para avaliar a necessidade de manejo do habitat e a saúde geral do ecossistema com base nas necessidades da biota aquática (Maddock, 1999).

Este trabalho mede a diversidade e disponibilidade do habitat físico em sistemas aquáticos naturais e artificiais no nordeste do Brasil. Este artigo tem como objetivo levantar variáveis físicas em escala local e de captação e avaliar sua importância como marco básico para caracterização e avaliação de habitats aquáticos em duas áreas do semiárido brasileiro.

\hypertarget{desenho-amostral-1}{%
\subsection{Desenho Amostral}\label{desenho-amostral-1}}

Este estudo foi realizado na região semiárida brasileira, em bacias hidrográficas que fluem através de uma floresta aberta arbustiva seca decidual (a ``Caatinga''). A amplitude térmica é baixa na área de estudo, com médias variando de aproximadamente 25 a 30 °C, e a precipitação média anual varia entre 600 e 1100 mm. As altitudes variam entre 100 e 1000 m (Figura \ref{fig:202mapa} e Figura \ref{fig:202hab1}).

\begin{figure}
\includegraphics[width=19.14in]{imagens/bases/com-mapa} \caption{Área de estudo mostrando os estados do Rio Grande do Norte (RN), Paraíba (PB), Pernambuco (PE) e Alagoas (AL), principais sistemas fluviais e pontos de amostragem na região semiárida do Brasil. TA, açude Recanto; CP, riacho Cipó; CT, riacho Catureré; PC, riacho Poço da Cruz; MU, açude Mulungu e GU, açude Gurjão.}\label{fig:202mapa}
\end{figure}

\textbf{Duas regiões} de importância biológica (\textbf{Seridó, S} e \textbf{Buíque, B}) foram escolhidas e nelas, \textbf{seis locais} ou sítios de coleta (\textbf{TA}, \textbf{CP}, \textbf{CT}, \textbf{PC}, \textbf{MU}, e \textbf{GU}) foram selecionados para representar ambientes Unidades Amostrais de ambientes temporários naturais e artificiais típicos (Figura \ref{fig:202mapa} e Figura \ref{fig:202habcodes}).

Os locais consistiam em trechos de \textbf{riachos (-R)} com fluxo de água superficial (durante a estação chuvosa) ou poças temporárias isoladas (durante a estação seca) e \textbf{reservatórios ou açudes (-A)} artificiais criados a partir do represamento de riachos. A amostragem foi realizada durante o ano de 2006 em \textbf{quatro ocasiões} durante as estações chuvosa (\textbf{abril, 1} e \textbf{junho, 2}) e seca (\textbf{setembro, 3} e \textbf{dezembro, 4}) (Figura \ref{fig:202mapa} e Figura \ref{fig:202habcodes}).

\begin{figure}
\includegraphics[width=11.39in]{imagens/bases/hab1} \caption{Características gerais dos locais de estudo. Os critérios a priori para agrupar a estrutura do habitat no DCA são indicados com um (*).}\label{fig:202hab1}
\end{figure}

\hypertarget{sobre-os-dados-do-ppbio-1}{%
\subsection{Sobre os dados do PPBio}\label{sobre-os-dados-do-ppbio-1}}

A planilha \texttt{ppbio} contém os dados das variáveis ambientais em diferentes unidades amostrais (UA's) (Figura \ref{fig:202habitat}) (Veja \href{https://ppbio.inpa.gov.br/Sobre}{Programa de Pesquisa em Biodiversidade -- PPBio}). Essa é a \textbf{matriz bruta de dados}, porque os valores ainda não foram ajustados (não foram relativizados ou transformados). Outros tipos de arquivos existem para os dados do PPBio (\ref{arqs}), que fazem parte de um estudo mais amplo sobre ecologia de rios do semiárido.

\begin{figure}
\includegraphics[width=18.19in]{imagens/bases/habitat} \caption{Parte da planilha de dados brutos do PPBio para o habitat.}\label{fig:202habitat}
\end{figure}

\hypertarget{codificauxe7uxe3o-para-variuxe1veis}{%
\section{Codificação para variáveis}\label{codificauxe7uxe3o-para-variuxe1veis}}

Os arquivos da base de dados do projeto são fornecidos em formato Excel (.xlsx). Por exemplo, o arquivo \texttt{ppbio06*-habitat.xlsx}, traz os dados brutos que serão usados nas análises. A matriz de dados brutos contem mais de 20 localidades (n=linhas ou objetos) em estações do ano diferentes, e cerca de 30 variáveis ambientais (m=colunas ou atributos), antes de qualquer modificação. Portando é uma matriz bruta. Os valores incluem diversis tipos de variáveis medidas em unidades diferentes (\%, °C, mg/L, etc), e apresentam uma alta amplitude de variação, portanto, a aplicação de uma transformação é recomendado (Figura \ref{fig:202habitat}). Nos nomes dos objetos e dos atributos são codificados de acordo com a tabela mostrada na Figura \ref{fig:202habcodes}.

\begin{figure}
\includegraphics[width=15.51in]{imagens/bases/hab-codes} \caption{Codificação para as variáveis ambentais descritoras da estrutura do habitat (atributos) e os sítios e períodos de amostragem (objetos).}\label{fig:202habcodes}
\end{figure}

\begin{figure}
\includegraphics[width=15.56in]{imagens/bases/hab-str} \caption{Imagem representativa da codificação das variáveis, para a medição da estrutura do habitat (atributos).}\label{fig:202habstr}
\end{figure}

\hypertarget{uxe1rea-de-estudo-e-amostragem-dos-dados}{%
\section{Área de estudo e amostragem dos dados}\label{uxe1rea-de-estudo-e-amostragem-dos-dados}}

Este estudo foi realizado em duas áreas diferentes do semiárido brasileiro: Seridó/Borborema e Buíque/Vale do Ipojuca (conforme Tabarelli e Silva, 2003) (Tabela 1; Figura 1). Essas áreas são classificadas como de extrema importância biológica e foram identificadas como áreas prioritárias para conservação da biodiversidade na Caatinga por Silva et al.~(2003), por apresentarem alta diversidade de espécies e serem ricas em endemismos. A área do Seridó está localizada no sul do Rio Grande do Norte (RN) e norte da Paraíba (PB) entre Patos e Caicó (Figura \ref{fig:202mapa}). A temperatura média anual é de 30,7°C, com máxima média mensal em outubro (31,0°C) e mínima média em fevereiro (29,3°C). A precipitação concentra-se entre janeiro e abril, com 350 a 800 mm por ano e média anual de 600 mm (Amorim et al., 2005). A altitude no Seridó varia entre 100 e 800 m (Governo do Estado da Paraíba, 1985). A área de Buíque está localizada no entorno da cidade de Buíque no centro de Pernambuco (PE) (Figura \ref{fig:202mapa}. A temperatura média anual e a precipitação são 25°C e 1095,9 mm, respectivamente. As chuvas concentram-se entre abril e junho. Altitude varia entre 800 e 1000 m (Rodal et al., 1998).

Dentro de cada área de estudo, três locais foram selecionados para representar ambientes típicos artificiais e naturais temporários e semipermanentes (Tabela 1). Os locais consistiam em trechos de córregos, geralmente de 100 a 500 m de comprimento, e reservatórios artificiais criados a partir de represamento de córregos. A amostragem foi realizada durante um ano em quatro ocasiões durante as estações chuvosa (abril e junho de 2006) e seca (setembro e dezembro de 2006). Variáveis de escala de alcance de captação e córrego foram selecionadas com base nas características locais e em Pusey et al.~(2004) (entre outros, Richards et al., 1996; Davies et al., 2000; Thomson et al., 2001; Mugodo et al., 2006). As variáveis de escala de captação foram quantificadas com base nas características topográficas gerais da área de estudo e nas características da bacia hidrográfica usando mapas topográficos 1:500.000, imagens de satélite e receptor GPS. As variáveis do local em escala de alcance apresentam características morfométricas e de qualidade (físico-químicas) da água. A estrutura do habitat compreendeu (i) elementos marginais subaquáticos e litorâneos (representando a abundância de elementos marginais de microhabitat) e (ii) a composição do substrato.

A qualidade da água em escala de alcance, a estrutura do habitat e algumas das características morfométricas foram estimadas em pontos de levantamento de um metro quadrado e distribuídas em categorias. Normalmente, 3 a 12 pontos de pesquisa aleatórios foram avaliados dentro de cada trecho de rio ou reservatório. A composição do substrato e os elementos do habitat foram estimados como sua contribuição proporcional ao perímetro molhado do local. Para a avaliação da morfologia do local, a largura foi medida com fita métrica para distâncias de até 100 m e receptor de satélite GPS para distâncias maiores. A profundidade foi medida com um bastão a distâncias aproximadamente equivalentes ao longo de um transecto para representar a profundidade do habitat (profundidade média dos primeiros 3 m das margens) e a profundidade máxima. A velocidade da água foi medida usando o método de flutuação (Maitland, 1990). A inclinação da margem foi estimada visualmente e alocada em categorias de inclinação (\textless{} 30°, 30-60° e 60-90°). A temperatura da água (°C) e o oxigênio dissolvido (mg/l) foram medidos com um medidor de oxigênio (Lutron DO-5510), e a transparência (cm) foi medida com um disco Secchi.

\hypertarget{anuxe1lises-estatuxedsticas-1}{%
\section{Análises estatísticas}\label{anuxe1lises-estatuxedsticas-1}}

A variação na composição do habitat entre os locais e as ocasiões de amostragem foi investigada usando a Análise de Corresponência Destendida (DCA) dos dados padronizados transformados em arcsine-squareroot. O Multi-Response Permutation Procedure (MRPP) (Biondini et al., 1985; McCune e Grace, 2002) foi usado para testar diferenças na composição de habitats entre as duas áreas de estudo (Seridó e Buíque), tipos de habitats (córrego e reservatório) e estações do ano (chuvoso e seco). Para todas as análises de MRPP, a concordância dentro do grupo corrigida pela chance (A) é apresentada como uma medida do grau de homogeneidade dentro do grupo, em comparação com a expectativa aleatória. Onde o MRPP detectou diferenças significativas, uma análise adicional foi realizada para revelar quais elementos particulares do habitat contribuíram significativamente como a fonte da diferença na composição do habitat, usando a Análise de Espécies Indicadoras de McCune e Mefford (1999). Os Valores Indicadores (IV) foram calculados pelo método de Dufrene e Legendre (1997). Estes foram testados para significância estatística (p\textless0,05) usando uma técnica de Monte Carlo com 1000 corridas. A influência das variáveis de alcance e escala de captação na estrutura do habitat foi avaliada usando a Análise de Correspondência Canônica (CCA) seguindo McCune e Grace (2002). Os dados foram centralizados e normalizados e testados com randomização de Monte Carlo (100 corridas). Todas as estatísticas foram realizadas em PC-ORD 4.27 (McCune e Mefford, 1999).

\hypertarget{resultados}{%
\section{Resultados}\label{resultados}}

A estrutura do habitat para todos os locais de estudo consistiu geralmente em plantas C4 crescendo nas margens do litoral (nomeadamente gramíneas), macrófitas aquáticas, incluindo plantas flutuantes (geralmente \emph{Salvinia} sp., \emph{Pistia} sp. e \emph{Azolla} sp.), plantas emergentes (\emph{Nymphaea} spp.) e plantas submersas (\emph{Ceratophyllum} sp. e \emph{Egeria} sp.) e algas (aderidas ao substrato e filamentosas) (Figura \ref{fig:202hab2}). Vegetação ciliar pendendo, restos lenhosos, serapilheira e massas de raízes também estiveram presentes nos locais estudados.

\begin{figure}
\includegraphics[width=7.57in]{imagens/bases/hab2} \caption{Percentagens (± desvio padrão) dos elementos do habitat de ambos os elementos marginais e composição do substrato que ocupam o perímetro molhado marginal dos locais de estudo e parâmetros médios de qualidade da água.}\label{fig:202hab2}
\end{figure}

A composição do substrato foi composta principalmente de lama e areia, com menor contribuição de seixos, pedras e cascalho (Figura \ref{fig:202hab2}). A diversidade de estruturas de habitat foi geralmente maior nos locais dos rios, na área do Seridó e durante a estação chuvosa, apesar de uma maior riqueza de elementos de habitat nos locais dos reservatórios (Figura \ref{fig:202hab3}). A composição do substrato também foi mais diversa e rica nos locais do rio e na área do Seridó, enquanto o substrato foi mais rico e diversificado durante a estação seca (Figura \ref{fig:202hab3}). A análise de correspondência sem tendência (Figura \ref{fig:202hab4}) mostrou que as diferenças na estrutura do habitat foram significativas apenas entre os locais de riachos e reservatórios (MRPP, A = 0,06, p = 0,03) e estações (MRPP, A = 0,06, p = 0,03).

\begin{figure}
\includegraphics[width=7.56in]{imagens/bases/hab3} \caption{Diversidade e riqueza de elementos do habitat em locais de estudo em todos os tipos de habitat, área de estudo e estações do ano.}\label{fig:202hab3}
\end{figure}

Diferenças na estrutura do habitat entre as duas áreas de estudo não foram significativas (MRPP, A = 0,01, p = 0,26). A análise do valor indicador mostrou que os elementos importantes que separam esses grupos foram macrófitas para reservatórios (IV = 53,2, p = 0,01) e areia e pedras para locais fluviais (IV = 61,8, p = 0,02 e IV = 82,6, p = 0,002, respectivamente). Importantes características de habitat que separam as estações foram vegetação submersa (IV = 48,4, 0,01) e vegetação pendente (IV = 48,9, p = 0,02) para a estação chuvosa. Nenhum elemento do substrato foi significativamente importante na separação dos locais de acordo com as estações do ano (Figura \ref{fig:202hab4}).

\begin{figure}
\includegraphics[width=7.58in]{imagens/bases/hab4} \caption{Resultados da DCA para a composição do habitat nos locais de estudo. A caixa de inserção mostra elementos de habitat correlacionados (r2>0,2) com locais/ocasiões de amostragem no espaço de ordenação (indicados por vetores). A direção e o comprimento dos vetores indicam a força da correlação. Consulte a Tabela 2 para obter os nomes completos dos elementos do habitat. Os locais (S1-S6) são codificados de acordo com a Tabela 1 e a ocasião da amostragem. Abreviaturas: APR, abril; JUN, junho; SET, setembro; DEZ, dezembro.}\label{fig:202hab4}
\end{figure}

A análise de correspondência canônica mostrou que as variáveis de alcance e escala de captação não estavam fortemente correlacionadas, exceto para a distância do local à nascente do rio e comprimento do córrego da localização do local (correlação de Pearson = 0,965), elevação e comprimento do rio principal (Correlação de Pearson = 0,853) e elevação e largura do local (correlação de Pearson = 0,906). A variância total dos dados (``inércia'') foi de 1,19. A variância total explicada pela CCA foi de 46,2\%, sendo a maior parte dessa variação explicada pelo primeiro eixo (18,4\%) (Figura \ref{fig:202hab5}. No entanto, os eixos 2 e 3 não podem ser desconsiderados, pois também explicaram parte substancial da variação do conjunto de dados. Os resultados da análise mostram uma relação significativa entre os dados de estrutura do habitat e as variáveis de alcance e escala de captação, com o autovalor do primeiro eixo muito alto para ser esperado por acaso (Figura \ref{fig:202hab5}, valores de p para autovalor e correlação para o primeiro eixo = 0,01). O primeiro eixo foi fortemente correlacionado com variáveis morfológicas da escala de captação, nomeadamente comprimento do rio principal, elevação e a distância altamente correlacionada do local à nascente do rio e comprimento do córrego da localização do local. O primeiro eixo também foi correlacionado com a inclinação da margem, uma variável morfológica da escala de alcance. Eixos 2 e 3 foram correlacionados com variáveis de escala de alcance de morfologia (largura do local, eixo 2) e qualidade da água (velocidade da água e oxigênio dissolvido, eixo 3) (Figura \ref{fig:202hab5}).

\begin{figure}
\includegraphics[width=7.62in]{imagens/bases/hab5} \caption{Estatísticas de sumário dos eixos da Análise de Correspondência Canônica para a estrutura do habitat e variáveis de escala de captação e alcance. Correlações para variáveis de escala de captação e alcance com os eixos de ordenação após Análise de Correspondência Canônica.}\label{fig:202hab5}
\end{figure}

A Figura \ref{fig:202hab6} indica que o eixo 1 representa um gradiente de sítios com estrutura de habitat influenciada por variáveis de escala de captação e o eixo 2 representa sítios influenciados por variáveis de escala de alcance.

\begin{figure}
\includegraphics[width=6.69in]{imagens/bases/hab6} \caption{Biplot de locais de ordenação (Δ) (com base na estrutura do habitat) na bacia e alcance- variáveis de escala espaço, conforme definido pelo CCA. Os vetores representam o comprimento do rio (RIVER), distância do local da fonte (DISTSOUR), elevação, inclinação da margem (SLOPE) e largura do local (WIDTH). (+) indica a posição dos elementos do habitat no espaço de ordenação.}\label{fig:202hab6}
\end{figure}

\hypertarget{discussuxe3o-1}{%
\section{Discussão}\label{discussuxe3o-1}}

O presente estudo mostrou que o habitat aquático no semiárido brasileiro é diversificado e dinâmico, com uma gama de elementos de habitat disponíveis para colonização pela biota aquática. Sítios de riachos apresentaram uma composição de substrato e elementos de habitats marginais semelhante a maior quando comparados a reservatórios, e a composição do habitat variou com o tipo de habitat (rio/reservatório) e estações (seco/úmido). Mudanças sazonais nas características físicas de ambientes variáveis foram relatadas para regiões secas em outros lugares es (Mackey, 1991; Davis et al., 2002). Essas mudanças geralmente estão associadas a flutuações nos níveis de água e morfologia do habitat (Osborne et al., 1987; Medeiros, 2005). No presente estudo, as variações sazonais do nível da água e do regime hidrológico influenciaram a morfometria e a estrutura do habitat dos locais de estudo. Dadas as características lóticas dos locais de riachos, areia e seixos foram as características de habitat mais importantes neste tipo de habitat, pois espera-se que substratos finos como lama sejam realizados durante as inundações. A Figura 2 também dá indicação de que rochas, cascalho e algas anexadas eram elementos importantes de habitat em locais de riachos. Já os reservatórios foram caracterizados principalmente pela presença de macrófitas, vegetação submersa e, em menor grau, lama. O fato de as macrófitas aquáticas não terem sido elementos importantes na definição de habitat nos córregos intermitentes estudados está relacionado aos extremos de inundação durante o período úmido. Maltchik e Pedro (2001) e Pedro et al.~(2006) constataram que a inundação limitou a ocorrência de macrófitas aquáticas em riachos no semiárido brasileiro, onde as comunidades de macrófitas aquáticas sujeitas a inundação apresentaram menor riqueza de espécies do que as comunidades sem inundações. Os ciclos secos e úmidos em riachos semiáridos do nordeste do Brasil são caracterizados por extremos de inundações e secas e considerados como os mais importantes impulsionadores da estrutura da comunidade nesses sistemas (Maltchik e Medeiros, 2001; Maltchik e Florin, 2002). Os resultados do presente estudo mostram que os ciclos seco e úmido também desempenham um papel importante na estrutura do habitat, uma vez que a composição do habitat marginal e do substrato foram diferentes entre as estações. Curiosamente, o habitat marginal foi mais diverso durante a estação chuvosa do que durante a estação seca. Esse resultado discorda de outros estudos sobre estrutura de habitat em sistemas temporários ou semipermanentes (por exemplo, Kennard, 1995; Medeiros, 2005), que constataram que a riqueza e a diversidade de estruturas de habitat foram maiores durante a estação seca, quando a água recua níveis provocam a exposição de estruturas físicas anteriormente submersas, como galhos e troncos. Os dados apresentados neste estudo indicam um efeito oposto, pois a vegetação submersa e a vegetação pendente foram elementos característicos do habitat durante a estação chuvosa. Dado o aumento da área dos sistemas aquáticos durante o período úmido, a vegetação que outrora ocupou as margens secas do litoral de rios e reservatórios é inundada, agregando ainda mais complexidade ao habitat. Além disso, à medida que os riachos inundados atingem suas margens, o contato com a vegetação ciliar atinge seu máximo, aumentando ainda mais a complexidade do habitat (por exemplo, Pusey e Arthington, 2003). Como indicado pela alta importância da vegetação pendente durante a estação chuvosa, proximidade com a vegetação ciliar gera sombra, limitando o crescimento de macrófitas, mas acrescenta elementos ao habitat litorâneo como massas de raízes, serrapilheira e detritos. A composição do substrato, por outro lado, é mais provável de ser descoberta durante a estação seca, mais importante ainda nos riachos intermitentes, onde características como pedras e cascalho tendem a ser mais frequentes em trechos mais profundos.

Apesar das diferenças na diversidade e riqueza de elementos de habitat entre Seridó e Buíque, a estrutura de habitat não foi diferente entre as duas áreas de estudo. Como o desenho amostral não é equilibrado (dois reservatórios na área do Buíque e apenas um na área do Seridó), a diversidade e a riqueza de elementos do habitat tendem a ser maiores no Seridó, devido à dominância de sites do tipo riacho nessa região. Pensa-se que os sistemas lóticos são organizados como uma hierarquia aninhada, tanto temporal quanto espacialmente (Johnson et al., 1995; Poff, 1997). Em uma hierarquia aninhada, processos de grande escala, como clima e geomorfologia, definem níveis mais altos de organização dos aspectos físicos e biológicos (por exemplo, morfologia do rio, regime de fluxo, pool de espécies) do ecossistema fluvial (Johnson et al., 1995). Esses níveis mais altos de organização influenciam fatores físicos e biológicos em uma variedade de escalas espaciais e temporais mais baixas, que podem ser particularmente relevantes para sistemas semiáridos (Poff, 1997; Maltchik e Medeiros, 2006). Portanto, em escalas temporais e espaciais relativamente curtas (ou seja, durante as estações secas e úmidas e através de bacias hidrográficas ou trechos de rios), vários fatores podem influenciar as variações no habitat físico disponível para os colonizadores.

No contexto do presente estudo, a Análise de Correspondência Canônica identificou vários níveis de variáveis de captação e alcance de riachos correlacionados com a estrutura do habitat. Ao nível da bacia, as variáveis que representam a ordem do riacho, como comprimento do rio, comprimento do córrego da localização do local e distância do local da fonte, foram importantes determinantes da estrutura do habitat. Na escala de alcance do córrego, a estrutura do habitat foi relacionada com a inclinação da margem e a largura do local, enquanto a profundidade não foi um determinante importante da estrutura do habitat. Como todas as variáveis de habitat foram medidas nas margens, não se deve esperar que a profundidade geral do local seja um elemento importante que afete a estrutura do habitat (a profundidade média foi de 34,4 ± 19,4 cm). Entretanto, a inclinação e a largura podem afetar o habitat físico através de sua influência no grau de conectividade entre o habitat aquático e terrestre. É importante lembrar que a declividade foi medida nas margens, compreendendo a porção terrestre das margens. Os parâmetros de qualidade da água em escala local também foram considerados importantes descritores da estrutura do habitat, principalmente aqueles associados ao conteúdo de oxigênio dissolvido na água (nomeadamente, oxigênio dissolvido e velocidade da água). Embora a temperatura e a transparência também possam estar relacionadas ao oxigênio dissolvido, a primeira foi relativamente constante em todos os locais de estudo (a temperatura média da água foi de 29,5 ± 2,9 °C) e a última (estimada a partir das profundidades de Secchi) foi muito alta em todas as áreas de estudo. como com médias por local variando de 16 a 90 cm. O modelo de habitat foi sugerido como uma estrutura onde fatores bióticos influenciam a biota (Southwood, 1977). Em ecossistemas de riachos, a inundação e a variabilidade do fluxo têm sido sugeridas como os principais eixos no modelo de habitat (Minshall, 1988; Poff e Ward, 1989). Embora os resultados apresentados neste estudo indiquem que as variações sazonais associadas ao regime de inundação e fluxo são elementos importantes que afetam a estrutura do habitat, há indícios de que o modelo de habitat pode ser multidimensional. Nos sistemas semiáridos do Brasil, a estrutura do habitat está sendo impulsionada por muitos componentes em diferentes níveis, onde o papel das características de captação e morfologia local e variáveis de água aumentariam em predominância em suas respectivas escalas. Isto realça a importância da ligação entre a estrutura do habitat e a diversidade biótica a nível local e regional.

O presente estudo mostra que a estrutura do habitat em sistemas semiáridos no nordeste do Brasil é composta por uma gama de elementos que variam de acordo com o tipo de habitat e estação do ano, enquanto a heterogeneidade do habitat foi mais forte nas escalas local e de captação do que entre áreas de captação. Portanto, vários níveis da hierarquia de captação são importantes para a estrutura do habitat, desde variáveis de escala de captação, como tamanho e elevação do rio, até características locais, como largura, inclinação e oxigênio dissolvido na água. Tais feições devem ser avaliadas e utilizadas como uma estrutura básica para caracterização e avaliação de habitats aquáticos no semiárido brasileiro. Os resultados apresentados têm implicações para a conservação e manejo dos sistemas semiáridos brasileiros. Dado que o habitat é a estrutura básica para a colonização de organismos aquáticos, os mecanismos potenciais que mantêm a diversidade biótica estão em todos os níveis da bacia hidrográfica. É fundamental, portanto, identificar as partes dos ecossistemas ribeirinhos que são vitais para a manutenção de sua saúde.

\hypertarget{apuxeandices-2}{%
\section*{Apêndices}\label{apuxeandices-2}}
\addcontentsline{toc}{section}{Apêndices}

\hypertarget{referuxeancias-3}{%
\section*{Referências}\label{referuxeancias-3}}
\addcontentsline{toc}{section}{Referências}

\hypertarget{part-estatuxedsticas-descritivas}{%
\part{ESTATÍSTICAS DESCRITIVAS}\label{part-estatuxedsticas-descritivas}}

\hypertarget{xls}{%
\chapter{R Modulo 1 - Importando planilhas de dados}\label{xls}}

\hypertarget{resumo-4}{%
\subsection*{RESUMO}\label{resumo-4}}
\addcontentsline{toc}{subsection}{RESUMO}

A importação de planilhas do Excel para o ambiente de programação R é uma tarefa fundamental para análise de dados e estatísticas. Através da importação de planilhas do Excel, é possível transformar dados armazenados em formatos familiares em estruturas que podem ser manipuladas e exploradas de maneira eficaz no R. Isso permite a aplicação de diversas técnicas estatísticas e criação de visualizações informativas, contribuindo para a tomada de decisões embasadas em dados. Neste contexto, entender como importar dados do Excel para o R é um passo crucial para realizar análises de alta qualidade e obter insights significativos a partir dos conjuntos de dados disponíveis.

\hypertarget{apresentauxe7uxe3o-3}{%
\subsection*{Apresentação}\label{apresentauxe7uxe3o-3}}
\addcontentsline{toc}{subsection}{Apresentação}

A importação de planilhas do Excel para o ambiente de programação R é uma tarefa fundamental para análise de dados e estatísticas. O R é uma linguagem de programação amplamente utilizada por cientistas de dados, pesquisadores e analistas para manipular, visualizar e modelar informações. Através da importação de planilhas do Excel, é possível transformar dados armazenados em formatos familiares em estruturas que podem ser manipuladas e exploradas de maneira eficaz no R. Isso permite a aplicação de diversas técnicas estatísticas e criação de visualizações informativas, contribuindo para a tomada de decisões embasadas em dados. Neste contexto, entender como importar dados do Excel para o R é um passo crucial para realizar análises de alta qualidade e obter insights significativos a partir dos conjuntos de dados disponíveis.

\hypertarget{instalauxe7uxe3o-do-r-e-rstudio}{%
\section{Instalação do R e RStudio}\label{instalauxe7uxe3o-do-r-e-rstudio}}

\hypertarget{r-base}{%
\section{R base}\label{r-base}}

\begin{enumerate}
\def\labelenumi{\arabic{enumi}.}
\tightlist
\item
  O primeiro passo é entrar na página do projeto CRAN (Comprehensive R Archive Network) (\url{https://www.r-project.org/}).\\
\item
  Do lado esquerdo da página clique sobre o link CRAN abaixo de Download. 3. Uma nova página com uma série de links irá se abrir. Esses links são chamados de ``espelhos'' e servem para que você possa escolher o local mais próximo de onde você está para fazer o download do programa. Escolha um espelho no Brasil.\\
\item
  Na seção Download and Install R, clique sobre o link \texttt{Download\ R\ for\ Windows} para baixar a versão para esse sistema\ldots{} (MacOS??\ldots{} o que é isso?)\\
\item
  Clique sobre o link \texttt{base}.
\item
  Clique sobre o link \texttt{Download\ R\ 4.x.x\ for\ Windows} para fazer o download do arquivo R.exe.
\item
  A instalação segue o formato padrão de instalação de programas no Windows, e portanto não são necessários maiores detalhes.
\end{enumerate}

\hypertarget{rstudio}{%
\section{RStudio}\label{rstudio}}

\begin{enumerate}
\def\labelenumi{\arabic{enumi}.}
\tightlist
\item
  Para baixar o RStudio entre no endereço (\url{https://posit.co/download/rstudio-desktop/})\\
\item
  Clique no link \texttt{Products\ \textgreater{}\ RStudio}\\
\item
  Selecione a versão Desktop.\\
\item
  Clique em \texttt{DOWNLOAD\ RSTUDIO\ DESKTOP}\\
\item
  Será exibida uma página com a recomendação para você baixar o RStudio FREE versão mais recente - Windows\\
\item
  Clicando nesse link, você irá baixar o arquivo RStudio atual (.exe)
\item
  Depois é só clicar e instalar da forma convencional do Windows.
\end{enumerate}

Após a instalação, você pode abrir o \texttt{RStudio} pelo seu respectivo ícone, e o RStudio estará pronto para ser utilizado. O ``R base'' continuará instalado mas será acessado pelo RStudio. Não o desinstale.

\hypertarget{rstudio-na-nuvem}{%
\subsection{RStudio na nuvem}\label{rstudio-na-nuvem}}

Usar o RStudio Cloud (\url{https://login.rstudio.cloud/}) é uma opção para quem não quer instalar a versão para PC. O RStudio Cloud é uma plataforma online que fornece um ambiente de desenvolvimento integrado para o R, permitindo que os usuários executem análises, desenvolvam código e colaborem com outras pessoas, sem a necessidade de instalar o R e o RStudio em seus próprios computadores. É uma solução conveniente e acessível, especialmente para iniciantes ou usuários que desejam compartilhar projetos e colaborar de forma eficiente.

\hypertarget{sobre-os-dados-do-ppbio-2}{%
\section{Sobre os dados do PPBio}\label{sobre-os-dados-do-ppbio-2}}

Usaremos para esse tutorial dados coletados no Programa de Pesquisa em Biodiversidade - PPBio (veja \href{https://ppbio.inpa.gov.br/Sobre}{Programa de Pesquisa em Biodiversidade -- PPBio}). Nesta base de dados estão armazenadas informações sobre diversos grupos taxonômicos dstribuidos em diversas unidades amostrais (UA's ou sítios), como peixes, macroinverbrebrados bentônicos, quironomídeos e zooplâncotn, além de dados do habitat, como variáveis físicas e químicas, morfologia do habitat, composição do substrato, estrutura de habitat marginal, entre outros (Figura \ref{fig:301ppbio}). Essa é a \textbf{matriz bruta de dados}, porque os valores ainda não foram ajustados para os valores de Captura Por Unidade de Esforço (CPUE), nem foram relativizados ou transformados (Tabela \ref{tab:301arqs}).

\begin{figure}
\includegraphics[width=13.21in]{imagens/ppbio} \caption{Parte da planilha de dados brutos do PPBio.}\label{fig:301ppbio}
\end{figure}

\begin{verbatim}
## Warning in table_info$align_vector[column] <-
## unlist(lapply(table_info$align_vector_origin[column], : número de itens para
## para substituir não é um múltiplo do comprimento do substituto
\end{verbatim}

\begin{table}

\caption{\label{tab:301arqs}Matrizes disponíveis para análises, com suas descrições e tipos de dados recomendados.}
\centering
\begin{tabular}[t]{>{\raggedright\arraybackslash}p{2cm}>{\raggedright\arraybackslash}p{2cm}>{\raggedright\arraybackslash}p{7cm}>{\raggedright\arraybackslash}p{5cm}}
\toprule
Arquivo (.xlsx) & Tipo de matriz & Descrição & Tipo de dados\\
\midrule
ppbio06c & Matriz comunitária & O arquivo ppbio06 traz os dados brutos que serão usados nas análises. A matriz de dados brutos contendo 26 localidades em estações do ano diferentes (objetos) x 35 espécies (atributos), antes de qualquer modificação. & Contagens de indivíduos com alta amplitude de variação, sugerido uso de matriz relativizada.\\
ppbio06h & Matriz ambiental & O arquivo ppbio06h traz os dados brutos que serão usados nas análises. A matriz de dados brutos contendo 26 localidades em estações diferentes (objetos) x 35 variáveis ambienteis (atributos) medidas em diferentes escalas espaciais, antes de qualquer modificação. & Unidades de medição diferentes (cm, m, °C, mg/L, etc.), com uma alta amplitude de variação, sugerido uso de matriz transformada e/ou reescalada.\\
ppbio06 & Matriz comunitária & O arquivo ppbio06 traz os dados brutos que serão usados nessa análise. A matriz de dados brutos contendo 26 locais/ocasiões (objetos) x 35 espécies (atributos), antes de qualquer modificação. & Contagens de indivíduos com alta amplitude de variação, sugerido uso de matriz relativizada.\\
ppbio06cpue & Matriz comunitária & O arquivo ppbio06cpue traz os valores depois de terem sidos ajustados pela Captura Por Unidade de Esforço (CPUE), onde o número de indivíduos de cada espécie em uma determinada UA é dividido pelo esforço de captura daquela UA. Isso significa que os dados foram relativizados pela CPUE. A matriz de dados brutos contendo 26 localidades em estações do ano diferentes (objetos) x 35 espécies (atributos), antes de qualquer modificação. & Densidades de indivíduos (no. de indivíduos por Unidade de Esforço de Captura) com alta amplitude de variação, sugerido uso de matriz relativizada.\\
\bottomrule
\end{tabular}
\end{table}

A planilha \texttt{ppbio} contém o delineamento amostral de um dos estudos do Projeto PPBio (Figura \ref{fig:301rm1}). Nas linhas são apresentadas as abreviações dos nomes das unidades amostrais (UA's) e nas colunas são apresentados os nomes abreviados das espécies - temos portando uma matriz comunitária (Tabela \ref{tab:301arqs}). No corpo da planilha temos os valores para o tipo de dados amostrado. Quantitativo, semi-quatitativo ou qualitativo.\\
Qual desses tipos de dados você acha que é apresentado na planilha?

\begin{figure}
\includegraphics[width=15.44in]{imagens/rm1.fig3} \caption{Associação entre a planilha de dados brutos do PPBio e o delineamento amostral do estudo.}\label{fig:301rm1}
\end{figure}

Várias das espécies nessa matriz tem grande importância ecológica, como é o caso de \emph{Astyanax bimaculatus} \footnote{A etimologia do gênero \emph{Astyanax} vem da mitologia Grega. Heitor personagem da ``Ilíada'', tinha um filho chamado Astíanax.} (Figura \ref{fig:301asbim}), que é muito comum em rios intermitentes e serve de alimento para predadores maiores como a espécie \emph{Hoplias malabaricus} \footnote{Do Grego, \emph{hoplon}, arma ou armadura, em referência aos dentes caniniformes muito desenvolvidos, e forte estrutura óssea na cabeça.} (Figura \ref{fig:301hmala}).

\begin{figure}
\includegraphics[width=8.89in]{imagens/Asbim_u4} \caption{*Astyanax bimaculatus*, a espécie mais comum da matriz de dados ppbio. Peru, by Eakins, R. Fonte: <https://www.fishbase.se/summary/Astianax-bimaculatus.html>}\label{fig:301asbim}
\end{figure}

\begin{figure}
\includegraphics[width=8.89in]{imagens/Homal_u6} \caption{*Hoplias malabaricus*, espécie que cresce para se tornar um importante predador. Brazil, by Roselet, F.F.G. Fonte: <https://www.fishbase.se/summary/Hoplias-malabaricus.html>}\label{fig:301hmala}
\end{figure}

\hypertarget{importando-a-planilha-de-trabalho}{%
\section{Importando a planilha de trabalho}\label{importando-a-planilha-de-trabalho}}

Para começar a usar o R e analisar os dados do Projeto PPBio, abra o RStudio, verifique sua interface (Figura \ref{fig:301interface}) e siga as instruções a seguir.

\begin{figure}
\includegraphics[width=18in]{imagens/interfaceR} \caption{Interface típica do RStudio e nome dos paineis  ou janelas.}\label{fig:301interface}
\end{figure}

\hypertarget{mensagens-de-erro-e-avisos-no-r}{%
\section{Mensagens de erro e avisos no R}\label{mensagens-de-erro-e-avisos-no-r}}

No contexto da linguagem de programação R, mensagens de erro (errors) e mensagens de aviso (warnings) que {aparecem em vermelho no painel de console}. Elas são formas de feedback do sistema que indicam problemas ou situações potencialmente problemáticas durante a execução do código. Aqui está uma breve explicação de cada um:

\begin{enumerate}
\def\labelenumi{\arabic{enumi}.}
\tightlist
\item
  \textbf{Erro ({Error}):}

  \begin{itemize}
  \tightlist
  \item
    Um erro ocorre quando algo no código não está correto ou não pode ser executado como esperado.
  \item
    Isso pode ser causado por sintaxe incorreta, uso incorreto de funções, operações inválidas, referências a objetos que não existem, entre outros problemas.
  \item
    Quando ocorre um erro, a execução do código é interrompida e uma mensagem de erro é exibida no console em vermelho, indicando o tipo de erro e, muitas vezes, a linha onde ocorreu.
  \end{itemize}
\item
  \textbf{Aviso ({Warning}):}

  \begin{itemize}
  \tightlist
  \item
    Não indica erro. Um aviso é emitido quando algo no código pode resultar em um comportamento indesejado ou em resultados inesperados, mas não interrompe necessariamente a execução do código.
  \item
    Os avisos geralmente indicam situações que merecem atenção, como conversões de tipos de dados que podem perder informações ou funções que estão sendo usadas de maneira que pode levar a resultados questionáveis.
  \item
    Os avisos são exibidos em vermelho no console e fornecem informações sobre a natureza do aviso e, possivelmente, como abordá-lo.
  \end{itemize}
\end{enumerate}

É importante prestar atenção a mensagens de erro e avisos, pois eles fornecem insights sobre problemas em seu código ou potenciais fontes de comportamento inesperado. Resolver erros é fundamental para que o código funcione conforme o esperado. Embora os avisos não interrompam a execução, investigá-los pode ajudar a evitar problemas futuros ou melhorar a qualidade do código.

\hypertarget{organizauxe7uxe3o-buxe1sica}{%
\section{Organização básica}\label{organizauxe7uxe3o-buxe1sica}}

\leavevmode\vadjust pre{\hypertarget{hello}{}}%
{ATENÇÃO}
Os links para baixar as planilhas necessárias para repetir esse tutorial podem ser encontrados na seção \ref{download401}

No ambiente do RStudio no painel de edição de código execute (\texttt{Ctrl+Enter} com o teclado ou \texttt{Run} no editor de código) os comandos a seguir, para instalar os pacotes necessários para este módulo.

\begin{Shaded}
\begin{Highlighting}[]
\FunctionTok{install.packages}\NormalTok{(}\StringTok{"readxl"}\NormalTok{) }\CommentTok{\#importa arquivos do excel}
\end{Highlighting}
\end{Shaded}

E em seguida,

\begin{Shaded}
\begin{Highlighting}[]
\FunctionTok{library}\NormalTok{(readxl)}
\end{Highlighting}
\end{Shaded}

Os códigos acima, são usados para instalar e carregar os pacotes necessários para este módulo. Esses códigos são comandos para instalar pacotes no R. Um pacote é uma coleção de funções, dados e documentação que ampliam as capacidades do R (\href{https://cran.r-project.org/}{R CRAN}) (\protect\hyperlink{ref-RN2774}{TEAM, 2017}), e \href{https://posit.co/download/rstudio-desktop/}{RStudio} (\protect\hyperlink{ref-RN358}{TEAM, 2022}). No exemplo acima, o pacote \texttt{readxl} permite ler e escrever arquivos Excel no R.

Para instalar um pacote no R, você precisa usar a função \texttt{install.packages()}. Depois de instalar um pacote, você precisa carregá-lo na sua sessão R com a função \texttt{library()}.

Por exemplo, para carregar o pacote \texttt{readxl}, você precisa executar a função \texttt{library(readxl)}. Isso irá permitir que você use as funções do pacote na sua sessão R. Você precisa carregar um pacote toda vez que iniciar uma nova sessão R e quiser usar um pacote instalado.

Agora vamos \textbf{definir o diretório de trabalho}. Esse código é usado para obter e definir o diretório de trabalho atual no R. O comando \texttt{getwd()} retorna o caminho do diretório onde o R está lendo e salvando arquivos. O comando \texttt{setwd()} muda esse diretório de trabalho para o caminho especificado entre aspas. No seu caso, você deve ajustar o caminho para o seu próprio diretório de trabalho. \textbf{Lembre de usar a barra ``/'' entre os diretórios. E não a contra-barra ``\textbackslash{}''.}

Usaremos uma matriz multivariada (sítios x espécies, matriz comunitária) do Projeto PPBio chamada ppbio**.xlsx que está no diretório {``C:/Meu/Diretório/De/Trabalho/Planilha.xlsx''}

Note que o símbolo \texttt{\#} em programação R significa que o texto que vem depois dele é um comentário e não será executado pelo programa. Isso é útil para explicar o código ou deixar anotações.

Ajuste a segunda linha do código abaixo para refletir {``C:/Seu/Diretório/De/Trabalho/Planilha.xlsx''}.

Definindo o diretório de trabalho e installando os pacotes necessários:

\begin{Shaded}
\begin{Highlighting}[]
\FunctionTok{getwd}\NormalTok{()}
\FunctionTok{setwd}\NormalTok{(}\StringTok{"C:/Seu/Diretório/De/Trabalho"}\NormalTok{)}
\end{Highlighting}
\end{Shaded}

Alternativamente você pode ir na barra de tarefas e escolhes as opções:\textbackslash SESSION -\textgreater{} SET WORKING DIRECTORY -\textgreater{} CHOOSE DIRECTORY

\hypertarget{prefira-sempre-cuxf3digos-e-scripts-do-que-mouse-e-menus-de-janelas-no-r}{%
\subsection{Prefira sempre códigos e scripts do que mouse e menus de janelas no R}\label{prefira-sempre-cuxf3digos-e-scripts-do-que-mouse-e-menus-de-janelas-no-r}}

Optar pelo uso de scripts e comandos de teclado no R, em vez das opções baseadas em mouse e menus das janelas, oferece várias vantagens significativas para quem está envolvido em análises de dados e programação. Aqui estão algumas justificativas para essa abordagem:

\begin{enumerate}
\def\labelenumi{\arabic{enumi}.}
\item
  \textbf{Reprodutibilidade:} O uso de scripts permite que todas as etapas de análise e manipulação de dados sejam documentadas em um único lugar. Isso facilita a reexecução de todo o processo, tornando a análise reprodutível e permitindo que outras pessoas compreendam e validem o trabalho realizado.
\item
  \textbf{Automação:} Comandos de script podem ser facilmente repetidos ou adaptados para diferentes conjuntos de dados. Isso possibilita a automação de tarefas complexas, economizando tempo e reduzindo a possibilidade de erros manuais.
\item
  \textbf{Flexibilidade:} Enquanto as opções de mouse e menus podem ser limitadas em termos das ações específicas que permitem, os scripts oferecem uma flexibilidade muito maior. Você pode personalizar cada etapa do processo de análise de acordo com suas necessidades específicas.
\item
  \textbf{Eficiência:} A digitação de comandos é geralmente mais rápida do que navegar por menus e clicar em botões, especialmente quando se trata de tarefas repetitivas e/ou complexas.
\item
  \textbf{Controle total:} Ao utilizar scripts, você tem controle total sobre cada etapa do processo. Isso é particularmente importante em análises estatísticas, onde pequenas variações nos parâmetros podem ter um grande impacto nos resultados.
\item
  \textbf{Aprendizado contínuo:} Escrever e modificar scripts permite um maior aprendizado e domínio da linguagem R. Conforme você ganha experiência, poderá realizar análises mais sofisticadas e explorar recursos avançados.
\item
  \textbf{Portabilidade:} Scripts podem ser facilmente compartilhados com outros pesquisadores ou colegas, independentemente do sistema operacional utilizado. Isso torna a colaboração mais fluida e ajuda a evitar problemas de compatibilidade.
\item
  \textbf{Melhor entendimento:} Ao escrever e ler scripts, você desenvolve uma compreensão mais profunda dos processos que está realizando. Isso é importante para identificar possíveis erros e interpretar corretamente os resultados.
\item
  \textbf{Documentação clara:} Ao escrever um script, você pode adicionar comentários explicativos que descrevem cada passo e sua lógica. Isso resulta em uma documentação clara e autoexplicativa do trabalho realizado.
\item
  \textbf{Consistência:} O uso de scripts promove a adoção de práticas consistentes em toda a análise, reduzindo a chance de erros causados por abordagens diferentes em momentos distintos.
\end{enumerate}

Em resumo, a abordagem baseada em scripts e comandos de teclado oferece mais controle, flexibilidade, eficiência e reprodutibilidade, tornando-a a escolha preferida para profissionais que buscam análises de dados precisas, consistentes e de alta qualidade no ambiente R.

\hypertarget{importando-a-planilha}{%
\section{Importando a planilha}\label{importando-a-planilha}}

\begin{Shaded}
\begin{Highlighting}[]
\FunctionTok{library}\NormalTok{(readxl)}
\NormalTok{ppbio06 }\OtherTok{\textless{}{-}} \FunctionTok{read\_excel}\NormalTok{(}\StringTok{"D:/Elvio/OneDrive/Disciplinas/\_EcoNumerica/5.Matrizes/ppbio06{-}peixes.xlsx"}\NormalTok{,}
                      \AttributeTok{sheet =} \StringTok{"Sheet1"}\NormalTok{,}
                      \AttributeTok{na =} \StringTok{"NA"}\NormalTok{)}
\FunctionTok{str}\NormalTok{(ppbio06)}
\FunctionTok{class}\NormalTok{(ppbio06)}
\end{Highlighting}
\end{Shaded}

Com essas linhas de código a primeira coluna da matriz importada apresenta texto. Não queremos assim porque vamos fazer cálculos matemáticos na matriz.

Resolvemos o problema com mais algumas linhas de código.

\begin{Shaded}
\begin{Highlighting}[]
\NormalTok{ppbio06 }\OtherTok{\textless{}{-}} \FunctionTok{as.data.frame}\NormalTok{(ppbio06)}
\FunctionTok{class}\NormalTok{(ppbio06)}
\FunctionTok{rownames}\NormalTok{(ppbio06) }\OtherTok{\textless{}{-}}\NormalTok{ ppbio06[,}\DecValTok{1}\NormalTok{] }\CommentTok{\#tem  que ser um df}
\NormalTok{ppbio06[,}\DecValTok{1}\NormalTok{] }\OtherTok{\textless{}{-}} \ConstantTok{NULL}
\end{Highlighting}
\end{Shaded}

Ou podemos instalar esse pacote de importação de arquivos .xlsx para o R.

\begin{Shaded}
\begin{Highlighting}[]
\FunctionTok{install.packages}\NormalTok{(}\StringTok{"openxlsx"}\NormalTok{)}
\end{Highlighting}
\end{Shaded}

\begin{Shaded}
\begin{Highlighting}[]
\FunctionTok{library}\NormalTok{(openxlsx)}
\end{Highlighting}
\end{Shaded}

\begin{verbatim}
## Warning: package 'openxlsx' was built under R version 4.3.2
\end{verbatim}

\begin{Shaded}
\begin{Highlighting}[]
\NormalTok{ppbio }\OtherTok{\textless{}{-}} \FunctionTok{read.xlsx}\NormalTok{(}\StringTok{"D:/Elvio/OneDrive/Disciplinas/\_EcoNumerica/5.Matrizes/ppbio06{-}peixes.xlsx"}\NormalTok{,}
                   \AttributeTok{rowNames =}\NormalTok{ T,}
                   \AttributeTok{colNames =}\NormalTok{ T,}
                   \AttributeTok{sheet =} \StringTok{"Sheet1"}\NormalTok{)}
\FunctionTok{str}\NormalTok{(ppbio)}
\FunctionTok{class}\NormalTok{(ppbio)}
\NormalTok{ppbio\_ma }\OtherTok{\textless{}{-}} \FunctionTok{as.matrix}\NormalTok{(ppbio) }\CommentTok{\#lê ppbio como uma matriz}
\FunctionTok{str}\NormalTok{(ppbio\_ma)}
\FunctionTok{class}\NormalTok{(ppbio\_ma)}
\CommentTok{\#ppbio}
\CommentTok{\#ppbio\_ma}
\end{Highlighting}
\end{Shaded}

Compare as diferenças. Agora podemos exportar os dados como uma matriz de dados em formato de \texttt{valores\ separados\ por\ vírgula\ (.csv)}.

\begin{Shaded}
\begin{Highlighting}[]
\FunctionTok{write.table}\NormalTok{(ppbio, }\StringTok{"ppbiocsv.txt"}\NormalTok{, }\AttributeTok{append =}\NormalTok{ F, }\AttributeTok{quote =}\NormalTok{ T, }\StringTok{";"}\NormalTok{, }\AttributeTok{row.names =}\NormalTok{ T)}
\NormalTok{dir }\OtherTok{\textless{}{-}} \FunctionTok{getwd}\NormalTok{()}
\FunctionTok{shell.exec}\NormalTok{(dir) }\CommentTok{\#abre o diretorio de trabalho no Windows Explorer}
\end{Highlighting}
\end{Shaded}

Podemos abrir o arquivo .csv criado \texttt{ppbiocsv.txt} usando os códigos abaixo.

\begin{Shaded}
\begin{Highlighting}[]
\NormalTok{ppbiocsv }\OtherTok{\textless{}{-}} \FunctionTok{read.csv}\NormalTok{(}\StringTok{"ppbiocsv.txt"}\NormalTok{,}
                     \AttributeTok{sep =} \StringTok{";"}\NormalTok{, }\AttributeTok{dec =} \StringTok{","}\NormalTok{, }\CommentTok{\#definimos o dígito separador}
                     \AttributeTok{header =}\NormalTok{ T,}
                     \AttributeTok{row.names =} \DecValTok{1}\NormalTok{,}
                     \AttributeTok{na.strings =} \ConstantTok{NA}\NormalTok{)}
\FunctionTok{str}\NormalTok{(ppbiocsv)}
\NormalTok{ppbiocsv}
\end{Highlighting}
\end{Shaded}

Lembre de prestar atenção no dígito separador de decimais '' , '' ou '' . '' . Além disso, só estaamos usando \texttt{ppbio**.***} porque o diretório de trabalho ja fo definido no início. Se não deveríamos estar usando \texttt{C:/Seu/Diretório/De/Trabalho/ppbio**.***}

Alguns comandos para exibir a planilha são ``case-sensitive'' \texttt{(ignore.case(object))}

\begin{Shaded}
\begin{Highlighting}[]
\CommentTok{\#View(ppbio)}
\FunctionTok{print}\NormalTok{(ppbio)}
\NormalTok{ppbio}
\FunctionTok{str}\NormalTok{(ppbio)}
\CommentTok{\#?View}
\CommentTok{\#?view}
\CommentTok{\#?remove}
\end{Highlighting}
\end{Shaded}

\hypertarget{outra-forma-de-achar-e-importar-uma-planilha}{%
\subsection{Outra forma de achar e importar uma planilha}\label{outra-forma-de-achar-e-importar-uma-planilha}}

Essa forma é desaconselhavel porque é demorada e sujeita a erros. Além de precisar ser refeita sempre que se quiser abrir uma nova planilha ou reabrir a última planilha importada. Veja o tópico \protect\hyperlink{prefira-sempre-cuxf3digos-e-scripts-do-que-mouse-e-menus-de-janelas-no-r}{Prefira sempre códigos e scripts do que mouse e menus de janelas no R}

\begin{Shaded}
\begin{Highlighting}[]
\FunctionTok{getwd}\NormalTok{()}
\NormalTok{ppbio }\OtherTok{\textless{}{-}} \FunctionTok{read.xlsx}\NormalTok{(}\FunctionTok{file.choose}\NormalTok{(),  }\CommentTok{\#abre o windows explorer}
                   \AttributeTok{rowNames =}\NormalTok{ T, }\AttributeTok{colNames =}\NormalTok{ T,}
                   \AttributeTok{sheet =} \StringTok{"Sheet1"}\NormalTok{)}
\end{Highlighting}
\end{Shaded}

\hypertarget{importando-.ods-do-libreoffice-calc}{%
\section{\texorpdfstring{Importando \texttt{.ods} do LibreOffice Calc}{Importando .ods do LibreOffice Calc}}\label{importando-.ods-do-libreoffice-calc}}

A planilha a seguir pode ser baixada da seção \ref{download401}

\begin{Shaded}
\begin{Highlighting}[]
\FunctionTok{install.packages}\NormalTok{(}\StringTok{"readODS"}\NormalTok{)}
\end{Highlighting}
\end{Shaded}

\begin{Shaded}
\begin{Highlighting}[]
\FunctionTok{library}\NormalTok{(readODS)}
\end{Highlighting}
\end{Shaded}

\begin{verbatim}
## Warning: package 'readODS' was built under R version 4.3.3
\end{verbatim}

\begin{Shaded}
\begin{Highlighting}[]
\NormalTok{rebio23.ods }\OtherTok{\textless{}{-}} \FunctionTok{read\_ods}\NormalTok{(}\StringTok{"D:/Elvio/OneDrive/Disciplinas/\_EcoNumerica/5.Matrizes/rebio23{-}zoopl\_5{-}20.ods"}\NormalTok{,}
                   \AttributeTok{row\_names =} \ConstantTok{TRUE}\NormalTok{,}
                   \AttributeTok{col\_names =} \ConstantTok{TRUE}\NormalTok{,}
                   \AttributeTok{sheet =} \StringTok{"densidade\_hab\_indL"}\NormalTok{,}
                   \AttributeTok{as\_tibble =} \ConstantTok{FALSE}\NormalTok{,}
                   \AttributeTok{na =} \StringTok{"n/a"}\NormalTok{) }\CommentTok{\# porque existem celulas vazias (n/a)}
\NormalTok{rebio23.ods }\OtherTok{\textless{}{-}} \FunctionTok{na.omit}\NormalTok{(rebio23.ods)}
\FunctionTok{str}\NormalTok{(rebio23.ods)}
\FunctionTok{class}\NormalTok{(rebio23.ods)}
\NormalTok{rebio23.ods\_ma }\OtherTok{\textless{}{-}} \FunctionTok{as.matrix}\NormalTok{(rebio23.ods) }\CommentTok{\#lê como uma matriz}
\FunctionTok{str}\NormalTok{(rebio23.ods\_ma)}
\FunctionTok{class}\NormalTok{(rebio23.ods\_ma)}
\CommentTok{\#rebio23.ods}
\CommentTok{\#rebio23.ods\_ma}
\end{Highlighting}
\end{Shaded}

\hypertarget{colunas-com-o-mesmo-nome}{%
\subsection{Colunas com o mesmo nome}\label{colunas-com-o-mesmo-nome}}

Em algumas situações é necessário encontrar colunas com o mesmo nome em um conjunto de dados e soma-las ou fazer sua média. Isso pode ocorrer quando se está trabalhando com dados de várias fontes e é necessário combinar esses diferentes conjuntos de dados.

Considere por exemplo um cenário onde se está trabalhando em um projeto de análise de dados ecológicos e você recebeu conjuntos de dados de diferentes locais de amostragens enviados por diferentes pesquisadores, e cada pesquiador enviou seu conjunto de dados que contém informações sobre as mesmas espécies (ou variáveis ambientais). Devido a diferentes sistemas de registro ou a falta de comunicação entre os pesquisadores, pode haver repetição nos nomes das colunas.

\begin{Shaded}
\begin{Highlighting}[]
\CommentTok{\#Dados de mamíferos roedores}
\NormalTok{df }\OtherTok{\textless{}{-}} \FunctionTok{data.frame}\NormalTok{(}
  \AttributeTok{Rato =} \FunctionTok{c}\NormalTok{(}\DecValTok{1}\NormalTok{, }\DecValTok{2}\NormalTok{, }\DecValTok{3}\NormalTok{),}
  \AttributeTok{Musaranho =} \FunctionTok{c}\NormalTok{(}\DecValTok{4}\NormalTok{, }\DecValTok{5}\NormalTok{, }\DecValTok{6}\NormalTok{),}
  \AttributeTok{Rato =} \FunctionTok{c}\NormalTok{(}\DecValTok{7}\NormalTok{, }\DecValTok{8}\NormalTok{, }\DecValTok{9}\NormalTok{), }\CommentTok{\#nome duplicado}
  \AttributeTok{Esquilo =} \FunctionTok{c}\NormalTok{(}\DecValTok{0}\NormalTok{, }\DecValTok{0}\NormalTok{, }\DecValTok{1}\NormalTok{),}
\NormalTok{  Ratão }\OtherTok{=} \FunctionTok{c}\NormalTok{(}\DecValTok{1}\NormalTok{, }\DecValTok{1}\NormalTok{, }\DecValTok{0}\NormalTok{),}
  \AttributeTok{Castor =} \FunctionTok{c}\NormalTok{(}\DecValTok{1}\NormalTok{, }\DecValTok{0}\NormalTok{, }\DecValTok{0}\NormalTok{),}
\NormalTok{  Tâmia }\OtherTok{=} \FunctionTok{c}\NormalTok{(}\DecValTok{11}\NormalTok{, }\DecValTok{12}\NormalTok{, }\DecValTok{13}\NormalTok{),}
  \AttributeTok{Marmota =} \FunctionTok{c}\NormalTok{(}\DecValTok{1}\NormalTok{, }\DecValTok{2}\NormalTok{, }\DecValTok{0}\NormalTok{),}
  \AttributeTok{Castor =} \FunctionTok{c}\NormalTok{(}\DecValTok{2}\NormalTok{, }\DecValTok{1}\NormalTok{, }\DecValTok{1}\NormalTok{), }\CommentTok{\#nome duplicado}
  \AttributeTok{check.names =} \ConstantTok{FALSE}
\NormalTok{)}
\NormalTok{df}
\end{Highlighting}
\end{Shaded}

\begin{verbatim}
##   Rato Musaranho Rato Esquilo Ratão Castor Tâmia Marmota Castor
## 1    1         4    7       0     1      1    11       1      2
## 2    2         5    8       0     1      0    12       2      1
## 3    3         6    9       1     0      0    13       0      1
\end{verbatim}

Por exemplo, no estudo sobre mamíferos roedores acima, quando se combina o conjunto geral de dados para realizar uma análise abrangente, depara-se com colunas duplicadas, onde a matriz com o conjunto total de dados contém espécies repetidas.

Nesse caso, encontrar e resolver colunas com o mesmo nome é crucial para garantir a integridade dos dados e realizar uma análise precisa. Você deve consolidar essas colunas duplicadas, somando-as ou fazendo sua média.

\begin{Shaded}
\begin{Highlighting}[]
\CommentTok{\# Achando colunas com nomes duplicados}
\NormalTok{dup\_cols }\OtherTok{\textless{}{-}} \FunctionTok{names}\NormalTok{(df)[}\FunctionTok{duplicated}\NormalTok{(}\FunctionTok{names}\NormalTok{(df))]}
\CommentTok{\# Somando colunas com o mesmo nome}
\ControlFlowTok{for}\NormalTok{ (col\_name }\ControlFlowTok{in} \FunctionTok{unique}\NormalTok{(dup\_cols)) \{}
  \CommentTok{\# Get indices of columns with the same name}
\NormalTok{  col\_indices }\OtherTok{\textless{}{-}} \FunctionTok{which}\NormalTok{(}\FunctionTok{names}\NormalTok{(df) }\SpecialCharTok{==}\NormalTok{ col\_name)}
  \CommentTok{\# Sum columns with the same name}
\NormalTok{  df[[col\_name]] }\OtherTok{\textless{}{-}} \FunctionTok{rowSums}\NormalTok{(df[, col\_indices, }\AttributeTok{drop =} \ConstantTok{FALSE}\NormalTok{])}
\NormalTok{\}}
\CommentTok{\# Remove as colunas duplicadas originais e mantem as novas colunas que são a soma ("except for the first occurrence")}
\NormalTok{df }\OtherTok{\textless{}{-}}\NormalTok{ df[, }\SpecialCharTok{!}\FunctionTok{duplicated}\NormalTok{(}\FunctionTok{names}\NormalTok{(df))]}
\CommentTok{\# Mostra a nova tabela com colunas repetidas somadas}
\FunctionTok{print}\NormalTok{(df)}
\end{Highlighting}
\end{Shaded}

\begin{verbatim}
##   Rato Musaranho Esquilo Ratão Castor Tâmia Marmota
## 1    8         4       0     1      3    11       1
## 2   10         5       0     1      1    12       2
## 3   12         6       1     0      1    13       0
\end{verbatim}

\hypertarget{removendo-linhas-ou-colunas-por-nome}{%
\subsection{Removendo linhas ou colunas por nome}\label{removendo-linhas-ou-colunas-por-nome}}

\begin{Shaded}
\begin{Highlighting}[]
\CommentTok{\#Colunas}
\NormalTok{df }\OtherTok{\textless{}{-}} \FunctionTok{subset}\NormalTok{(df, }\AttributeTok{select =} \SpecialCharTok{{-}}\NormalTok{Lec.mono) }\CommentTok{\#escolhendo uma coluna pelo nome}

\CommentTok{\#Linhas}
\NormalTok{del\_rows }\OtherTok{\textless{}{-}} \FunctionTok{c}\NormalTok{(}\StringTok{"ST20"}\NormalTok{)}
\NormalTok{del\_rows}
\NormalTok{m\_part }\OtherTok{\textless{}{-}}\NormalTok{ df[}\SpecialCharTok{!}\NormalTok{(}\FunctionTok{row.names}\NormalTok{(entorno) }\SpecialCharTok{\%in\%} \FunctionTok{c}\NormalTok{(del\_rows)),]}
\NormalTok{m\_part}
\end{Highlighting}
\end{Shaded}

\hypertarget{criando-uma-matriz-de-muxe9dias}{%
\subsection{Criando uma matriz de médias}\label{criando-uma-matriz-de-muxe9dias}}

Por razões diferentes o precedimento anterior pode vir a ser necessário de ser aplicado às linhas. Por exemplo, quando se quer somar ou fazer a média de amostras diferentes do mesmo ambiente de coleta. Veja a matriz abaixo.

\begin{Shaded}
\begin{Highlighting}[]
\NormalTok{data }\OtherTok{\textless{}{-}} \FunctionTok{read.table}\NormalTok{(}\AttributeTok{text =} \StringTok{"}
\StringTok{Sp1 Sp2 Sp3 Sp4 Sp5 Sp6 Sp7 Sp8}
\StringTok{A1 0 0 0 0 0 0 6 1}
\StringTok{A2 0 0 0 2 0 0 10 2}
\StringTok{B1 93 2 0 177 0 260 2 5}
\StringTok{B2 0 4 0 8 0 0 83 7}
\StringTok{C1 0 0 0 0 1 0 0 1}
\StringTok{C2 0 0 1 0 0 1 0 1}
\StringTok{C3 0 2 0 2 0 0 0 1}
\StringTok{"}\NormalTok{, }\AttributeTok{header =} \ConstantTok{TRUE}\NormalTok{, }\AttributeTok{row.names =} \DecValTok{1}\NormalTok{)}
\NormalTok{data}
\end{Highlighting}
\end{Shaded}

\begin{verbatim}
##    Sp1 Sp2 Sp3 Sp4 Sp5 Sp6 Sp7 Sp8
## A1   0   0   0   0   0   0   6   1
## A2   0   0   0   2   0   0  10   2
## B1  93   2   0 177   0 260   2   5
## B2   0   4   0   8   0   0  83   7
## C1   0   0   0   0   1   0   0   1
## C2   0   0   1   0   0   1   0   1
## C3   0   2   0   2   0   0   0   1
\end{verbatim}

\begin{Shaded}
\begin{Highlighting}[]
\FunctionTok{library}\NormalTok{(}\StringTok{"tidyverse"}\NormalTok{)}
\CommentTok{\#Inserindo coluna para agrupamentos}
\FunctionTok{nrow}\NormalTok{(data); }\FunctionTok{ncol}\NormalTok{(data) }\CommentTok{\#no. de N colunas x M linhas}
\NormalTok{data\_g }\OtherTok{\textless{}{-}} \FunctionTok{cbind}\NormalTok{(}\AttributeTok{Grupos =} \FunctionTok{rownames}\NormalTok{(data), data)}
\NormalTok{data\_g}

\NormalTok{grps }\OtherTok{\textless{}{-}} \FunctionTok{substr}\NormalTok{(data\_g[, }\DecValTok{1}\NormalTok{], }\DecValTok{1}\NormalTok{,}\DecValTok{3}\NormalTok{)}
\NormalTok{grps}

\NormalTok{data\_g }\OtherTok{\textless{}{-}}\NormalTok{ data\_g }\SpecialCharTok{\%\textgreater{}\%} \FunctionTok{mutate}\NormalTok{(}\AttributeTok{Grupos=}\FunctionTok{c}\NormalTok{(grps))}

\CommentTok{\#data\_avg \textless{}{-} aggregate(data\_g[, 9:9], list(data\_g$Grupos), mean)}
\CommentTok{\#data\_avg}

\NormalTok{data\_avg }\OtherTok{\textless{}{-}}\NormalTok{ data\_g }\SpecialCharTok{\%\textgreater{}\%} 
  \FunctionTok{group\_by}\NormalTok{(Grupos) }\SpecialCharTok{\%\textgreater{}\%}
  \FunctionTok{summarise}\NormalTok{(}\FunctionTok{across}\NormalTok{(}\AttributeTok{.cols =} \FunctionTok{everything}\NormalTok{(), }\SpecialCharTok{\textasciitilde{}} \FunctionTok{mean}\NormalTok{(.x, }\AttributeTok{na.rm =} \ConstantTok{TRUE}\NormalTok{)))}
\NormalTok{data\_avg}

\NormalTok{data\_dp }\OtherTok{\textless{}{-}}\NormalTok{ data\_g }\SpecialCharTok{\%\textgreater{}\%} 
  \FunctionTok{group\_by}\NormalTok{(Grupos) }\SpecialCharTok{\%\textgreater{}\%}
  \FunctionTok{summarise}\NormalTok{(}\FunctionTok{across}\NormalTok{(}\AttributeTok{.cols =} \FunctionTok{everything}\NormalTok{(), }\FunctionTok{list}\NormalTok{(}\AttributeTok{mean =}\NormalTok{ mean, }\AttributeTok{sd =}\NormalTok{ sd)))}
\CommentTok{\#?across}
\NormalTok{data\_dp}

\CommentTok{\#Primeira coluna para nomes das linhas }
\NormalTok{data\_dp }\OtherTok{\textless{}{-}} \FunctionTok{as.data.frame}\NormalTok{(data\_dp)}
\FunctionTok{class}\NormalTok{(data\_dp)}
\FunctionTok{rownames}\NormalTok{(data\_dp) }\OtherTok{\textless{}{-}}\NormalTok{ data\_dp[,}\DecValTok{1}\NormalTok{]}
\NormalTok{data\_dp[,}\DecValTok{1}\NormalTok{] }\OtherTok{\textless{}{-}} \ConstantTok{NULL}
\NormalTok{data\_dp }\OtherTok{\textless{}{-}} \FunctionTok{round}\NormalTok{(data\_dp, }\DecValTok{1}\NormalTok{)}
\NormalTok{data\_dp}
\CommentTok{\#Salvando a matriz}
\FunctionTok{write.table}\NormalTok{(data\_dp,}
            \StringTok{"data\_dp.csv"}\NormalTok{,}
            \AttributeTok{append =}\NormalTok{ F,}
            \AttributeTok{quote =} \ConstantTok{TRUE}\NormalTok{,}
            \AttributeTok{sep =} \StringTok{";"}\NormalTok{, }\AttributeTok{dec =} \StringTok{","}\NormalTok{,}
            \AttributeTok{row.names =}\NormalTok{ T)}
\NormalTok{data\_dp\_csv }\OtherTok{\textless{}{-}} \FunctionTok{read.csv}\NormalTok{(}\StringTok{"data\_dp.csv"}\NormalTok{,}
                    \AttributeTok{sep =} \StringTok{";"}\NormalTok{, }\AttributeTok{dec =} \StringTok{","}\NormalTok{,}
                    \AttributeTok{header =}\NormalTok{ T,}
                    \AttributeTok{row.names =} \DecValTok{1}\NormalTok{,}
                    \AttributeTok{na.strings =} \ConstantTok{NA}\NormalTok{)}
\end{Highlighting}
\end{Shaded}

\hypertarget{apuxeandices-3}{%
\section*{Apêndices}\label{apuxeandices-3}}
\addcontentsline{toc}{section}{Apêndices}

\hypertarget{sites-consultados-1}{%
\subsection*{Sites consultados}\label{sites-consultados-1}}
\addcontentsline{toc}{subsection}{Sites consultados}

{[}\url{https://youtu.be/U6ksXvvY6Q0}{]}\\
{[}\url{https://youtu.be/a7EJE_2mtGk}{]}

\hypertarget{script-limpo}{%
\section*{Script limpo}\label{script-limpo}}
\addcontentsline{toc}{section}{Script limpo}

Aqui apresento o scrip na íntegra sem os textos ou outros comentários. Você pode copiar e colar no R para executa-lo. Lembre de remover os {\texttt{\#}} ou {\texttt{\#\#}} caso necessite executar essas linhas.

\begin{Shaded}
\begin{Highlighting}[]
\DocumentationTok{\#\# install.packages("readxl") \#importa arquivos do excel}
\DocumentationTok{\#\# library(readxl)}
\DocumentationTok{\#\# getwd()}
\DocumentationTok{\#\# setwd("C:/Seu/Diretório/De/Trabalho")}
\FunctionTok{library}\NormalTok{(readxl)}
\NormalTok{ppbio06 }\OtherTok{\textless{}{-}} \FunctionTok{read\_excel}\NormalTok{(}\StringTok{"D:/Elvio/OneDrive/Disciplinas/\_EcoNumerica/5.Matrizes/ppbio06.xlsx"}\NormalTok{,}
                      \AttributeTok{sheet =} \StringTok{"Sheet1"}\NormalTok{,}
                      \AttributeTok{na =} \StringTok{"NA"}\NormalTok{)}
\FunctionTok{str}\NormalTok{(ppbio06)}
\FunctionTok{class}\NormalTok{(ppbio06)}
\NormalTok{ppbio06 }\OtherTok{\textless{}{-}} \FunctionTok{as.data.frame}\NormalTok{(ppbio06)}
\FunctionTok{class}\NormalTok{(ppbio06)}
\FunctionTok{rownames}\NormalTok{(ppbio06) }\OtherTok{\textless{}{-}}\NormalTok{ ppbio06[,}\DecValTok{1}\NormalTok{] }\CommentTok{\#tem  que ser um df}
\NormalTok{ppbio06[,}\DecValTok{1}\NormalTok{] }\OtherTok{\textless{}{-}} \ConstantTok{NULL}
\CommentTok{\#install.packages("openxlsx")}
\FunctionTok{library}\NormalTok{(openxlsx)}
\NormalTok{ppbio }\OtherTok{\textless{}{-}} \FunctionTok{read.xlsx}\NormalTok{(}\StringTok{"D:/Elvio/OneDrive/Disciplinas/\_EcoNumerica/5.Matrizes/bentos06.xlsx"}\NormalTok{,}
                   \AttributeTok{rowNames =}\NormalTok{ T,}
                   \AttributeTok{colNames =}\NormalTok{ T,}
                   \AttributeTok{sheet =} \StringTok{"contagem"}\NormalTok{)}
\FunctionTok{str}\NormalTok{(ppbio)}
\FunctionTok{class}\NormalTok{(ppbio)}
\NormalTok{ppbio\_ma }\OtherTok{\textless{}{-}} \FunctionTok{as.matrix}\NormalTok{(ppbio) }\CommentTok{\#lê ppbio como uma matriz}
\FunctionTok{str}\NormalTok{(ppbio\_ma)}
\FunctionTok{class}\NormalTok{(ppbio\_ma)}
\CommentTok{\#ppbio}
\CommentTok{\#ppbio\_ma}
\DocumentationTok{\#\# write.table(ppbio, "ppbiocsv.txt", append = F, quote = T, ";", row.names = T)}
\DocumentationTok{\#\# dir \textless{}{-} getwd()}
\DocumentationTok{\#\# shell.exec(dir) \#abre o diretorio de trabalho no Windows Explorer}
\DocumentationTok{\#\# ppbiocsv \textless{}{-} read.csv("ppbiocsv.txt",}
\DocumentationTok{\#\#                      sep = ";", dec = ",", \#definimos o dígito separador}
\DocumentationTok{\#\#                      header = T,}
\DocumentationTok{\#\#                      row.names = 1,}
\DocumentationTok{\#\#                      na.strings = NA)}
\DocumentationTok{\#\# str(ppbiocsv)}
\DocumentationTok{\#\# ppbiocsv}
\DocumentationTok{\#\# \#View(ppbio)}
\DocumentationTok{\#\# print(ppbio)}
\DocumentationTok{\#\# ppbio}
\DocumentationTok{\#\# str(ppbio)}
\DocumentationTok{\#\# \#?View}
\DocumentationTok{\#\# \#?view}
\DocumentationTok{\#\# \#?remove}
\DocumentationTok{\#\# getwd()}
\DocumentationTok{\#\# ppbio \textless{}{-} read.xlsx(file.choose(),  \#abre o windows explorer}
\DocumentationTok{\#\#                    rowNames = T, colNames = T,}
\DocumentationTok{\#\#                    sheet = "Sheet1")}
\end{Highlighting}
\end{Shaded}

\hypertarget{referuxeancias-4}{%
\section*{Referências}\label{referuxeancias-4}}
\addcontentsline{toc}{section}{Referências}

\hypertarget{tiposR}{%
\chapter{R Modulo 2 - Tipos de Dados no R}\label{tiposR}}

\hypertarget{apresentauxe7uxe3o-4}{%
\subsection*{Apresentação}\label{apresentauxe7uxe3o-4}}
\addcontentsline{toc}{subsection}{Apresentação}

O R é uma \textbf{linguagem de expressão} com uma sintaxe muito simples (veja TEAM (\protect\hyperlink{ref-RN1787}{1999-2024})). Ela é sensível a maiúsculas e minúsculas, como a maioria dos pacotes baseados em UNIX, então \texttt{A} e \texttt{a} são símbolos diferentes e se refeririam a variáveis diferentes. O conjunto de símbolos que podem ser usados nos nomes do R depende do sistema operacional e do país em que o R está sendo executado (tecnicamente, da localidade em uso). Normalmente, todos os símbolos alfanuméricos são permitidos (e em alguns países isso inclui letras acentuadas), além de \texttt{.} e \texttt{\_}, com a restrição de que um nome deve começar com \texttt{.} ou uma letra, e se começar com \texttt{.}, o segundo caractere não deve ser um dígito. Os nomes são efetivamente ilimitados em comprimento.

\leavevmode\vadjust pre{\hypertarget{hello}{}}%
{ATENÇÃO}
Evite usar nomes longos e complicados

\textbf{Comandos elementares consistem em expressões ou atribuições}. Se uma \texttt{expressão} for dada como um comando, ela é avaliada, impressa (a menos que seja especificamente tornada invisível) e o valor é perdido. Uma \texttt{atribuição} também avalia uma expressão e passa o valor para uma variável, mas o resultado não é automaticamente impresso.

Os comandos são separados por um ponto e vírgula (\texttt{;}) ou por uma nova linha (\texttt{Enter}). Comandos elementares podem ser agrupados em uma única expressão composta por chaves (\texttt{\{\textquotesingle{}\ and\ \textquotesingle{}\}}).

\textbf{Comentários} podem ser colocados em quase qualquer lugar, começando com uma marca de hash (\texttt{\#}), e tudo até o final da linha é um comentário.

Se um comando não estiver completo no final de uma linha, o R fornecerá um prompt diferente, por padrão \texttt{+} (ao invés do tradiocional \texttt{\textgreater{}}, nas linhas seguintes e continuará a ler a entrada até que o comando seja sintaticamente completo. Este prompt pode ser alterado pelo usuário. Geralmente, omitiremos o prompt de continuação e indicaremos a continuação apenas recuando.

As linhas de comando digitadas no console são limitadas a cerca de 4095 bytes (não caracteres).

No R, uma linguagem de programação amplamente utilizada para análise de dados e estatísticas, os tipos e estruturas de dados desempenham um papel fundamental na manipulação e organização das informações.

\hypertarget{tipos-e-estruturas-de-dados-no-r}{%
\section{Tipos e estruturas de dados no R}\label{tipos-e-estruturas-de-dados-no-r}}

No R, uma linguagem de programação amplamente utilizada para análise de dados e estatísticas, os tipos e estruturas de dados desempenham um papel fundamental na manipulação e organização das informações. Você vai conhecer esses tipos de estruturas de dados a seguir, mas antes se familiarize com como o R funciona.

Para começar a usar o R e analisar os dados do Projeto PPBio, abra o RStudio, verifique sua interface (Figura \ref{fig:interface2}) e siga as instruções a seguir.

\begin{figure}
\includegraphics[width=18in]{imagens/interfaceR} \caption{Interface típica do RStudio e nome dos paineis  ou janelas.}\label{fig:interface2}
\end{figure}

\hypertarget{mensagens-de-erro-e-avisos-no-r-1}{%
\section{Mensagens de erro e avisos no R}\label{mensagens-de-erro-e-avisos-no-r-1}}

No contexto da linguagem de programação R, mensagens de erro (errors) e mensagens de aviso (warnings) que {aparecem em vermelho no painel de console}. Elas são formas de feedback do sistema que indicam problemas ou situações potencialmente problemáticas durante a execução do código. Aqui está uma breve explicação de cada um:

\begin{enumerate}
\def\labelenumi{\arabic{enumi}.}
\tightlist
\item
  \textbf{Erro ({Error}):}

  \begin{itemize}
  \tightlist
  \item
    Um erro ocorre quando algo no código não está correto ou não pode ser executado como esperado.
  \item
    Isso pode ser causado por sintaxe incorreta, uso incorreto de funções, operações inválidas, referências a objetos que não existem, entre outros problemas.
  \item
    Quando ocorre um erro, a execução do código é interrompida e uma mensagem de erro é exibida no console em vermelho, indicando o tipo de erro e, muitas vezes, a linha onde ocorreu.
  \end{itemize}
\item
  \textbf{Aviso ({Warning}):}

  \begin{itemize}
  \tightlist
  \item
    Não indica erro. Um aviso é emitido quando algo no código pode resultar em um comportamento indesejado ou em resultados inesperados, mas não interrompe necessariamente a execução do código.
  \item
    Os avisos geralmente indicam situações que merecem atenção, como conversões de tipos de dados que podem perder informações ou funções que estão sendo usadas de maneira que pode levar a resultados questionáveis.
  \item
    Os avisos são exibidos em vermelho no console e fornecem informações sobre a natureza do aviso e, possivelmente, como abordá-lo.
  \end{itemize}
\end{enumerate}

É importante prestar atenção a mensagens de erro e avisos, pois eles fornecem insights sobre problemas em seu código ou potenciais fontes de comportamento inesperado. Resolver erros é fundamental para que o código funcione conforme o esperado. Embora os avisos não interrompam a execução, investigá-los pode ajudar a evitar problemas futuros ou melhorar a qualidade do código.

\hypertarget{o-que-uxe9-um-script-e-qual-sua-diferenuxe7a-para-um-cuxf3digo}{%
\section{\texorpdfstring{O que é um \texttt{script} e qual sua diferença para um \texttt{código}?}{O que é um script e qual sua diferença para um código?}}\label{o-que-uxe9-um-script-e-qual-sua-diferenuxe7a-para-um-cuxf3digo}}

Em programação R, tanto ``código'' quanto ``script'' se referem a sequências de instruções escritas na linguagem de programação R (veja também \protect\hyperlink{snippet-e-chunk}{Snippet e chunk}). No entanto, há uma diferença sutil em como esses termos são geralmente usados:

\begin{enumerate}
\def\labelenumi{\arabic{enumi}.}
\tightlist
\item
  Código:
\end{enumerate}

Em R, ``código'' geralmente se refere a linhas individuais ou blocos de instruções de programação R que realizam tarefas ou operações específicas.
O código pode ser escrito de forma interativa em um console R ou dentro de um arquivo de script.
O código pode consistir em declarações simples, como atribuir valores a variáveis, realizar cálculos, definir funções ou chamar funções de pacotes.

\begin{enumerate}
\def\labelenumi{\arabic{enumi}.}
\setcounter{enumi}{1}
\tightlist
\item
  Script:
\end{enumerate}

Um ``script'' em R refere-se a um arquivo contendo uma série de instruções de código R.
Scripts R são essencialmente arquivos de texto contendo uma sequência de comandos e declarações R que podem ser executados em conjunto.
Os scripts R permitem que os usuários organizem seu código em unidades reutilizáveis e estruturadas.
Os scripts R geralmente têm a extensão de arquivo ``.R''.

Em resumo, enquanto ``código'' se refere a instruções individuais de programação R, ``script'' se refere a um arquivo contendo uma coleção de código R, geralmente usado para executar tarefas ou análises maiores de forma estruturada e organizada.

Agora você pode começar a escrever seu código no console. Digite ``Início do R Módulo'' no console clique em \texttt{Run} (no canto superior direito do painel de edição de código) ou \texttt{Ctrl+Enter} e veja o que acontece:

\begin{Shaded}
\begin{Highlighting}[]
\StringTok{"Início do R Módulo"}
\end{Highlighting}
\end{Shaded}

\begin{verbatim}
## [1] "Início do R Módulo"
\end{verbatim}

O R mostra o que você escreveu, e apresenta este \texttt{{[}1{]}} do lado. Isso significa que seu resultado foi apresentado na primeira linha. Como você pode ver, o R imprime de volta (no Console) o resultado de comandos que você dá. Você agora pode tentar as operações basicas.

\begin{Shaded}
\begin{Highlighting}[]
\DecValTok{600}\SpecialCharTok{+}\DecValTok{66}
\end{Highlighting}
\end{Shaded}

\begin{verbatim}
## [1] 666
\end{verbatim}

Cálculos repetitivos podem ser automatizados usando um \texttt{loop}

\begin{Shaded}
\begin{Highlighting}[]
\NormalTok{x }\OtherTok{\textless{}{-}} \DecValTok{10}
\NormalTok{x}
\end{Highlighting}
\end{Shaded}

\begin{verbatim}
## [1] 10
\end{verbatim}

\begin{Shaded}
\begin{Highlighting}[]
\ControlFlowTok{for}\NormalTok{ (i }\ControlFlowTok{in} \DecValTok{1}\SpecialCharTok{:}\DecValTok{5}\NormalTok{) \{x}\SpecialCharTok{+}\DecValTok{1}\OtherTok{{-}\textgreater{}}\NormalTok{x\}}
\NormalTok{x}
\end{Highlighting}
\end{Shaded}

\begin{verbatim}
## [1] 15
\end{verbatim}

Aqui o número 1 foi somado cinco vezes ao valor de \texttt{x} que foi armazenado cada vez como \texttt{x} (substituindo o valor anterior) até chegar em 15. Na prática o R calculou 10+1=11, 11+1=12, 12+1=13, 13+1=14, 14+1=15.

Note que os \texttt{:} foi usado para produzir uma seqüência de números de 1 a 5.

\begin{Shaded}
\begin{Highlighting}[]
\DecValTok{1}\SpecialCharTok{:}\DecValTok{5}
\end{Highlighting}
\end{Shaded}

\begin{verbatim}
## [1] 1 2 3 4 5
\end{verbatim}

O R usa os seguintes símbolos para as operações básicas:

\begin{verbatim}
`+` Adição
`-` Subtração
`/` Divisão
`*` Multiplicação
`^` Potência
`sqrt` Raiz quadrada (função)
`log` Logarítmo (função)
\end{verbatim}

Teste agora outras possibilidades com as operações básicas (veja BEASLEY (\protect\hyperlink{ref-RN1484}{2004}) pags 7-10 para outros exemplos de cáculos básicos).

\leavevmode\vadjust pre{\hypertarget{hello}{}}%
Você já deve estar percebendo o potencial da programação em R como ferramenta para aprender estatística ou análise de dados, duas áreas com muita demanda de profissionais.

Em sua apostila ``\emph{Conversando com o R usando 57 palavras}'' MELO (\protect\hyperlink{ref-RN2785}{2014}) ressalta que a programação em R ``ajuda no sentido de prover métodos para resolvermos um problema ou entendermos uma análise''. Isso porque, primeiro, temos que reunir os elementos fundamentais e informações necessários, depois identificar e dar nomes a esses elementos, para apenas depois disso, decompor nosso problema em suas partes e, finalmente, colocar essas partes em sequência lógica (\protect\hyperlink{ref-RN2785}{MELO, 2014}).

A forma linear (passo-a-passo), mas hierarquicamente estruturada, da programação em R, permite uma construção intuitiva e lógica do raciocínio associado a cada análise, que, com o tempo, se torna parte cotidiano das nossas análises de dados. Assim como, na Língua Portuguesa, aprendemos o alfabeto, depois as palavras e eventualmente conversamos sem precisar pensar sobre a gramática e construção das frases.

\hypertarget{tipos-e-estruturas-de-dados-no-r-1}{%
\section{Tipos e estruturas de dados no R}\label{tipos-e-estruturas-de-dados-no-r-1}}

O R oferece uma variedade de tipos e estruturas de dados que permitem aos cientistas de dados e analistas manipular e organizar informações de forma eficiente. A escolha da estrutura de dados correta depende das necessidades específicas de análise e do tipo de dados que você está lidando.

\begin{enumerate}
\def\labelenumi{\arabic{enumi}.}
\tightlist
\item
  \textbf{Tipos de Dados Básicos}:

  \begin{itemize}
  \tightlist
  \item
    \textbf{Numéricos}: Representam valores numéricos, como inteiros ou números de ponto flutuante.
  \item
    \textbf{Caracteres}: Armazenam texto e são usados para representar sequências de caracteres.
  \item
    \textbf{Lógicos}: Armazenam valores lógicos (TRUE ou FALSE), úteis para expressar afirmações condicionais.
  \item
    \textbf{Complexos}: Armazenam números complexos.
  \end{itemize}
\item
  \textbf{Vetores}:

  \begin{itemize}
  \tightlist
  \item
    Os vetores são a estrutura de dados mais básica no R.
  \item
    Eles podem conter elementos de um único tipo de dado (exemplo: vetor numérico, de caracteres ou lógico).
  \item
    Os vetores são criados usando a função \texttt{c()}.
  \end{itemize}
\item
  \textbf{Matrizes}:

  \begin{itemize}
  \tightlist
  \item
    Matrizes são vetores bidimensionais, onde os elementos são organizados em linhas e colunas.
  \item
    Todos os elementos de uma matriz devem ser do mesmo tipo de dado.
  \item
    Matrizes podem ser criadas usando a função \texttt{matrix()}.
  \end{itemize}
\item
  \textbf{Data Frames}:

  \begin{itemize}
  \tightlist
  \item
    Data frames são estruturas de dados tabulares semelhantes a planilhas ou tabelas de banco de dados.
  \item
    Cada coluna de um data frame pode conter um tipo de dado diferente.
  \item
    Data frames são amplamente usados para armazenar e manipular conjuntos de dados.
  \item
    São frequentemente criados com funções como \texttt{data.frame()} ou lidos de arquivos externos.
  \end{itemize}
\item
  \textbf{Fatores}:

  \begin{itemize}
  \tightlist
  \item
    Fatores são usados para representar variáveis categóricas ou de fatores.
  \item
    São úteis para análises estatísticas e gráficos.
  \item
    Podem ser criados com a função \texttt{factor()}.
  \end{itemize}
\item
  \textbf{Listas}:

  \begin{itemize}
  \tightlist
  \item
    Listas são estruturas de dados flexíveis que podem conter elementos de diferentes tipos.
  \item
    Os elementos de uma lista podem ser acessados por meio de índices ou nomes.
  \item
    Listas são criadas usando a função \texttt{list()}.
  \end{itemize}
\item
  \textbf{Arrays}:

  \begin{itemize}
  \tightlist
  \item
    Arrays são estruturas multidimensionais que podem conter elementos de um único tipo de dado.
  \item
    São semelhantes às matrizes, mas podem ter mais de duas dimensões.
  \item
    Arrays são criados com a função \texttt{array()}.
  \end{itemize}
\item
  \textbf{Tabelas Hash}:

  \begin{itemize}
  \tightlist
  \item
    Tabelas hash são estruturas de dados que associam chaves a valores.
  \item
    Não são nativas do R, mas podem ser implementadas com pacotes como o \texttt{hash}.
  \end{itemize}
\end{enumerate}

\hypertarget{tipos-de-dados-no-r}{%
\section{Tipos de dados no R}\label{tipos-de-dados-no-r}}

\hypertarget{variuxe1veis-e-tipos-buxe1sicos-de-dados-figura-reffignucleos}{%
\subsection{Variáveis e tipos básicos de dados (Figura \ref{fig:nucleos})}\label{variuxe1veis-e-tipos-buxe1sicos-de-dados-figura-reffignucleos}}

\begin{enumerate}
\def\labelenumi{\arabic{enumi}.}
\tightlist
\item
  \textbf{Tipos de Dados Básicos}:

  \begin{itemize}
  \tightlist
  \item
    \textbf{Numéricos}: Representam valores numéricos, como inteiros ou números de ponto flutuante.
  \item
    \textbf{Caracteres}: Armazenam texto e são usados para representar sequências de caracteres.
  \item
    \textbf{Lógicos}: Armazenam valores lógicos (TRUE ou FALSE), úteis para expressar afirmações condicionais.
  \item
    \textbf{Complexos}: Armazenam números complexos.
  \end{itemize}
\end{enumerate}

Para guardar os valores em uma variável em R, usamos \texttt{\textless{}-} (operador de atribuição). Neste caso, o print automático não ocorre, e caso queira checar qual é o valor daquela variável, basta digitar seu nome novamente, ou utilizar o comando \texttt{print(nome\_da\_variavel)}.

Vejamos:

\begin{Shaded}
\begin{Highlighting}[]
\NormalTok{a }\OtherTok{\textless{}{-}} \DecValTok{5} \SpecialCharTok{+} \DecValTok{3}
\NormalTok{a}
\FunctionTok{print}\NormalTok{(a)}
\FunctionTok{class}\NormalTok{(a)}
\end{Highlighting}
\end{Shaded}

\begin{verbatim}
## [1] 8
## [1] 8
## [1] "numeric"
\end{verbatim}

Criamos um \textbf{objeto} do R chamado \texttt{a} que armazena o resultado de \texttt{5+3}. Objetos são palavras ou letras às quais são atribuídos dados. A atribuição possibilita a manipulação de dados ou armazenamento dos resultados de análises (\protect\hyperlink{ref-RN2940}{SILVA et al., 2022}).

O objeto \texttt{a} é uma variável numérica, como você pode ver ao usar a função \texttt{class()}. Se quiser trabalhar com inteiros (\texttt{integer}), você deve defini-los como \texttt{integer}, através da função \texttt{as.integer()}.

Vejamos:

\begin{Shaded}
\begin{Highlighting}[]
\NormalTok{var1 }\OtherTok{\textless{}{-}} \DecValTok{3}
\NormalTok{var2 }\OtherTok{\textless{}{-}} \FunctionTok{as.integer}\NormalTok{(}\DecValTok{3}\NormalTok{)}
\FunctionTok{class}\NormalTok{(var1)}
\FunctionTok{class}\NormalTok{(var2)}
\end{Highlighting}
\end{Shaded}

\begin{verbatim}
## [1] "numeric"
## [1] "integer"
\end{verbatim}

A função \texttt{class()} tem o mesmo efeito que a função \texttt{mode()}. Faça o teste.

Em R, os dados também podem ser do tipo \texttt{character}, que é uma variável de texto (em outras linguagens pode ser conhecida como \texttt{string}) e \texttt{logical} (que em outras linguagens pode ser conhecida como \texttt{boolean}), e que assume os valores \texttt{TRUE} ou \texttt{FALSE}.

Exemplos:

\begin{Shaded}
\begin{Highlighting}[]
\NormalTok{b }\OtherTok{\textless{}{-}} \StringTok{"Ola mundo"}
\NormalTok{c }\OtherTok{\textless{}{-}} \ConstantTok{TRUE}
\FunctionTok{class}\NormalTok{(b)}
\FunctionTok{class}\NormalTok{(c)}
\end{Highlighting}
\end{Shaded}

\begin{verbatim}
## [1] "character"
## [1] "logical"
\end{verbatim}

\leavevmode\vadjust pre{\hypertarget{hello}{}}%
Como você já deve ter percebido, uma grande parte da sua interação com o R é feita por comandos ou \texttt{funções}, onde cada função realiza um determinado papel. Seja comandar o R a mostrar uma variável recem criada, função \texttt{print()}, seja para informar o resultado de uma operação, função \texttt{log()}.

Uma \texttt{função} vem sempre seguida de parênteses \texttt{()} (no formato, \texttt{função(x)}), dentro dos quais colocamos os \texttt{argumentos} da função. Os argumentos, por sua vês, devem ser separados por vírgula \texttt{,}. Os argumentos são informações adicionais que especificam certas particularidades para serem executadas pelas \texttt{funções}.

Usando a função \texttt{log()} como exemplo, temos:

\begin{Shaded}
\begin{Highlighting}[]
\FunctionTok{log}\NormalTok{(}\AttributeTok{x=}\DecValTok{20}\NormalTok{, }\AttributeTok{base=}\DecValTok{10}\NormalTok{)}
\end{Highlighting}
\end{Shaded}

\begin{verbatim}
## [1] 1.30103
\end{verbatim}

O argumento \texttt{x=} indica de qual valor queremos o log e o argumento \texttt{base=} indica qual base para o logaritmo o R vai usar.

Ou, em um segundo caso, podemos ter simplesmente a função sem nenhuma indicação de argumento:

\begin{Shaded}
\begin{Highlighting}[]
\FunctionTok{log}\NormalTok{(}\DecValTok{20}\NormalTok{)}
\end{Highlighting}
\end{Shaded}

\begin{verbatim}
## [1] 2.995732
\end{verbatim}

Muitas funções possuem opções padrões (\emph{default}) para alguns de seus argumentos. No caso do log, se não for dado nenhum argumento o \texttt{default} do R é usar o log natural do valor entre parêntesis.

\hypertarget{estruturas-de-dados-no-r}{%
\section{Estruturas de dados no R}\label{estruturas-de-dados-no-r}}

Agora vamos falar das estruturas de dados no R (Figura \ref{fig:tipos}).

\begin{enumerate}
\def\labelenumi{\arabic{enumi}.}
\setcounter{enumi}{1}
\tightlist
\item
  \textbf{Vetores}:

  \begin{itemize}
  \tightlist
  \item
    Os vetores são a estrutura de dados mais básica no R.
  \item
    Eles podem conter elementos de um único tipo de dado (exemplo: numérico, caracteres ou lógico).
  \item
    Os vetores são criados usando a função \texttt{c()}.
  \end{itemize}
\end{enumerate}

Como ja foi dito, os vetores são uma sequência simples de elementos do mesmo tipo. Quando definimos uma variável como fizemos anteriormente, é criado um vetor com um elemento (vetor numérico, vetor de caractere, etc.).

O R opera em \textbf{estruturas de dados nomeadas}. A estrutura mais simples é o \textbf{vetor numérico}, que é uma entidade única consistindo de uma coleção ordenada de números. Para configurar um vetor chamado x, por exemplo, consistindo de cinco números, a saber, 1.4, 6.5, 3, 4 e 71.2, usa-se o comando R:

\begin{Shaded}
\begin{Highlighting}[]
\NormalTok{x }\OtherTok{\textless{}{-}} \FunctionTok{c}\NormalTok{(}\FloatTok{1.4}\NormalTok{, }\FloatTok{6.5}\NormalTok{, }\DecValTok{3}\NormalTok{, }\DecValTok{4}\NormalTok{, }\FloatTok{71.2}\NormalTok{)}
\end{Highlighting}
\end{Shaded}

Esta é uma instrução de atribuição usando a função \texttt{c()} (``combinar''), que neste contexto pode receber um número arbitrário de argumentos de vetor e cujo valor é um vetor obtido pela concatenação de seus argumentos de ponta a ponta.

Observe que o operador de atribuição (\texttt{\textless{}-}), que consiste nos dois caracteres `\textless{}' (``menor que'') e `-' (``menos'') ocorrendo estritamente lado a lado, e ele ``aponta'' para o objeto que recebe o valor da expressão. Na maioria dos contextos, o operador `=' pode ser usado como uma alternativa.

Observe:

\begin{Shaded}
\begin{Highlighting}[]
\NormalTok{var1 }\OtherTok{\textless{}{-}} \DecValTok{3}
\FunctionTok{is.vector}\NormalTok{(var1)}
\end{Highlighting}
\end{Shaded}

\begin{verbatim}
## [1] TRUE
\end{verbatim}

Isso explica aquele \texttt{{[}1{]}} que vimos no início do módulo. Quando você associa uma variável dessa forma, o R cria um vetor, e o valor que você definiu será o primeiro item deste vetor.

Para criarmos vetores com mais de um elemento, incluiremos os valores desejados dentro de um \texttt{c()} que significa \texttt{combinar}.

Veja o exemplo:

\begin{Shaded}
\begin{Highlighting}[]
\NormalTok{var1 }\OtherTok{\textless{}{-}} \FunctionTok{c}\NormalTok{(}\DecValTok{3}\NormalTok{,}\DecValTok{6}\NormalTok{, }\FloatTok{7.8}\NormalTok{, }\DecValTok{332}\NormalTok{)}
\FunctionTok{print}\NormalTok{(var1)}
\NormalTok{var2 }\OtherTok{\textless{}{-}} \FunctionTok{c}\NormalTok{(}\StringTok{"Ola"}\NormalTok{,}\StringTok{"tudo"}\NormalTok{,}\StringTok{"bem"}\NormalTok{)}
\FunctionTok{print}\NormalTok{(var2)}
\NormalTok{var3 }\OtherTok{\textless{}{-}} \FunctionTok{c}\NormalTok{(}\ConstantTok{TRUE}\NormalTok{, }\ConstantTok{FALSE}\NormalTok{, }\ConstantTok{TRUE}\NormalTok{, }\ConstantTok{TRUE}\NormalTok{, }\ConstantTok{FALSE}\NormalTok{, }\ConstantTok{FALSE}\NormalTok{, }\ConstantTok{TRUE}\NormalTok{, }\ConstantTok{TRUE}\NormalTok{, }\ConstantTok{TRUE}\NormalTok{)}
\FunctionTok{print}\NormalTok{(var3)}
\end{Highlighting}
\end{Shaded}

\begin{verbatim}
## [1]   3.0   6.0   7.8 332.0
## [1] "Ola"  "tudo" "bem" 
## [1]  TRUE FALSE  TRUE  TRUE FALSE FALSE  TRUE  TRUE  TRUE
\end{verbatim}

Caso você tente misturar os tipos, o R irá forçar para que os elementos sejam
todos do mesmo tipo. Veja alguns exemplos:

\begin{enumerate}
\def\labelenumi{\arabic{enumi}.}
\setcounter{enumi}{2}
\tightlist
\item
  \textbf{Matrizes}:

  \begin{itemize}
  \tightlist
  \item
    Matrizes são vetores bidimensionais, onde os elementos são organizados em linhas e colunas.
  \item
    Todos os elementos de uma matriz devem ser do mesmo tipo de dado.
  \item
    Matrizes podem ser criadas usando a função \texttt{matrix()}.
  \end{itemize}
\end{enumerate}

As matrizes são estruturas que correspondem às matrizes matemáticas, conjuntos de elementos com linhas e colunas. Assim como os vetores, todos os seus elementos sao do mesmo tipo. Existem algumas formas de criar uma matriz:

\begin{Shaded}
\begin{Highlighting}[]
\NormalTok{mat1 }\OtherTok{\textless{}{-}} \FunctionTok{matrix}\NormalTok{(}\FunctionTok{c}\NormalTok{(}\DecValTok{1}\NormalTok{,}\DecValTok{5}\NormalTok{,}\DecValTok{10}\NormalTok{,}\DecValTok{30}\NormalTok{,}\DecValTok{15}\NormalTok{,}\DecValTok{8}\NormalTok{),}
 \AttributeTok{nrow=}\DecValTok{3}\NormalTok{,}
 \AttributeTok{ncol=}\DecValTok{2}\NormalTok{,}
 \AttributeTok{byrow=}\ConstantTok{TRUE}\NormalTok{)}
\FunctionTok{print}\NormalTok{(mat1)}
\end{Highlighting}
\end{Shaded}

\begin{verbatim}
##      [,1] [,2]
## [1,]    1    5
## [2,]   10   30
## [3,]   15    8
\end{verbatim}

Ou

\begin{Shaded}
\begin{Highlighting}[]
\NormalTok{vec1 }\OtherTok{\textless{}{-}} \FunctionTok{c}\NormalTok{(}\DecValTok{1}\NormalTok{, }\DecValTok{5}\NormalTok{)}
\NormalTok{vec2 }\OtherTok{\textless{}{-}} \FunctionTok{c}\NormalTok{(}\DecValTok{10}\NormalTok{, }\DecValTok{30}\NormalTok{)}
\NormalTok{vec3 }\OtherTok{\textless{}{-}} \FunctionTok{c}\NormalTok{(}\DecValTok{15}\NormalTok{, }\DecValTok{8}\NormalTok{) }
\NormalTok{mat2 }\OtherTok{\textless{}{-}} \FunctionTok{rbind}\NormalTok{(vec1, vec2, vec3)}
\FunctionTok{print}\NormalTok{(mat2)}
\FunctionTok{class}\NormalTok{(mat2)}
\end{Highlighting}
\end{Shaded}

\begin{verbatim}
##      [,1] [,2]
## vec1    1    5
## vec2   10   30
## vec3   15    8
## [1] "matrix" "array"
\end{verbatim}

Ou

\begin{Shaded}
\begin{Highlighting}[]
\NormalTok{a }\OtherTok{\textless{}{-}} \FunctionTok{sample}\NormalTok{(}\FunctionTok{c}\NormalTok{(}\DecValTok{0}\NormalTok{,}\DecValTok{1}\NormalTok{,}\DecValTok{5}\NormalTok{),}\DecValTok{10}\NormalTok{,}\AttributeTok{replace=}\ConstantTok{TRUE}\NormalTok{)}
\NormalTok{b }\OtherTok{\textless{}{-}} \FunctionTok{sample}\NormalTok{(}\FunctionTok{c}\NormalTok{(}\DecValTok{0}\NormalTok{,}\DecValTok{10}\NormalTok{,}\DecValTok{20}\NormalTok{),}\DecValTok{10}\NormalTok{,}\AttributeTok{replace=}\ConstantTok{TRUE}\NormalTok{)}
\NormalTok{c }\OtherTok{\textless{}{-}} \FunctionTok{sample}\NormalTok{(}\FunctionTok{c}\NormalTok{(}\DecValTok{0}\NormalTok{,}\DecValTok{5}\NormalTok{,}\DecValTok{10}\NormalTok{),}\DecValTok{10}\NormalTok{,}\AttributeTok{replace=}\ConstantTok{TRUE}\NormalTok{)}
\NormalTok{df1 }\OtherTok{\textless{}{-}} \FunctionTok{data.frame}\NormalTok{(a,b,c)}
\NormalTok{df1}
\end{Highlighting}
\end{Shaded}

\begin{verbatim}
##    a  b  c
## 1  0 10  0
## 2  1 20  0
## 3  1  0 10
## 4  0 10 10
## 5  1 10 10
## 6  5 10  0
## 7  0 20  0
## 8  1  0  5
## 9  5 10  5
## 10 5  0  5
\end{verbatim}

Na primeira forma, definimos os elementos da matriz, o número de linhas e colunas através dos \texttt{argumentos} \texttt{nrow} e \texttt{ncol}, e se os elementos serão preenchidos ao longo das linhas (\texttt{byrow\ =\ TRUE}) ou ao longo das colunas (\texttt{byrow\ =\ FALSE}).

Na segunda forma, criamos dois vetores e então os juntamos através da função \texttt{rbind()}. Repare que na segunda matriz, o R nomeou automaticamente as linhas da matriz com os nomes dos vetores, ao invés de definir seus números.

Para selecionar itens de matrizes também utilizamos os colchetes, primeiro especificando a linha e depois a coluna que desejamos retornar:

\begin{Shaded}
\begin{Highlighting}[]
\NormalTok{mat1[}\DecValTok{2}\NormalTok{][}\DecValTok{1}\NormalTok{]}
\end{Highlighting}
\end{Shaded}

\begin{verbatim}
## [1] 10
\end{verbatim}

\begin{enumerate}
\def\labelenumi{\arabic{enumi}.}
\setcounter{enumi}{3}
\tightlist
\item
  \textbf{Data Frames}:

  \begin{itemize}
  \tightlist
  \item
    Data frames são estruturas de dados tabulares semelhantes a planilhas ou tabelas de banco de dados.
  \item
    Cada coluna de um data frame pode conter um tipo de dado diferente.
  \item
    Data frames são amplamente usados para armazenar e manipular conjuntos de dados.
  \item
    São frequentemente criados com funções como \texttt{data.frame()} ou lidos de arquivos externos.
  \end{itemize}
\end{enumerate}

O \texttt{Data\ Frame} é a estrutura do R utilizada para armazenar elementos em forma de tabela, organizados em linhas e colunas. As colunas e linhas podem ser nomeadas. Você pode criar um data frame com a função \texttt{data.frame()}:

\begin{Shaded}
\begin{Highlighting}[]
\NormalTok{df1 }\OtherTok{\textless{}{-}} \FunctionTok{data.frame}\NormalTok{(}\FunctionTok{c}\NormalTok{(}\DecValTok{1}\NormalTok{,}\DecValTok{2}\NormalTok{,}\DecValTok{3}\NormalTok{),}\FunctionTok{c}\NormalTok{(}\StringTok{"baixo"}\NormalTok{,}\StringTok{"medio"}\NormalTok{,}\StringTok{"alto"}\NormalTok{),}\FunctionTok{c}\NormalTok{(}\ConstantTok{TRUE}\NormalTok{, }\ConstantTok{TRUE}\NormalTok{, }\ConstantTok{FALSE}\NormalTok{))}
\FunctionTok{print}\NormalTok{(df1)}
\end{Highlighting}
\end{Shaded}

\begin{verbatim}
##   c.1..2..3. c..baixo....medio....alto.. c.TRUE..TRUE..FALSE.
## 1          1                       baixo                 TRUE
## 2          2                       medio                 TRUE
## 3          3                        alto                FALSE
\end{verbatim}

Ou

\begin{Shaded}
\begin{Highlighting}[]
\NormalTok{df2 }\OtherTok{\textless{}{-}} \FunctionTok{data.frame}\NormalTok{(}\AttributeTok{Sp1=}\FunctionTok{c}\NormalTok{(}\DecValTok{1}\NormalTok{,}\DecValTok{5}\NormalTok{,}\DecValTok{2}\NormalTok{,}\DecValTok{5}\NormalTok{,}\DecValTok{3}\NormalTok{), }\AttributeTok{Sp2=}\FunctionTok{c}\NormalTok{(}\DecValTok{21}\NormalTok{,}\DecValTok{8}\NormalTok{,}\DecValTok{10}\NormalTok{,}\DecValTok{7}\NormalTok{,}\DecValTok{18}\NormalTok{)) }
\FunctionTok{row.names}\NormalTok{(df2) }\OtherTok{\textless{}{-}}\NormalTok{ LETTERS[}\DecValTok{1}\SpecialCharTok{:}\DecValTok{5}\NormalTok{]}
\NormalTok{df2}
\end{Highlighting}
\end{Shaded}

\begin{verbatim}
##   Sp1 Sp2
## A   1  21
## B   5   8
## C   2  10
## D   5   7
## E   3  18
\end{verbatim}

É possivel plotar esse dataframe:

\begin{Shaded}
\begin{Highlighting}[]
\FunctionTok{plot}\NormalTok{(df2, }\AttributeTok{type=}\StringTok{\textquotesingle{}n\textquotesingle{}}\NormalTok{); }\FunctionTok{text}\NormalTok{(df2, }\FunctionTok{row.names}\NormalTok{(df2))}
\end{Highlighting}
\end{Shaded}

\includegraphics{fundamentos_de_ecologia_numerica_files/figure-latex/unnamed-chunk-39-1.pdf}

\texttt{Dataframes} ou \texttt{Matrizes} muito grandes são inviáveis de serem inseridas no ambiente do R através de digitação simples como mostrado anteriormente. Uma opção é usar a função \texttt{read.table(text="")} e colar a matriz de dados entre as aspas \texttt{"..."}

\begin{Shaded}
\begin{Highlighting}[]
\NormalTok{data }\OtherTok{\textless{}{-}} \FunctionTok{read.table}\NormalTok{(}\AttributeTok{text =} \StringTok{"}
\StringTok{Sp1 Sp2 Sp3 Sp4 Sp5 Sp6 Sp7}
\StringTok{A 0 0 0 0 0 0 6}
\StringTok{B 0 0 0 2 0 0 10}
\StringTok{C 93 2 0 177 0 260 2}
\StringTok{D 0 4 0 8 0 0 83}
\StringTok{E 0 0 0 0 1 0 0}
\StringTok{F 0 0 1 0 0 1 0}
\StringTok{G 0 2 0 2 0 0 0}
\StringTok{"}\NormalTok{, }\AttributeTok{header =} \ConstantTok{TRUE}\NormalTok{, }\AttributeTok{row.names =} \DecValTok{1}\NormalTok{)}
\NormalTok{data}
\end{Highlighting}
\end{Shaded}

\begin{verbatim}
##   Sp1 Sp2 Sp3 Sp4 Sp5 Sp6 Sp7
## A   0   0   0   0   0   0   6
## B   0   0   0   2   0   0  10
## C  93   2   0 177   0 260   2
## D   0   4   0   8   0   0  83
## E   0   0   0   0   1   0   0
## F   0   0   1   0   0   1   0
## G   0   2   0   2   0   0   0
\end{verbatim}

O R vem com alguns conjuntos de dados organizados em Data Frame na sua base, para a finalidade de testes e aprendizado. Um deles é o \texttt{mtcars}:

\begin{Shaded}
\begin{Highlighting}[]
\FunctionTok{print}\NormalTok{(mtcars)}
\end{Highlighting}
\end{Shaded}

\begin{verbatim}
##                      mpg cyl  disp  hp drat    wt  qsec vs am gear carb
## Mazda RX4           21.0   6 160.0 110 3.90 2.620 16.46  0  1    4    4
## Mazda RX4 Wag       21.0   6 160.0 110 3.90 2.875 17.02  0  1    4    4
## Datsun 710          22.8   4 108.0  93 3.85 2.320 18.61  1  1    4    1
## Hornet 4 Drive      21.4   6 258.0 110 3.08 3.215 19.44  1  0    3    1
## Hornet Sportabout   18.7   8 360.0 175 3.15 3.440 17.02  0  0    3    2
## Valiant             18.1   6 225.0 105 2.76 3.460 20.22  1  0    3    1
## Duster 360          14.3   8 360.0 245 3.21 3.570 15.84  0  0    3    4
## Merc 240D           24.4   4 146.7  62 3.69 3.190 20.00  1  0    4    2
## Merc 230            22.8   4 140.8  95 3.92 3.150 22.90  1  0    4    2
## Merc 280            19.2   6 167.6 123 3.92 3.440 18.30  1  0    4    4
## Merc 280C           17.8   6 167.6 123 3.92 3.440 18.90  1  0    4    4
## Merc 450SE          16.4   8 275.8 180 3.07 4.070 17.40  0  0    3    3
## Merc 450SL          17.3   8 275.8 180 3.07 3.730 17.60  0  0    3    3
## Merc 450SLC         15.2   8 275.8 180 3.07 3.780 18.00  0  0    3    3
## Cadillac Fleetwood  10.4   8 472.0 205 2.93 5.250 17.98  0  0    3    4
## Lincoln Continental 10.4   8 460.0 215 3.00 5.424 17.82  0  0    3    4
## Chrysler Imperial   14.7   8 440.0 230 3.23 5.345 17.42  0  0    3    4
## Fiat 128            32.4   4  78.7  66 4.08 2.200 19.47  1  1    4    1
## Honda Civic         30.4   4  75.7  52 4.93 1.615 18.52  1  1    4    2
## Toyota Corolla      33.9   4  71.1  65 4.22 1.835 19.90  1  1    4    1
## Toyota Corona       21.5   4 120.1  97 3.70 2.465 20.01  1  0    3    1
## Dodge Challenger    15.5   8 318.0 150 2.76 3.520 16.87  0  0    3    2
## AMC Javelin         15.2   8 304.0 150 3.15 3.435 17.30  0  0    3    2
## Camaro Z28          13.3   8 350.0 245 3.73 3.840 15.41  0  0    3    4
## Pontiac Firebird    19.2   8 400.0 175 3.08 3.845 17.05  0  0    3    2
## Fiat X1-9           27.3   4  79.0  66 4.08 1.935 18.90  1  1    4    1
## Porsche 914-2       26.0   4 120.3  91 4.43 2.140 16.70  0  1    5    2
## Lotus Europa        30.4   4  95.1 113 3.77 1.513 16.90  1  1    5    2
## Ford Pantera L      15.8   8 351.0 264 4.22 3.170 14.50  0  1    5    4
## Ferrari Dino        19.7   6 145.0 175 3.62 2.770 15.50  0  1    5    6
## Maserati Bora       15.0   8 301.0 335 3.54 3.570 14.60  0  1    5    8
## Volvo 142E          21.4   4 121.0 109 4.11 2.780 18.60  1  1    4    2
\end{verbatim}

Algumas funções úteis para conjuntos muito grandes de dados que você deve conhecer:

\begin{Shaded}
\begin{Highlighting}[]
\NormalTok{mtcars[}\DecValTok{1}\NormalTok{,}\DecValTok{2}\NormalTok{]}
\end{Highlighting}
\end{Shaded}

\begin{verbatim}
## [1] 6
\end{verbatim}

\begin{Shaded}
\begin{Highlighting}[]
\NormalTok{mtcars[}\StringTok{"Mazda RX4"}\NormalTok{,}\StringTok{"gear"}\NormalTok{]}
\end{Highlighting}
\end{Shaded}

\begin{verbatim}
## [1] 4
\end{verbatim}

\begin{Shaded}
\begin{Highlighting}[]
\NormalTok{mtcars[}\DecValTok{1}\SpecialCharTok{:}\DecValTok{5}\NormalTok{,}\DecValTok{1}\NormalTok{]}
\end{Highlighting}
\end{Shaded}

\begin{verbatim}
## [1] 21.0 21.0 22.8 21.4 18.7
\end{verbatim}

\begin{Shaded}
\begin{Highlighting}[]
\NormalTok{mtcars[,}\DecValTok{2}\NormalTok{]}
\end{Highlighting}
\end{Shaded}

\begin{verbatim}
##  [1] 6 6 4 6 8 6 8 4 4 6 6 8 8 8 8 8 8 4 4 4 4 8 8 8 8 4 4 4 8 6 8 4
\end{verbatim}

\begin{Shaded}
\begin{Highlighting}[]
\NormalTok{mtcars[}\DecValTok{3}\NormalTok{,]}
\end{Highlighting}
\end{Shaded}

\begin{verbatim}
##             mpg cyl disp hp drat   wt  qsec vs am gear carb
## Datsun 710 22.8   4  108 93 3.85 2.32 18.61  1  1    4    1
\end{verbatim}

\begin{Shaded}
\begin{Highlighting}[]
\FunctionTok{rownames}\NormalTok{(mtcars)}
\end{Highlighting}
\end{Shaded}

\begin{verbatim}
##  [1] "Mazda RX4"           "Mazda RX4 Wag"       "Datsun 710"         
##  [4] "Hornet 4 Drive"      "Hornet Sportabout"   "Valiant"            
##  [7] "Duster 360"          "Merc 240D"           "Merc 230"           
## [10] "Merc 280"            "Merc 280C"           "Merc 450SE"         
## [13] "Merc 450SL"          "Merc 450SLC"         "Cadillac Fleetwood" 
## [16] "Lincoln Continental" "Chrysler Imperial"   "Fiat 128"           
## [19] "Honda Civic"         "Toyota Corolla"      "Toyota Corona"      
## [22] "Dodge Challenger"    "AMC Javelin"         "Camaro Z28"         
## [25] "Pontiac Firebird"    "Fiat X1-9"           "Porsche 914-2"      
## [28] "Lotus Europa"        "Ford Pantera L"      "Ferrari Dino"       
## [31] "Maserati Bora"       "Volvo 142E"
\end{verbatim}

\begin{Shaded}
\begin{Highlighting}[]
\NormalTok{mtcars}\SpecialCharTok{$}\NormalTok{gear}
\end{Highlighting}
\end{Shaded}

\begin{verbatim}
##  [1] 4 4 4 3 3 3 3 4 4 4 4 3 3 3 3 3 3 4 4 4 3 3 3 3 3 4 5 5 5 5 5 4
\end{verbatim}

\begin{Shaded}
\begin{Highlighting}[]
\FunctionTok{t}\NormalTok{(mtcars)}
\end{Highlighting}
\end{Shaded}

\begin{verbatim}
##      Mazda RX4 Mazda RX4 Wag Datsun 710 Hornet 4 Drive Hornet Sportabout
## mpg      21.00        21.000      22.80         21.400             18.70
## cyl       6.00         6.000       4.00          6.000              8.00
## disp    160.00       160.000     108.00        258.000            360.00
## hp      110.00       110.000      93.00        110.000            175.00
## drat      3.90         3.900       3.85          3.080              3.15
## wt        2.62         2.875       2.32          3.215              3.44
## qsec     16.46        17.020      18.61         19.440             17.02
## vs        0.00         0.000       1.00          1.000              0.00
## am        1.00         1.000       1.00          0.000              0.00
## gear      4.00         4.000       4.00          3.000              3.00
## carb      4.00         4.000       1.00          1.000              2.00
##      Valiant Duster 360 Merc 240D Merc 230 Merc 280 Merc 280C Merc 450SE
## mpg    18.10      14.30     24.40    22.80    19.20     17.80      16.40
## cyl     6.00       8.00      4.00     4.00     6.00      6.00       8.00
## disp  225.00     360.00    146.70   140.80   167.60    167.60     275.80
## hp    105.00     245.00     62.00    95.00   123.00    123.00     180.00
## drat    2.76       3.21      3.69     3.92     3.92      3.92       3.07
## wt      3.46       3.57      3.19     3.15     3.44      3.44       4.07
## qsec   20.22      15.84     20.00    22.90    18.30     18.90      17.40
## vs      1.00       0.00      1.00     1.00     1.00      1.00       0.00
## am      0.00       0.00      0.00     0.00     0.00      0.00       0.00
## gear    3.00       3.00      4.00     4.00     4.00      4.00       3.00
## carb    1.00       4.00      2.00     2.00     4.00      4.00       3.00
##      Merc 450SL Merc 450SLC Cadillac Fleetwood Lincoln Continental
## mpg       17.30       15.20              10.40              10.400
## cyl        8.00        8.00               8.00               8.000
## disp     275.80      275.80             472.00             460.000
## hp       180.00      180.00             205.00             215.000
## drat       3.07        3.07               2.93               3.000
## wt         3.73        3.78               5.25               5.424
## qsec      17.60       18.00              17.98              17.820
## vs         0.00        0.00               0.00               0.000
## am         0.00        0.00               0.00               0.000
## gear       3.00        3.00               3.00               3.000
## carb       3.00        3.00               4.00               4.000
##      Chrysler Imperial Fiat 128 Honda Civic Toyota Corolla Toyota Corona
## mpg             14.700    32.40      30.400         33.900        21.500
## cyl              8.000     4.00       4.000          4.000         4.000
## disp           440.000    78.70      75.700         71.100       120.100
## hp             230.000    66.00      52.000         65.000        97.000
## drat             3.230     4.08       4.930          4.220         3.700
## wt               5.345     2.20       1.615          1.835         2.465
## qsec            17.420    19.47      18.520         19.900        20.010
## vs               0.000     1.00       1.000          1.000         1.000
## am               0.000     1.00       1.000          1.000         0.000
## gear             3.000     4.00       4.000          4.000         3.000
## carb             4.000     1.00       2.000          1.000         1.000
##      Dodge Challenger AMC Javelin Camaro Z28 Pontiac Firebird Fiat X1-9
## mpg             15.50      15.200      13.30           19.200    27.300
## cyl              8.00       8.000       8.00            8.000     4.000
## disp           318.00     304.000     350.00          400.000    79.000
## hp             150.00     150.000     245.00          175.000    66.000
## drat             2.76       3.150       3.73            3.080     4.080
## wt               3.52       3.435       3.84            3.845     1.935
## qsec            16.87      17.300      15.41           17.050    18.900
## vs               0.00       0.000       0.00            0.000     1.000
## am               0.00       0.000       0.00            0.000     1.000
## gear             3.00       3.000       3.00            3.000     4.000
## carb             2.00       2.000       4.00            2.000     1.000
##      Porsche 914-2 Lotus Europa Ford Pantera L Ferrari Dino Maserati Bora
## mpg          26.00       30.400          15.80        19.70         15.00
## cyl           4.00        4.000           8.00         6.00          8.00
## disp        120.30       95.100         351.00       145.00        301.00
## hp           91.00      113.000         264.00       175.00        335.00
## drat          4.43        3.770           4.22         3.62          3.54
## wt            2.14        1.513           3.17         2.77          3.57
## qsec         16.70       16.900          14.50        15.50         14.60
## vs            0.00        1.000           0.00         0.00          0.00
## am            1.00        1.000           1.00         1.00          1.00
## gear          5.00        5.000           5.00         5.00          5.00
## carb          2.00        2.000           4.00         6.00          8.00
##      Volvo 142E
## mpg       21.40
## cyl        4.00
## disp     121.00
## hp       109.00
## drat       4.11
## wt         2.78
## qsec      18.60
## vs         1.00
## am         1.00
## gear       4.00
## carb       2.00
\end{verbatim}

\begin{Shaded}
\begin{Highlighting}[]
\FunctionTok{nrow}\NormalTok{(mtcars)}
\end{Highlighting}
\end{Shaded}

\begin{verbatim}
## [1] 32
\end{verbatim}

\begin{Shaded}
\begin{Highlighting}[]
\FunctionTok{ncol}\NormalTok{(mtcars)}
\end{Highlighting}
\end{Shaded}

\begin{verbatim}
## [1] 11
\end{verbatim}

\begin{Shaded}
\begin{Highlighting}[]
\NormalTok{mtcars[mtcars}\SpecialCharTok{\textless{}=}\DecValTok{3}\NormalTok{]}
\end{Highlighting}
\end{Shaded}

\begin{verbatim}
##   [1] 2.760 2.930 3.000 2.760 2.620 2.875 2.320 2.200 1.615 1.835 2.465 1.935
##  [13] 2.140 1.513 2.770 2.780 0.000 0.000 1.000 1.000 0.000 1.000 0.000 1.000
##  [25] 1.000 1.000 1.000 0.000 0.000 0.000 0.000 0.000 0.000 1.000 1.000 1.000
##  [37] 1.000 0.000 0.000 0.000 0.000 1.000 0.000 1.000 0.000 0.000 0.000 1.000
##  [49] 1.000 1.000 1.000 0.000 0.000 0.000 0.000 0.000 0.000 0.000 0.000 0.000
##  [61] 0.000 0.000 0.000 0.000 0.000 1.000 1.000 1.000 0.000 0.000 0.000 0.000
##  [73] 0.000 1.000 1.000 1.000 1.000 1.000 1.000 1.000 3.000 3.000 3.000 3.000
##  [85] 3.000 3.000 3.000 3.000 3.000 3.000 3.000 3.000 3.000 3.000 3.000 1.000
##  [97] 1.000 2.000 1.000 2.000 2.000 3.000 3.000 3.000 1.000 2.000 1.000 1.000
## [109] 2.000 2.000 2.000 1.000 2.000 2.000 2.000
\end{verbatim}

\begin{Shaded}
\begin{Highlighting}[]
\NormalTok{mtcars[mtcars}\SpecialCharTok{==}\DecValTok{0}\NormalTok{] }
\end{Highlighting}
\end{Shaded}

\begin{verbatim}
##  [1] 0 0 0 0 0 0 0 0 0 0 0 0 0 0 0 0 0 0 0 0 0 0 0 0 0 0 0 0 0 0 0 0 0 0 0 0 0
\end{verbatim}

\begin{Shaded}
\begin{Highlighting}[]
\NormalTok{mtcars[mtcars}\SpecialCharTok{!=}\DecValTok{0}\NormalTok{]}
\end{Highlighting}
\end{Shaded}

\begin{verbatim}
##   [1]  21.000  21.000  22.800  21.400  18.700  18.100  14.300  24.400  22.800
##  [10]  19.200  17.800  16.400  17.300  15.200  10.400  10.400  14.700  32.400
##  [19]  30.400  33.900  21.500  15.500  15.200  13.300  19.200  27.300  26.000
##  [28]  30.400  15.800  19.700  15.000  21.400   6.000   6.000   4.000   6.000
##  [37]   8.000   6.000   8.000   4.000   4.000   6.000   6.000   8.000   8.000
##  [46]   8.000   8.000   8.000   8.000   4.000   4.000   4.000   4.000   8.000
##  [55]   8.000   8.000   8.000   4.000   4.000   4.000   8.000   6.000   8.000
##  [64]   4.000 160.000 160.000 108.000 258.000 360.000 225.000 360.000 146.700
##  [73] 140.800 167.600 167.600 275.800 275.800 275.800 472.000 460.000 440.000
##  [82]  78.700  75.700  71.100 120.100 318.000 304.000 350.000 400.000  79.000
##  [91] 120.300  95.100 351.000 145.000 301.000 121.000 110.000 110.000  93.000
## [100] 110.000 175.000 105.000 245.000  62.000  95.000 123.000 123.000 180.000
## [109] 180.000 180.000 205.000 215.000 230.000  66.000  52.000  65.000  97.000
## [118] 150.000 150.000 245.000 175.000  66.000  91.000 113.000 264.000 175.000
## [127] 335.000 109.000   3.900   3.900   3.850   3.080   3.150   2.760   3.210
## [136]   3.690   3.920   3.920   3.920   3.070   3.070   3.070   2.930   3.000
## [145]   3.230   4.080   4.930   4.220   3.700   2.760   3.150   3.730   3.080
## [154]   4.080   4.430   3.770   4.220   3.620   3.540   4.110   2.620   2.875
## [163]   2.320   3.215   3.440   3.460   3.570   3.190   3.150   3.440   3.440
## [172]   4.070   3.730   3.780   5.250   5.424   5.345   2.200   1.615   1.835
## [181]   2.465   3.520   3.435   3.840   3.845   1.935   2.140   1.513   3.170
## [190]   2.770   3.570   2.780  16.460  17.020  18.610  19.440  17.020  20.220
## [199]  15.840  20.000  22.900  18.300  18.900  17.400  17.600  18.000  17.980
## [208]  17.820  17.420  19.470  18.520  19.900  20.010  16.870  17.300  15.410
## [217]  17.050  18.900  16.700  16.900  14.500  15.500  14.600  18.600   1.000
## [226]   1.000   1.000   1.000   1.000   1.000   1.000   1.000   1.000   1.000
## [235]   1.000   1.000   1.000   1.000   1.000   1.000   1.000   1.000   1.000
## [244]   1.000   1.000   1.000   1.000   1.000   1.000   1.000   1.000   4.000
## [253]   4.000   4.000   3.000   3.000   3.000   3.000   4.000   4.000   4.000
## [262]   4.000   3.000   3.000   3.000   3.000   3.000   3.000   4.000   4.000
## [271]   4.000   3.000   3.000   3.000   3.000   3.000   4.000   5.000   5.000
## [280]   5.000   5.000   5.000   4.000   4.000   4.000   1.000   1.000   2.000
## [289]   1.000   4.000   2.000   2.000   4.000   4.000   3.000   3.000   3.000
## [298]   4.000   4.000   4.000   1.000   2.000   1.000   1.000   2.000   2.000
## [307]   4.000   2.000   1.000   2.000   2.000   4.000   6.000   8.000   2.000
\end{verbatim}

\begin{Shaded}
\begin{Highlighting}[]
\FunctionTok{length}\NormalTok{(}\FunctionTok{as.matrix}\NormalTok{(mtcars))}
\end{Highlighting}
\end{Shaded}

\begin{verbatim}
## [1] 352
\end{verbatim}

\begin{Shaded}
\begin{Highlighting}[]
\FunctionTok{dim}\NormalTok{(mtcars)}
\end{Highlighting}
\end{Shaded}

\begin{verbatim}
## [1] 32 11
\end{verbatim}

\begin{Shaded}
\begin{Highlighting}[]
\FunctionTok{length}\NormalTok{(mtcars[mtcars}\SpecialCharTok{==}\DecValTok{0}\NormalTok{])}
\end{Highlighting}
\end{Shaded}

\begin{verbatim}
## [1] 37
\end{verbatim}

\begin{Shaded}
\begin{Highlighting}[]
\FunctionTok{str}\NormalTok{(mtcars)}
\end{Highlighting}
\end{Shaded}

\begin{verbatim}
## 'data.frame':    32 obs. of  11 variables:
##  $ mpg : num  21 21 22.8 21.4 18.7 18.1 14.3 24.4 22.8 19.2 ...
##  $ cyl : num  6 6 4 6 8 6 8 4 4 6 ...
##  $ disp: num  160 160 108 258 360 ...
##  $ hp  : num  110 110 93 110 175 105 245 62 95 123 ...
##  $ drat: num  3.9 3.9 3.85 3.08 3.15 2.76 3.21 3.69 3.92 3.92 ...
##  $ wt  : num  2.62 2.88 2.32 3.21 3.44 ...
##  $ qsec: num  16.5 17 18.6 19.4 17 ...
##  $ vs  : num  0 0 1 1 0 1 0 1 1 1 ...
##  $ am  : num  1 1 1 0 0 0 0 0 0 0 ...
##  $ gear: num  4 4 4 3 3 3 3 4 4 4 ...
##  $ carb: num  4 4 1 1 2 1 4 2 2 4 ...
\end{verbatim}

\begin{Shaded}
\begin{Highlighting}[]
\FunctionTok{summary}\NormalTok{(mtcars)}
\end{Highlighting}
\end{Shaded}

\begin{verbatim}
##       mpg             cyl             disp             hp       
##  Min.   :10.40   Min.   :4.000   Min.   : 71.1   Min.   : 52.0  
##  1st Qu.:15.43   1st Qu.:4.000   1st Qu.:120.8   1st Qu.: 96.5  
##  Median :19.20   Median :6.000   Median :196.3   Median :123.0  
##  Mean   :20.09   Mean   :6.188   Mean   :230.7   Mean   :146.7  
##  3rd Qu.:22.80   3rd Qu.:8.000   3rd Qu.:326.0   3rd Qu.:180.0  
##  Max.   :33.90   Max.   :8.000   Max.   :472.0   Max.   :335.0  
##       drat             wt             qsec             vs        
##  Min.   :2.760   Min.   :1.513   Min.   :14.50   Min.   :0.0000  
##  1st Qu.:3.080   1st Qu.:2.581   1st Qu.:16.89   1st Qu.:0.0000  
##  Median :3.695   Median :3.325   Median :17.71   Median :0.0000  
##  Mean   :3.597   Mean   :3.217   Mean   :17.85   Mean   :0.4375  
##  3rd Qu.:3.920   3rd Qu.:3.610   3rd Qu.:18.90   3rd Qu.:1.0000  
##  Max.   :4.930   Max.   :5.424   Max.   :22.90   Max.   :1.0000  
##        am              gear            carb      
##  Min.   :0.0000   Min.   :3.000   Min.   :1.000  
##  1st Qu.:0.0000   1st Qu.:3.000   1st Qu.:2.000  
##  Median :0.0000   Median :4.000   Median :2.000  
##  Mean   :0.4062   Mean   :3.688   Mean   :2.812  
##  3rd Qu.:1.0000   3rd Qu.:4.000   3rd Qu.:4.000  
##  Max.   :1.0000   Max.   :5.000   Max.   :8.000
\end{verbatim}

As funções \texttt{str()} e \texttt{summary()} trazem algumas informaçoes úteis sobre o data frame, de uma forma geral. É comum utiliza-las ao importar uma base de dados para o R, para se ter uma ideia do que está sendo analisado.
O \texttt{str()} conta as linhas e colunas e mostra os seus tipos e alguns valores de exemplo. Já o \texttt{summary()} traz médias, medianas, máximos, mínimos e quantidade de valores indisponíveis em cada coluna.

\begin{enumerate}
\def\labelenumi{\arabic{enumi}.}
\setcounter{enumi}{4}
\tightlist
\item
  \textbf{Fatores}:

  \begin{itemize}
  \tightlist
  \item
    Fatores são usados para representar variáveis categóricas ou de fatores.
  \item
    São úteis para análises estatísticas e gráficos.
  \item
    Podem ser criados com a função \texttt{factor()}.
  \end{itemize}
\end{enumerate}

Por fim, temos os fatores (\texttt{factors}). Fatores são usados para representar categorias. Isso quer dizer que em uma variável do tipo \texttt{factor}, aquele valor terá uma quantidade limitada de valores. Pense, por exemplo, nos meses do ano ou letras do alfabeto. Existe um número definido de meses, e o valor estará dentro destas possibilidades. Um nome de uma pessoa, ao contrário, tem infinitas possibilidades, pois não existe uma lista pré-definida de nomes de pessoas, e qualquer um pode criar um novo nome. Para criar um fator, vamos aproveitar o exemplo dos meses do ano e utilizar a função \texttt{factor()}:

\begin{Shaded}
\begin{Highlighting}[]
\NormalTok{meses }\OtherTok{\textless{}{-}} \FunctionTok{c}\NormalTok{(}\StringTok{"Jan"}\NormalTok{,}\StringTok{"Jan"}\NormalTok{,}\StringTok{"Feb"}\NormalTok{,}\StringTok{"Mar"}\NormalTok{,}\StringTok{"Apr"}\NormalTok{,}\StringTok{"Apr"}\NormalTok{,}\StringTok{"May"}\NormalTok{,}\StringTok{"Jan"}\NormalTok{,}\StringTok{"Jan"}\NormalTok{,}\StringTok{"May"}\NormalTok{,}\StringTok{"May"}\NormalTok{,}\StringTok{"Jun"}\NormalTok{)}
\NormalTok{meses}
\end{Highlighting}
\end{Shaded}

\begin{verbatim}
##  [1] "Jan" "Jan" "Feb" "Mar" "Apr" "Apr" "May" "Jan" "Jan" "May" "May" "Jun"
\end{verbatim}

\begin{Shaded}
\begin{Highlighting}[]
\NormalTok{meses }\OtherTok{\textless{}{-}} \FunctionTok{factor}\NormalTok{(meses)}
\FunctionTok{print}\NormalTok{(meses)}
\end{Highlighting}
\end{Shaded}

\begin{verbatim}
##  [1] Jan Jan Feb Mar Apr Apr May Jan Jan May May Jun
## Levels: Apr Feb Jan Jun Mar May
\end{verbatim}

\begin{Shaded}
\begin{Highlighting}[]
\FunctionTok{table}\NormalTok{(meses)}
\end{Highlighting}
\end{Shaded}

\begin{verbatim}
## meses
## Apr Feb Jan Jun Mar May 
##   2   1   4   1   1   3
\end{verbatim}

Um \texttt{vector} ou ``vetor'' é uma estrutura de dados fundamental que armazena uma sequência ordenada de elementos do mesmo tipo. Ao imprimir (\texttt{print()}) o vetor \texttt{meses}, vemos que o R nos indica os níveis (\texttt{levels}), que nada mais são do que os valores únicos que existem naquele vetor. Usando a função \texttt{table()}, podemos ver a contagem de cada uma das categorias. Vemos que elas estão ordenadas em ordem alfabética.

Como isso pode ser aplicado?

\begin{Shaded}
\begin{Highlighting}[]
\NormalTok{meses }\OtherTok{\textless{}{-}} \FunctionTok{c}\NormalTok{(}\StringTok{"Jan"}\NormalTok{, }\StringTok{"Jan"}\NormalTok{, }\StringTok{"Jan"}\NormalTok{, }\StringTok{"Feb"}\NormalTok{, }\StringTok{"Feb"}\NormalTok{, }\StringTok{"Feb"}\NormalTok{, }\StringTok{"Feb"}\NormalTok{, }\StringTok{"Mar"}\NormalTok{, }\StringTok{"Mar"}\NormalTok{, }\StringTok{"Mar"}\NormalTok{, }\StringTok{"Apr"}\NormalTok{, }\StringTok{"Apr"}\NormalTok{, }\StringTok{"Apr"}\NormalTok{, }\StringTok{"May"}\NormalTok{, }\StringTok{"May"}\NormalTok{, }\StringTok{"May"}\NormalTok{, }\StringTok{"Jun"}\NormalTok{, }\StringTok{"Jun"}\NormalTok{, }\StringTok{"Jun"}\NormalTok{)}
\NormalTok{meses}
\end{Highlighting}
\end{Shaded}

\begin{verbatim}
##  [1] "Jan" "Jan" "Jan" "Feb" "Feb" "Feb" "Feb" "Mar" "Mar" "Mar" "Apr" "Apr"
## [13] "Apr" "May" "May" "May" "Jun" "Jun" "Jun"
\end{verbatim}

\begin{Shaded}
\begin{Highlighting}[]
\FunctionTok{factor}\NormalTok{(meses)}
\end{Highlighting}
\end{Shaded}

\begin{verbatim}
##  [1] Jan Jan Jan Feb Feb Feb Feb Mar Mar Mar Apr Apr Apr May May May Jun Jun Jun
## Levels: Apr Feb Jan Jun Mar May
\end{verbatim}

\begin{Shaded}
\begin{Highlighting}[]
\FunctionTok{print}\NormalTok{(meses)}
\end{Highlighting}
\end{Shaded}

\begin{verbatim}
##  [1] "Jan" "Jan" "Jan" "Feb" "Feb" "Feb" "Feb" "Mar" "Mar" "Mar" "Apr" "Apr"
## [13] "Apr" "May" "May" "May" "Jun" "Jun" "Jun"
\end{verbatim}

\begin{Shaded}
\begin{Highlighting}[]
\FunctionTok{table}\NormalTok{(meses)}
\end{Highlighting}
\end{Shaded}

\begin{verbatim}
## meses
## Apr Feb Jan Jun Mar May 
##   3   4   3   3   3   3
\end{verbatim}

\begin{Shaded}
\begin{Highlighting}[]
\NormalTok{riq }\OtherTok{\textless{}{-}} \FunctionTok{c}\NormalTok{(}\DecValTok{3}\NormalTok{, }\DecValTok{5}\NormalTok{, }\DecValTok{7}\NormalTok{, }\DecValTok{6}\NormalTok{, }\DecValTok{8}\NormalTok{, }\DecValTok{9}\NormalTok{, }\DecValTok{0}\NormalTok{, }\DecValTok{4}\NormalTok{, }\DecValTok{6}\NormalTok{, }\DecValTok{9}\NormalTok{, }\DecValTok{2}\NormalTok{, }\DecValTok{6}\NormalTok{, }\DecValTok{3}\NormalTok{, }\DecValTok{9}\NormalTok{, }\DecValTok{6}\NormalTok{, }\DecValTok{0}\NormalTok{, }\DecValTok{1}\NormalTok{, }\DecValTok{6}\NormalTok{, }\DecValTok{7}\NormalTok{)}
\NormalTok{dados }\OtherTok{\textless{}{-}} \FunctionTok{data.frame}\NormalTok{(}\AttributeTok{Mes =}\NormalTok{ meses, }\AttributeTok{Riqueza =}\NormalTok{ riq)}
\NormalTok{riq}
\end{Highlighting}
\end{Shaded}

\begin{verbatim}
##  [1] 3 5 7 6 8 9 0 4 6 9 2 6 3 9 6 0 1 6 7
\end{verbatim}

\begin{Shaded}
\begin{Highlighting}[]
\NormalTok{dados}
\end{Highlighting}
\end{Shaded}

\begin{verbatim}
##    Mes Riqueza
## 1  Jan       3
## 2  Jan       5
## 3  Jan       7
## 4  Feb       6
## 5  Feb       8
## 6  Feb       9
## 7  Feb       0
## 8  Mar       4
## 9  Mar       6
## 10 Mar       9
## 11 Apr       2
## 12 Apr       6
## 13 Apr       3
## 14 May       9
## 15 May       6
## 16 May       0
## 17 Jun       1
## 18 Jun       6
## 19 Jun       7
\end{verbatim}

\begin{Shaded}
\begin{Highlighting}[]
\FunctionTok{factor}\NormalTok{(meses)}
\end{Highlighting}
\end{Shaded}

\begin{verbatim}
##  [1] Jan Jan Jan Feb Feb Feb Feb Mar Mar Mar Apr Apr Apr May May May Jun Jun Jun
## Levels: Apr Feb Jan Jun Mar May
\end{verbatim}

\begin{Shaded}
\begin{Highlighting}[]
\FunctionTok{plot}\NormalTok{(}\FunctionTok{factor}\NormalTok{(meses),riq)}
\end{Highlighting}
\end{Shaded}

\includegraphics{fundamentos_de_ecologia_numerica_files/figure-latex/unnamed-chunk-44-1.pdf}

\begin{Shaded}
\begin{Highlighting}[]
\FunctionTok{by}\NormalTok{(riq, meses, mean)}
\end{Highlighting}
\end{Shaded}

\begin{verbatim}
## meses: Apr
## [1] 3.666667
## ------------------------------------------------------------ 
## meses: Feb
## [1] 5.75
## ------------------------------------------------------------ 
## meses: Jan
## [1] 5
## ------------------------------------------------------------ 
## meses: Jun
## [1] 4.666667
## ------------------------------------------------------------ 
## meses: Mar
## [1] 6.333333
## ------------------------------------------------------------ 
## meses: May
## [1] 5
\end{verbatim}

Uma situação diferente ocorre quando tem-se colunas com o mesmo nome.

\begin{Shaded}
\begin{Highlighting}[]
\NormalTok{df }\OtherTok{\textless{}{-}} \FunctionTok{data.frame}\NormalTok{(}
  \AttributeTok{A =} \FunctionTok{c}\NormalTok{(}\DecValTok{1}\NormalTok{, }\DecValTok{2}\NormalTok{, }\DecValTok{3}\NormalTok{),}
  \AttributeTok{B =} \FunctionTok{c}\NormalTok{(}\DecValTok{4}\NormalTok{, }\DecValTok{5}\NormalTok{, }\DecValTok{6}\NormalTok{),}
  \AttributeTok{A =} \FunctionTok{c}\NormalTok{(}\DecValTok{7}\NormalTok{, }\DecValTok{8}\NormalTok{, }\DecValTok{9}\NormalTok{), }\CommentTok{\#nome duplicado}
  \AttributeTok{C =} \FunctionTok{c}\NormalTok{(}\DecValTok{0}\NormalTok{, }\DecValTok{0}\NormalTok{, }\DecValTok{1}\NormalTok{),}
  \AttributeTok{D =} \FunctionTok{c}\NormalTok{(}\DecValTok{1}\NormalTok{, }\DecValTok{1}\NormalTok{, }\DecValTok{0}\NormalTok{),}
  \AttributeTok{E =} \FunctionTok{c}\NormalTok{(}\DecValTok{1}\NormalTok{, }\DecValTok{0}\NormalTok{, }\DecValTok{0}\NormalTok{),}
  \AttributeTok{F =} \FunctionTok{c}\NormalTok{(}\DecValTok{11}\NormalTok{, }\DecValTok{12}\NormalTok{, }\DecValTok{13}\NormalTok{),}
  \AttributeTok{G =} \FunctionTok{c}\NormalTok{(}\DecValTok{1}\NormalTok{, }\DecValTok{2}\NormalTok{, }\DecValTok{0}\NormalTok{),}
  \AttributeTok{E =} \FunctionTok{c}\NormalTok{(}\DecValTok{2}\NormalTok{, }\DecValTok{1}\NormalTok{, }\DecValTok{1}\NormalTok{), }\CommentTok{\#nome duplicado}
  \AttributeTok{check.names =} \ConstantTok{FALSE}
\NormalTok{)}
\NormalTok{df}
\end{Highlighting}
\end{Shaded}

\begin{verbatim}
##   A B A C D E  F G E
## 1 1 4 7 0 1 1 11 1 2
## 2 2 5 8 0 1 0 12 2 1
## 3 3 6 9 1 0 0 13 0 1
\end{verbatim}

Por exemplo, nesse dataframe temos colunas duplicadas. Nesse caso, encontrar e resolver colunas com o mesmo nome é crucial para garantir a integridade dos dados e realizar uma análise precisa. Você deve consolidar essas colunas duplicadas, somando-as ou fazendo sua média.

\begin{Shaded}
\begin{Highlighting}[]
\CommentTok{\# Achando colunas com nomes duplicados}
\NormalTok{dup\_cols }\OtherTok{\textless{}{-}} \FunctionTok{names}\NormalTok{(df)[}\FunctionTok{duplicated}\NormalTok{(}\FunctionTok{names}\NormalTok{(df))]}
\CommentTok{\# Somando colunas com o mesmo nome}
\ControlFlowTok{for}\NormalTok{ (col\_name }\ControlFlowTok{in} \FunctionTok{unique}\NormalTok{(dup\_cols)) \{}
  \CommentTok{\# Get indices of columns with the same name}
\NormalTok{  col\_indices }\OtherTok{\textless{}{-}} \FunctionTok{which}\NormalTok{(}\FunctionTok{names}\NormalTok{(df) }\SpecialCharTok{==}\NormalTok{ col\_name)}
  \CommentTok{\# Sum columns with the same name}
\NormalTok{  df[[col\_name]] }\OtherTok{\textless{}{-}} \FunctionTok{rowSums}\NormalTok{(df[, col\_indices, }\AttributeTok{drop =} \ConstantTok{FALSE}\NormalTok{])}
\NormalTok{\}}
\CommentTok{\# Remove as colunas duplicadas originais e mantem as novas colunas que são a soma ("except for the first occurrence")}
\NormalTok{df }\OtherTok{\textless{}{-}}\NormalTok{ df[, }\SpecialCharTok{!}\FunctionTok{duplicated}\NormalTok{(}\FunctionTok{names}\NormalTok{(df))]}
\CommentTok{\# Mostra a nova tabela com colunas repetidas somadas}
\FunctionTok{print}\NormalTok{(df)}
\end{Highlighting}
\end{Shaded}

\begin{verbatim}
##    A B C D E  F G
## 1  8 4 0 1 3 11 1
## 2 10 5 0 1 1 12 2
## 3 12 6 1 0 1 13 0
\end{verbatim}

\begin{enumerate}
\def\labelenumi{\arabic{enumi}.}
\setcounter{enumi}{5}
\tightlist
\item
  Listas
\end{enumerate}

Listas são um tipo especial de vetor, que podem conter elementos de diferentes tipos, incluindo vetores. Veja a lista baseada em alguns vetores, a seguir:

\begin{Shaded}
\begin{Highlighting}[]
\NormalTok{a }\OtherTok{\textless{}{-}} \FunctionTok{c}\NormalTok{(}\DecValTok{3}\NormalTok{,}\DecValTok{6}\NormalTok{,}\DecValTok{9}\NormalTok{)}
\NormalTok{b }\OtherTok{\textless{}{-}} \FunctionTok{c}\NormalTok{(}\StringTok{"a"}\NormalTok{,}\StringTok{"b"}\NormalTok{,}\StringTok{"c"}\NormalTok{,}\StringTok{"d"}\NormalTok{)}
\NormalTok{c }\OtherTok{\textless{}{-}} \FunctionTok{c}\NormalTok{(}\ConstantTok{TRUE}\NormalTok{, }\ConstantTok{FALSE}\NormalTok{, }\ConstantTok{TRUE}\NormalTok{, }\ConstantTok{TRUE}\NormalTok{)}
\NormalTok{lista1 }\OtherTok{\textless{}{-}} \FunctionTok{list}\NormalTok{(a,b,c)}
\FunctionTok{print}\NormalTok{(lista1)}
\end{Highlighting}
\end{Shaded}

\begin{verbatim}
## [[1]]
## [1] 3 6 9
## 
## [[2]]
## [1] "a" "b" "c" "d"
## 
## [[3]]
## [1]  TRUE FALSE  TRUE  TRUE
\end{verbatim}

\textbf{RESUMO} (De \url{http://venus.ifca.unican.es/Rintro/dataStruct.html})

• Vetores: arrays unidimensionais usados para armazenar coleções de dados do mesmo tipo

○ Vetores Numéricos (mode: numeric)\\
○ Vetores Complexos (mode: complex)\\
○ Vetores Lógicos (mode: logical)\\
○ Vetor de Caracteres ou strings de texto (mode: character)

• Matrizes: arrays bidimensionais para armazenar coleções de dados do mesmo modo. São acessados por dois índices inteiros.\\
• Arrays: semelhantes a matrizes, mas podem ser multidimensionais (mais de duas dimensões).\\
• Fatores: vetores de variáveis categóricas projetados para agrupar os componentes de outro vetor com o mesmo tamanho.\\
• Listas: coleção ordenada de objetos, onde os elementos podem ser de tipos diferentes.\\
• Data Frames: generalização de matrizes onde diferentes colunas podem armazenar dados de modo diferente.\\
• Funcões: objetos criados pelo usuário e reutilizados para realizar operações específicas.

\begin{figure}
\includegraphics[width=10.71in]{imagens/nucleos} \caption{Resumo visual dos tipos de dados no ambiente de programação R. (De: https://www.tutorialkart.com/r-tutorial/r-data-types)}\label{fig:nucleos}
\end{figure}

\begin{figure}
\includegraphics[width=9.82in]{imagens/tipos} \caption{Resumo visual dos tipos de dados no ambiente de programação R. (De: https://lhmet.github.io/adar-ebook/estrutura-dados.html)}\label{fig:tipos}
\end{figure}

\hypertarget{como-pedir-ajuda}{%
\section{Como pedir ajuda?}\label{como-pedir-ajuda}}

\hypertarget{dentro-do-r}{%
\subsection{Dentro do R}\label{dentro-do-r}}

No R, você pode solicitar ajuda de várias maneiras:

\begin{enumerate}
\def\labelenumi{\arabic{enumi}.}
\item
  \textbf{\texttt{help()} ou \texttt{?}}:

  Você pode usar a função \texttt{help()} seguida pelo nome da função ou pacote sobre o qual você precisa de ajuda. Por exemplo:
\end{enumerate}

\begin{Shaded}
\begin{Highlighting}[]
\FunctionTok{help}\NormalTok{(mean)}
\end{Highlighting}
\end{Shaded}

Ou, de forma mais concisa:

\begin{Shaded}
\begin{Highlighting}[]
\NormalTok{?mean}
\end{Highlighting}
\end{Shaded}

\begin{enumerate}
\def\labelenumi{\arabic{enumi}.}
\setcounter{enumi}{1}
\tightlist
\item
  \textbf{\texttt{help.search()}}:
\end{enumerate}

Você pode pesquisar por termos relacionados à sua dúvida usando a função \texttt{help.search()}.

\begin{Shaded}
\begin{Highlighting}[]
\FunctionTok{help.search}\NormalTok{(}\StringTok{"linear regression"}\NormalTok{)}
\end{Highlighting}
\end{Shaded}

\begin{enumerate}
\def\labelenumi{\arabic{enumi}.}
\setcounter{enumi}{2}
\tightlist
\item
  \textbf{\texttt{example()}}:
\end{enumerate}

Para ver exemplos de como usar uma função, você pode usar a função \texttt{example()}.

\begin{Shaded}
\begin{Highlighting}[]
\FunctionTok{example}\NormalTok{(mean)}
\end{Highlighting}
\end{Shaded}

\hypertarget{fora-do-r}{%
\subsection{Fora do R}\label{fora-do-r}}

\begin{enumerate}
\def\labelenumi{\arabic{enumi}.}
\setcounter{enumi}{3}
\tightlist
\item
  \textbf{R Site e Documentação Online}:
\end{enumerate}

O site oficial do R (\url{https://www.r-project.org/}) oferece uma vasta documentação, tutoriais e recursos para ajudá-lo a aprender e resolver problemas.

\begin{enumerate}
\def\labelenumi{\arabic{enumi}.}
\setcounter{enumi}{4}
\tightlist
\item
  \textbf{Fóruns e Comunidades Online}:
\end{enumerate}

Existem várias comunidades online onde você pode fazer perguntas e obter ajuda de outros usuários do R, como o Stack Overflow (\url{https://stackoverflow.com/}), o RStudio Community (\url{https://community.rstudio.com/}), entre outros.

\begin{enumerate}
\def\labelenumi{\arabic{enumi}.}
\setcounter{enumi}{5}
\tightlist
\item
  \textbf{Livros e Tutoriais}:
\end{enumerate}

Existem muitos livros e tutoriais disponíveis que podem ajudá-lo a aprender R, desde os conceitos básicos até técnicas avançadas. Os livros em Português mais importantes são FERREIRA; OLIVEIRA (\protect\hyperlink{ref-RN2942}{2020}).

\leavevmode\vadjust pre{\hypertarget{hello}{}}%
Ao usar esses recursos, certifique-se de fornecer informações claras sobre sua dúvida ou problema, incluindo detalhes como o código que você está usando, mensagens de erro (se houver) e o que você já tentou fazer para resolver o problema. Isso ajudará os outros a entenderem melhor sua situação e fornecer uma resposta mais útil.

\hypertarget{chatgpt}{%
\subsection{ChatGPT}\label{chatgpt}}

Pergunte ao ChatGPT
\url{https://chat.openai.com/}

\emph{Para usar o ChatGPT para pedir ajuda ao R, você pode solicitar assistência na elaboração de código, depurar problemas, ou mesmo pedir orientação sobre como realizar uma determinada tarefa em R. Aqui está um exemplo de como você pode pedir ajuda ao R:}

\begin{verbatim}
Oi ChatGPT! Estou enfrentando um problema ao tentar realizar uma análise de regressão linear no R.
Eu tenho um conjunto de dados chamado "dados.csv" e estou tentando ajustar um modelo linear simples
com a função lm(). No entanto, estou recebendo um erro que não consigo entender.
Você poderia me ajudar a resolver isso?
\end{verbatim}

\emph{Ao fazer uma pergunta específica e fornecer detalhes sobre o que você está tentando alcançar e quais problemas está enfrentando, o ChatGPT pode oferecer orientações úteis e sugestões para ajudá-lo com o seu código em R.}

Detalhe, o texto acima foi gerado pelo próprio ChatGPT (Figura \ref{fig:22chat}). Você pode chamar o ChatGPT pelo próprio R:

\begin{Shaded}
\begin{Highlighting}[]
\CommentTok{\# Abrir o link para o site do ChatGPT}
\FunctionTok{browseURL}\NormalTok{(}\StringTok{"https://chat.openai.com/"}\NormalTok{)}
\end{Highlighting}
\end{Shaded}

\begin{figure}
\includegraphics[width=10.24in]{imagens/chatgpt} \caption{Usando o ChatGPT para pedir ajuda sobre o R.}\label{fig:22chat}
\end{figure}

\hypertarget{work-flow}{%
\section{Work flow}\label{work-flow}}

\leavevmode\vadjust pre{\hypertarget{hello}{}}%
Você já deve estar percebendo que a programação em R pode se tornar confusa, pela quantidade de possibilidades disponíveis e formas diferentes de se obter o mesmo resultado.

Por isso é necessário que o usuário tenha um workflow que permita o reconhecimento de uma \textbf{sequência lógica para estruturar e aplicar} os devidos testes estatísticos e construção de gráficos e tabelas, após a organização e manipulação dos dados. Uma sugestão de workflow preliminar e vista na figura \ref{fig:22wf} (modificada de OLIVEIRA; GUERRA; MCDONNELL (\protect\hyperlink{ref-RN2941}{2018})):

• Carregar os dados\\
• Limpar os dados\\
• Transformar, visualizar e modelar (fase exploratória)\\
• Comunicar o resultado

\begin{figure}
\includegraphics[width=19.32in]{imagens/wf} \caption{Exemplo de workflow para análises no R (modificada de [@RN2941])}\label{fig:22wf}
\end{figure}

\hypertarget{reconhecendo-a-estrutura-dos-dados-do-r-em-uma-matriz-multivariada}{%
\section{Reconhecendo a estrutura dos dados do R em uma matriz multivariada}\label{reconhecendo-a-estrutura-dos-dados-do-r-em-uma-matriz-multivariada}}

\hypertarget{organizauxe7uxe3o-buxe1sica-1}{%
\subsection{Organização básica}\label{organizauxe7uxe3o-buxe1sica-1}}

\begin{Shaded}
\begin{Highlighting}[]
\FunctionTok{rm}\NormalTok{(}\AttributeTok{list=}\FunctionTok{ls}\NormalTok{(}\AttributeTok{all=}\ConstantTok{TRUE}\NormalTok{)) }\CommentTok{\#limpa a memória}
\end{Highlighting}
\end{Shaded}

Instalando os pacotes necessários para esse módulo

\begin{Shaded}
\begin{Highlighting}[]
\FunctionTok{install.packages}\NormalTok{(}\StringTok{"openxlsx"}\NormalTok{) }\CommentTok{\#importa arquivos do excel}
\end{Highlighting}
\end{Shaded}

\begin{Shaded}
\begin{Highlighting}[]
\FunctionTok{library}\NormalTok{(openxlsx)}
\end{Highlighting}
\end{Shaded}

Os códigos acima, são usados para instalar e carregar os pacotes necessários para este módulo. Esses códigos são comandos para instalar pacotes no R. Um pacote é uma coleção de funções, dados e documentação que ampliam as capacidades do R (\href{https://cran.r-project.org/}{R CRAN} (\protect\hyperlink{ref-RN2774}{TEAM, 2017}) e \href{https://posit.co/download/rstudio-desktop/}{RStudio}) (\protect\hyperlink{ref-RN358}{TEAM, 2022}). No exemplo acima, o pacote \texttt{openxlsx} permite ler e escrever arquivos Excel no R. Para instalar um pacote no R, você precisa usar a função \texttt{install.packages()}.

Depois de instalar um pacote, você precisa carregá-lo na sua sessão R com a função \texttt{library()}. Por exemplo, para carregar o pacote \texttt{openxlsx}, você precisa executar a função \texttt{library(openxlsx)}. Isso irá permitir que você use as funções do pacote na sua sessão R. Você precisa carregar um pacote toda vez que iniciar uma nova sessão R e quiser usar um pacote instalado.

Agora vamos \textbf{definir o diretório de trabalho}. Esse código é usado para obter e definir o diretório de trabalho atual no R. O comando \texttt{getwd()} retorna o caminho do diretório onde o R está lendo e salvando arquivos. O comando \texttt{setwd()} muda esse diretório de trabalho para o caminho especificado entre aspas. No seu caso, você deve ajustar o caminho para o seu próprio diretório de trabalho. \textbf{Lembre de usar a barra ``/'' entre os diretórios. E não a contra-barra ``\textbackslash{}''.}

Usaremos uma matriz multivariada (sítios x espécies, matriz comunitária) do Projeto PPBio chamada ppbio**.xlsx que está no diretório {``C:/Meu/Diretório/De/Trabalho/Planilha.xlsx''}

Note que o sómbolo \texttt{\#} em programação R significa que o texto que vem depois dele é um comentário e não será executado pelo programa. Isso é útil para explicar o código ou deixar anotações.

Ajuste a segunda linha do código abaixo para refletir {``C:/Seu/Diretório/De/Trabalho/Planilha.xlsx''}.

Definindo o diretório de trabalho e installando os pacotes necessários:

\begin{Shaded}
\begin{Highlighting}[]
\FunctionTok{getwd}\NormalTok{()}
\FunctionTok{setwd}\NormalTok{(}\StringTok{"C:/Seu/Diretório/De/Trabalho"}\NormalTok{)}
\end{Highlighting}
\end{Shaded}

O símbolo \texttt{?} é usado para acessar a documentação de uma função ou um pacote no R. Como mostrado acima você pode saber mais sobre a função \texttt{getwd()}, usando o comando \texttt{?getwd}. Isso vai abrir uma página no menu de ajuda com a descrição, os argumentos, os valores de retorno e os exemplos da função \texttt{getwd()}. Você também pode usar o símbolo \texttt{?} para obter informações sobre um pacote inteiro. Por exemplo, se você quiser saber mais sobre o pacote \texttt{openxlsx}, você pode digitar \texttt{?openxlsx}. Isso vai abrir uma página com a visão geral, a instalação, os recursos e as referências do pacote solicitado.

\hypertarget{importando-a-planilha-1}{%
\section{Importando a planilha}\label{importando-a-planilha-1}}

Note que o sómbolo \texttt{\#} em programação R significa que o texto que vem depois dele é um comentário e não será executado pelo programa. Isso é útil para explicar o código ou deixar anotações. Ajuste a segunda linha do código abaixo para refletir {``C:/Seu/Diretório/De/Trabalho/Planilha.xlsx''}.

\begin{Shaded}
\begin{Highlighting}[]
\FunctionTok{library}\NormalTok{(openxlsx)}
\NormalTok{ppbio }\OtherTok{\textless{}{-}} \FunctionTok{read.xlsx}\NormalTok{(}\StringTok{"D:/Elvio/OneDrive/Disciplinas/\_EcoNumerica/5.Matrizes/ppbio06{-}peixes.xlsx"}\NormalTok{,}
                   \AttributeTok{rowNames =}\NormalTok{ T,}
                   \AttributeTok{colNames =}\NormalTok{ T,}
                   \AttributeTok{sheet =} \StringTok{"Sheet1"}\NormalTok{)}
\NormalTok{ppbio[}\DecValTok{1}\SpecialCharTok{:}\DecValTok{10}\NormalTok{, }\DecValTok{1}\SpecialCharTok{:}\DecValTok{10}\NormalTok{] }\CommentTok{\#primeiras 10 linhas e 10 colunas da matriz}
\end{Highlighting}
\end{Shaded}

\begin{verbatim}
##         ap-davis as-bimac as-fasci ch-bimac ci-ocela ci-orien co-macro co-heter
## S-A-ZA1        0        1        0        0        0        0        0        0
## S-R-CC1        0       99        0        0        0        0        0        0
## S-R-CT1        0      194       55        0        0        5        0        1
## S-R-CP1        0       19        0        0        0        0        0        0
## S-A-TA1        0       23        1       13        0        0        0        0
## S-R-CT2        0      142        3        3        0       69        0        0
## S-R-CP2        0        5        1        0       40        9        0        0
## S-A-TA2        0       46        0      178        0        0        0        0
## S-R-CT3        0      206       64        0        0       25        0        0
## S-R-CP3        0       16        0        0       13       24        0        0
##         cr-menez cu-lepid
## S-A-ZA1        0        0
## S-R-CC1        0        0
## S-R-CT1       14        0
## S-R-CP1        0        0
## S-A-TA1        0        0
## S-R-CT2        4        0
## S-R-CP2        0        0
## S-A-TA2        0        0
## S-R-CT3        8        0
## S-R-CP3        0        0
\end{verbatim}

Atente para os resultados dos comandos a seguir.

\begin{Shaded}
\begin{Highlighting}[]
\CommentTok{\#View(ppbio)}
\FunctionTok{print}\NormalTok{(ppbio[}\DecValTok{1}\SpecialCharTok{:}\DecValTok{10}\NormalTok{, }\DecValTok{1}\SpecialCharTok{:}\DecValTok{10}\NormalTok{])}
\CommentTok{\#ppbio}
\FunctionTok{str}\NormalTok{(ppbio)}
\CommentTok{\#?str}
\FunctionTok{mode}\NormalTok{(ppbio)}
\CommentTok{\#?mode}
\FunctionTok{class}\NormalTok{(ppbio)}
\CommentTok{\#?class}
\NormalTok{ppbio\_ma }\OtherTok{\textless{}{-}} \FunctionTok{as.matrix}\NormalTok{(ppbio) }\CommentTok{\#lê ppbio como uma matriz}
\FunctionTok{str}\NormalTok{(ppbio\_ma)}
\end{Highlighting}
\end{Shaded}

\begin{verbatim}
##         ap-davis as-bimac as-fasci ch-bimac ci-ocela ci-orien co-macro co-heter
## S-A-ZA1        0        1        0        0        0        0        0        0
## S-R-CC1        0       99        0        0        0        0        0        0
## S-R-CT1        0      194       55        0        0        5        0        1
## S-R-CP1        0       19        0        0        0        0        0        0
## S-A-TA1        0       23        1       13        0        0        0        0
## S-R-CT2        0      142        3        3        0       69        0        0
## S-R-CP2        0        5        1        0       40        9        0        0
## S-A-TA2        0       46        0      178        0        0        0        0
## S-R-CT3        0      206       64        0        0       25        0        0
## S-R-CP3        0       16        0        0       13       24        0        0
##         cr-menez cu-lepid
## S-A-ZA1        0        0
## S-R-CC1        0        0
## S-R-CT1       14        0
## S-R-CP1        0        0
## S-A-TA1        0        0
## S-R-CT2        4        0
## S-R-CP2        0        0
## S-A-TA2        0        0
## S-R-CT3        8        0
## S-R-CP3        0        0
## 'data.frame':    26 obs. of  35 variables:
##  $ ap-davis : num  0 0 0 0 0 0 0 0 0 0 ...
##  $ as-bimac : num  1 99 194 19 23 142 5 46 206 16 ...
##  $ as-fasci : num  0 0 55 0 1 3 1 0 64 0 ...
##  $ ch-bimac : num  0 0 0 0 13 3 0 178 0 0 ...
##  $ ci-ocela : num  0 0 0 0 0 0 40 0 0 13 ...
##  $ ci-orien : num  0 0 5 0 0 69 9 0 25 24 ...
##  $ co-macro : num  0 0 0 0 0 0 0 0 0 0 ...
##  $ co-heter : num  0 0 1 0 0 0 0 0 0 0 ...
##  $ cr-menez : num  0 0 14 0 0 4 0 0 8 0 ...
##  $ cu-lepid : num  0 0 0 0 0 0 0 0 0 0 ...
##  $ cy-gilbe : num  0 0 0 0 0 0 0 0 0 0 ...
##  $ ge-brasi : num  0 0 3 0 0 0 0 0 1 0 ...
##  $ he-margi : num  0 0 0 0 0 1 0 0 0 0 ...
##  $ ho-malab : num  0 0 1 5 0 17 10 2 31 4 ...
##  $ hy-pusar : num  0 0 9 2 0 43 2 0 11 0 ...
##  $ le-melan : num  0 0 0 0 0 0 0 0 0 0 ...
##  $ le-piau  : num  0 0 3 0 0 1 3 0 2 1 ...
##  $ le-taeni : num  0 0 0 0 0 0 0 0 0 0 ...
##  $ mo-costa : num  0 0 0 0 0 0 0 0 0 0 ...
##  $ mo-lepid : num  0 1 39 0 0 1 0 0 0 0 ...
##  $ or-nilot : num  0 2 36 0 0 77 0 0 138 0 ...
##  $ pa-manag : num  0 0 0 0 0 0 0 0 0 0 ...
##  $ pimel-sp : num  0 0 6 0 0 0 0 0 0 0 ...
##  $ po-retic : num  0 0 0 0 0 20 0 0 5 0 ...
##  $ po-vivip : num  0 0 47 15 0 221 32 0 326 10 ...
##  $ pr-brevi : num  9 0 5 0 1 15 5 2 164 0 ...
##  $ ps-rhomb : num  0 0 0 0 0 0 0 0 1 0 ...
##  $ ps-genise: num  0 0 0 0 0 0 0 0 1 0 ...
##  $ se-heter : num  0 0 40 14 4 60 0 0 38 0 ...
##  $ se-piaba : num  0 0 68 0 0 0 0 0 0 0 ...
##  $ se-spilo : num  0 0 0 0 0 0 0 0 1 0 ...
##  $ st-noton : num  0 0 1 0 0 25 0 0 115 0 ...
##  $ sy-marmo : num  0 0 0 0 0 0 1 0 0 0 ...
##  $ te-chalc : num  0 0 0 0 0 0 0 0 0 0 ...
##  $ tr-signa : num  0 0 18 0 0 15 0 0 7 0 ...
## [1] "list"
## [1] "data.frame"
##  num [1:26, 1:35] 0 0 0 0 0 0 0 0 0 0 ...
##  - attr(*, "dimnames")=List of 2
##   ..$ : chr [1:26] "S-A-ZA1" "S-R-CC1" "S-R-CT1" "S-R-CP1" ...
##   ..$ : chr [1:35] "ap-davis" "as-bimac" "as-fasci" "ch-bimac" ...
\end{verbatim}

\hypertarget{apuxeandices-4}{%
\section*{Apêndices}\label{apuxeandices-4}}
\addcontentsline{toc}{section}{Apêndices}

\hypertarget{snippet-e-chunk}{%
\subsection*{Snippet e chunk}\label{snippet-e-chunk}}
\addcontentsline{toc}{subsection}{Snippet e chunk}

Em programação R, tanto \texttt{snippet} (``trecho de código'') quanto \texttt{chunk} ``pedaço de código'' referem-se a porções de código usadas dentro de um contexto maior, geralmente no contexto de sessões interativas ou documentos estruturados como R Markdown. Aqui está como eles diferem:

\begin{enumerate}
\def\labelenumi{\arabic{enumi}.}
\tightlist
\item
  \textbf{Snippet}:

  \begin{itemize}
  \tightlist
  \item
    Um ``trecho de código'' em R geralmente se refere a um pedaço pequeno e isolado de código que executa uma tarefa ou operação específica.
  \item
    Trechos de código são frequentemente usados de forma interativa, seja em um console R ou dentro de um ambiente de desenvolvimento integrado (IDE) como o RStudio.
  \item
    Esses trechos são geralmente segmentos curtos de código usados para experimentação, testes ou cálculos rápidos.
  \end{itemize}
\item
  \textbf{Chunk}:

  \begin{itemize}
  \tightlist
  \item
    Um ``trecho'' em R é um termo comumente usado no contexto da programação literária, especialmente em documentos escritos usando R Markdown.
  \item
    Trechos são blocos maiores de código que podem ser executados independentemente dentro de um documento R Markdown.
  \item
    Os trechos são delimitados por delimitadores especiais, como \texttt{\{r\}}, que indicam ao processador R Markdown que o conteúdo incluído é código R.
  \item
    Documentos R Markdown permitem a integração de trechos de código R com texto narrativo, permitindo que os usuários combinem código, resultados e texto explicativo em um único documento.
  \end{itemize}
\end{enumerate}

Em resumo, enquanto um ``trecho de código'' geralmente se refere a uma pequena e isolada peça de código usada para testes ou experimentação, um ``trecho'' se refere a um bloco maior de código usado dentro do contexto da programação literária, particularmente em documentos escritos usando R Markdown. Os trechos permitem a integração de código e texto narrativo em documentos estruturados.

\hypertarget{sites-consultados-2}{%
\section*{Sites consultados}\label{sites-consultados-2}}
\addcontentsline{toc}{section}{Sites consultados}

Recomendo visitar esses sites e ver demais conteúdos relacionados

Visite:\\
\url{https://felipegalvao.com.br/pt/blog/basic-r-introduction-data-types-and-structures/}\strut \\
\url{https://lhmet.github.io/adar-ebook/datatype.html}

Vídeos:\\
Objetos, atributos e tipos de dados em R: \url{https://www.youtube.com/watch?v=YdUUQapPYzs\&t=1s}

Outros materiais de apoio:\\
Tipos de dados e operadores I (Parte 8): \url{https://www.youtube.com/watch?v=Dy2cI02WbMQ}\\
Tipos de dados e operadores II (Parte 9) \url{https://www.youtube.com/watch?v=E6ZTBbicTvg}\\
Estrutura de dados I (Parte 10): \url{https://www.youtube.com/watch?v=xzlM33Tqvic}\\
Estrutura de dados I (Parte 11): \url{https://www.youtube.com/watch?v=aEaSzlDMcuI}

\hypertarget{script-limpo-1}{%
\section*{Script limpo}\label{script-limpo-1}}
\addcontentsline{toc}{section}{Script limpo}

Aqui apresento o scrip na íntegra sem os textos ou outros comentários. Você pode copiar e colar no R para executa-lo. Lembre de remover os {\texttt{\#}} ou {\texttt{\#\#}} caso necessite executar essas linhas.

\begin{Shaded}
\begin{Highlighting}[]
\StringTok{"Início do R Módulo"}
\DecValTok{600}\SpecialCharTok{+}\DecValTok{66}
\NormalTok{a }\OtherTok{\textless{}{-}} \DecValTok{5} \SpecialCharTok{+} \DecValTok{3}
\NormalTok{a}
\FunctionTok{print}\NormalTok{(a)}
\FunctionTok{class}\NormalTok{(a)}
\NormalTok{var1 }\OtherTok{\textless{}{-}} \DecValTok{3}
\NormalTok{var2 }\OtherTok{\textless{}{-}} \FunctionTok{as.integer}\NormalTok{(}\DecValTok{3}\NormalTok{)}
\FunctionTok{class}\NormalTok{(var1)}
\FunctionTok{class}\NormalTok{(var2)}
\NormalTok{b }\OtherTok{\textless{}{-}} \StringTok{"Ola mundo"}
\NormalTok{c }\OtherTok{\textless{}{-}} \ConstantTok{TRUE}
\FunctionTok{class}\NormalTok{(b)}
\FunctionTok{class}\NormalTok{(c)}
\NormalTok{militares }\OtherTok{\textless{}{-}} \FunctionTok{c}\NormalTok{(}\StringTok{"Soldado"}\NormalTok{,}\StringTok{"Soldado"}\NormalTok{,}\StringTok{"Coronel"}\NormalTok{,}\StringTok{"General"}\NormalTok{,}\StringTok{"Tenente"}\NormalTok{,}\StringTok{"Tenente"}\NormalTok{,}\StringTok{"Cabo"}\NormalTok{,}\StringTok{"Soldado"}\NormalTok{,}\StringTok{"Soldado"}\NormalTok{,}\StringTok{"Cabo"}\NormalTok{,}\StringTok{"Cabo"}\NormalTok{,}\StringTok{"Soldado"}\NormalTok{)}
\NormalTok{militares}
\NormalTok{militares }\OtherTok{\textless{}{-}} \FunctionTok{factor}\NormalTok{(militares)}
\FunctionTok{print}\NormalTok{(militares)}
\FunctionTok{table}\NormalTok{(militares)}
\NormalTok{var1 }\OtherTok{\textless{}{-}} \DecValTok{3}
\FunctionTok{is.vector}\NormalTok{(var1)}
\NormalTok{var1 }\OtherTok{\textless{}{-}} \FunctionTok{c}\NormalTok{(}\DecValTok{3}\NormalTok{,}\DecValTok{6}\NormalTok{, }\FloatTok{7.8}\NormalTok{, }\DecValTok{332}\NormalTok{)}
\FunctionTok{print}\NormalTok{(var1)}
\NormalTok{var2 }\OtherTok{\textless{}{-}} \FunctionTok{c}\NormalTok{(}\StringTok{"Ola"}\NormalTok{,}\StringTok{"tudo"}\NormalTok{,}\StringTok{"bem"}\NormalTok{)}
\FunctionTok{print}\NormalTok{(var2)}
\NormalTok{var3 }\OtherTok{\textless{}{-}} \FunctionTok{c}\NormalTok{(}\ConstantTok{TRUE}\NormalTok{, }\ConstantTok{FALSE}\NormalTok{, }\ConstantTok{TRUE}\NormalTok{, }\ConstantTok{TRUE}\NormalTok{, }\ConstantTok{FALSE}\NormalTok{, }\ConstantTok{FALSE}\NormalTok{, }\ConstantTok{TRUE}\NormalTok{, }\ConstantTok{TRUE}\NormalTok{, }\ConstantTok{TRUE}\NormalTok{)}
\FunctionTok{print}\NormalTok{(var3)}
\NormalTok{mat1 }\OtherTok{\textless{}{-}} \FunctionTok{matrix}\NormalTok{(}
 \FunctionTok{c}\NormalTok{(}\DecValTok{1}\NormalTok{,}\DecValTok{5}\NormalTok{,}\DecValTok{10}\NormalTok{,}\DecValTok{30}\NormalTok{,}\DecValTok{15}\NormalTok{,}\DecValTok{8}\NormalTok{),}
 \AttributeTok{nrow=}\DecValTok{3}\NormalTok{,}
 \AttributeTok{ncol=}\DecValTok{2}\NormalTok{,}
 \AttributeTok{byrow=}\ConstantTok{TRUE}\NormalTok{)}
\FunctionTok{print}\NormalTok{(mat1)}
\NormalTok{vec1 }\OtherTok{\textless{}{-}} \FunctionTok{c}\NormalTok{(}\DecValTok{1}\NormalTok{, }\DecValTok{5}\NormalTok{)}
\NormalTok{vec2 }\OtherTok{\textless{}{-}} \FunctionTok{c}\NormalTok{(}\DecValTok{10}\NormalTok{, }\DecValTok{30}\NormalTok{)}
\NormalTok{vec3 }\OtherTok{\textless{}{-}} \FunctionTok{c}\NormalTok{(}\DecValTok{15}\NormalTok{, }\DecValTok{8}\NormalTok{) }
\NormalTok{mat2 }\OtherTok{\textless{}{-}} \FunctionTok{rbind}\NormalTok{(vec1, vec2, vec3)}
\FunctionTok{print}\NormalTok{(mat2)}
\FunctionTok{class}\NormalTok{(mat2)}
\NormalTok{mat1[}\DecValTok{2}\NormalTok{][}\DecValTok{1}\NormalTok{]}
\NormalTok{a }\OtherTok{\textless{}{-}} \FunctionTok{c}\NormalTok{(}\DecValTok{3}\NormalTok{,}\DecValTok{6}\NormalTok{,}\DecValTok{9}\NormalTok{)}
\NormalTok{b }\OtherTok{\textless{}{-}} \FunctionTok{c}\NormalTok{(}\StringTok{"a"}\NormalTok{,}\StringTok{"b"}\NormalTok{,}\StringTok{"c"}\NormalTok{,}\StringTok{"d"}\NormalTok{)}
\NormalTok{c }\OtherTok{\textless{}{-}} \FunctionTok{c}\NormalTok{(}\ConstantTok{TRUE}\NormalTok{, }\ConstantTok{FALSE}\NormalTok{, }\ConstantTok{TRUE}\NormalTok{, }\ConstantTok{TRUE}\NormalTok{)}
\NormalTok{lista1 }\OtherTok{\textless{}{-}} \FunctionTok{list}\NormalTok{(a,b,c)}
\FunctionTok{print}\NormalTok{(lista1)}
\NormalTok{df1 }\OtherTok{\textless{}{-}} \FunctionTok{data.frame}\NormalTok{(}\FunctionTok{c}\NormalTok{(}\DecValTok{1}\NormalTok{,}\DecValTok{2}\NormalTok{,}\DecValTok{3}\NormalTok{),}\FunctionTok{c}\NormalTok{(}\StringTok{"baixo"}\NormalTok{,}\StringTok{"medio"}\NormalTok{,}\StringTok{"alto"}\NormalTok{),}\FunctionTok{c}\NormalTok{(}\ConstantTok{TRUE}\NormalTok{, }\ConstantTok{TRUE}\NormalTok{, }\ConstantTok{FALSE}\NormalTok{))}
\FunctionTok{print}\NormalTok{(df1)}
\FunctionTok{print}\NormalTok{(mtcars)}
\NormalTok{mtcars[}\DecValTok{1}\NormalTok{,}\DecValTok{2}\NormalTok{]}
\NormalTok{mtcars[}\StringTok{"Mazda RX4"}\NormalTok{,}\StringTok{"gear"}\NormalTok{]}
\NormalTok{mtcars[}\DecValTok{1}\SpecialCharTok{:}\DecValTok{5}\NormalTok{,}\DecValTok{1}\NormalTok{]}
\NormalTok{mtcars[,}\DecValTok{2}\NormalTok{]}
\NormalTok{mtcars[}\DecValTok{3}\NormalTok{,]}
\FunctionTok{nrow}\NormalTok{(mtcars)}
\FunctionTok{ncol}\NormalTok{(mtcars)}
\FunctionTok{dim}\NormalTok{(mtcars)}
\FunctionTok{str}\NormalTok{(mtcars)}
\FunctionTok{summary}\NormalTok{(mtcars)}
\DocumentationTok{\#\# rm(list=ls(all=TRUE)) \#limpa a memória}
\DocumentationTok{\#\# install.packages("openxlsx") \#importa arquivos do excel}
\FunctionTok{library}\NormalTok{(openxlsx)}
\DocumentationTok{\#\# getwd()}
\DocumentationTok{\#\# setwd("C:/Seu/Diretório/De/Trabalho")}
\FunctionTok{library}\NormalTok{(openxlsx)}
\NormalTok{ppbio }\OtherTok{\textless{}{-}} \FunctionTok{read.xlsx}\NormalTok{(}\StringTok{"D:/Elvio/OneDrive/Disciplinas/\_EcoNumerica/5.Matrizes/bentos06.xlsx"}\NormalTok{,}
                   \AttributeTok{rowNames =}\NormalTok{ T,}
                   \AttributeTok{colNames =}\NormalTok{ T,}
                   \AttributeTok{sheet =} \StringTok{"contagem"}\NormalTok{)}
\NormalTok{ppbio[}\DecValTok{1}\SpecialCharTok{:}\DecValTok{10}\NormalTok{, }\DecValTok{1}\SpecialCharTok{:}\DecValTok{10}\NormalTok{] }\CommentTok{\#primeiras 10 linhas e 10 colunas da matriz}
\CommentTok{\#View(ppbio)}
\FunctionTok{print}\NormalTok{(ppbio[}\DecValTok{1}\SpecialCharTok{:}\DecValTok{10}\NormalTok{, }\DecValTok{1}\SpecialCharTok{:}\DecValTok{10}\NormalTok{])}
\CommentTok{\#ppbio}
\FunctionTok{str}\NormalTok{(ppbio)}
\CommentTok{\#?str}
\FunctionTok{mode}\NormalTok{(ppbio)}
\CommentTok{\#?mode}
\FunctionTok{class}\NormalTok{(ppbio)}
\CommentTok{\#?class}
\NormalTok{ppbio\_ma }\OtherTok{\textless{}{-}} \FunctionTok{as.matrix}\NormalTok{(ppbio) }\CommentTok{\#lê ppbio como uma matriz}
\FunctionTok{str}\NormalTok{(ppbio\_ma)}
\end{Highlighting}
\end{Shaded}

\hypertarget{referuxeancias-5}{%
\section*{Referências}\label{referuxeancias-5}}
\addcontentsline{toc}{section}{Referências}

\hypertarget{univ}{%
\chapter{R Modulo 3.1 - Estatísticas descritivas e Normalidade}\label{univ}}

\hypertarget{resumo-5}{%
\subsection*{RESUMO}\label{resumo-5}}
\addcontentsline{toc}{subsection}{RESUMO}

A estatística descritiva tem um papel importante a desempenhar na ciência. Quando problemas específicos são tratados na ciência, os dados precisam ser coletados, analisados e apresentados de forma concisa para que outros possam se beneficiar do que foi encontrado.

\hypertarget{apresentauxe7uxe3o-5}{%
\subsection*{Apresentação}\label{apresentauxe7uxe3o-5}}
\addcontentsline{toc}{subsection}{Apresentação}

A estatística descritiva tem um papel importante a desempenhar na ciência. Quando problemas específicos são tratados na ciência, os dados precisam ser coletados, analisados e apresentados de forma concisa para que outros possam se beneficiar do que foi encontrado. Geralmente não é possível apresentar um conjunto de dados completo em uma publicação ou em um seminário e, mesmo que fosse, é improvável isso permitisse uma boa comunicação dos resultados da pesquisa. Em vez disso, os dados são geralmente resumidos como tabelas de frequência, histogramas e estatísticas descritivas que os leitores ou ouvintes podem assimilar prontamente, mas que ainda transmitem os elementos essenciais do conjunto de dados original. O principal objetivo do cálculo das estatísticas descritivas é transmitir informações essenciais contidas em um conjunto de dados da forma mais concisa e clara possível.

\hypertarget{sobre-os-dados}{%
\section{Sobre os dados}\label{sobre-os-dados}}

Considere os dados sobre o comprimento dos brotos de \emph{Banksia ericifolia} de charnecas\footnote{A etimologia do gênero \emph{Astyanax} vem da mitologia Grega. Heitor personagem da ``Ilíada'', tinha um filho chamado Astíanax.} na Baía de Jervis, Austrália (\protect\hyperlink{ref-RN146}{BRADSTOCK; TOZER; KEITH, 1997}) (Figura \ref{fig:rmd3banks}). Existem 500 medições, um conjunto de dados formidável. Por inspeção, o comprimento mínimo do broto é 10,0 e o máximo é 44,9 cm. Esses valores definem o intervalo da amostra. Agora precisamos subdividir o intervalo em intervalos ou classes, cada um com o mesmo tamanho. Geralmente, é aconselhável arredondar o valor mínimo para baixo e o valor máximo para valores apropriados ao decidir as classes de intervalos. Nesse caso, parece sensato dividir a faixa de 10 a 45 cm em sete intervalos a cada 5 cm de largura. Se contarmos o número de brotos que se encontram em cada um dos sete intervalos, temos a base para a tabulação da frequência. A coluna de frequência foi obtida contando o número de medições que existem dentro de cada classe. A coluna de frequência percentual foi obtida representando cada contagem como uma porcentagem da contagem total. A frequência cumulativa e as frequências percentuais cumulativas foram obtidas somando progressivamente as frequências correspondentes.

\begin{figure}
\includegraphics[width=8.5in]{imagens/banksia} \caption{Comprimentos (cm) de 500 brotos do arbusto *Banksia ericifolia* de charnecas na Baía de Jervis, Austrália.}\label{fig:rmd3banks}
\end{figure}

\hypertarget{organizauxe7uxe3o-buxe1sica-2}{%
\section{Organização básica}\label{organizauxe7uxe3o-buxe1sica-2}}

\begin{Shaded}
\begin{Highlighting}[]
\FunctionTok{rm}\NormalTok{(}\AttributeTok{list=}\FunctionTok{ls}\NormalTok{(}\AttributeTok{all=}\ConstantTok{TRUE}\NormalTok{)) }\CommentTok{\#limpa a memória}
\end{Highlighting}
\end{Shaded}

Instalando os pacotes necessários para esse módulo

\begin{Shaded}
\begin{Highlighting}[]
\FunctionTok{install.packages}\NormalTok{(}\StringTok{"openxlsx"}\NormalTok{) }\CommentTok{\#importa arquivos do excel}
\FunctionTok{install.packages}\NormalTok{(}\StringTok{"fdth"}\NormalTok{)}
\end{Highlighting}
\end{Shaded}

\begin{Shaded}
\begin{Highlighting}[]
\FunctionTok{library}\NormalTok{(openxlsx)}
\end{Highlighting}
\end{Shaded}

Os códigos acima, são usados para instalar e carregar os pacotes necessários para este módulo. Esses códigos são comandos para instalar pacotes no R. Um pacote é uma coleção de funções, dados e documentação que ampliam as capacidades do R (\href{https://cran.r-project.org/}{R CRAN} (\protect\hyperlink{ref-RN2774}{TEAM, 2017}) e \href{https://posit.co/download/rstudio-desktop/}{RStudio} (\protect\hyperlink{ref-RN358}{TEAM, 2022})). No exemplo acima, o pacote \texttt{openxlsx} permite ler e escrever arquivos Excel no R. Para instalar um pacote no R, você precisa usar a função \texttt{install.packages()}.

Depois de instalar um pacote, você precisa carregá-lo na sua sessão R com a função \texttt{library()}. Por exemplo, para carregar o pacote \texttt{openxlsx}, você precisa executar a função \texttt{library(openxlsx)}. Isso irá permitir que você use as funções do pacote na sua sessão R. Você precisa carregar um pacote toda vez que iniciar uma nova sessão R e quiser usar um pacote instalado.

Agora vamos \textbf{definir o diretório de trabalho}. Esse código é usado para obter e definir o diretório de trabalho atual no R. O comando \texttt{getwd()} retorna o caminho do diretório onde o R está lendo e salvando arquivos. O comando \texttt{setwd()} muda esse diretório de trabalho para o caminho especificado entre aspas. No seu caso, você deve ajustar o caminho para o seu próprio diretório de trabalho. \textbf{Lembre de usar a barra ``/'' entre os diretórios. E não a contra-barra ``\textbackslash{}''.}

\begin{Shaded}
\begin{Highlighting}[]
\FunctionTok{getwd}\NormalTok{()}
\FunctionTok{setwd}\NormalTok{(}\StringTok{"C:/Seu/Diretório/De/Trabalho"}\NormalTok{)}
\end{Highlighting}
\end{Shaded}

\hypertarget{importando-a-planilha-2}{%
\section{Importando a planilha}\label{importando-a-planilha-2}}

Note que o sómbolo {\texttt{\#}} em programação R significa que o texto que vem depois dele é um comentário e não será executado pelo programa. Isso é útil para explicar o código ou deixar anotações. Ajuste a segunda linha do código abaixo para refletir {``C:/Seu/Diretório/De/Trabalho/Planilha.xlsx''}.

\begin{Shaded}
\begin{Highlighting}[]
\FunctionTok{library}\NormalTok{(openxlsx)}
\NormalTok{brotos }\OtherTok{\textless{}{-}} \FunctionTok{read.xlsx}\NormalTok{(}\StringTok{"D:/Elvio/OneDrive/Disciplinas/\_EcoNumerica/5.Matrizes/brotos.xlsx"}\NormalTok{,}
                    \AttributeTok{rowNames =}\NormalTok{ F, }\AttributeTok{colNames =}\NormalTok{ F,}
                    \AttributeTok{sheet =} \StringTok{"Planilha1"}\NormalTok{)}
\FunctionTok{head}\NormalTok{(brotos,}\DecValTok{10}\NormalTok{)}
\FunctionTok{head}\NormalTok{(brotos[, }\DecValTok{1}\SpecialCharTok{:}\DecValTok{5}\NormalTok{], }\DecValTok{10}\NormalTok{)}
\end{Highlighting}
\end{Shaded}

\begin{verbatim}
##      X1   X2   X3   X4   X5   X6   X7   X8   X9  X10
## 1  28.4 29.1 25.7 26.2 25.8 30.2 26.2 35.8 33.4 29.0
## 2  25.1 36.8 29.9 38.3 27.1 32.3 29.0 27.7 28.1 28.9
## 3  29.9 36.3 26.0 21.2 19.8 36.7 21.1 34.6 29.6 32.6
## 4  25.8 23.2 28.8 38.2 32.7 38.7 33.2 24.5 21.6 28.6
## 5  35.0 25.1 24.7 29.5 24.9 29.2 19.5 20.1 30.3 38.9
## 6  36.4 31.8 31.0 39.4 28.8 31.8 28.7 37.0 25.5 19.3
## 7  29.1 21.1 30.4 31.2 38.0 39.0 19.3 27.6 19.1 32.5
## 8  33.2 26.5 38.1 14.9 33.2 27.8 24.7 24.9 25.0 33.1
## 9  26.9 22.5 25.5 33.0 19.4 26.8 24.6 37.5 19.8 43.7
## 10 34.2 33.6 42.5 19.0 25.8 34.0 34.4 42.0 35.4 31.5
##      X1   X2   X3   X4   X5
## 1  28.4 29.1 25.7 26.2 25.8
## 2  25.1 36.8 29.9 38.3 27.1
## 3  29.9 36.3 26.0 21.2 19.8
## 4  25.8 23.2 28.8 38.2 32.7
## 5  35.0 25.1 24.7 29.5 24.9
## 6  36.4 31.8 31.0 39.4 28.8
## 7  29.1 21.1 30.4 31.2 38.0
## 8  33.2 26.5 38.1 14.9 33.2
## 9  26.9 22.5 25.5 33.0 19.4
## 10 34.2 33.6 42.5 19.0 25.8
\end{verbatim}

Exibindo os dados importados (esses comando são ``case-sensitive'' \texttt{ignore.case(object)}).

\begin{Shaded}
\begin{Highlighting}[]
\CommentTok{\#View(brotos)}
\FunctionTok{print}\NormalTok{(brotos[}\DecValTok{1}\SpecialCharTok{:}\DecValTok{5}\NormalTok{,}\DecValTok{1}\SpecialCharTok{:}\DecValTok{5}\NormalTok{])}
\NormalTok{brotos}
\FunctionTok{str}\NormalTok{(brotos)}
\FunctionTok{mode}\NormalTok{(brotos)}
\FunctionTok{class}\NormalTok{(brotos)}
\end{Highlighting}
\end{Shaded}

Precisaremos converter para um vetor.

\begin{Shaded}
\begin{Highlighting}[]
\NormalTok{brotos\_v }\OtherTok{\textless{}{-}} \FunctionTok{c}\NormalTok{(}\FunctionTok{t}\NormalTok{(brotos))}
\NormalTok{brotos\_v}
\end{Highlighting}
\end{Shaded}

\begin{verbatim}
##   [1] 28.4 29.1 25.7 26.2 25.8 30.2 26.2 35.8 33.4 29.0 25.1 36.8 29.9 38.3 27.1
##  [16] 32.3 29.0 27.7 28.1 28.9 29.9 36.3 26.0 21.2 19.8 36.7 21.1 34.6 29.6 32.6
##  [31] 25.8 23.2 28.8 38.2 32.7 38.7 33.2 24.5 21.6 28.6 35.0 25.1 24.7 29.5 24.9
##  [46] 29.2 19.5 20.1 30.3 38.9 36.4 31.8 31.0 39.4 28.8 31.8 28.7 37.0 25.5 19.3
##  [61] 29.1 21.1 30.4 31.2 38.0 39.0 19.3 27.6 19.1 32.5 33.2 26.5 38.1 14.9 33.2
##  [76] 27.8 24.7 24.9 25.0 33.1 26.9 22.5 25.5 33.0 19.4 26.8 24.6 37.5 19.8 43.7
##  [91] 34.2 33.6 42.5 19.0 25.8 34.0 34.4 42.0 35.4 31.5 32.0 38.4 29.1 29.4 29.3
## [106] 26.8 32.4 25.2 28.5 29.8 31.7 22.4 21.7 20.1 21.6 23.5 33.2 33.0 29.6 36.9
## [121] 23.6 16.2 27.3 33.3 21.6 30.2 22.5 33.0 38.3 29.5 31.1 31.7 31.6 41.7 25.9
## [136] 32.3 35.7 31.6 26.0 26.6 29.1 15.8 22.1 23.2 25.4 28.4 20.2 25.5 26.9 32.7
## [151] 29.8 27.1 36.9 32.7 24.8 18.0 40.2 28.0 26.8 41.9 31.5 27.4 37.2 30.6 32.2
## [166] 34.8 28.2 31.3 34.3 32.0 20.6 27.2 25.0 26.1 34.5 44.9 40.5 32.1 40.4 35.7
## [181] 29.3 24.7 37.0 36.9 34.8 29.8 22.8 32.3 34.8 29.6 24.0 30.5 31.6 28.7 20.8
## [196] 14.6 23.4 26.3 31.9 32.5 32.4 36.4 24.1 33.1 26.3 35.7 26.4 34.7 27.5 39.6
## [211] 30.5 30.1 21.3 27.1 19.0 25.4 36.5 22.6 25.5 30.0 29.2 30.2 20.8 30.3 27.8
## [226] 32.9 28.2 20.6 33.6 22.2 26.1 29.7 32.0 22.2 29.2 21.5 31.4 43.1 35.9 14.9
## [241] 38.8 21.9 25.6 29.7 29.9 32.5 30.4 29.2 40.9 14.1 22.4 19.9 34.8 33.4 28.0
## [256] 29.1 27.2 18.8 36.2 27.8 29.9 21.8 33.1 30.4 30.8 33.9 27.1 27.6 37.2 30.9
## [271] 28.3 34.3 34.0 29.0 30.9 24.4 29.0 25.4 30.5 31.1 26.3 38.3 24.8 23.5 29.3
## [286] 37.8 29.9 28.6 27.4 29.9 24.3 20.7 22.5 39.5 32.0 27.6 36.3 22.0 28.4 19.1
## [301] 16.8 43.9 27.9 44.4 29.7 23.0 26.8 43.4 29.4 26.7 27.8 33.1 34.9 20.5 25.4
## [316] 10.0 28.2 31.0 10.6 28.4 21.9 27.0 26.5 29.2 24.9 18.4 24.1 28.3 29.0 29.1
## [331] 26.2 32.6 22.3 31.7 37.1 35.6 19.5 26.9 24.8 19.2 29.9 42.1 36.6 25.5 34.2
## [346] 22.4 40.5 21.2 32.3 31.5 29.3 34.8 34.8 22.6 33.6 23.0 33.8 21.9 31.6 16.5
## [361] 31.6 26.8 37.3 23.2 20.5 17.6 25.0 41.9 21.8 34.4 23.2 27.6 34.3 26.7 28.1
## [376] 24.7 34.2 15.7 27.2 37.5 32.1 25.5 34.7 43.7 31.6 37.9 27.4 17.0 19.4 24.6
## [391] 20.3 22.4 37.7 29.4 36.0 39.2 26.2 25.7 28.2 22.1 27.8 36.5 23.7 28.6 23.8
## [406] 16.2 38.0 22.1 44.0 20.2 22.9 41.5 30.4 36.1 32.7 30.1 39.9 22.3 26.8 31.4
## [421] 32.5 31.4 18.8 19.1 24.2 29.3 19.7 36.7 24.1 33.9 24.5 22.8 29.9 34.5 33.4
## [436] 22.8 30.8 22.1 38.1 33.5 36.5 33.5 28.6 24.9 24.3 25.7 19.2 34.5 40.6 25.8
## [451] 39.4 33.3 21.9 29.6 27.0 30.2 17.1 15.8 22.8 16.5 24.2 21.2 25.8 29.8 24.7
## [466] 30.6 38.1 33.3 26.8 16.5 23.2 24.5 29.9 26.0 40.5 30.0 30.3 33.9 34.9 18.8
## [481] 28.3 40.2 30.9 22.0 34.1 26.2 19.6 33.6 26.3 25.1 29.3 31.9 43.0 31.9 33.9
## [496] 20.0 39.6 34.3 28.8 34.2
\end{verbatim}

E agora visualizando nossos dados.

\begin{Shaded}
\begin{Highlighting}[]
\CommentTok{\#View(brotos\_v)}
\FunctionTok{print}\NormalTok{(brotos\_v)}
\NormalTok{brotos\_v}
\FunctionTok{str}\NormalTok{(brotos\_v)}
\FunctionTok{mode}\NormalTok{(brotos\_v)}
\FunctionTok{class}\NormalTok{(brotos\_v)}
\end{Highlighting}
\end{Shaded}

Tendo em mãos o conjunto total de 500 comprimentos medidos de brotos de \emph{Banksia}, agora \textbf{vamos tirar uma subamostra aleatória de uma parte dos 500 valores}. Essa subamostra é tirada usando o comando \texttt{size=} no código subsequente, que estabelece o tamanho da subamostra a ser tirada do total de dados. Esse tamanho é estabelecido no quiz. Inclua esse valor de acordo com o que é pedido no quiz. Por exemplo, se for pedido uma subamostra de 150 comprimentos, então \texttt{size\ =\ 150}. Nesse tutorial usaremos todos os 500 comprimentos.

\begin{Shaded}
\begin{Highlighting}[]
\FunctionTok{set.seed}\NormalTok{(}\DecValTok{666}\NormalTok{)}
\NormalTok{brotos\_sub }\OtherTok{\textless{}{-}} \FunctionTok{sample}\NormalTok{(brotos\_v, }\AttributeTok{size =} \DecValTok{500}\NormalTok{, }\AttributeTok{replace =}\NormalTok{ F) }\CommentTok{\#atualize o valor de \textquotesingle{}size=\textquotesingle{} se necessário}
\NormalTok{brotos\_sub}
\end{Highlighting}
\end{Shaded}

\begin{verbatim}
##   [1] 21.1 30.2 34.0 36.0 27.3 34.6 15.8 19.4 31.6 26.6 31.5 38.9 31.1 32.0 29.9
##  [16] 24.5 21.2 19.3 22.1 24.7 24.7 37.0 22.4 32.5 35.9 36.9 24.2 33.0 27.6 22.8
##  [31] 26.5 31.6 32.3 38.1 27.4 32.3 24.8 25.5 33.1 34.5 24.4 27.1 43.9 28.6 28.4
##  [46] 33.4 29.3 24.9 36.2 24.6 30.0 32.6 25.6 26.0 25.8 37.0 26.4 22.5 23.2 38.2
##  [61] 23.0 44.4 27.6 20.2 29.0 25.9 22.3 27.4 14.9 33.3 40.9 33.5 36.3 28.0 28.4
##  [76] 41.5 28.5 34.8 29.1 31.6 36.7 30.1 35.4 19.2 39.4 16.5 25.1 28.4 31.9 16.2
##  [91] 31.7 29.3 33.9 22.3 33.6 25.5 33.0 30.0 33.3 34.0 44.0 29.5 33.6 26.7 42.0
## [106] 33.2 24.8 26.9 25.4 43.0 29.4 15.8 28.8 39.5 31.8 20.3 38.3 19.8 37.8 26.7
## [121] 23.0 19.8 39.6 18.4 31.0 29.1 24.0 32.4 22.6 30.9 26.2 29.1 28.2 33.4 25.7
## [136] 30.9 34.8 34.1 24.1 29.9 20.0 29.3 28.7 28.6 36.5 25.1 17.1 24.7 29.2 38.7
## [151] 26.8 30.6 17.0 25.5 26.2 31.5 32.5 34.7 30.2 36.6 36.5 29.1 32.3 35.0 33.9
## [166] 19.1 25.0 26.1 40.2 30.4 31.4 33.6 37.5 29.8 20.8 29.3 39.6 28.4 31.0 24.9
## [181] 27.5 33.4 33.3 34.3 21.6 40.5 29.4 26.3 22.5 42.5 20.5 44.9 29.8 30.4 28.6
## [196] 27.6 39.4 33.2 34.2 20.1 18.8 26.8 22.9 25.7 21.9 34.9 34.8 33.2 40.6 26.9
## [211] 20.2 19.1 27.2 23.2 37.3 32.5 25.4 32.0 34.5 29.7 28.1 31.2 27.2 25.8 27.0
## [226] 26.3 32.2 43.7 42.1 38.4 19.0 28.9 31.8 34.5 22.1 29.0 25.2 38.1 29.2 34.3
## [241] 27.9 24.5 34.4 40.2 36.1 29.6 33.1 28.8 22.8 38.0 20.1 28.6 34.4 30.5 19.5
## [256] 30.2 30.4 22.4 35.6 17.6 28.2 29.5 37.7 19.5 28.0 37.1 20.6 34.3 40.5 27.8
## [271] 29.9 27.1 36.9 25.0 29.9 32.7 33.2 24.7 30.9 27.0 29.9 29.6 29.7 23.8 22.2
## [286] 34.8 38.8 31.1 26.8 19.0 22.4 36.9 35.7 22.0 24.1 26.1 34.3 25.5 18.0 41.7
## [301] 29.8 32.0 31.4 21.8 27.7 33.1 28.7 29.2 31.6 23.6 34.2 25.8 21.3 26.3 23.5
## [316] 21.6 32.7 27.8 32.7 30.3 16.8 33.0 10.0 30.4 25.4 26.2 22.1 37.5 26.0 33.5
## [331] 29.0 36.7 29.8 24.8 31.9 32.6 36.4 32.1 31.5 33.8 43.4 29.4 24.2 25.0 34.9
## [346] 22.0 24.9 16.2 28.3 16.5 29.1 24.6 19.3 25.8 29.0 29.3 29.9 32.7 35.8 32.3
## [361] 19.9 30.2 21.2 24.3 32.5 27.6 25.8 39.9 34.7 25.5 29.3 21.9 23.2 26.2 19.4
## [376] 19.2 32.1 26.3 26.8 38.0 28.3 34.8 19.6 35.7 35.7 23.2 32.9 27.4 22.8 40.4
## [391] 37.9 28.2 43.7 21.7 21.1 31.6 24.9 25.5 29.6 40.5 39.2 39.0 29.1 33.9 25.7
## [406] 29.6 29.7 19.7 38.3 22.6 36.4 16.5 28.3 30.8 21.2 14.9 34.8 33.6 24.1 25.4
## [421] 43.1 20.7 30.3 41.9 31.3 27.1 34.2 27.8 18.8 31.7 30.6 27.8 21.6 30.1 30.8
## [436] 27.2 28.8 37.2 14.6 18.8 26.9 22.1 14.1 25.1 22.2 36.5 23.4 29.9 33.9 24.3
## [451] 33.1 26.2 28.1 24.5 34.2 15.7 32.0 22.8 21.9 26.0 23.2 20.8 21.5 38.1 30.3
## [466] 19.1 31.6 27.1 32.4 23.5 36.8 37.2 21.8 26.8 31.9 22.4 24.7 20.6 30.5 26.8
## [481] 30.5 29.0 26.5 29.2 31.4 20.5 26.8 10.6 27.8 21.9 28.2 23.7 29.9 29.9 36.3
## [496] 31.7 38.3 29.2 22.5 41.9
\end{verbatim}

Para fazermos uma tabela de frequência do comprimento dos brotos carregamos o pacote \texttt{fdth} e pedimos a função \texttt{range} que retorna o valor máximo e mínimo no vetor.

\begin{Shaded}
\begin{Highlighting}[]
\FunctionTok{library}\NormalTok{(fdth)}
\end{Highlighting}
\end{Shaded}

\begin{verbatim}
## 
## Attaching package: 'fdth'
\end{verbatim}

\begin{verbatim}
## The following objects are masked from 'package:stats':
## 
##     sd, var
\end{verbatim}

\begin{Shaded}
\begin{Highlighting}[]
\FunctionTok{range}\NormalTok{(brotos\_sub) }\CommentTok{\#retorna o valor máximo e mínimo}
\CommentTok{\#?range}
\end{Highlighting}
\end{Shaded}

\begin{verbatim}
## [1] 10.0 44.9
\end{verbatim}

Agora é necessário que você substitua os valores de range 10, 44.9 nos valores de início e fim da distribuição de frequência. Os comandos abaixo criam uma tabela de frequência chamada \texttt{tf} com valores máximos e mínimos definidos por \texttt{range(brotos\_v)} em intervalor definidos por \texttt{h=5}, e o comando \texttt{print(tf)} exibe a tabela de frequência.

\begin{Shaded}
\begin{Highlighting}[]
\NormalTok{tf }\OtherTok{\textless{}{-}} \FunctionTok{fdt}\NormalTok{(brotos\_sub, }\AttributeTok{start=}\DecValTok{10}\NormalTok{, }\AttributeTok{end=}\DecValTok{45}\NormalTok{, }\AttributeTok{h=}\DecValTok{5}\NormalTok{) }\CommentTok{\#tabela de frequência}
\CommentTok{\#?fdt \#atente para o uso de k, por exemplo tf \textless{}{-} fdt(brotosv, k=5)}
\FunctionTok{print}\NormalTok{(tf)}
\end{Highlighting}
\end{Shaded}

\begin{verbatim}
##  Class limits   f   rf rf(%)  cf cf(%)
##       [10,15)   6 0.01   1.2   6   1.2
##       [15,20)  35 0.07   7.0  41   8.2
##       [20,25)  93 0.19  18.6 134  26.8
##       [25,30) 155 0.31  31.0 289  57.8
##       [30,35) 130 0.26  26.0 419  83.8
##       [35,40)  57 0.11  11.4 476  95.2
##       [40,45)  24 0.05   4.8 500 100.0
\end{verbatim}

Class limits = limites de classe, f = frequência de classe, rf = frequência relativa da classe, rf(\%) = frequência relativa percentual da classe, cf = frequência cumulativa da classe, cf(\%) = frequência cumulativa percentual da classe.

Se contarmos o número de brotos que se encontram em cada um dos sete intervalos, temos a base para a tabulação da frequência. Essa tabulação é mostrada na tabela gerada pelos códigos acima. A coluna de frequência foi obtida contando o número de medições que existem dentro de cada classe. A coluna de frequência percentual foi obtida representando cada contagem como uma porcentagem da contagem total. A frequência cumulativa e as frequências percentuais cumulativas foram obtidas somando progressivamente as frequências correspondentes.

\hypertarget{gruxe1ficos-de-histograma-e-boxplot}{%
\section{Gráficos de histograma e boxplot}\label{gruxe1ficos-de-histograma-e-boxplot}}

No código abaixo se define o layout dos gráficos para serem exibidos lado a lado, e na sequência criamos um gráfico de histograma com base na tabela de frequência \texttt{tf}. O comando \texttt{boxplot()} cria um gráfico de boxplot para os valores em \texttt{brotos\_v}. O argumento ``horizontal = TRUE'' indica que o boxplot deve ser horizontal.

\begin{Shaded}
\begin{Highlighting}[]
\FunctionTok{par}\NormalTok{(}\AttributeTok{mfrow =} \FunctionTok{c}\NormalTok{(}\DecValTok{1}\NormalTok{,}\DecValTok{2}\NormalTok{)) }\CommentTok{\#gráficos lado a lado}
\FunctionTok{plot}\NormalTok{(tf) }\CommentTok{\#distribuição de frequências}
\FunctionTok{boxplot}\NormalTok{(brotos\_sub, }\AttributeTok{horizontal =} \ConstantTok{TRUE}\NormalTok{,}
        \AttributeTok{xlab=}\StringTok{"Class limits"}\NormalTok{) }\CommentTok{\#boxplot}
\end{Highlighting}
\end{Shaded}

\includegraphics{fundamentos_de_ecologia_numerica_files/figure-latex/unnamed-chunk-72-1.pdf}

\begin{Shaded}
\begin{Highlighting}[]
\FunctionTok{par}\NormalTok{(}\AttributeTok{mfrow =} \FunctionTok{c}\NormalTok{(}\DecValTok{1}\NormalTok{,}\DecValTok{1}\NormalTok{)) }\CommentTok{\#gráficos de volta ao normal}
\end{Highlighting}
\end{Shaded}

Se você recebeu a mensagem de erro {``Error in plot.new() : figure margins too large''}, aumente o tamanho da janela do gráfico e execute as últimas três linhas de comando novamente.

\hypertarget{sumuxe1rio-estatuxedstico-geral}{%
\section{Sumário estatístico geral}\label{sumuxe1rio-estatuxedstico-geral}}

\begin{Shaded}
\begin{Highlighting}[]
\FunctionTok{summary}\NormalTok{(brotos\_v)}
\CommentTok{\#?summary}
\FunctionTok{sd}\NormalTok{(brotos\_v) }\CommentTok{\#desvio padrão}
\FunctionTok{var}\NormalTok{(brotos\_v) }\CommentTok{\#variância}
\end{Highlighting}
\end{Shaded}

\begin{verbatim}
##    Min. 1st Qu.  Median    Mean 3rd Qu.    Max. 
##   10.00   24.70   29.10   28.97   33.23   44.90 
## [1] 6.373581
## [1] 40.62254
\end{verbatim}

\hypertarget{apuxeandices-5}{%
\section*{Apêndices}\label{apuxeandices-5}}
\addcontentsline{toc}{section}{Apêndices}

\hypertarget{sites-consultados-3}{%
\section*{Sites consultados}\label{sites-consultados-3}}
\addcontentsline{toc}{section}{Sites consultados}

\hypertarget{script-limpo-2}{%
\section*{Script limpo}\label{script-limpo-2}}
\addcontentsline{toc}{section}{Script limpo}

Aqui apresento o scrip na íntegra sem os textos ou outros comentários. Você pode copiar e colar no R para executa-lo. Lembre de remover os {\texttt{\#}} ou {\texttt{\#\#}} caso necessite executar essas linhas.

\begin{Shaded}
\begin{Highlighting}[]
\DocumentationTok{\#\# rm(list=ls(all=TRUE)) \#limpa a memória}
\DocumentationTok{\#\# install.packages("openxlsx") \#importa arquivos do excel}
\DocumentationTok{\#\# install.packages("fdth")}
\FunctionTok{library}\NormalTok{(openxlsx)}
\DocumentationTok{\#\# getwd()}
\DocumentationTok{\#\# setwd("C:/Seu/Diretório/De/Trabalho")}
\FunctionTok{library}\NormalTok{(openxlsx)}
\NormalTok{brotos }\OtherTok{\textless{}{-}} \FunctionTok{read.xlsx}\NormalTok{(}\StringTok{"D:/Elvio/OneDrive/Disciplinas/\_EcoNumerica/5.Matrizes/brotos.xlsx"}\NormalTok{,}
                    \AttributeTok{rowNames =}\NormalTok{ F, }\AttributeTok{colNames =}\NormalTok{ F,}
                    \AttributeTok{sheet =} \StringTok{"Planilha1"}\NormalTok{)}
\FunctionTok{head}\NormalTok{(brotos,}\DecValTok{10}\NormalTok{)}
\FunctionTok{head}\NormalTok{(brotos[, }\DecValTok{1}\SpecialCharTok{:}\DecValTok{5}\NormalTok{], }\DecValTok{10}\NormalTok{)}
\DocumentationTok{\#\# \#View(brotos)}
\DocumentationTok{\#\# print(brotos[1:5,1:5])}
\DocumentationTok{\#\# brotos}
\DocumentationTok{\#\# str(brotos)}
\DocumentationTok{\#\# mode(brotos)}
\DocumentationTok{\#\# class(brotos)}
\NormalTok{brotos\_v }\OtherTok{\textless{}{-}} \FunctionTok{c}\NormalTok{(}\FunctionTok{t}\NormalTok{(brotos))}
\NormalTok{brotos\_v}
\DocumentationTok{\#\# \#View(brotosv)}
\DocumentationTok{\#\# print(brotos\_v)}
\DocumentationTok{\#\# brotos\_v}
\DocumentationTok{\#\# str(brotos\_v)}
\DocumentationTok{\#\# mode(brotos\_v)}
\DocumentationTok{\#\# class(brotos\_v)}
\FunctionTok{set.seed}\NormalTok{(}\DecValTok{666}\NormalTok{)}
\NormalTok{brotos\_sub }\OtherTok{\textless{}{-}} \FunctionTok{sample}\NormalTok{(brotos\_v, }\AttributeTok{size =} \DecValTok{500}\NormalTok{, }\AttributeTok{replace =}\NormalTok{ F) }\CommentTok{\#atualize o valor de \textquotesingle{}size=\textquotesingle{} se necessário}
\NormalTok{brotos\_sub}
\FunctionTok{library}\NormalTok{(fdth)}
\FunctionTok{range}\NormalTok{(brotos\_sub) }\CommentTok{\#retorna o valor máximo e mínimo}
\CommentTok{\#?range}
\NormalTok{tf }\OtherTok{\textless{}{-}} \FunctionTok{fdt}\NormalTok{(brotos\_sub, }\AttributeTok{start=}\DecValTok{10}\NormalTok{, }\AttributeTok{end=}\DecValTok{45}\NormalTok{, }\AttributeTok{h=}\DecValTok{5}\NormalTok{) }\CommentTok{\#tabela de frequência}
\CommentTok{\#?fdt \#atente para o uso de k, por exemplo tf \textless{}{-} fdt(brotosv, k=5)}
\FunctionTok{print}\NormalTok{(tf)}
\FunctionTok{par}\NormalTok{(}\AttributeTok{mfrow =} \FunctionTok{c}\NormalTok{(}\DecValTok{1}\NormalTok{,}\DecValTok{2}\NormalTok{)) }\CommentTok{\#gráficos lado a lado}
\FunctionTok{plot}\NormalTok{(tf) }\CommentTok{\#distribuição de frequências}
\FunctionTok{boxplot}\NormalTok{(brotos\_sub, }\AttributeTok{horizontal =} \ConstantTok{TRUE}\NormalTok{,}
        \AttributeTok{xlab=}\StringTok{"Class limits"}\NormalTok{) }\CommentTok{\#boxplot}
\FunctionTok{par}\NormalTok{(}\AttributeTok{mfrow =} \FunctionTok{c}\NormalTok{(}\DecValTok{1}\NormalTok{,}\DecValTok{1}\NormalTok{)) }\CommentTok{\#gráficos de volta ao normal}
\FunctionTok{summary}\NormalTok{(brotos\_v)}
\CommentTok{\#?summary}
\FunctionTok{sd}\NormalTok{(brotos\_v) }\CommentTok{\#desvio padrão}
\FunctionTok{var}\NormalTok{(brotos\_v) }\CommentTok{\#variância}
\end{Highlighting}
\end{Shaded}

\hypertarget{referuxeancias-6}{%
\section*{Referências}\label{referuxeancias-6}}
\addcontentsline{toc}{section}{Referências}

\hypertarget{univ2}{%
\chapter{R Modulo 3.2 - Testes Estatísticos}\label{univ2}}

\hypertarget{resumo-6}{%
\subsection*{RESUMO}\label{resumo-6}}
\addcontentsline{toc}{subsection}{RESUMO}

A estatística descritiva tem um papel importante a desempenhar na ciência. Quando problemas específicos são tratados na ciência, os dados precisam ser coletados, analisados e apresentados de forma concisa para que outros possam se beneficiar do que foi encontrado.

\hypertarget{apresentauxe7uxe3o-6}{%
\subsection*{Apresentação}\label{apresentauxe7uxe3o-6}}
\addcontentsline{toc}{subsection}{Apresentação}

A estatística descritiva tem um papel importante a desempenhar na ciência. Quando problemas específicos são tratados na ciência, os dados precisam ser coletados, analisados e apresentados de forma concisa para que outros possam se beneficiar do que foi encontrado. Geralmente não é possível apresentar um conjunto de dados completo em uma publicação ou em um seminário e, mesmo que fosse, é improvável isso permitisse uma boa comunicação dos resultados da pesquisa. Em vez disso, os dados são geralmente resumidos como tabelas de frequência, histogramas e estatísticas descritivas que os leitores ou ouvintes podem assimilar prontamente, mas que ainda transmitem os elementos essenciais do conjunto de dados original. O principal objetivo do cálculo das estatísticas descritivas é transmitir informações essenciais contidas em um conjunto de dados da forma mais concisa e clara possível.

\hypertarget{sobre-os-dados-1}{%
\section{Sobre os dados}\label{sobre-os-dados-1}}

Considere os dados merísticos (ou médições morfológicas) da espécie de peixe \emph{Cichla ocellaris} (tucunaré amarelo) do reservatório da barragem de Gramame, PB (\protect\hyperlink{ref-RN146}{BRADSTOCK; TOZER; KEITH, 1997}) (Figura \ref{fig:rmd32tucu}). Existem 421 medições do comprimemto total (CT), comprimento padrão (CP) e peso total (PT), além do sexo (MACHO, FÊMEA ou imaturo), e outros descritores da estrutura populacional da pespécie, um conjunto de dados formidável.

\begin{figure}
\includegraphics[width=7.24in]{imagens/tucuna} \caption{Dados merísticos da espécie de peixe *Cichla ocellaris* (tucunaré amarelo) do reservatório da barragem  de Gramame, PB.}\label{fig:rmd232tucu}
\end{figure}

\hypertarget{organizauxe7uxe3o-buxe1sica-3}{%
\section{Organização básica}\label{organizauxe7uxe3o-buxe1sica-3}}

\begin{Shaded}
\begin{Highlighting}[]
\FunctionTok{dev.off}\NormalTok{() }\CommentTok{\#apaga os graficos, se houver algum}
\FunctionTok{rm}\NormalTok{(}\AttributeTok{list=}\FunctionTok{ls}\NormalTok{(}\AttributeTok{all=}\ConstantTok{TRUE}\NormalTok{)) }\CommentTok{\#limpa a memória}
\FunctionTok{cat}\NormalTok{(}\StringTok{"}\SpecialCharTok{\textbackslash{}014}\StringTok{"}\NormalTok{) }\CommentTok{\#limpa o console }
\end{Highlighting}
\end{Shaded}

Instalando os pacotes necessários para esse módulo

\begin{Shaded}
\begin{Highlighting}[]
\FunctionTok{install.packages}\NormalTok{(}\StringTok{"openxlsx"}\NormalTok{) }\CommentTok{\#importa arquivos do excel}
\FunctionTok{install.packages}\NormalTok{(}\StringTok{"fdth"}\NormalTok{)}
\FunctionTok{install.packages}\NormalTok{(}\StringTok{"ggpubr"}\NormalTok{)}
\FunctionTok{install.packages}\NormalTok{(}\StringTok{"multcomp"}\NormalTok{)}
\end{Highlighting}
\end{Shaded}

\begin{Shaded}
\begin{Highlighting}[]
\FunctionTok{library}\NormalTok{(openxlsx)}
\end{Highlighting}
\end{Shaded}

Os códigos acima, são usados para instalar e carregar os pacotes necessários para este módulo. Esses códigos são comandos para instalar pacotes no R. Um pacote é uma coleção de funções, dados e documentação que ampliam as capacidades do R (\href{https://cran.r-project.org/}{R CRAN} (\protect\hyperlink{ref-RN2774}{TEAM, 2017}) e \href{https://posit.co/download/rstudio-desktop/}{RStudio} (\protect\hyperlink{ref-RN358}{TEAM, 2022})). No exemplo acima, o pacote \texttt{openxlsx} permite ler e escrever arquivos Excel no R. Para instalar um pacote no R, você precisa usar a função \texttt{install.packages()}.

Depois de instalar um pacote, você precisa carregá-lo na sua sessão R com a função \texttt{library()}. Por exemplo, para carregar o pacote \texttt{openxlsx}, você precisa executar a função \texttt{library(openxlsx)}. Isso irá permitir que você use as funções do pacote na sua sessão R. Você precisa carregar um pacote toda vez que iniciar uma nova sessão R e quiser usar um pacote instalado.

Agora vamos \textbf{definir o diretório de trabalho}. Esse código é usado para obter e definir o diretório de trabalho atual no R. O comando \texttt{getwd()} retorna o caminho do diretório onde o R está lendo e salvando arquivos. O comando \texttt{setwd()} muda esse diretório de trabalho para o caminho especificado entre aspas. No seu caso, você deve ajustar o caminho para o seu próprio diretório de trabalho. \textbf{Lembre de usar a barra ``/'' entre os diretórios. E não a contra-barra ``\textbackslash{}''.}

\begin{Shaded}
\begin{Highlighting}[]
\FunctionTok{getwd}\NormalTok{()}
\FunctionTok{setwd}\NormalTok{(}\StringTok{"C:/Seu/Diretório/De/Trabalho"}\NormalTok{)}
\end{Highlighting}
\end{Shaded}

\hypertarget{importando-a-planilha-3}{%
\section{Importando a planilha}\label{importando-a-planilha-3}}

Vamos importar a planilha de dados univariados \texttt{univ*.xlsx}. Note que o símbolo {\texttt{\#}} em programação R significa que o texto que vem depois dele é um comentário e não será executado pelo programa. Isso é útil para explicar o código ou deixar anotações. Ajuste a segunda linha do código abaixo para refletir {``C:/Seu/Diretório/De/Trabalho/Planilha.xlsx''}.

\begin{Shaded}
\begin{Highlighting}[]
\FunctionTok{library}\NormalTok{(openxlsx)}
\NormalTok{univ }\OtherTok{\textless{}{-}} \FunctionTok{read.xlsx}\NormalTok{(}\StringTok{"D:/Elvio/OneDrive/Disciplinas/\_EcoNumerica/5.Matrizes/univ.xlsx"}\NormalTok{,}
                    \AttributeTok{rowNames =}\NormalTok{ F, }\AttributeTok{colNames =}\NormalTok{ T,}
                    \AttributeTok{sheet =} \StringTok{"tucuna"}\NormalTok{)}
\FunctionTok{head}\NormalTok{(univ,}\DecValTok{10}\NormalTok{)}
\FunctionTok{head}\NormalTok{(univ[, }\DecValTok{1}\SpecialCharTok{:}\DecValTok{5}\NormalTok{], }\DecValTok{10}\NormalTok{)}
\end{Highlighting}
\end{Shaded}

\begin{verbatim}
##    n.=.434 CT_cm  PT_g CP_cm Ctubo_cm  PC_g      %PT Pest_g Cest_cm gr_est
## 1    TU001  32.4 468.8  27.2     39.8 458.9 2.111775    3.9     7.7      I
## 2    TU002  33.4 520.0  28.8     14.3 507.4 2.423077    5.9    10.0      I
## 3    TU003  27.3 301.5  23.8     13.0 283.4 6.003317   15.5    10.2    III
## 4    TU004  13.2  28.2  11.0     16.5  27.7 1.773050    0.3     4.3      I
## 5    TU005  14.3  38.9  11.9     15.5  37.7 3.084833    0.8     4.5    III
## 6    TU006  22.7 431.7  20.5     24.2 418.7 3.011350    9.9     6.4    III
## 7    TU007  23.2 544.0  19.2     25.0 520.1 4.393382    6.8     7.5     II
## 8    TU008  13.5 161.6  11.5     17.5 157.0 2.846535    4.1     6.8     II
## 9    TU009  24.6 200.5  20.5     25.7 195.9 2.294264    2.5     7.6     II
## 10   TU010  19.4  86.7  16.0     18.3  84.5 2.537486    0.7     5.1      I
##       ir_est Pint_g Cint_cm gr_int    ir_int Pgon_g Cgon_cm          emg
## 1  0.8319113    4.8    38.3     II 1.0238908    1.2       7      IMATURO
## 2  1.1346154    5.3    13.3     II 1.0192308    1.4     6.5       MADURO
## 3  5.1409619    2.4    12.0     II 0.7960199    0.2     7.7 EM MATURACAO
## 4  1.0638298    0.1    16.0     II 0.3546099    0.1       5      IMATURO
## 5  2.0565553    0.3    15.0     II 0.7712082    0.1     3.2      IMATURO
## 6  2.2932592    2.7    23.7     II 0.6254343    0.4     7.8 EM MATURACAO
## 7  1.2500000    3.3    24.5     II 0.6066176   13.8     7.9       MADURO
## 8  2.5371287    0.4    17.0      I 0.2475248    0.1     2.5      IMATURO
## 9  1.2468828    2.0    25.3     II 0.9975062    0.1     6.5 EM MATURACAO
## 10 0.8073818    1.4    17.7     II 1.6147636    0.1       6      IMATURO
##            ig mes periodo estação    sexo
## 1  0.25597270 ago chuvoso inverno   MACHO
## 2  0.26923077 ago chuvoso inverno   MACHO
## 3  0.06633499 ago chuvoso inverno   MACHO
## 4  0.35460993 ago chuvoso inverno   MACHO
## 5  0.25706941 ago chuvoso inverno   MACHO
## 6  0.09265694 set chuvoso inverno   MACHO
## 7  2.53676471 set chuvoso inverno   FEMEA
## 8  0.06188119 set chuvoso inverno imaturo
## 9  0.04987531 set chuvoso inverno   MACHO
## 10 0.11534025 set chuvoso inverno   MACHO
##    n.=.434 CT_cm  PT_g CP_cm Ctubo_cm
## 1    TU001  32.4 468.8  27.2     39.8
## 2    TU002  33.4 520.0  28.8     14.3
## 3    TU003  27.3 301.5  23.8     13.0
## 4    TU004  13.2  28.2  11.0     16.5
## 5    TU005  14.3  38.9  11.9     15.5
## 6    TU006  22.7 431.7  20.5     24.2
## 7    TU007  23.2 544.0  19.2     25.0
## 8    TU008  13.5 161.6  11.5     17.5
## 9    TU009  24.6 200.5  20.5     25.7
## 10   TU010  19.4  86.7  16.0     18.3
\end{verbatim}

Exibindo os dados importados (esses comando são ``case-sensitive'' \texttt{ignore.case(object)}).

\begin{Shaded}
\begin{Highlighting}[]
\CommentTok{\#View(univ)}
\FunctionTok{print}\NormalTok{(univ[}\DecValTok{1}\SpecialCharTok{:}\DecValTok{5}\NormalTok{,}\DecValTok{1}\SpecialCharTok{:}\DecValTok{5}\NormalTok{])}
\NormalTok{univ}
\FunctionTok{str}\NormalTok{(univ)}
\FunctionTok{mode}\NormalTok{(univ)}
\FunctionTok{class}\NormalTok{(univ)}
\end{Highlighting}
\end{Shaded}

Vamos escolher uma coluna como a variável de interesse para trabalhar com ela. No código abaixo, essa coluna é descrita pelo seu nome apresentado depois do \texttt{\$}. Antes do \texttt{\$} especificamos e qual data frame está a variável. Depois disso, a convertemos para um vertor.

\begin{Shaded}
\begin{Highlighting}[]
\NormalTok{var }\OtherTok{\textless{}{-}}\NormalTok{ univ}\SpecialCharTok{$}\NormalTok{CP\_cm}
\NormalTok{var\_v }\OtherTok{\textless{}{-}} \FunctionTok{as.vector}\NormalTok{(var)}
\FunctionTok{range}\NormalTok{(var\_v)}
\end{Highlighting}
\end{Shaded}

\begin{verbatim}
## [1]  3.5 36.1
\end{verbatim}

E agora visualizando nossos dados.

\begin{Shaded}
\begin{Highlighting}[]
\CommentTok{\#View(var)}
\FunctionTok{print}\NormalTok{(var\_v)}
\NormalTok{var\_v}
\FunctionTok{str}\NormalTok{(var\_v)}
\FunctionTok{mode}\NormalTok{(var\_v)}
\FunctionTok{class}\NormalTok{(var\_v)}
\FunctionTok{range}\NormalTok{(var\_v)}
\end{Highlighting}
\end{Shaded}

Por inspeção, o comprimento total mínimo do peixe é 3.5 cm e o máximo é 36.1 cm. Esses valores definem o intervalo da amostra. Agora precisamos subdividir os dados em intervalos ou classes, cada um com o mesmo tamanho. Geralmente, é aconselhável arredondar o valor mínimo para baixo e o valor máximo para cima, para valores apropriados ao decidir as classes de intervalos. Nesse caso, parece sensato dividir a faixa de \texttt{range(var\_v)} cm em sete intervalos a cada 5 cm de largura. Se contarmos o número de peixes que se encontram em cada um dos sete intervalos, temos a base para a tabulação da frequência.

A coluna de frequência será obtida contando o número de medições que existem dentro de cada classe. A coluna de frequência percentual será obtida representando cada contagem como uma porcentagem da contagem total. A frequência cumulativa e as frequências percentuais cumulativas serão obtidas somando progressivamente as frequências correspondentes.

Tendo em mãos o conjunto total de 434 valores merísticos da variável de interesse agora \textbf{podemos tirar uma subamostra aleatória de uma parte dos 434 valores}. Essa subamostra é tirada usando o comando \texttt{size=} no código subsequente, que estabelece o tamanho da subamostra a ser tirada do total de dados. Esse tamanho é estabelecido no quiz. Inclua esse valor de acordo com o que é pedido no quiz. Por exemplo, se for pedido uma subamostra de 150 comprimentos, então \texttt{size\ =\ 150}. Nesse tutorial usaremos todos os 434 comprimentos. Podemos ainda escolher uma das colunas da base de dados, no caso desse vamos usar a coluna

Subsitua em \texttt{n\ \textless{}-} o valor de \texttt{size=} desejado. Aqui \texttt{n\ \textless{}-} será todo o conjunto de dados \texttt{length(var\_v)}.

\begin{Shaded}
\begin{Highlighting}[]
\NormalTok{n }\OtherTok{\textless{}{-}} \FunctionTok{length}\NormalTok{(var\_v)}
\FunctionTok{set.seed}\NormalTok{(}\DecValTok{666}\NormalTok{)}
\NormalTok{var\_sub }\OtherTok{\textless{}{-}} \FunctionTok{sample}\NormalTok{(var\_v, }\AttributeTok{size =}\NormalTok{ n, }\AttributeTok{replace =}\NormalTok{ F) }\CommentTok{\#atualize o valor de \textquotesingle{}size=\textquotesingle{} se necessário}
\CommentTok{\#OU}
\CommentTok{\#var\_sub \textless{}{-} univ[univ$CP\_cm \textgreater{}0 \& univ$CP\_cm \textless{}20,] \#data.frame}
\CommentTok{\#var\_sub \textless{}{-} var\_v[var\_v\textgreater{}=5 \& var\_v \textless{}=15] \#vector}
\end{Highlighting}
\end{Shaded}

Para fazermos uma tabela de frequência dos valores merísticos carregamos o pacote \texttt{fdth} e pedimos a função \texttt{range} que retorna o valor máximo e mínimo no vetor.

\begin{Shaded}
\begin{Highlighting}[]
\FunctionTok{library}\NormalTok{(fdth)}
\FunctionTok{range}\NormalTok{(var\_sub) }\CommentTok{\#retorna o valor máximo e mínimo}
\CommentTok{\#?range}
\end{Highlighting}
\end{Shaded}

\begin{verbatim}
## [1]  3.5 36.1
\end{verbatim}

Agora é necessário que você substitua os valores de range 3.5, 36.1 nos valores de início e fim da distribuição de frequência. Os comandos abaixo criam uma tabela de frequência chamada \texttt{tf} com valores máximos e mínimos definidos por \texttt{range(var\_v)} em intervalor definidos por \texttt{h=5}, e o comando \texttt{print(tf)} exibe a tabela de frequência.

\begin{Shaded}
\begin{Highlighting}[]
\NormalTok{tf }\OtherTok{\textless{}{-}} \FunctionTok{fdt}\NormalTok{(var\_sub, }\AttributeTok{start=}\DecValTok{0}\NormalTok{, }\AttributeTok{end=}\DecValTok{40}\NormalTok{, }\AttributeTok{h=}\DecValTok{5}\NormalTok{) }\CommentTok{\#tabela de frequência}
\CommentTok{\#?fdt \#atente para o uso de k, por exemplo tf \textless{}{-} fdt(var\_sub, k=5)}
\FunctionTok{print}\NormalTok{(tf)}
\end{Highlighting}
\end{Shaded}

\begin{verbatim}
##  Class limits   f   rf rf(%)  cf  cf(%)
##         [0,5)  57 0.13 13.13  57  13.13
##        [5,10) 154 0.35 35.48 211  48.62
##       [10,15) 109 0.25 25.12 320  73.73
##       [15,20)  38 0.09  8.76 358  82.49
##       [20,25)  33 0.08  7.60 391  90.09
##       [25,30)  23 0.05  5.30 414  95.39
##       [30,35)  15 0.03  3.46 429  98.85
##       [35,40)   5 0.01  1.15 434 100.00
\end{verbatim}

Class limits = limites de classe, f = frequência de classe, rf = frequência relativa da classe, rf(\%) = frequência relativa percentual da classe, cf = frequência cumulativa da classe, cf(\%) = frequência cumulativa percentual da classe.

Se contarmos o número de brotos que se encontram em cada um dos sete intervalos, temos a base para a tabulação da frequência. Essa tabulação é mostrada na tabela gerada pelos códigos acima. A coluna de frequência foi obtida contando o número de medições que existem dentro de cada classe. A coluna de frequência percentual foi obtida representando cada contagem como uma porcentagem da contagem total. A frequência cumulativa e as frequências percentuais cumulativas foram obtidas somando progressivamente as frequências correspondentes.

\hypertarget{gruxe1ficos-de-histograma-e-boxplot-1}{%
\section{Gráficos de histograma e boxplot}\label{gruxe1ficos-de-histograma-e-boxplot-1}}

No código abaixo se define o layout dos gráficos para serem exibidos lado a lado, e na sequência criamos um gráfico de histograma com base na tabela de frequência \texttt{tf}. O comando \texttt{boxplot()} cria um gráfico de boxplot para os valores em \texttt{var\_v}. O argumento ``horizontal = TRUE'' indica que o boxplot deve ser horizontal.

\begin{Shaded}
\begin{Highlighting}[]
\FunctionTok{par}\NormalTok{(}\AttributeTok{mfrow =} \FunctionTok{c}\NormalTok{(}\DecValTok{1}\NormalTok{,}\DecValTok{2}\NormalTok{)) }\CommentTok{\#gráficos lado a lado}
\FunctionTok{plot}\NormalTok{(tf) }\CommentTok{\#distribuição de frequências}
\FunctionTok{boxplot}\NormalTok{(var\_sub, }\AttributeTok{horizontal =} \ConstantTok{TRUE}\NormalTok{,}
        \AttributeTok{xlab=}\StringTok{"Class limits"}\NormalTok{) }\CommentTok{\#boxplot}
\end{Highlighting}
\end{Shaded}

\includegraphics{fundamentos_de_ecologia_numerica_files/figure-latex/unnamed-chunk-87-1.pdf}

\begin{Shaded}
\begin{Highlighting}[]
\FunctionTok{par}\NormalTok{(}\AttributeTok{mfrow =} \FunctionTok{c}\NormalTok{(}\DecValTok{1}\NormalTok{,}\DecValTok{1}\NormalTok{)) }\CommentTok{\#gráficos de volta ao normal}
\end{Highlighting}
\end{Shaded}

\leavevmode\vadjust pre{\hypertarget{hello}{}}%
{IMPORTANTE}
Se você recebeu a mensagem de erro {``Error in plot.new() : figure margins too large''}, aumente o tamanho da janela do gráfico e execute as últimas três linhas de comando novamente.

\hypertarget{sumuxe1rio-estatuxedstico-geral-1}{%
\section{Sumário estatístico geral}\label{sumuxe1rio-estatuxedstico-geral-1}}

\begin{Shaded}
\begin{Highlighting}[]
\FunctionTok{summary}\NormalTok{(var\_sub)}
\CommentTok{\#?summary}
\FunctionTok{sd}\NormalTok{(var\_sub) }\CommentTok{\#desvio padrão}
\FunctionTok{var}\NormalTok{(var\_sub) }\CommentTok{\#variância}
\end{Highlighting}
\end{Shaded}

\begin{verbatim}
##    Min. 1st Qu.  Median    Mean 3rd Qu.    Max. 
##    3.50    6.00   10.10   12.32   15.70   36.10 
## [1] 7.845708
## [1] 61.55514
\end{verbatim}

\hypertarget{testando-normalidade}{%
\section{Testando normalidade}\label{testando-normalidade}}

\hypertarget{q-q-plots}{%
\subsection{Q-Q plots}\label{q-q-plots}}

\begin{Shaded}
\begin{Highlighting}[]
\FunctionTok{par}\NormalTok{(}\AttributeTok{mfrow=}\FunctionTok{c}\NormalTok{(}\DecValTok{3}\NormalTok{,}\DecValTok{1}\NormalTok{))}
\FunctionTok{qqnorm}\NormalTok{(var\_sub, }\AttributeTok{main=}\StringTok{\textquotesingle{}Normal\textquotesingle{}}\NormalTok{)}
\FunctionTok{qqline}\NormalTok{(var\_sub)}
\FunctionTok{boxplot}\NormalTok{(var\_sub, }\AttributeTok{horizontal =} \ConstantTok{TRUE}\NormalTok{,}
        \AttributeTok{xlab=}\StringTok{"Class limits"}\NormalTok{) }\CommentTok{\#boxplot}
\FunctionTok{hist}\NormalTok{(var\_sub)}
\end{Highlighting}
\end{Shaded}

\includegraphics{fundamentos_de_ecologia_numerica_files/figure-latex/unnamed-chunk-89-1.pdf}

\begin{Shaded}
\begin{Highlighting}[]
\FunctionTok{par}\NormalTok{(}\AttributeTok{mfrow=}\FunctionTok{c}\NormalTok{(}\DecValTok{1}\NormalTok{,}\DecValTok{1}\NormalTok{))}
\FunctionTok{library}\NormalTok{(ggpubr)}
\end{Highlighting}
\end{Shaded}

\begin{verbatim}
## Warning: package 'ggpubr' was built under R version 4.3.3
\end{verbatim}

\begin{verbatim}
## Carregando pacotes exigidos: ggplot2
\end{verbatim}

\begin{verbatim}
## Warning: package 'ggplot2' was built under R version 4.3.3
\end{verbatim}

\begin{Shaded}
\begin{Highlighting}[]
\FunctionTok{ggdensity}\NormalTok{(var\_sub)}
\end{Highlighting}
\end{Shaded}

\includegraphics{fundamentos_de_ecologia_numerica_files/figure-latex/unnamed-chunk-89-2.pdf}

\begin{Shaded}
\begin{Highlighting}[]
\FunctionTok{ggqqplot}\NormalTok{(var\_sub)}
\end{Highlighting}
\end{Shaded}

\includegraphics{fundamentos_de_ecologia_numerica_files/figure-latex/unnamed-chunk-89-3.pdf}

\begin{Shaded}
\begin{Highlighting}[]
\FunctionTok{library}\NormalTok{(car)}
\end{Highlighting}
\end{Shaded}

\begin{verbatim}
## Carregando pacotes exigidos: carData
\end{verbatim}

\begin{Shaded}
\begin{Highlighting}[]
\FunctionTok{qqPlot}\NormalTok{(var\_sub)}
\end{Highlighting}
\end{Shaded}

\includegraphics{fundamentos_de_ecologia_numerica_files/figure-latex/unnamed-chunk-89-4.pdf}

\begin{verbatim}
## [1] 265 434
\end{verbatim}

\hypertarget{testes-de-shapiro-wilk-e-kolmogorov-smirnov}{%
\subsection{Testes de Shapiro-Wilk e Kolmogorov-Smirnov}\label{testes-de-shapiro-wilk-e-kolmogorov-smirnov}}

\begin{Shaded}
\begin{Highlighting}[]
\NormalTok{shap }\OtherTok{\textless{}{-}} \FunctionTok{shapiro.test}\NormalTok{(var\_sub)}
\NormalTok{shap}
\NormalTok{p }\OtherTok{\textless{}{-}} \FunctionTok{format}\NormalTok{(shap}\SpecialCharTok{$}\NormalTok{p.value, }\AttributeTok{scientific =} \ConstantTok{FALSE}\NormalTok{)}
\NormalTok{p}
\FunctionTok{ks.test}\NormalTok{(var\_sub, }\StringTok{"pnorm"}\NormalTok{)}
\end{Highlighting}
\end{Shaded}

\begin{verbatim}
## Warning in ks.test.default(var_sub, "pnorm"): ties should not be present for
## the Kolmogorov-Smirnov test
\end{verbatim}

\begin{verbatim}
## 
##  Shapiro-Wilk normality test
## 
## data:  var_sub
## W = 0.86837, p-value < 2.2e-16
## 
## [1] "0.00000000000000000107347"
## 
##  Asymptotic one-sample Kolmogorov-Smirnov test
## 
## data:  var_sub
## D = 0.99977, p-value < 2.2e-16
## alternative hypothesis: two-sided
\end{verbatim}

\hypertarget{testando-diferenuxe7as-entre-muxe9dias}{%
\section{Testando diferenças entre médias}\label{testando-diferenuxe7as-entre-muxe9dias}}

\hypertarget{testando-homogeneidade-de-variuxe2ncias}{%
\subsection{Testando homogeneidade de variâncias}\label{testando-homogeneidade-de-variuxe2ncias}}

\begin{Shaded}
\begin{Highlighting}[]
\CommentTok{\# F{-}test}
\CommentTok{\#var.test(CT\_cm \textasciitilde{} sexo, data = univ) com erro}
\CommentTok{\#Interpretação: Um valor de p maior que o nível de significância de 0.05 significa que, NÃO HÁ diferença significativa entre as duas variâncias.}

\CommentTok{\# Levene}
\FunctionTok{library}\NormalTok{(car)}
\CommentTok{\#univ$sexo \textless{}{-} as.factor(univ$sexo) \#evita o Warning de "group coerced to factor"  }
\NormalTok{lev }\OtherTok{\textless{}{-}} \FunctionTok{leveneTest}\NormalTok{(CT\_cm }\SpecialCharTok{\textasciitilde{}}\NormalTok{ sexo, }\AttributeTok{data =}\NormalTok{ univ)}
\end{Highlighting}
\end{Shaded}

\begin{verbatim}
## Warning in leveneTest.default(y = y, group = group, ...): group coerced to
## factor.
\end{verbatim}

\begin{Shaded}
\begin{Highlighting}[]
\NormalTok{lev}

\CommentTok{\# Teste de Levene entre dois de tres grupos}
\NormalTok{machos\_CT\_cm }\OtherTok{\textless{}{-}} \FunctionTok{na.omit}\NormalTok{(univ}\SpecialCharTok{$}\NormalTok{CT\_cm[univ}\SpecialCharTok{$}\NormalTok{sexo }\SpecialCharTok{==} \StringTok{"MACHO"}\NormalTok{])}
\NormalTok{femeas\_CT\_cm }\OtherTok{\textless{}{-}} \FunctionTok{na.omit}\NormalTok{(univ}\SpecialCharTok{$}\NormalTok{CT\_cm[univ}\SpecialCharTok{$}\NormalTok{sexo }\SpecialCharTok{==} \StringTok{"FEMEA"}\NormalTok{])}
\NormalTok{imat\_CT\_cm }\OtherTok{\textless{}{-}} \FunctionTok{na.omit}\NormalTok{(univ}\SpecialCharTok{$}\NormalTok{CT\_cm[univ}\SpecialCharTok{$}\NormalTok{sexo }\SpecialCharTok{==} \StringTok{"imaturo"}\NormalTok{])}
\NormalTok{lev2 }\OtherTok{\textless{}{-}} \FunctionTok{leveneTest}\NormalTok{(CT\_cm }\SpecialCharTok{\textasciitilde{}}\NormalTok{ sexo, }\AttributeTok{data =}\NormalTok{ univ[univ}\SpecialCharTok{$}\NormalTok{sexo }\SpecialCharTok{\%in\%} \FunctionTok{c}\NormalTok{(}\StringTok{"MACHO"}\NormalTok{, }\StringTok{"FEMEA"}\NormalTok{), ])}
\end{Highlighting}
\end{Shaded}

\begin{verbatim}
## Warning in leveneTest.default(y = y, group = group, ...): group coerced to
## factor.
\end{verbatim}

\begin{Shaded}
\begin{Highlighting}[]
\NormalTok{lev2}
\CommentTok{\#Interpretação: Um valor de p maior que o nível de significância de 0.05 significa que, a hipótese nula é mantida e NÃO HÁ diferença significativa entre as variâncias.}
\NormalTok{univ}\SpecialCharTok{$}\NormalTok{sexo }\OtherTok{\textless{}{-}} \FunctionTok{factor}\NormalTok{(univ}\SpecialCharTok{$}\NormalTok{sexo, }\AttributeTok{levels =} \FunctionTok{c}\NormalTok{(}\StringTok{"MACHO"}\NormalTok{, }\StringTok{"FEMEA"}\NormalTok{, }\StringTok{"imaturo"}\NormalTok{))}
\FunctionTok{boxplot}\NormalTok{(CT\_cm }\SpecialCharTok{\textasciitilde{}}\NormalTok{ sexo, }\AttributeTok{data =}\NormalTok{ univ) }
\end{Highlighting}
\end{Shaded}

\includegraphics{fundamentos_de_ecologia_numerica_files/figure-latex/unnamed-chunk-91-1.pdf}

\begin{verbatim}
## Levene's Test for Homogeneity of Variance (center = median)
##        Df F value    Pr(>F)    
## group   2  45.395 < 2.2e-16 ***
##       431                      
## ---
## Signif. codes:  0 '***' 0.001 '**' 0.01 '*' 0.05 '.' 0.1 ' ' 1
## Levene's Test for Homogeneity of Variance (center = median)
##        Df F value   Pr(>F)   
## group   1  10.826 0.001173 **
##       211                    
## ---
## Signif. codes:  0 '***' 0.001 '**' 0.01 '*' 0.05 '.' 0.1 ' ' 1
\end{verbatim}

\hypertarget{teste-entre-duas-muxe9dias-t-student}{%
\subsection{Teste entre duas médias (t-Student)}\label{teste-entre-duas-muxe9dias-t-student}}

\begin{Shaded}
\begin{Highlighting}[]
\FunctionTok{t.test}\NormalTok{(machos\_CT\_cm, femeas\_CT\_cm,}
      \AttributeTok{alternative =} \FunctionTok{c}\NormalTok{(}\StringTok{"two.sided"}\NormalTok{, }\StringTok{"less"}\NormalTok{, }\StringTok{"greater"}\NormalTok{),}
      \AttributeTok{mu =} \DecValTok{0}\NormalTok{, }\AttributeTok{paired =} \ConstantTok{FALSE}\NormalTok{, }\AttributeTok{var.equal =} \ConstantTok{FALSE}\NormalTok{,}
      \AttributeTok{conf.level =} \FloatTok{0.95}\NormalTok{)}
\end{Highlighting}
\end{Shaded}

\begin{verbatim}
## 
##  Welch Two Sample t-test
## 
## data:  machos_CT_cm and femeas_CT_cm
## t = 3.4342, df = 210.89, p-value = 0.0007155
## alternative hypothesis: true difference in means is not equal to 0
## 95 percent confidence interval:
##  1.747453 6.456834
## sample estimates:
## mean of x mean of y 
##  22.70984  18.60769
\end{verbatim}

\hypertarget{teste-entre-truxeas-muxe9dias-anova}{%
\subsection{Teste entre três médias (ANOVA)}\label{teste-entre-truxeas-muxe9dias-anova}}

\href{https://www.r-bloggers.com/2022/05/one-way-anova-example-in-r-quick-guide/}{ANOVA}

\begin{Shaded}
\begin{Highlighting}[]
\FunctionTok{levels}\NormalTok{(univ}\SpecialCharTok{$}\NormalTok{sexo)}
\NormalTok{univ}\SpecialCharTok{$}\NormalTok{sexo }\OtherTok{\textless{}{-}} \FunctionTok{ordered}\NormalTok{(univ}\SpecialCharTok{$}\NormalTok{sexo,}
                         \AttributeTok{levels =} \FunctionTok{c}\NormalTok{(}\StringTok{"MACHO"}\NormalTok{, }\StringTok{"FEMEA"}\NormalTok{, }\StringTok{"imaturo"}\NormalTok{))}
\FunctionTok{library}\NormalTok{(dplyr)}
\end{Highlighting}
\end{Shaded}

\begin{verbatim}
## 
## Attaching package: 'dplyr'
\end{verbatim}

\begin{verbatim}
## The following object is masked from 'package:car':
## 
##     recode
\end{verbatim}

\begin{verbatim}
## The following object is masked from 'package:kableExtra':
## 
##     group_rows
\end{verbatim}

\begin{verbatim}
## The following objects are masked from 'package:stats':
## 
##     filter, lag
\end{verbatim}

\begin{verbatim}
## The following objects are masked from 'package:base':
## 
##     intersect, setdiff, setequal, union
\end{verbatim}

\begin{Shaded}
\begin{Highlighting}[]
\FunctionTok{group\_by}\NormalTok{(univ, sexo) }\SpecialCharTok{\%\textgreater{}\%}
  \FunctionTok{summarise}\NormalTok{(}
    \AttributeTok{count =} \FunctionTok{n}\NormalTok{(),}
    \AttributeTok{mean =} \FunctionTok{mean}\NormalTok{(CP\_cm, }\AttributeTok{na.rm =} \ConstantTok{TRUE}\NormalTok{),}
    \AttributeTok{sd =} \FunctionTok{sd}\NormalTok{(CP\_cm, }\AttributeTok{na.rm =} \ConstantTok{TRUE}\NormalTok{)}
\NormalTok{  )}
\CommentTok{\# Conjunto de gráficos}
\FunctionTok{library}\NormalTok{(}\StringTok{"ggpubr"}\NormalTok{)}
\FunctionTok{ggboxplot}\NormalTok{(univ, }\AttributeTok{x =} \StringTok{"sexo"}\NormalTok{, }\AttributeTok{y =} \StringTok{"CP\_cm"}\NormalTok{,}
          \AttributeTok{color =} \StringTok{"sexo"}\NormalTok{, }\AttributeTok{palette =} \FunctionTok{c}\NormalTok{(}\StringTok{"\#00AFBB"}\NormalTok{, }\StringTok{"\#E7B800"}\NormalTok{, }\StringTok{"\#FC4E07"}\NormalTok{),}
          \AttributeTok{order =} \FunctionTok{c}\NormalTok{(}\StringTok{"MACHO"}\NormalTok{, }\StringTok{"FEMEA"}\NormalTok{, }\StringTok{"imaturo"}\NormalTok{),}
          \AttributeTok{ylab =} \StringTok{"CP\_cm"}\NormalTok{, }\AttributeTok{xlab =} \StringTok{"Sexo"}\NormalTok{)}
\end{Highlighting}
\end{Shaded}

\includegraphics{fundamentos_de_ecologia_numerica_files/figure-latex/unnamed-chunk-93-1.pdf}

\begin{Shaded}
\begin{Highlighting}[]
\FunctionTok{ggline}\NormalTok{(univ, }\AttributeTok{x =} \StringTok{"sexo"}\NormalTok{, }\AttributeTok{y =} \StringTok{"CP\_cm"}\NormalTok{,}
       \AttributeTok{add =} \FunctionTok{c}\NormalTok{(}\StringTok{"mean\_se"}\NormalTok{, }\StringTok{"jitter"}\NormalTok{),}
       \AttributeTok{order =} \FunctionTok{c}\NormalTok{(}\StringTok{"MACHO"}\NormalTok{, }\StringTok{"FEMEA"}\NormalTok{, }\StringTok{"imaturo"}\NormalTok{),}
       \AttributeTok{ylab =} \StringTok{"Weight"}\NormalTok{, }\AttributeTok{xlab =} \StringTok{"Treatment"}\NormalTok{)}
\end{Highlighting}
\end{Shaded}

\includegraphics{fundamentos_de_ecologia_numerica_files/figure-latex/unnamed-chunk-93-2.pdf}

\begin{Shaded}
\begin{Highlighting}[]
\FunctionTok{boxplot}\NormalTok{(CP\_cm }\SpecialCharTok{\textasciitilde{}}\NormalTok{ sexo, }\AttributeTok{data =}\NormalTok{ univ,}
        \AttributeTok{xlab =} \StringTok{"Sexo"}\NormalTok{, }\AttributeTok{ylab =} \StringTok{"CP\_cm"}\NormalTok{,}
        \AttributeTok{frame =} \ConstantTok{FALSE}\NormalTok{, }\AttributeTok{col =} \FunctionTok{c}\NormalTok{(}\StringTok{"\#00AFBB"}\NormalTok{, }\StringTok{"\#E7B800"}\NormalTok{, }\StringTok{"\#FC4E07"}\NormalTok{))}
\end{Highlighting}
\end{Shaded}

\includegraphics{fundamentos_de_ecologia_numerica_files/figure-latex/unnamed-chunk-93-3.pdf}

\begin{Shaded}
\begin{Highlighting}[]
\FunctionTok{library}\NormalTok{(gplots)}
\end{Highlighting}
\end{Shaded}

\begin{verbatim}
## 
## Attaching package: 'gplots'
\end{verbatim}

\begin{verbatim}
## The following object is masked from 'package:stats':
## 
##     lowess
\end{verbatim}

\begin{Shaded}
\begin{Highlighting}[]
\FunctionTok{plotmeans}\NormalTok{(CP\_cm }\SpecialCharTok{\textasciitilde{}}\NormalTok{ sexo, }\AttributeTok{data =}\NormalTok{ univ, }\AttributeTok{frame =} \ConstantTok{FALSE}\NormalTok{,}
          \AttributeTok{xlab =} \StringTok{"Sexo"}\NormalTok{, }\AttributeTok{ylab =} \StringTok{"CP\_cm"}\NormalTok{,}
          \AttributeTok{main=}\StringTok{"Mean Plot with 95\% CI"}\NormalTok{)}
\end{Highlighting}
\end{Shaded}

\begin{verbatim}
## Warning in arrows(x, li, x, pmax(y - gap, li), col = barcol, lwd = lwd, :
## zero-length arrow is of indeterminate angle and so skipped
\end{verbatim}

\begin{verbatim}
## Warning in arrows(x, ui, x, pmin(y + gap, ui), col = barcol, lwd = lwd, :
## zero-length arrow is of indeterminate angle and so skipped
\end{verbatim}

\begin{verbatim}
## Warning in plot.xy(xy.coords(x, y), type = type, ...): "frame" não é um
## parâmetro gráfico
\end{verbatim}

\begin{verbatim}
## Warning in axis(1, at = 1:length(means), labels = legends, ...): "frame" não é
## um parâmetro gráfico
\end{verbatim}

\begin{verbatim}
## Warning in plot.xy(xy.coords(x, y), type = type, ...): "frame" não é um
## parâmetro gráfico
\end{verbatim}

\includegraphics{fundamentos_de_ecologia_numerica_files/figure-latex/unnamed-chunk-93-4.pdf}

\begin{Shaded}
\begin{Highlighting}[]
\CommentTok{\# ANOVA}
\NormalTok{anova }\OtherTok{\textless{}{-}} \FunctionTok{aov}\NormalTok{(CP\_cm }\SpecialCharTok{\textasciitilde{}}\NormalTok{ sexo, }\AttributeTok{data =}\NormalTok{ univ)}
\FunctionTok{summary}\NormalTok{(anova)}
\CommentTok{\#Interpretação: Um valor de p MENOR que o nível de significância de 0.05 significa que,}
\CommentTok{\#EXISTE diferença significativa entre as três grupos de médias.}

\CommentTok{\# Comparações multiplas}
\FunctionTok{TukeyHSD}\NormalTok{(anova)}
\FunctionTok{library}\NormalTok{(multcomp)}
\end{Highlighting}
\end{Shaded}

\begin{verbatim}
## Warning: package 'multcomp' was built under R version 4.3.3
\end{verbatim}

\begin{verbatim}
## Carregando pacotes exigidos: mvtnorm
\end{verbatim}

\begin{verbatim}
## Carregando pacotes exigidos: survival
\end{verbatim}

\begin{verbatim}
## Carregando pacotes exigidos: TH.data
\end{verbatim}

\begin{verbatim}
## Warning: package 'TH.data' was built under R version 4.3.3
\end{verbatim}

\begin{verbatim}
## Carregando pacotes exigidos: MASS
\end{verbatim}

\begin{verbatim}
## 
## Attaching package: 'MASS'
\end{verbatim}

\begin{verbatim}
## The following object is masked from 'package:dplyr':
## 
##     select
\end{verbatim}

\begin{verbatim}
## 
## Attaching package: 'TH.data'
\end{verbatim}

\begin{verbatim}
## The following object is masked from 'package:MASS':
## 
##     geyser
\end{verbatim}

\begin{Shaded}
\begin{Highlighting}[]
\FunctionTok{library}\NormalTok{(multcomp)}
\FunctionTok{summary}\NormalTok{(}\FunctionTok{glht}\NormalTok{(anova, }\AttributeTok{linfct =} \FunctionTok{mcp}\NormalTok{(}\AttributeTok{sexo =} \StringTok{"Tukey"}\NormalTok{)))}
\CommentTok{\# T{-}test entre pares}
\FunctionTok{pairwise.t.test}\NormalTok{(univ}\SpecialCharTok{$}\NormalTok{CP\_cm, univ}\SpecialCharTok{$}\NormalTok{sexo,}
                 \AttributeTok{p.adjust.method =} \StringTok{"BH"}\NormalTok{)}
\CommentTok{\# Pressupostos}
\DocumentationTok{\#\# Homogeneidade de variâncias }
\FunctionTok{plot}\NormalTok{(anova, }\DecValTok{1}\NormalTok{) }\CommentTok{\#valores numerados são outliers}
\end{Highlighting}
\end{Shaded}

\includegraphics{fundamentos_de_ecologia_numerica_files/figure-latex/unnamed-chunk-93-5.pdf}

\begin{Shaded}
\begin{Highlighting}[]
\FunctionTok{library}\NormalTok{(car)}
\FunctionTok{leveneTest}\NormalTok{(CP\_cm }\SpecialCharTok{\textasciitilde{}}\NormalTok{ sexo, }\AttributeTok{data =}\NormalTok{ univ)}
\CommentTok{\# ANOVA sem o pressuposto de equalidade de variâncias}
\FunctionTok{oneway.test}\NormalTok{(CP\_cm }\SpecialCharTok{\textasciitilde{}}\NormalTok{ sexo, }\AttributeTok{data =}\NormalTok{ univ)}
\CommentTok{\# Testes pareados sem o pressuposto de equalidade de variâncias}
\FunctionTok{pairwise.t.test}\NormalTok{(univ}\SpecialCharTok{$}\NormalTok{CP\_cm, univ}\SpecialCharTok{$}\NormalTok{sexo,}
                \AttributeTok{p.adjust.method =} \StringTok{"BH"}\NormalTok{, }\AttributeTok{pool.sd =} \ConstantTok{FALSE}\NormalTok{)}
\CommentTok{\# Normalidade pelos resíduos (Q{-}Q plot)}
\FunctionTok{plot}\NormalTok{(anova, }\DecValTok{2}\NormalTok{)}
\end{Highlighting}
\end{Shaded}

\includegraphics{fundamentos_de_ecologia_numerica_files/figure-latex/unnamed-chunk-93-6.pdf}

\begin{Shaded}
\begin{Highlighting}[]
\CommentTok{\# Extraindo os resíduos e rodando o Shapiro{-}Wilk neles}
\NormalTok{anova\_residuals }\OtherTok{\textless{}{-}} \FunctionTok{residuals}\NormalTok{(}\AttributeTok{object =}\NormalTok{ anova)}
\FunctionTok{shapiro.test}\NormalTok{(}\AttributeTok{x =}\NormalTok{ anova\_residuals)}
\CommentTok{\# ANOVA não{-}paramétrica (Kruskal{-}Wallis)}
\FunctionTok{kruskal.test}\NormalTok{(CP\_cm }\SpecialCharTok{\textasciitilde{}}\NormalTok{ sexo, }\AttributeTok{data =}\NormalTok{ univ)}
\end{Highlighting}
\end{Shaded}

\begin{verbatim}
## [1] "MACHO"   "FEMEA"   "imaturo"
## # A tibble: 3 x 4
##   sexo    count  mean    sd
##   <ord>   <int> <dbl> <dbl>
## 1 MACHO     122 18.7   8.16
## 2 FEMEA      91 15.3   6.25
## 3 imaturo   221  7.57  4.44
##              Df Sum Sq Mean Sq F value Pr(>F)    
## sexo          2  10743    5371   145.5 <2e-16 ***
## Residuals   431  15911      37                   
## ---
## Signif. codes:  0 '***' 0.001 '**' 0.01 '*' 0.05 '.' 0.1 ' ' 1
##   Tukey multiple comparisons of means
##     95% family-wise confidence level
## 
## Fit: aov(formula = CP_cm ~ sexo, data = univ)
## 
## $sexo
##                     diff        lwr       upr     p adj
## FEMEA-MACHO    -3.424716  -5.403982 -1.445451 0.0001655
## imaturo-MACHO -11.127884 -12.739590 -9.516177 0.0000000
## imaturo-FEMEA  -7.703167  -9.482984 -5.923350 0.0000000
## 
## 
##   Simultaneous Tests for General Linear Hypotheses
## 
## Multiple Comparisons of Means: Tukey Contrasts
## 
## 
## Fit: aov(formula = CP_cm ~ sexo, data = univ)
## 
## Linear Hypotheses:
##                      Estimate Std. Error t value Pr(>|t|)    
## FEMEA - MACHO == 0    -3.4247     0.8416  -4.069 0.000161 ***
## imaturo - MACHO == 0 -11.1279     0.6853 -16.238  < 1e-04 ***
## imaturo - FEMEA == 0  -7.7032     0.7568 -10.179  < 1e-04 ***
## ---
## Signif. codes:  0 '***' 0.001 '**' 0.01 '*' 0.05 '.' 0.1 ' ' 1
## (Adjusted p values reported -- single-step method)
## 
## 
##  Pairwise comparisons using t tests with pooled SD 
## 
## data:  univ$CP_cm and univ$sexo 
## 
##         MACHO   FEMEA  
## FEMEA   5.6e-05 -      
## imaturo < 2e-16 < 2e-16
## 
## P value adjustment method: BH 
## Levene's Test for Homogeneity of Variance (center = median)
##        Df F value    Pr(>F)    
## group   2  45.138 < 2.2e-16 ***
##       431                      
## ---
## Signif. codes:  0 '***' 0.001 '**' 0.01 '*' 0.05 '.' 0.1 ' ' 1
## 
##  One-way analysis of means (not assuming equal variances)
## 
## data:  CP_cm and sexo
## F = 134.26, num df = 2.00, denom df = 180.92, p-value < 2.2e-16
## 
## 
##  Pairwise comparisons using t tests with non-pooled SD 
## 
## data:  univ$CP_cm and univ$sexo 
## 
##         MACHO   FEMEA  
## FEMEA   0.00063 -      
## imaturo < 2e-16 < 2e-16
## 
## P value adjustment method: BH 
## 
##  Shapiro-Wilk normality test
## 
## data:  anova_residuals
## W = 0.93021, p-value = 2.35e-13
## 
## 
##  Kruskal-Wallis rank sum test
## 
## data:  CP_cm by sexo
## Kruskal-Wallis chi-squared = 217.39, df = 2, p-value < 2.2e-16
\end{verbatim}

\hypertarget{teste-entre-truxeas-muxe9dias-e-dois-fatores-two-way-anova}{%
\subsection{Teste entre três médias e dois fatores (Two-Way ANOVA)}\label{teste-entre-truxeas-muxe9dias-e-dois-fatores-two-way-anova}}

\href{https://www.r-bloggers.com/2022/05/two-way-anova-example-in-r-quick-guide/}{Two-way ANOVA}

\begin{Shaded}
\begin{Highlighting}[]
\NormalTok{univ}\SpecialCharTok{$}\NormalTok{sexo }\OtherTok{\textless{}{-}} \FunctionTok{factor}\NormalTok{(univ}\SpecialCharTok{$}\NormalTok{sexo,}
                  \AttributeTok{labels =} \FunctionTok{c}\NormalTok{(}\StringTok{"MACHO"}\NormalTok{, }\StringTok{"FEMEA"}\NormalTok{, }\StringTok{"imaturo"}\NormalTok{))}
\NormalTok{univ}\SpecialCharTok{$}\NormalTok{periodo }\OtherTok{\textless{}{-}} \FunctionTok{factor}\NormalTok{(univ}\SpecialCharTok{$}\NormalTok{periodo,}
                  \AttributeTok{labels =} \FunctionTok{c}\NormalTok{(}\StringTok{"chuvoso"}\NormalTok{, }\StringTok{"seco"}\NormalTok{))}
\FunctionTok{str}\NormalTok{(univ)}
\FunctionTok{table}\NormalTok{(univ}\SpecialCharTok{$}\NormalTok{sexo,univ}\SpecialCharTok{$}\NormalTok{periodo)}

\FunctionTok{library}\NormalTok{(}\StringTok{"ggpubr"}\NormalTok{)}
\FunctionTok{ggboxplot}\NormalTok{(univ, }\AttributeTok{x =} \StringTok{"sexo"}\NormalTok{, }\AttributeTok{y =} \StringTok{"CP\_cm"}\NormalTok{, }\AttributeTok{color =} \StringTok{"periodo"}\NormalTok{,}
          \AttributeTok{palette =} \FunctionTok{c}\NormalTok{(}\StringTok{"\#00AFBB"}\NormalTok{, }\StringTok{"\#E7B800"}\NormalTok{))}
\end{Highlighting}
\end{Shaded}

\includegraphics{fundamentos_de_ecologia_numerica_files/figure-latex/unnamed-chunk-94-1.pdf}

\begin{Shaded}
\begin{Highlighting}[]
\FunctionTok{ggline}\NormalTok{(univ, }\AttributeTok{x =} \StringTok{"sexo"}\NormalTok{, }\AttributeTok{y =} \StringTok{"CP\_cm"}\NormalTok{, }\AttributeTok{color =} \StringTok{"periodo"}\NormalTok{,}
       \AttributeTok{add =} \FunctionTok{c}\NormalTok{(}\StringTok{"mean\_se"}\NormalTok{, }\StringTok{"dotplot"}\NormalTok{),}
       \AttributeTok{palette =} \FunctionTok{c}\NormalTok{(}\StringTok{"\#00AFBB"}\NormalTok{, }\StringTok{"\#E7B800"}\NormalTok{))}
\end{Highlighting}
\end{Shaded}

\begin{verbatim}
## Bin width defaults to 1/30 of the range of the data. Pick better value with
## `binwidth`.
\end{verbatim}

\includegraphics{fundamentos_de_ecologia_numerica_files/figure-latex/unnamed-chunk-94-2.pdf}

\begin{Shaded}
\begin{Highlighting}[]
\FunctionTok{boxplot}\NormalTok{(CP\_cm }\SpecialCharTok{\textasciitilde{}}\NormalTok{ periodo }\SpecialCharTok{*}\NormalTok{ sexo, }\AttributeTok{data=}\NormalTok{univ, }\AttributeTok{frame =} \ConstantTok{FALSE}\NormalTok{,}
        \AttributeTok{col =} \FunctionTok{c}\NormalTok{(}\StringTok{"\#00AFBB"}\NormalTok{, }\StringTok{"\#E7B800"}\NormalTok{), }\AttributeTok{ylab=}\StringTok{"CP\_cm"}\NormalTok{)}
\end{Highlighting}
\end{Shaded}

\includegraphics{fundamentos_de_ecologia_numerica_files/figure-latex/unnamed-chunk-94-3.pdf}

\begin{Shaded}
\begin{Highlighting}[]
\CommentTok{\# Gráfico das interações}
\FunctionTok{interaction.plot}\NormalTok{(}\AttributeTok{x.factor =}\NormalTok{ univ}\SpecialCharTok{$}\NormalTok{sexo, }\AttributeTok{trace.factor =}\NormalTok{ univ}\SpecialCharTok{$}\NormalTok{periodo,}
                 \AttributeTok{response =}\NormalTok{ univ}\SpecialCharTok{$}\NormalTok{CP\_cm, }\AttributeTok{fun =}\NormalTok{ mean,}
                 \AttributeTok{type =} \StringTok{"b"}\NormalTok{, }\AttributeTok{legend =} \ConstantTok{TRUE}\NormalTok{,}
                 \AttributeTok{xlab =} \StringTok{"Sexo"}\NormalTok{, }\AttributeTok{ylab=}\StringTok{"CP\_cm"}\NormalTok{,}
                 \AttributeTok{pch=}\FunctionTok{c}\NormalTok{(}\DecValTok{1}\NormalTok{,}\DecValTok{19}\NormalTok{), }\AttributeTok{col =} \FunctionTok{c}\NormalTok{(}\StringTok{"\#00AFBB"}\NormalTok{, }\StringTok{"\#E7B800"}\NormalTok{))}
\end{Highlighting}
\end{Shaded}

\includegraphics{fundamentos_de_ecologia_numerica_files/figure-latex/unnamed-chunk-94-4.pdf}

\begin{Shaded}
\begin{Highlighting}[]
\CommentTok{\# ANOVA Two{-}way}
\NormalTok{anova2 }\OtherTok{\textless{}{-}} \FunctionTok{aov}\NormalTok{(CP\_cm }\SpecialCharTok{\textasciitilde{}}\NormalTok{ periodo }\SpecialCharTok{+}\NormalTok{ sexo, }\AttributeTok{data =}\NormalTok{ univ)}
\FunctionTok{summary}\NormalTok{(anova2)}

\CommentTok{\#The above{-}fitted model is not referred to as an additive model.}
\CommentTok{\#It is presumptively assumed that the two{-}factor variables are unrelated.}
\CommentTok{\#Replace the plus symbol (+) with an asterisk (*) if you think}
\CommentTok{\#these two variables will interact to create a synergistic effect.}

\NormalTok{anova3 }\OtherTok{\textless{}{-}} \FunctionTok{aov}\NormalTok{(CP\_cm }\SpecialCharTok{\textasciitilde{}}\NormalTok{ periodo }\SpecialCharTok{*}\NormalTok{ sexo, }\AttributeTok{data =}\NormalTok{ univ)}
\FunctionTok{summary}\NormalTok{(anova3)}

\FunctionTok{require}\NormalTok{(}\StringTok{"dplyr"}\NormalTok{)}
\FunctionTok{group\_by}\NormalTok{(univ, periodo, sexo) }\SpecialCharTok{\%\textgreater{}\%}
  \FunctionTok{summarise}\NormalTok{(}
    \AttributeTok{count =} \FunctionTok{n}\NormalTok{(),}
    \AttributeTok{mean =} \FunctionTok{mean}\NormalTok{(CP\_cm, }\AttributeTok{na.rm =} \ConstantTok{TRUE}\NormalTok{),}
    \AttributeTok{sd =} \FunctionTok{sd}\NormalTok{(CP\_cm, }\AttributeTok{na.rm =} \ConstantTok{TRUE}\NormalTok{)}
\NormalTok{  )}
\end{Highlighting}
\end{Shaded}

\begin{verbatim}
## `summarise()` has grouped output by 'periodo'. You can override using the
## `.groups` argument.
\end{verbatim}

\begin{Shaded}
\begin{Highlighting}[]
\CommentTok{\# Comparações múltiplas pares: Tukey}
\FunctionTok{TukeyHSD}\NormalTok{(anova3, }\AttributeTok{which =} \StringTok{"sexo"}\NormalTok{)}

\CommentTok{\# Comparações múltiplas pares}
\FunctionTok{library}\NormalTok{(multcomp)}
\FunctionTok{summary}\NormalTok{(}\FunctionTok{glht}\NormalTok{(anova2, }\AttributeTok{linfct =} \FunctionTok{mcp}\NormalTok{(}\AttributeTok{sexo =} \StringTok{"Tukey"}\NormalTok{)))}

\CommentTok{\# T{-}test entre pares}
\FunctionTok{pairwise.t.test}\NormalTok{(univ}\SpecialCharTok{$}\NormalTok{CP\_cm, univ}\SpecialCharTok{$}\NormalTok{sexo,}
                \AttributeTok{p.adjust.method =} \StringTok{"BH"}\NormalTok{)}

\CommentTok{\# Pressupostos}
\DocumentationTok{\#\# Homogeneidade de variâncias}
\FunctionTok{plot}\NormalTok{(anova3, }\DecValTok{1}\NormalTok{)}
\end{Highlighting}
\end{Shaded}

\includegraphics{fundamentos_de_ecologia_numerica_files/figure-latex/unnamed-chunk-94-5.pdf}

\begin{Shaded}
\begin{Highlighting}[]
\FunctionTok{library}\NormalTok{(car)}
\FunctionTok{leveneTest}\NormalTok{(CP\_cm }\SpecialCharTok{\textasciitilde{}}\NormalTok{ periodo}\SpecialCharTok{*}\NormalTok{sexo, }\AttributeTok{data =}\NormalTok{ univ)}
\CommentTok{\# Normalidade pelos resíduos (Q{-}Q plot)}
\FunctionTok{plot}\NormalTok{(anova3, }\DecValTok{2}\NormalTok{)}
\end{Highlighting}
\end{Shaded}

\includegraphics{fundamentos_de_ecologia_numerica_files/figure-latex/unnamed-chunk-94-6.pdf}

\begin{Shaded}
\begin{Highlighting}[]
\CommentTok{\# Extraindo os resíduos e rodando o Shapiro{-}Wilk neles}
\NormalTok{anova3\_residuals }\OtherTok{\textless{}{-}} \FunctionTok{residuals}\NormalTok{(}\AttributeTok{object =}\NormalTok{ anova3)}
\FunctionTok{shapiro.test}\NormalTok{(}\AttributeTok{x =}\NormalTok{ anova3\_residuals )}

\CommentTok{\# ANOVA para desenhos não balanceados}
\FunctionTok{library}\NormalTok{(car)}
\NormalTok{nb\_anova }\OtherTok{\textless{}{-}} \FunctionTok{aov}\NormalTok{(CP\_cm }\SpecialCharTok{\textasciitilde{}}\NormalTok{ periodo }\SpecialCharTok{*}\NormalTok{ sexo, }\AttributeTok{data =}\NormalTok{ univ)}
\FunctionTok{Anova}\NormalTok{(nb\_anova, }\AttributeTok{type =} \StringTok{"III"}\NormalTok{)}
\end{Highlighting}
\end{Shaded}

\begin{verbatim}
## 'data.frame':    434 obs. of  23 variables:
##  $ n.=.434 : chr  "TU001" "TU002" "TU003" "TU004" ...
##  $ CT_cm   : num  32.4 33.4 27.3 13.2 14.3 22.7 23.2 13.5 24.6 19.4 ...
##  $ PT_g    : num  468.8 520 301.5 28.2 38.9 ...
##  $ CP_cm   : num  27.2 28.8 23.8 11 11.9 20.5 19.2 11.5 20.5 16 ...
##  $ Ctubo_cm: num  39.8 14.3 13 16.5 15.5 24.2 25 17.5 25.7 18.3 ...
##  $ PC_g    : num  458.9 507.4 283.4 27.7 37.7 ...
##  $ %PT     : num  2.11 2.42 6 1.77 3.08 ...
##  $ Pest_g  : num  3.9 5.9 15.5 0.3 0.8 9.9 6.8 4.1 2.5 0.7 ...
##  $ Cest_cm : num  7.7 10 10.2 4.3 4.5 6.4 7.5 6.8 7.6 5.1 ...
##  $ gr_est  : chr  "I" "I" "III" "I" ...
##  $ ir_est  : num  0.832 1.135 5.141 1.064 2.057 ...
##  $ Pint_g  : num  4.8 5.3 2.4 0.1 0.3 2.7 3.3 0.4 2 1.4 ...
##  $ Cint_cm : num  38.3 13.3 12 16 15 23.7 24.5 17 25.3 17.7 ...
##  $ gr_int  : chr  "II" "II" "II" "II" ...
##  $ ir_int  : num  1.024 1.019 0.796 0.355 0.771 ...
##  $ Pgon_g  : num  1.2 1.4 0.2 0.1 0.1 0.4 13.8 0.1 0.1 0.1 ...
##  $ Cgon_cm : chr  "7" "6.5" "7.7" "5" ...
##  $ emg     : chr  "IMATURO" "MADURO" "EM MATURACAO" "IMATURO" ...
##  $ ig      : num  0.256 0.2692 0.0663 0.3546 0.2571 ...
##  $ mes     : chr  "ago" "ago" "ago" "ago" ...
##  $ periodo : Factor w/ 2 levels "chuvoso","seco": 1 1 1 1 1 1 1 1 1 1 ...
##  $ estação : chr  "inverno" "inverno" "inverno" "inverno" ...
##  $ sexo    : Ord.factor w/ 3 levels "MACHO"<"FEMEA"<..: 1 1 1 1 1 1 2 3 1 1 ...
##          
##           chuvoso seco
##   MACHO        68   54
##   FEMEA        38   53
##   imaturo      60  161
##              Df Sum Sq Mean Sq F value Pr(>F)    
## periodo       1    231     231    6.29 0.0125 *  
## sexo          2  10634    5317  144.80 <2e-16 ***
## Residuals   430  15789      37                   
## ---
## Signif. codes:  0 '***' 0.001 '**' 0.01 '*' 0.05 '.' 0.1 ' ' 1
##               Df Sum Sq Mean Sq F value Pr(>F)    
## periodo        1    231     231   6.287 0.0125 *  
## sexo           2  10634    5317 144.714 <2e-16 ***
## periodo:sexo   2     64      32   0.874 0.4180    
## Residuals    428  15725      37                   
## ---
## Signif. codes:  0 '***' 0.001 '**' 0.01 '*' 0.05 '.' 0.1 ' ' 1
## # A tibble: 6 x 5
## # Groups:   periodo [2]
##   periodo sexo    count  mean    sd
##   <fct>   <ord>   <int> <dbl> <dbl>
## 1 chuvoso MACHO      68 17.7   8.41
## 2 chuvoso FEMEA      38 14.9   6.78
## 3 chuvoso imaturo    60  7.18  6.25
## 4 seco    MACHO      54 20.0   7.70
## 5 seco    FEMEA      53 15.5   5.90
## 6 seco    imaturo   161  7.72  3.56
##   Tukey multiple comparisons of means
##     95% family-wise confidence level
## 
## Fit: aov(formula = CP_cm ~ periodo * sexo, data = univ)
## 
## $sexo
##                     diff        lwr       upr     p adj
## FEMEA-MACHO    -3.214875  -5.189460 -1.240291 0.0004334
## imaturo-MACHO -10.698753 -12.306648 -9.090859 0.0000000
## imaturo-FEMEA  -7.483878  -9.259486 -5.708270 0.0000000
## 
## 
##   Simultaneous Tests for General Linear Hypotheses
## 
## Multiple Comparisons of Means: Tukey Contrasts
## 
## 
## Fit: aov(formula = CP_cm ~ periodo + sexo, data = univ)
## 
## Linear Hypotheses:
##                      Estimate Std. Error t value Pr(>|t|)    
## FEMEA - MACHO == 0    -3.5823     0.8438  -4.246  7.4e-05 ***
## imaturo - MACHO == 0 -11.4502     0.7060 -16.219  < 1e-05 ***
## imaturo - FEMEA == 0  -7.8679     0.7601 -10.351  < 1e-05 ***
## ---
## Signif. codes:  0 '***' 0.001 '**' 0.01 '*' 0.05 '.' 0.1 ' ' 1
## (Adjusted p values reported -- single-step method)
## 
## 
##  Pairwise comparisons using t tests with pooled SD 
## 
## data:  univ$CP_cm and univ$sexo 
## 
##         MACHO   FEMEA  
## FEMEA   5.6e-05 -      
## imaturo < 2e-16 < 2e-16
## 
## P value adjustment method: BH 
## Levene's Test for Homogeneity of Variance (center = median)
##        Df F value   Pr(>F)    
## group   5  16.044 1.67e-14 ***
##       428                     
## ---
## Signif. codes:  0 '***' 0.001 '**' 0.01 '*' 0.05 '.' 0.1 ' ' 1
## 
##  Shapiro-Wilk normality test
## 
## data:  anova3_residuals
## W = 0.92359, p-value = 4.653e-14
## 
## Anova Table (Type III tests)
## 
## Response: CP_cm
##               Sum Sq  Df  F value Pr(>F)    
## (Intercept)  27390.0   1 745.5129 <2e-16 ***
## periodo        122.3   1   3.3281 0.0688 .  
## sexo          3641.5   2  49.5584 <2e-16 ***
## periodo:sexo    64.2   2   0.8741 0.4180    
## Residuals    15724.6 428                    
## ---
## Signif. codes:  0 '***' 0.001 '**' 0.01 '*' 0.05 '.' 0.1 ' ' 1
\end{verbatim}

\hypertarget{correlauxe7uxe3o-e-regressuxe3o-simples}{%
\section{Correlação e regressão simples}\label{correlauxe7uxe3o-e-regressuxe3o-simples}}

\href{https://www.r-bloggers.com/2015/11/correlation-and-linear-regression/}{Correlação e regressão}

\begin{Shaded}
\begin{Highlighting}[]
\NormalTok{df}\OtherTok{\textless{}{-}}\FunctionTok{data.frame}\NormalTok{(univ}\SpecialCharTok{$}\NormalTok{CT\_cm, univ}\SpecialCharTok{$}\NormalTok{CP\_cm, univ}\SpecialCharTok{$}\NormalTok{PT\_g)}
\FunctionTok{plot}\NormalTok{(df[,}\DecValTok{1}\SpecialCharTok{:}\DecValTok{3}\NormalTok{])}
\end{Highlighting}
\end{Shaded}

\includegraphics{fundamentos_de_ecologia_numerica_files/figure-latex/unnamed-chunk-95-1.pdf}

\begin{Shaded}
\begin{Highlighting}[]
\FunctionTok{cor}\NormalTok{(df,}\AttributeTok{method=}\StringTok{"pearson"}\NormalTok{)}
\FunctionTok{cor}\NormalTok{(df[,}\DecValTok{1}\SpecialCharTok{:}\DecValTok{3}\NormalTok{], }\AttributeTok{method=}\StringTok{"spearman"}\NormalTok{)}
\FunctionTok{cor.test}\NormalTok{(univ}\SpecialCharTok{$}\NormalTok{CT\_cm, univ}\SpecialCharTok{$}\NormalTok{CP\_cm, }\AttributeTok{method=}\StringTok{"pearson"}\NormalTok{)}
\FunctionTok{cor.test}\NormalTok{(univ}\SpecialCharTok{$}\NormalTok{CT\_cm, univ}\SpecialCharTok{$}\NormalTok{CP\_cm, }\AttributeTok{method=}\StringTok{"spearman"}\NormalTok{)}
\end{Highlighting}
\end{Shaded}

\begin{verbatim}
## Warning in cor.test.default(univ$CT_cm, univ$CP_cm, method = "spearman"):
## Impossível calcular o valor exato de p com empates
\end{verbatim}

\begin{Shaded}
\begin{Highlighting}[]
\NormalTok{m1 }\OtherTok{\textless{}{-}} \FunctionTok{lm}\NormalTok{(univ}\SpecialCharTok{$}\NormalTok{CT\_cm }\SpecialCharTok{\textasciitilde{}}\NormalTok{ univ}\SpecialCharTok{$}\NormalTok{CP\_cm, }\AttributeTok{data=}\NormalTok{univ)}
\FunctionTok{summary}\NormalTok{(m1)}
\FunctionTok{par}\NormalTok{(}\AttributeTok{mfrow=}\FunctionTok{c}\NormalTok{(}\DecValTok{2}\NormalTok{,}\DecValTok{2}\NormalTok{))}
\FunctionTok{plot}\NormalTok{(m1)}
\end{Highlighting}
\end{Shaded}

\includegraphics{fundamentos_de_ecologia_numerica_files/figure-latex/unnamed-chunk-95-2.pdf}

\begin{Shaded}
\begin{Highlighting}[]
\FunctionTok{par}\NormalTok{(}\AttributeTok{mfrow=}\FunctionTok{c}\NormalTok{(}\DecValTok{1}\NormalTok{,}\DecValTok{1}\NormalTok{))}
\NormalTok{univ}\SpecialCharTok{$}\NormalTok{CT\_cm }\OtherTok{\textless{}{-}} \FunctionTok{log}\NormalTok{(univ}\SpecialCharTok{$}\NormalTok{CT\_cm)}
\FunctionTok{plot}\NormalTok{(univ}\SpecialCharTok{$}\NormalTok{CT\_cm }\SpecialCharTok{\textasciitilde{}}\NormalTok{ CP\_cm, univ)}
\end{Highlighting}
\end{Shaded}

\includegraphics{fundamentos_de_ecologia_numerica_files/figure-latex/unnamed-chunk-95-3.pdf}

\begin{Shaded}
\begin{Highlighting}[]
\FunctionTok{anova}\NormalTok{(m1) }
\FunctionTok{library}\NormalTok{(car)}
\FunctionTok{Anova}\NormalTok{(m1)}
\FunctionTok{Anova}\NormalTok{(m1, }\AttributeTok{white.adjust=}\ConstantTok{TRUE}\NormalTok{)}
\end{Highlighting}
\end{Shaded}

\begin{verbatim}
## Coefficient covariances computed by hccm()
\end{verbatim}

\begin{Shaded}
\begin{Highlighting}[]
\NormalTok{univ }\OtherTok{\textless{}{-}}\NormalTok{ univ[univ}\SpecialCharTok{$}\NormalTok{CT\_cm}\SpecialCharTok{!=}\FunctionTok{max}\NormalTok{(univ}\SpecialCharTok{$}\NormalTok{CT\_cm),]}
\end{Highlighting}
\end{Shaded}

\begin{verbatim}
##            univ.CT_cm univ.CP_cm univ.PT_g
## univ.CT_cm  1.0000000  0.9990120 0.9057485
## univ.CP_cm  0.9990120  1.0000000 0.9039503
## univ.PT_g   0.9057485  0.9039503 1.0000000
##            univ.CT_cm univ.CP_cm univ.PT_g
## univ.CT_cm  1.0000000  0.9976789 0.9924823
## univ.CP_cm  0.9976789  1.0000000 0.9934510
## univ.PT_g   0.9924823  0.9934510 1.0000000
## 
##  Pearson's product-moment correlation
## 
## data:  univ$CT_cm and univ$CP_cm
## t = 467.22, df = 432, p-value < 2.2e-16
## alternative hypothesis: true correlation is not equal to 0
## 95 percent confidence interval:
##  0.9988068 0.9991819
## sample estimates:
##      cor 
## 0.999012 
## 
## 
##  Spearman's rank correlation rho
## 
## data:  univ$CT_cm and univ$CP_cm
## S = 31624, p-value < 2.2e-16
## alternative hypothesis: true rho is not equal to 0
## sample estimates:
##       rho 
## 0.9976789 
## 
## 
## Call:
## lm(formula = univ$CT_cm ~ univ$CP_cm, data = univ)
## 
## Residuals:
##     Min      1Q  Median      3Q     Max 
## -2.3777 -0.1281 -0.0147  0.1260  3.0354 
## 
## Coefficients:
##             Estimate Std. Error t value Pr(>|t|)    
## (Intercept) 0.147883   0.037700   3.923 0.000102 ***
## univ$CP_cm  1.206566   0.002582 467.221  < 2e-16 ***
## ---
## Signif. codes:  0 '***' 0.001 '**' 0.01 '*' 0.05 '.' 0.1 ' ' 1
## 
## Residual standard error: 0.4216 on 432 degrees of freedom
## Multiple R-squared:  0.998,  Adjusted R-squared:  0.998 
## F-statistic: 2.183e+05 on 1 and 432 DF,  p-value: < 2.2e-16
## 
## Analysis of Variance Table
## 
## Response: univ$CT_cm
##             Df Sum Sq Mean Sq F value    Pr(>F)    
## univ$CP_cm   1  38802   38802  218295 < 2.2e-16 ***
## Residuals  432     77       0                      
## ---
## Signif. codes:  0 '***' 0.001 '**' 0.01 '*' 0.05 '.' 0.1 ' ' 1
## Anova Table (Type II tests)
## 
## Response: univ$CT_cm
##            Sum Sq  Df F value    Pr(>F)    
## univ$CP_cm  38802   1  218295 < 2.2e-16 ***
## Residuals      77 432                      
## ---
## Signif. codes:  0 '***' 0.001 '**' 0.01 '*' 0.05 '.' 0.1 ' ' 1
## Analysis of Deviance Table (Type II tests)
## 
## Response: univ$CT_cm
##             Df     F    Pr(>F)    
## univ$CP_cm   1 75315 < 2.2e-16 ***
## Residuals  432                    
## ---
## Signif. codes:  0 '***' 0.001 '**' 0.01 '*' 0.05 '.' 0.1 ' ' 1
\end{verbatim}

\hypertarget{apuxeandices-6}{%
\section*{Apêndices}\label{apuxeandices-6}}
\addcontentsline{toc}{section}{Apêndices}

\hypertarget{sites-consultados-4}{%
\section*{Sites consultados}\label{sites-consultados-4}}
\addcontentsline{toc}{section}{Sites consultados}

\hypertarget{script-limpo-3}{%
\section*{Script limpo}\label{script-limpo-3}}
\addcontentsline{toc}{section}{Script limpo}

Aqui apresento o scrip na íntegra sem os textos ou outros comentários. Você pode copiar e colar no R para executa-lo. Lembre de remover os {\texttt{\#}} ou {\texttt{\#\#}} caso necessite executar essas linhas.

\hypertarget{referuxeancias-7}{%
\section*{Referências}\label{referuxeancias-7}}
\addcontentsline{toc}{section}{Referências}

\hypertarget{unigr}{%
\chapter{R Modulo 3.3 - Gráficos univariados}\label{unigr}}

\hypertarget{resumo-7}{%
\subsection*{RESUMO}\label{resumo-7}}
\addcontentsline{toc}{subsection}{RESUMO}

\hypertarget{apresentauxe7uxe3o-7}{%
\subsection*{Apresentação}\label{apresentauxe7uxe3o-7}}
\addcontentsline{toc}{subsection}{Apresentação}

\hypertarget{sobre-os-dados-2}{%
\section{Sobre os dados}\label{sobre-os-dados-2}}

Considere os dados merísticos (ou médições morfológicas) da espécie de peixe \emph{Cichla ocellaris} (tucunaré amarelo) do reservatório da barragem de Gramame, PB (\protect\hyperlink{ref-RN3024}{MEDEIROS; ROSA, 1994}) (Figura \ref{fig:rmd32tucu}). Existem 421 medições do comprimemto total (CT), comprimento padrão (CP) e peso total (PT), além do sexo (MACHO, FÊMEA ou imaturo), e outros descritores da estrutura populacional da espécie, um conjunto de dados formidável.

\begin{figure}
\includegraphics[width=7.24in]{imagens/tucuna} \caption{Dados merísticos da espécie de peixe *Cichla ocellaris* (tucunaré amarelo) do reservatório da barragem  de Gramame, PB.}\label{fig:rmd32tucu}
\end{figure}

\hypertarget{organizauxe7uxe3o-buxe1sica-4}{%
\section{Organização básica}\label{organizauxe7uxe3o-buxe1sica-4}}

\begin{Shaded}
\begin{Highlighting}[]
\FunctionTok{dev.off}\NormalTok{() }\CommentTok{\#apaga os graficos, se houver algum}
\FunctionTok{rm}\NormalTok{(}\AttributeTok{list=}\FunctionTok{ls}\NormalTok{(}\AttributeTok{all=}\ConstantTok{TRUE}\NormalTok{)) }\CommentTok{\#limpa a memória}
\FunctionTok{cat}\NormalTok{(}\StringTok{"}\SpecialCharTok{\textbackslash{}014}\StringTok{"}\NormalTok{) }\CommentTok{\#limpa o console }
\end{Highlighting}
\end{Shaded}

Instalando os pacotes necessários para esse módulo

\begin{Shaded}
\begin{Highlighting}[]
\FunctionTok{install.packages}\NormalTok{(}\StringTok{"mdatools"}\NormalTok{)}
\end{Highlighting}
\end{Shaded}

\begin{Shaded}
\begin{Highlighting}[]
\FunctionTok{library}\NormalTok{(openxlsx)}
\end{Highlighting}
\end{Shaded}

Os códigos acima, são usados para instalar e carregar os pacotes necessários para este módulo. Esses códigos são comandos para instalar pacotes no R. Um pacote é uma coleção de funções, dados e documentação que ampliam as capacidades do R (\href{https://cran.r-project.org/}{R CRAN} (\protect\hyperlink{ref-RN2774}{TEAM, 2017}) e \href{https://posit.co/download/rstudio-desktop/}{RStudio} (\protect\hyperlink{ref-RN358}{TEAM, 2022})). No exemplo acima, o pacote \texttt{openxlsx} permite ler e escrever arquivos Excel no R. Para instalar um pacote no R, você precisa usar a função \texttt{install.packages()}.

Depois de instalar um pacote, você precisa carregá-lo na sua sessão R com a função \texttt{library()}. Por exemplo, para carregar o pacote \texttt{openxlsx}, você precisa executar a função \texttt{library(openxlsx)}. Isso irá permitir que você use as funções do pacote na sua sessão R. Você precisa carregar um pacote toda vez que iniciar uma nova sessão R e quiser usar um pacote instalado.

Agora vamos \textbf{definir o diretório de trabalho}. Esse código é usado para obter e definir o diretório de trabalho atual no R. O comando \texttt{getwd()} retorna o caminho do diretório onde o R está lendo e salvando arquivos. O comando \texttt{setwd()} muda esse diretório de trabalho para o caminho especificado entre aspas. No seu caso, você deve ajustar o caminho para o seu próprio diretório de trabalho. \textbf{Lembre de usar a barra ``/'' entre os diretórios. E não a contra-barra ``\textbackslash{}''.}

\begin{Shaded}
\begin{Highlighting}[]
\FunctionTok{getwd}\NormalTok{()}
\FunctionTok{setwd}\NormalTok{(}\StringTok{"C:/Seu/Diretório/De/Trabalho"}\NormalTok{)}
\end{Highlighting}
\end{Shaded}

\hypertarget{importando-a-planilha-4}{%
\section{Importando a planilha}\label{importando-a-planilha-4}}

Vamos importar a planilha de dados univariados \texttt{univ*.xlsx}. Note que o símbolo {\texttt{\#}} em programação R significa que o texto que vem depois dele é um comentário e não será executado pelo programa. Isso é útil para explicar o código ou deixar anotações. Ajuste a segunda linha do código abaixo para refletir {``C:/Seu/Diretório/De/Trabalho/Planilha.xlsx''}.

\begin{Shaded}
\begin{Highlighting}[]
\FunctionTok{library}\NormalTok{(openxlsx)}
\NormalTok{univ\_all }\OtherTok{\textless{}{-}} \FunctionTok{read.xlsx}\NormalTok{(}\StringTok{"D:/Elvio/OneDrive/Disciplinas/\_EcoNumerica/5.Matrizes/univ.xlsx"}\NormalTok{,}
                    \AttributeTok{rowNames =}\NormalTok{ T, }\AttributeTok{colNames =}\NormalTok{ T,}
                    \AttributeTok{sheet =} \StringTok{"tucuna"}\NormalTok{)}
\FunctionTok{head}\NormalTok{(univ\_all,}\DecValTok{10}\NormalTok{)}
\FunctionTok{head}\NormalTok{(univ\_all[, }\DecValTok{1}\SpecialCharTok{:}\DecValTok{5}\NormalTok{], }\DecValTok{10}\NormalTok{)}
\end{Highlighting}
\end{Shaded}

\begin{verbatim}
##       CT_cm  PT_g CP_cm Ctubo_cm  PC_g      %PT Pest_g Cest_cm gr_est    ir_est
## TU001  32.4 468.8  27.2     39.8 458.9 2.111775    3.9     7.7      I 0.8319113
## TU002  33.4 520.0  28.8     14.3 507.4 2.423077    5.9    10.0      I 1.1346154
## TU003  27.3 301.5  23.8     13.0 283.4 6.003317   15.5    10.2    III 5.1409619
## TU004  13.2  28.2  11.0     16.5  27.7 1.773050    0.3     4.3      I 1.0638298
## TU005  14.3  38.9  11.9     15.5  37.7 3.084833    0.8     4.5    III 2.0565553
## TU006  22.7 431.7  20.5     24.2 418.7 3.011350    9.9     6.4    III 2.2932592
## TU007  23.2 544.0  19.2     25.0 520.1 4.393382    6.8     7.5     II 1.2500000
## TU008  13.5 161.6  11.5     17.5 157.0 2.846535    4.1     6.8     II 2.5371287
## TU009  24.6 200.5  20.5     25.7 195.9 2.294264    2.5     7.6     II 1.2468828
## TU010  19.4  86.7  16.0     18.3  84.5 2.537486    0.7     5.1      I 0.8073818
##       Pint_g Cint_cm gr_int    ir_int Pgon_g Cgon_cm          emg         ig
## TU001    4.8    38.3     II 1.0238908    1.2       7      IMATURO 0.25597270
## TU002    5.3    13.3     II 1.0192308    1.4     6.5       MADURO 0.26923077
## TU003    2.4    12.0     II 0.7960199    0.2     7.7 EM MATURACAO 0.06633499
## TU004    0.1    16.0     II 0.3546099    0.1       5      IMATURO 0.35460993
## TU005    0.3    15.0     II 0.7712082    0.1     3.2      IMATURO 0.25706941
## TU006    2.7    23.7     II 0.6254343    0.4     7.8 EM MATURACAO 0.09265694
## TU007    3.3    24.5     II 0.6066176   13.8     7.9       MADURO 2.53676471
## TU008    0.4    17.0      I 0.2475248    0.1     2.5      IMATURO 0.06188119
## TU009    2.0    25.3     II 0.9975062    0.1     6.5 EM MATURACAO 0.04987531
## TU010    1.4    17.7     II 1.6147636    0.1       6      IMATURO 0.11534025
##       mes periodo estação    sexo
## TU001 ago chuvoso inverno   MACHO
## TU002 ago chuvoso inverno   MACHO
## TU003 ago chuvoso inverno   MACHO
## TU004 ago chuvoso inverno   MACHO
## TU005 ago chuvoso inverno   MACHO
## TU006 set chuvoso inverno   MACHO
## TU007 set chuvoso inverno   FEMEA
## TU008 set chuvoso inverno imaturo
## TU009 set chuvoso inverno   MACHO
## TU010 set chuvoso inverno   MACHO
##       CT_cm  PT_g CP_cm Ctubo_cm  PC_g
## TU001  32.4 468.8  27.2     39.8 458.9
## TU002  33.4 520.0  28.8     14.3 507.4
## TU003  27.3 301.5  23.8     13.0 283.4
## TU004  13.2  28.2  11.0     16.5  27.7
## TU005  14.3  38.9  11.9     15.5  37.7
## TU006  22.7 431.7  20.5     24.2 418.7
## TU007  23.2 544.0  19.2     25.0 520.1
## TU008  13.5 161.6  11.5     17.5 157.0
## TU009  24.6 200.5  20.5     25.7 195.9
## TU010  19.4  86.7  16.0     18.3  84.5
\end{verbatim}

Exibindo os dados importados (esses comando são ``case-sensitive'' \texttt{ignore.case(object)}).

\begin{Shaded}
\begin{Highlighting}[]
\CommentTok{\#View(univ)}
\FunctionTok{print}\NormalTok{(univ\_all[}\DecValTok{1}\SpecialCharTok{:}\DecValTok{5}\NormalTok{,}\DecValTok{1}\SpecialCharTok{:}\DecValTok{5}\NormalTok{])}
\NormalTok{univ\_all}
\FunctionTok{str}\NormalTok{(univ\_all)}
\FunctionTok{mode}\NormalTok{(univ\_all)}
\FunctionTok{class}\NormalTok{(univ\_all)}
\end{Highlighting}
\end{Shaded}

\hypertarget{reset-point}{%
\section{Reset point}\label{reset-point}}

\begin{Shaded}
\begin{Highlighting}[]
\NormalTok{univ }\OtherTok{\textless{}{-}}\NormalTok{ univ\_all}
\NormalTok{univ }\OtherTok{\textless{}{-}}\NormalTok{ univ\_all[, }\DecValTok{1}\SpecialCharTok{:}\DecValTok{3}\NormalTok{]}
\CommentTok{\#univ\_all \textless{}{-} univ\_all}
\CommentTok{\#var \textless{}{-} ""}
\end{Highlighting}
\end{Shaded}

\hypertarget{gruxe1ficos-de-dispersuxe3o}{%
\section{Gráficos de dispersão}\label{gruxe1ficos-de-dispersuxe3o}}

(\url{https://mda.tools/docs/datasets--simple-plots.html})

\begin{Shaded}
\begin{Highlighting}[]
\FunctionTok{library}\NormalTok{(mdatools)}
\CommentTok{\#removendo linhas ou coluns por nome}
\FunctionTok{attr}\NormalTok{(univ, }\StringTok{"name"}\NormalTok{) }\OtherTok{=} \StringTok{"Tucunaré"}
\FunctionTok{attr}\NormalTok{(univ, }\StringTok{"xaxis.name"}\NormalTok{) }\OtherTok{=} \StringTok{"Eixo X"}
\FunctionTok{attr}\NormalTok{(univ, }\StringTok{"yaxis.name"}\NormalTok{) }\OtherTok{=} \StringTok{"Eixo Y"}
\FunctionTok{par}\NormalTok{(}\AttributeTok{mfrow =} \FunctionTok{c}\NormalTok{(}\DecValTok{1}\NormalTok{,}\DecValTok{2}\NormalTok{))}
\CommentTok{\# show plot for the whole dataset (columns 1 and 2 will be taken)}
\FunctionTok{mdaplot}\NormalTok{(univ, }\AttributeTok{type =} \StringTok{"p"}\NormalTok{)}
\CommentTok{\# subset the dataset and keep only columns 6 and 7 and then make a plot}
\FunctionTok{mdaplot}\NormalTok{(}\FunctionTok{mda.subset}\NormalTok{(univ, }\AttributeTok{select =} \FunctionTok{c}\NormalTok{(}\DecValTok{1}\NormalTok{,}\DecValTok{2}\NormalTok{)), }\AttributeTok{type =} \StringTok{"p"}\NormalTok{)}

\FunctionTok{par}\NormalTok{(}\AttributeTok{mfrow =} \FunctionTok{c}\NormalTok{(}\DecValTok{2}\NormalTok{,}\DecValTok{2}\NormalTok{))}
\CommentTok{\# show Height vs Weight and color points by the Beer consumption}
\FunctionTok{mdaplot}\NormalTok{(univ, }\AttributeTok{type =} \StringTok{"p"}\NormalTok{, }\AttributeTok{cgroup =}\NormalTok{ univ\_all[, var])}
\CommentTok{\# do the same but do not show colorbar}
\FunctionTok{mdaplot}\NormalTok{(univ, }\AttributeTok{type =} \StringTok{"p"}\NormalTok{, }\AttributeTok{cgroup =}\NormalTok{ univ\_all[, var],}
        \AttributeTok{show.colorbar =} \ConstantTok{FALSE}\NormalTok{)}
\CommentTok{\# do the same but use grayscale color map}
\FunctionTok{mdaplot}\NormalTok{(univ, }\AttributeTok{type =} \StringTok{"p"}\NormalTok{, }\AttributeTok{cgroup =}\NormalTok{ univ\_all[, var], }\AttributeTok{colmap =} \StringTok{"gray"}\NormalTok{)}
\CommentTok{\# do the same but using colormap with gradients between red, yellow and green colors}
\FunctionTok{mdaplot}\NormalTok{(univ, }\AttributeTok{type =} \StringTok{"p"}\NormalTok{, }\AttributeTok{cgroup =}\NormalTok{ univ\_all[, var],}
        \AttributeTok{colmap =} \FunctionTok{c}\NormalTok{(}\StringTok{"red"}\NormalTok{, }\StringTok{"yellow"}\NormalTok{, }\StringTok{"green"}\NormalTok{))}
\CommentTok{\# make a factor using values of variable Sex and define labels for the factor levels}
\NormalTok{g }\OtherTok{\textless{}{-}} \FunctionTok{factor}\NormalTok{(univ\_all[, }\StringTok{"sexo"}\NormalTok{],}
           \AttributeTok{levels =} \FunctionTok{c}\NormalTok{(}\StringTok{"MACHO"}\NormalTok{, }\StringTok{"FEMEA"}\NormalTok{, }\StringTok{"imaturo"}\NormalTok{))}
\NormalTok{g}
\FunctionTok{par}\NormalTok{(}\AttributeTok{mfrow =} \FunctionTok{c}\NormalTok{(}\DecValTok{1}\NormalTok{, }\DecValTok{2}\NormalTok{))}
\FunctionTok{mdaplot}\NormalTok{(univ, }\AttributeTok{type =} \StringTok{"p"}\NormalTok{, }\AttributeTok{cgroup =}\NormalTok{ g)}
\FunctionTok{mdaplot}\NormalTok{(univ, }\AttributeTok{type =} \StringTok{"p"}\NormalTok{, }\AttributeTok{cgroup =}\NormalTok{ g, }\AttributeTok{colmap =} \StringTok{"gray"}\NormalTok{)}

\FunctionTok{par}\NormalTok{(}\AttributeTok{mfrow =} \FunctionTok{c}\NormalTok{(}\DecValTok{1}\NormalTok{, }\DecValTok{2}\NormalTok{))}
\CommentTok{\# default way {-} color grouping is used for borders and "bg" for background}
\FunctionTok{mdaplot}\NormalTok{(univ, }\AttributeTok{type =} \StringTok{"p"}\NormalTok{, }\AttributeTok{cgroup =}\NormalTok{ univ\_all[, var], }\AttributeTok{pch =} \DecValTok{21}\NormalTok{, }\AttributeTok{bg =} \StringTok{"white"}\NormalTok{)}
\CommentTok{\# inverse {-} color grouping is used for background and "bg" for border}
\FunctionTok{mdaplot}\NormalTok{(univ, }\AttributeTok{type =} \StringTok{"p"}\NormalTok{, }\AttributeTok{cgroup =}\NormalTok{ univ\_all[, var], }\AttributeTok{pch =} \DecValTok{21}\NormalTok{, }\AttributeTok{bg =} \StringTok{"white"}\NormalTok{, }\AttributeTok{pch.colinv =} \ConstantTok{TRUE}\NormalTok{)}

\FunctionTok{par}\NormalTok{(}\AttributeTok{mfrow =} \FunctionTok{c}\NormalTok{(}\DecValTok{2}\NormalTok{, }\DecValTok{2}\NormalTok{))}
\CommentTok{\# by default row names will be used as labels}
\FunctionTok{mdaplot}\NormalTok{(univ, }\AttributeTok{type =} \StringTok{"p"}\NormalTok{, }\AttributeTok{show.labels =} \ConstantTok{TRUE}\NormalTok{)}
\CommentTok{\# here we tell to use indices as labels instead}
\FunctionTok{mdaplot}\NormalTok{(univ, }\AttributeTok{type =} \StringTok{"p"}\NormalTok{, }\AttributeTok{show.labels =} \ConstantTok{TRUE}\NormalTok{, }\AttributeTok{labels =} \StringTok{"indices"}\NormalTok{)}
\CommentTok{\# here we use names again but change color and size of the labels}
\FunctionTok{mdaplot}\NormalTok{(univ, }\AttributeTok{type =} \StringTok{"p"}\NormalTok{, }\AttributeTok{show.labels =} \ConstantTok{TRUE}\NormalTok{, }\AttributeTok{labels =} \StringTok{"names"}\NormalTok{, }\AttributeTok{lab.col =} \StringTok{"red"}\NormalTok{, }\AttributeTok{lab.cex =} \FloatTok{0.5}\NormalTok{)}
\CommentTok{\# finally we provide a vector with manual values to be used as the labels}
\FunctionTok{mdaplot}\NormalTok{(univ, }\AttributeTok{type =} \StringTok{"p"}\NormalTok{, }\AttributeTok{show.labels =} \ConstantTok{TRUE}\NormalTok{, }\AttributeTok{labels =} \FunctionTok{paste0}\NormalTok{(}\StringTok{"T"}\NormalTok{, }\FunctionTok{seq\_len}\NormalTok{(}\FunctionTok{nrow}\NormalTok{(univ))))}

\FunctionTok{par}\NormalTok{(}\AttributeTok{mfrow =} \FunctionTok{c}\NormalTok{(}\DecValTok{2}\NormalTok{, }\DecValTok{2}\NormalTok{))}
\CommentTok{\# manual values and tick labels for the x{-}axis}
\FunctionTok{mdaplot}\NormalTok{(univ, }\AttributeTok{xticks =} \FunctionTok{c}\NormalTok{(}\DecValTok{10}\NormalTok{,}\DecValTok{30}\NormalTok{,}\DecValTok{40}\NormalTok{), }\AttributeTok{xticklabels =} \FunctionTok{c}\NormalTok{(}\StringTok{"Small"}\NormalTok{, }\StringTok{"Medium"}\NormalTok{, }\StringTok{"Large"}\NormalTok{))}
\CommentTok{\# same but with rotation of the tick labels}
\FunctionTok{mdaplot}\NormalTok{(univ, }\AttributeTok{xticks =} \FunctionTok{c}\NormalTok{(}\DecValTok{10}\NormalTok{,}\DecValTok{30}\NormalTok{,}\DecValTok{40}\NormalTok{), }\AttributeTok{xticklabels =} \FunctionTok{c}\NormalTok{(}\StringTok{"Small"}\NormalTok{, }\StringTok{"Medium"}\NormalTok{, }\StringTok{"Large"}\NormalTok{), }\AttributeTok{xlas =} \DecValTok{2}\NormalTok{)}
\CommentTok{\# manual values and tick labels for the y{-}axis}
\FunctionTok{mdaplot}\NormalTok{(univ, }\AttributeTok{yticks =} \FunctionTok{c}\NormalTok{(}\DecValTok{200}\NormalTok{,}\DecValTok{600}\NormalTok{,}\DecValTok{900}\NormalTok{), }\AttributeTok{yticklabels =} \FunctionTok{c}\NormalTok{(}\StringTok{"Light"}\NormalTok{, }\StringTok{"Medium"}\NormalTok{, }\StringTok{"Heavy"}\NormalTok{))}
\CommentTok{\# same but with rotation of the tick labels}
\FunctionTok{mdaplot}\NormalTok{(univ, }\AttributeTok{yticks =} \FunctionTok{c}\NormalTok{(}\DecValTok{200}\NormalTok{,}\DecValTok{600}\NormalTok{,}\DecValTok{900}\NormalTok{), }\AttributeTok{yticklabels =} \FunctionTok{c}\NormalTok{(}\StringTok{"Light"}\NormalTok{, }\StringTok{"Medium"}\NormalTok{, }\StringTok{"Heavy"}\NormalTok{), }\AttributeTok{ylas =} \DecValTok{2}\NormalTok{)}

\FunctionTok{par}\NormalTok{(}\AttributeTok{mfrow =} \FunctionTok{c}\NormalTok{(}\DecValTok{1}\NormalTok{, }\DecValTok{2}\NormalTok{))}
\CommentTok{\# change margin for bottom part}
\FunctionTok{par}\NormalTok{(}\AttributeTok{mar =} \FunctionTok{c}\NormalTok{(}\DecValTok{6}\NormalTok{, }\DecValTok{4}\NormalTok{, }\DecValTok{4}\NormalTok{, }\DecValTok{2}\NormalTok{) }\SpecialCharTok{+} \FloatTok{0.1}\NormalTok{)}
\FunctionTok{mdaplot}\NormalTok{(univ, }\AttributeTok{xticks =} \FunctionTok{c}\NormalTok{(}\DecValTok{10}\NormalTok{,}\DecValTok{30}\NormalTok{,}\DecValTok{40}\NormalTok{),}
        \AttributeTok{xticklabels =} \FunctionTok{c}\NormalTok{(}\StringTok{"Small"}\NormalTok{, }\StringTok{"Medium"}\NormalTok{, }\StringTok{"Large"}\NormalTok{),}
        \AttributeTok{xlas =} \DecValTok{2}\NormalTok{, }\AttributeTok{xlab =} \StringTok{""}\NormalTok{)}
\FunctionTok{mtext}\NormalTok{(}\StringTok{"Comprimento"}\NormalTok{, }\AttributeTok{side =} \DecValTok{1}\NormalTok{, }\AttributeTok{line =} \DecValTok{5}\NormalTok{)}
\CommentTok{\# change margin for left part}
\FunctionTok{par}\NormalTok{(}\AttributeTok{mar =} \FunctionTok{c}\NormalTok{(}\DecValTok{5}\NormalTok{, }\DecValTok{6}\NormalTok{, }\DecValTok{4}\NormalTok{, }\DecValTok{1}\NormalTok{) }\SpecialCharTok{+} \FloatTok{0.1}\NormalTok{)}
\FunctionTok{mdaplot}\NormalTok{(univ, }\AttributeTok{yticks =} \FunctionTok{c}\NormalTok{(}\DecValTok{200}\NormalTok{,}\DecValTok{600}\NormalTok{,}\DecValTok{900}\NormalTok{), }\AttributeTok{yticklabels =} \FunctionTok{c}\NormalTok{(}\StringTok{"Light"}\NormalTok{, }\StringTok{"Medium"}\NormalTok{, }\StringTok{"Heavy"}\NormalTok{),}
        \AttributeTok{ylas =} \DecValTok{2}\NormalTok{, }\AttributeTok{ylab =} \StringTok{""}\NormalTok{)}
\FunctionTok{mtext}\NormalTok{(}\StringTok{"Peso"}\NormalTok{, }\AttributeTok{side =} \DecValTok{2}\NormalTok{, }\AttributeTok{line =} \DecValTok{5}\NormalTok{)}

\FunctionTok{par}\NormalTok{(}\AttributeTok{mfrow =} \FunctionTok{c}\NormalTok{(}\DecValTok{1}\NormalTok{,}\DecValTok{2}\NormalTok{))}
\FunctionTok{mdaplot}\NormalTok{(univ, }\AttributeTok{show.grid =} \ConstantTok{FALSE}\NormalTok{, }\AttributeTok{show.lines =} \FunctionTok{c}\NormalTok{(}\DecValTok{20}\NormalTok{,}\DecValTok{200}\NormalTok{))}
\FunctionTok{mdaplot}\NormalTok{(univ, }\AttributeTok{show.lines =} \FunctionTok{c}\NormalTok{(}\DecValTok{40}\NormalTok{, }\ConstantTok{NA}\NormalTok{))}
\CommentTok{\# define a factor using values of variable Sex and simple labels}
\NormalTok{g}
\FunctionTok{par}\NormalTok{(}\AttributeTok{mfrow =} \FunctionTok{c}\NormalTok{(}\DecValTok{1}\NormalTok{, }\DecValTok{2}\NormalTok{))}
\CommentTok{\# make a scatter plot grouping points by the factor and then show convex hull for each group}
\NormalTok{p }\OtherTok{\textless{}{-}} \FunctionTok{mdaplot}\NormalTok{(univ, }\AttributeTok{cgroup =}\NormalTok{ g)}
\FunctionTok{plotConvexHull}\NormalTok{(p)}
\CommentTok{\# make a scatter plot grouping points by the factor and then show 90\% confidence intervals}
\NormalTok{p }\OtherTok{\textless{}{-}} \FunctionTok{mdaplot}\NormalTok{(univ, }\AttributeTok{cgroup =}\NormalTok{ g)}
\FunctionTok{plotConfidenceEllipse}\NormalTok{(p, }\AttributeTok{conf.level =} \FloatTok{0.90}\NormalTok{)}

\CommentTok{\# média agrupada por fator}
\NormalTok{media }\OtherTok{\textless{}{-}} \FunctionTok{tapply}\NormalTok{(univ}\SpecialCharTok{$}\NormalTok{CT\_cm, g, mean)}
\NormalTok{media}
\NormalTok{anova }\OtherTok{\textless{}{-}} \FunctionTok{aov}\NormalTok{(CT\_cm }\SpecialCharTok{\textasciitilde{}}\NormalTok{ g, }\AttributeTok{data =}\NormalTok{ univ\_all)}
\FunctionTok{summary}\NormalTok{(anova)}
\FunctionTok{plot}\NormalTok{(anova,}\DecValTok{1}\NormalTok{)}
\end{Highlighting}
\end{Shaded}

\hypertarget{gruxe1ficos-de-linha}{%
\section{Gráficos de linha}\label{gruxe1ficos-de-linha}}

\begin{Shaded}
\begin{Highlighting}[]
\FunctionTok{library}\NormalTok{(mdatools)}
\end{Highlighting}
\end{Shaded}

\begin{verbatim}
## Warning: package 'mdatools' was built under R version 4.3.3
\end{verbatim}

\begin{verbatim}
## 
## Attaching package: 'mdatools'
\end{verbatim}

\begin{verbatim}
## The following object is masked from 'package:car':
## 
##     ellipse
\end{verbatim}

\begin{Shaded}
\begin{Highlighting}[]
\NormalTok{var }\OtherTok{\textless{}{-}}\NormalTok{ univ}\SpecialCharTok{$}\NormalTok{CT\_cm}
\FunctionTok{attr}\NormalTok{(univ}\SpecialCharTok{$}\NormalTok{CT\_cm, }\StringTok{"yaxis.name"}\NormalTok{) }\OtherTok{=} \StringTok{"CT\_cm"}
\FunctionTok{attr}\NormalTok{(univ}\SpecialCharTok{$}\NormalTok{CT\_cm, }\StringTok{"xaxis.name"}\NormalTok{) }\OtherTok{=} \StringTok{"Indivíduo"}

\FunctionTok{par}\NormalTok{(}\AttributeTok{mfrow =} \FunctionTok{c}\NormalTok{(}\DecValTok{2}\NormalTok{, }\DecValTok{1}\NormalTok{))}
\FunctionTok{mdaplot}\NormalTok{(univ}\SpecialCharTok{$}\NormalTok{CT\_cm, }\AttributeTok{type =} \StringTok{"l"}\NormalTok{)}
\FunctionTok{mdaplot}\NormalTok{(univ}\SpecialCharTok{$}\NormalTok{CT\_cm, }\AttributeTok{type =} \StringTok{"l"}\NormalTok{, }\AttributeTok{col =} \StringTok{"darkgray"}\NormalTok{, }\AttributeTok{lty =} \DecValTok{2}\NormalTok{)}
\end{Highlighting}
\end{Shaded}

\includegraphics{fundamentos_de_ecologia_numerica_files/figure-latex/unnamed-chunk-105-1.pdf}

\begin{Shaded}
\begin{Highlighting}[]
\CommentTok{\#mdaplot(univ$CT\_cm, type = "l", cgroup = g)}
\end{Highlighting}
\end{Shaded}

\hypertarget{gruxe1ficos-de-barra}{%
\section{Gráficos de barra}\label{gruxe1ficos-de-barra}}

\begin{Shaded}
\begin{Highlighting}[]
\CommentTok{\# make a simple two rows matrix with values}
\NormalTok{d }\OtherTok{=} \FunctionTok{rbind}\NormalTok{(}
   \FunctionTok{c}\NormalTok{(}\DecValTok{20}\NormalTok{, }\DecValTok{50}\NormalTok{, }\DecValTok{60}\NormalTok{, }\DecValTok{90}\NormalTok{),}
   \FunctionTok{c}\NormalTok{(}\DecValTok{14}\NormalTok{, }\DecValTok{45}\NormalTok{, }\DecValTok{59}\NormalTok{, }\DecValTok{88}\NormalTok{)}
\NormalTok{)}

\CommentTok{\# add some names and attributes}
\FunctionTok{colnames}\NormalTok{(d) }\OtherTok{=} \FunctionTok{paste0}\NormalTok{(}\StringTok{"PC"}\NormalTok{, }\DecValTok{1}\SpecialCharTok{:}\DecValTok{4}\NormalTok{)}
\FunctionTok{rownames}\NormalTok{(d) }\OtherTok{=} \FunctionTok{c}\NormalTok{(}\StringTok{"Cal"}\NormalTok{, }\StringTok{"CV"}\NormalTok{)}
\FunctionTok{attr}\NormalTok{(d, }\StringTok{"xaxis.name"}\NormalTok{) }\OtherTok{=} \StringTok{"Components"}
\FunctionTok{attr}\NormalTok{(d, }\StringTok{"name"}\NormalTok{) }\OtherTok{=} \StringTok{"Explained variance"}

\FunctionTok{par}\NormalTok{(}\AttributeTok{mfrow =} \FunctionTok{c}\NormalTok{(}\DecValTok{1}\NormalTok{, }\DecValTok{2}\NormalTok{))}

\CommentTok{\# make a default bar plot}
\FunctionTok{mdaplot}\NormalTok{(d, }\AttributeTok{type =} \StringTok{"h"}\NormalTok{)}

\CommentTok{\# make a bar plot with manual xtick labels, color and labels for data values}
\FunctionTok{mdaplot}\NormalTok{(d, }\AttributeTok{type =} \StringTok{"h"}\NormalTok{, }\AttributeTok{xticks =} \FunctionTok{seq\_len}\NormalTok{(}\FunctionTok{ncol}\NormalTok{(d)), }\AttributeTok{xticklabels =} \FunctionTok{colnames}\NormalTok{(d), }\AttributeTok{col =} \StringTok{"red"}\NormalTok{,}
   \AttributeTok{show.labels =} \ConstantTok{TRUE}\NormalTok{, }\AttributeTok{labels =} \StringTok{"values"}\NormalTok{, }\AttributeTok{xlas =} \DecValTok{2}\NormalTok{, }\AttributeTok{xlab =} \StringTok{""}\NormalTok{, }\AttributeTok{ylab =} \StringTok{"Variance, \%"}\NormalTok{)}
\end{Highlighting}
\end{Shaded}

\includegraphics{fundamentos_de_ecologia_numerica_files/figure-latex/unnamed-chunk-106-1.pdf}

\begin{Shaded}
\begin{Highlighting}[]
\NormalTok{d }\OtherTok{=} \FunctionTok{rbind}\NormalTok{(}
   \FunctionTok{c}\NormalTok{(}\DecValTok{20}\NormalTok{, }\DecValTok{60}\NormalTok{, }\DecValTok{70}\NormalTok{, }\DecValTok{75}\NormalTok{),}
   \FunctionTok{c}\NormalTok{(}\DecValTok{2}\NormalTok{, }\DecValTok{5}\NormalTok{, }\DecValTok{4}\NormalTok{, }\DecValTok{3}\NormalTok{)}
\NormalTok{)}

\CommentTok{\# add names and attributes}
\FunctionTok{rownames}\NormalTok{(d) }\OtherTok{=} \FunctionTok{c}\NormalTok{(}\StringTok{"Mean"}\NormalTok{, }\StringTok{"Std"}\NormalTok{)}
\FunctionTok{colnames}\NormalTok{(d) }\OtherTok{=} \FunctionTok{paste0}\NormalTok{(}\StringTok{"PC"}\NormalTok{, }\DecValTok{1}\SpecialCharTok{:}\DecValTok{4}\NormalTok{)}
\FunctionTok{attr}\NormalTok{(d, }\StringTok{\textquotesingle{}name\textquotesingle{}}\NormalTok{) }\OtherTok{=} \StringTok{"Statistics"}


\CommentTok{\# show the plots}
\FunctionTok{par}\NormalTok{(}\AttributeTok{mfrow =} \FunctionTok{c}\NormalTok{(}\DecValTok{1}\NormalTok{, }\DecValTok{2}\NormalTok{))}
\FunctionTok{mdaplot}\NormalTok{(d, }\AttributeTok{type =} \StringTok{"e"}\NormalTok{)}
\FunctionTok{mdaplot}\NormalTok{(d, }\AttributeTok{type =} \StringTok{"e"}\NormalTok{, }\AttributeTok{xticks =} \FunctionTok{seq\_len}\NormalTok{(}\FunctionTok{ncol}\NormalTok{(d)),}
   \AttributeTok{xticklabels =} \FunctionTok{colnames}\NormalTok{(d), }\AttributeTok{col =} \StringTok{"red"}\NormalTok{, }\AttributeTok{xlas =} \DecValTok{2}\NormalTok{, }\AttributeTok{xlab =} \StringTok{""}\NormalTok{)}
\end{Highlighting}
\end{Shaded}

\includegraphics{fundamentos_de_ecologia_numerica_files/figure-latex/unnamed-chunk-106-2.pdf}

\begin{Shaded}
\begin{Highlighting}[]
\FunctionTok{par}\NormalTok{(}\AttributeTok{mfrow =} \FunctionTok{c}\NormalTok{(}\DecValTok{1}\NormalTok{, }\DecValTok{2}\NormalTok{))}

\FunctionTok{mdaplot}\NormalTok{(}\FunctionTok{mda.subset}\NormalTok{(d, }\DecValTok{1}\NormalTok{), }\AttributeTok{type =} \StringTok{"h"}\NormalTok{, }\AttributeTok{col =} \StringTok{"lightgray"}\NormalTok{)}
\FunctionTok{mdaplot}\NormalTok{(d, }\AttributeTok{type =} \StringTok{"e"}\NormalTok{, }\AttributeTok{show.axes =} \ConstantTok{FALSE}\NormalTok{, }\AttributeTok{pch =} \ConstantTok{NA}\NormalTok{)}

\FunctionTok{mdaplot}\NormalTok{(}\FunctionTok{mda.subset}\NormalTok{(d, }\DecValTok{1}\NormalTok{), }\AttributeTok{type =} \StringTok{"b"}\NormalTok{)}
\FunctionTok{mdaplot}\NormalTok{(d, }\AttributeTok{type =} \StringTok{"e"}\NormalTok{, }\AttributeTok{show.axes =} \ConstantTok{FALSE}\NormalTok{)}
\end{Highlighting}
\end{Shaded}

\includegraphics{fundamentos_de_ecologia_numerica_files/figure-latex/unnamed-chunk-106-3.pdf}

\hypertarget{gruxe1ficos-de-grupos}{%
\section{Gráficos de grupos}\label{gruxe1ficos-de-grupos}}

\begin{Shaded}
\begin{Highlighting}[]
\CommentTok{\# let\textquotesingle{}s create a simple dataset with 3 rows}
\NormalTok{p }\OtherTok{=} \FunctionTok{rbind}\NormalTok{(}
   \FunctionTok{c}\NormalTok{(}\FloatTok{0.40}\NormalTok{, }\FloatTok{0.69}\NormalTok{, }\FloatTok{0.88}\NormalTok{, }\FloatTok{0.95}\NormalTok{),}
   \FunctionTok{c}\NormalTok{(}\FloatTok{0.34}\NormalTok{, }\FloatTok{0.64}\NormalTok{, }\FloatTok{0.81}\NormalTok{, }\FloatTok{0.92}\NormalTok{),}
   \FunctionTok{c}\NormalTok{(}\FloatTok{0.30}\NormalTok{, }\FloatTok{0.61}\NormalTok{, }\FloatTok{0.80}\NormalTok{, }\FloatTok{0.88}\NormalTok{)}
\NormalTok{)}

\CommentTok{\# add some names and attributes}
\FunctionTok{rownames}\NormalTok{(p) }\OtherTok{=} \FunctionTok{c}\NormalTok{(}\StringTok{"Cal"}\NormalTok{, }\StringTok{"CV"}\NormalTok{, }\StringTok{"Test"}\NormalTok{)}
\FunctionTok{colnames}\NormalTok{(p) }\OtherTok{=} \FunctionTok{paste0}\NormalTok{(}\StringTok{"PC"}\NormalTok{, }\DecValTok{1}\SpecialCharTok{:}\DecValTok{4}\NormalTok{)}
\FunctionTok{attr}\NormalTok{(p, }\StringTok{"name"}\NormalTok{) }\OtherTok{=} \StringTok{"Cumulative variance"}
\FunctionTok{attr}\NormalTok{(p, }\StringTok{"xaxis.name"}\NormalTok{) }\OtherTok{=} \StringTok{"Components"}

\CommentTok{\# and make group plots of different types}
\FunctionTok{par}\NormalTok{(}\AttributeTok{mfrow =} \FunctionTok{c}\NormalTok{(}\DecValTok{2}\NormalTok{, }\DecValTok{2}\NormalTok{))}
\FunctionTok{mdaplotg}\NormalTok{(p, }\AttributeTok{type =} \StringTok{"l"}\NormalTok{)}
\FunctionTok{mdaplotg}\NormalTok{(p, }\AttributeTok{type =} \StringTok{"b"}\NormalTok{)}
\FunctionTok{mdaplotg}\NormalTok{(p, }\AttributeTok{type =} \StringTok{"h"}\NormalTok{, }\AttributeTok{xticks =} \DecValTok{1}\SpecialCharTok{:}\DecValTok{4}\NormalTok{)}
\FunctionTok{mdaplotg}\NormalTok{(p, }\AttributeTok{type =} \StringTok{"b"}\NormalTok{, }\AttributeTok{lty =} \FunctionTok{c}\NormalTok{(}\DecValTok{1}\NormalTok{, }\DecValTok{2}\NormalTok{, }\DecValTok{1}\NormalTok{), }\AttributeTok{col =} \FunctionTok{c}\NormalTok{(}\StringTok{"red"}\NormalTok{, }\StringTok{"green"}\NormalTok{, }\StringTok{"blue"}\NormalTok{), }\AttributeTok{pch =} \DecValTok{1}\NormalTok{,}
         \AttributeTok{xticks =} \DecValTok{1}\SpecialCharTok{:}\DecValTok{4}\NormalTok{, }\AttributeTok{xticklabels =} \FunctionTok{colnames}\NormalTok{(p))}
\end{Highlighting}
\end{Shaded}

\includegraphics{fundamentos_de_ecologia_numerica_files/figure-latex/unnamed-chunk-107-1.pdf}

\begin{Shaded}
\begin{Highlighting}[]
\FunctionTok{par}\NormalTok{(}\AttributeTok{mfrow =} \FunctionTok{c}\NormalTok{(}\DecValTok{2}\NormalTok{, }\DecValTok{2}\NormalTok{))}
\FunctionTok{mdaplotg}\NormalTok{(p, }\AttributeTok{type =} \StringTok{"l"}\NormalTok{, }\AttributeTok{legend.position =} \StringTok{"top"}\NormalTok{)}
\FunctionTok{mdaplotg}\NormalTok{(p, }\AttributeTok{type =} \StringTok{"b"}\NormalTok{, }\AttributeTok{legend.position =} \StringTok{"bottomleft"}\NormalTok{)}
\FunctionTok{mdaplotg}\NormalTok{(p, }\AttributeTok{type =} \StringTok{"h"}\NormalTok{, }\AttributeTok{legend.position =} \StringTok{"bottom"}\NormalTok{)}
\FunctionTok{mdaplotg}\NormalTok{(p, }\AttributeTok{type =} \StringTok{"b"}\NormalTok{, }\AttributeTok{show.legend =} \ConstantTok{FALSE}\NormalTok{)}
\end{Highlighting}
\end{Shaded}

\includegraphics{fundamentos_de_ecologia_numerica_files/figure-latex/unnamed-chunk-107-2.pdf}

\begin{Shaded}
\begin{Highlighting}[]
\FunctionTok{par}\NormalTok{(}\AttributeTok{mfrow =} \FunctionTok{c}\NormalTok{(}\DecValTok{2}\NormalTok{, }\DecValTok{2}\NormalTok{))}
\FunctionTok{mdaplotg}\NormalTok{(p, }\AttributeTok{type =} \StringTok{"l"}\NormalTok{, }\AttributeTok{show.labels =} \ConstantTok{TRUE}\NormalTok{)}
\FunctionTok{mdaplotg}\NormalTok{(p, }\AttributeTok{type =} \StringTok{"b"}\NormalTok{, }\AttributeTok{show.labels =} \ConstantTok{TRUE}\NormalTok{, }\AttributeTok{labels =} \StringTok{"indices"}\NormalTok{)}
\FunctionTok{mdaplotg}\NormalTok{(p, }\AttributeTok{type =} \StringTok{"h"}\NormalTok{, }\AttributeTok{show.labels =} \ConstantTok{TRUE}\NormalTok{, }\AttributeTok{labels =} \StringTok{"values"}\NormalTok{)}
\FunctionTok{mdaplotg}\NormalTok{(p, }\AttributeTok{type =} \StringTok{"b"}\NormalTok{, }\AttributeTok{show.labels =} \ConstantTok{TRUE}\NormalTok{, }\AttributeTok{labels =} \StringTok{"values"}\NormalTok{)}
\end{Highlighting}
\end{Shaded}

\includegraphics{fundamentos_de_ecologia_numerica_files/figure-latex/unnamed-chunk-107-3.pdf}

\begin{Shaded}
\begin{Highlighting}[]
\CommentTok{\# load data and exclude column with income}
\FunctionTok{data}\NormalTok{(people)}
\NormalTok{people }\OtherTok{=} \FunctionTok{mda.exclcols}\NormalTok{(people, }\StringTok{"Income"}\NormalTok{)}

\CommentTok{\# use values of sex variable to split data into two subsets}
\NormalTok{sex }\OtherTok{=}\NormalTok{ people[, }\StringTok{"Sex"}\NormalTok{]}
\NormalTok{m }\OtherTok{=} \FunctionTok{mda.subset}\NormalTok{(people, }\AttributeTok{subset =}\NormalTok{ sex }\SpecialCharTok{==} \SpecialCharTok{{-}}\DecValTok{1}\NormalTok{)}
\NormalTok{f }\OtherTok{=} \FunctionTok{mda.subset}\NormalTok{(people, }\AttributeTok{subset =}\NormalTok{ sex }\SpecialCharTok{==}  \DecValTok{1}\NormalTok{)}

\CommentTok{\# combine the two subsets into a named list}
\NormalTok{d }\OtherTok{=} \FunctionTok{list}\NormalTok{(}\AttributeTok{male =}\NormalTok{ m, }\AttributeTok{female =}\NormalTok{ f)}

\CommentTok{\# make plots for the list}
\FunctionTok{par}\NormalTok{(}\AttributeTok{mfrow =} \FunctionTok{c}\NormalTok{(}\DecValTok{2}\NormalTok{, }\DecValTok{2}\NormalTok{))}
\FunctionTok{mdaplotg}\NormalTok{(d, }\AttributeTok{type =} \StringTok{"p"}\NormalTok{)}
\FunctionTok{mdaplotg}\NormalTok{(d, }\AttributeTok{type =} \StringTok{"b"}\NormalTok{)}
\FunctionTok{mdaplotg}\NormalTok{(d, }\AttributeTok{type =} \StringTok{"h"}\NormalTok{)}
\FunctionTok{mdaplotg}\NormalTok{(d, }\AttributeTok{type =} \StringTok{"b"}\NormalTok{, }\AttributeTok{lty =} \FunctionTok{c}\NormalTok{(}\DecValTok{1}\NormalTok{, }\DecValTok{2}\NormalTok{), }\AttributeTok{col =} \FunctionTok{c}\NormalTok{(}\StringTok{"red"}\NormalTok{, }\StringTok{"blue"}\NormalTok{), }\AttributeTok{pch =} \DecValTok{1}\NormalTok{)}
\end{Highlighting}
\end{Shaded}

\includegraphics{fundamentos_de_ecologia_numerica_files/figure-latex/unnamed-chunk-107-4.pdf}

\begin{Shaded}
\begin{Highlighting}[]
\NormalTok{sex }\OtherTok{=} \FunctionTok{factor}\NormalTok{(people[, }\StringTok{"Sex"}\NormalTok{], }\AttributeTok{labels =} \FunctionTok{c}\NormalTok{(}\StringTok{"M"}\NormalTok{, }\StringTok{"F"}\NormalTok{))}
\NormalTok{reg }\OtherTok{=} \FunctionTok{factor}\NormalTok{(people[, }\StringTok{"Region"}\NormalTok{], }\AttributeTok{labels =} \FunctionTok{c}\NormalTok{(}\StringTok{"S"}\NormalTok{, }\StringTok{"M"}\NormalTok{))}
\NormalTok{groups }\OtherTok{=} \FunctionTok{data.frame}\NormalTok{(sex, reg)}

\FunctionTok{par}\NormalTok{(}\AttributeTok{mfrow =} \FunctionTok{c}\NormalTok{(}\DecValTok{2}\NormalTok{, }\DecValTok{2}\NormalTok{))}
\FunctionTok{mdaplotg}\NormalTok{(people, }\AttributeTok{type =} \StringTok{"p"}\NormalTok{, }\AttributeTok{groupby =}\NormalTok{ groups)}
\FunctionTok{mdaplotg}\NormalTok{(people, }\AttributeTok{type =} \StringTok{"l"}\NormalTok{, }\AttributeTok{groupby =}\NormalTok{ groups)}
\FunctionTok{mdaplotg}\NormalTok{(people, }\AttributeTok{type =} \StringTok{"b"}\NormalTok{, }\AttributeTok{groupby =}\NormalTok{ groups)}
\FunctionTok{mdaplotg}\NormalTok{(people, }\AttributeTok{type =} \StringTok{"h"}\NormalTok{, }\AttributeTok{groupby =}\NormalTok{ groups)}
\end{Highlighting}
\end{Shaded}

\includegraphics{fundamentos_de_ecologia_numerica_files/figure-latex/unnamed-chunk-107-5.pdf}

\hypertarget{apuxeandices-7}{%
\section*{Apêndices}\label{apuxeandices-7}}
\addcontentsline{toc}{section}{Apêndices}

\hypertarget{sites-consultados-5}{%
\section*{Sites consultados}\label{sites-consultados-5}}
\addcontentsline{toc}{section}{Sites consultados}

\url{https://www.geeksforgeeks.org/break-axis-of-plot-in-r/}
\url{https://stackoverflow.com/questions/72498597/add-a-break-to-a-bar-chart-when-one-value-is-very-large}

\hypertarget{script-limpo-4}{%
\section*{Script limpo}\label{script-limpo-4}}
\addcontentsline{toc}{section}{Script limpo}

Aqui apresento o scrip na íntegra sem os textos ou outros comentários. Você pode copiar e colar no R para executa-lo. Lembre de remover os {\texttt{\#}} ou {\texttt{\#\#}} caso necessite executar essas linhas.

\hypertarget{referuxeancias-8}{%
\section*{Referências}\label{referuxeancias-8}}
\addcontentsline{toc}{section}{Referências}

\hypertarget{estrcom}{%
\chapter{R Modulo 4.1 - Estrutura da Comunidade}\label{estrcom}}

\hypertarget{resumo-8}{%
\subsection*{RESUMO}\label{resumo-8}}
\addcontentsline{toc}{subsection}{RESUMO}

Na ecologia de comunidades, a diversidade de espécies é uma medida importante para entender a complexidade e a estrutura de uma comunidade de organismos. Vamos explorar algumas das métricas comuns usadas para avaliar a diversidade de espécies.

\hypertarget{apresentauxe7uxe3o-8}{%
\subsection*{Apresentação}\label{apresentauxe7uxe3o-8}}
\addcontentsline{toc}{subsection}{Apresentação}

Na ecologia de comunidades, a diversidade de espécies é uma medida importante para entender a complexidade e a estrutura de uma comunidade de organismos. Vamos explorar algumas das métricas comuns usadas para avaliar a diversidade de espécies. Essas métricas são usadas para avaliar a diversidade e a estrutura de comunidades ecológicas. Elas podem fornecer informações importantes sobre como as diferentes espécies interagem em um ecossistema e como a diversidade de espécies pode ser afetada por mudanças ambientais ou distúrbios (\protect\hyperlink{ref-RN357}{MAGURRAN, 1988}).

\hypertarget{sobre-os-dados-3}{%
\section{Sobre os dados}\label{sobre-os-dados-3}}

Usaremos para esse tutorial dois conjuntos de dados. A \textbf{Matriz comunitária} (\href{https://docs.google.com/spreadsheets/d/1Olyew6L3SCFeSXWJC54Ak3MKmgHD3sTK/edit?usp=drive_link\&ouid=101766125969589673518\&rtpof=true\&sd=true}{ppbio06c-peixes.xlsx}) de dados coletados no Programa de Pesquisa em Biodiversidade - PPBio (Veja \href{https://ppbio.inpa.gov.br/Sobre}{Programa de Pesquisa em Biodiversidade -- PPBio}). Esses são dados de espécies de peixes distribuidas em diversas unidades amostrais (UA's ou sítios). Essa é a \textbf{matriz bruta de dados}, porque os valores ainda não foram ajustados para os valores de Captura Por Unidade de Esforço (CPUE), nem foram relativizados ou transformados.

Além disso usaremos a \textbf{tabela de agrupamentos} (\href{https://docs.google.com/spreadsheets/d/1H52eBBxZ1k-6Xgrse87etq6mq3mMUTZi/edit?usp=drive_link\&ouid=101766125969589673518\&rtpof=true\&sd=true}{ppbio06-grupos})

\textbf{Revise as informações sobre as bases de dados no Capítulo \ref{bases}. A matriz de dados para esse Módulo pode ser baixada na Seção \ref{arqs}.}

\hypertarget{organizauxe7uxe3o-buxe1sica-5}{%
\section{Organização básica}\label{organizauxe7uxe3o-buxe1sica-5}}

\begin{Shaded}
\begin{Highlighting}[]
\FunctionTok{dev.off}\NormalTok{() }\CommentTok{\#apaga os graficos, se houver algum}
\FunctionTok{rm}\NormalTok{(}\AttributeTok{list=}\FunctionTok{ls}\NormalTok{(}\AttributeTok{all=}\ConstantTok{TRUE}\NormalTok{)) }\CommentTok{\#limpa a memória}
\FunctionTok{cat}\NormalTok{(}\StringTok{"}\SpecialCharTok{\textbackslash{}014}\StringTok{"}\NormalTok{) }\CommentTok{\#limpa o console }
\end{Highlighting}
\end{Shaded}

Instalando os pacotes necessários para esse módulo.

\begin{Shaded}
\begin{Highlighting}[]
\FunctionTok{install.packages}\NormalTok{(}\StringTok{"vegan"}\NormalTok{)}
\FunctionTok{install.packages}\NormalTok{(}\StringTok{"moments"}\NormalTok{)}
\FunctionTok{install.packages}\NormalTok{(}\StringTok{"ggplot2"}\NormalTok{)}
\FunctionTok{install.packages}\NormalTok{(}\StringTok{"dplyr"}\NormalTok{)}
\FunctionTok{install.packages}\NormalTok{(}\StringTok{"tidyr"}\NormalTok{)}
\FunctionTok{install.packages}\NormalTok{(}\StringTok{"tibble"}\NormalTok{)}
\FunctionTok{install.packages}\NormalTok{(}\StringTok{"tidyverse"}\NormalTok{) }\CommentTok{\#atente para alguma msg de erro qdo executar essa linha}
\FunctionTok{install.packages}\NormalTok{(}\StringTok{"forcats"}\NormalTok{)}
\FunctionTok{install.packages}\NormalTok{(}\StringTok{"iNEXT"}\NormalTok{)}
\FunctionTok{install.packages}\NormalTok{(}\StringTok{"openxlsx"}\NormalTok{)}
\FunctionTok{install.packages}\NormalTok{(}\StringTok{"gt"}\NormalTok{)}
\end{Highlighting}
\end{Shaded}

Depois de instalados, carregue os pacotes a seguir no seu computador.

\begin{Shaded}
\begin{Highlighting}[]
\FunctionTok{library}\NormalTok{(tibble); }\FunctionTok{library}\NormalTok{(tidyverse); }\FunctionTok{library}\NormalTok{(forcats); }\FunctionTok{library}\NormalTok{(openxlsx); }\FunctionTok{library}\NormalTok{(Rcpp)}
\end{Highlighting}
\end{Shaded}

Os códigos acima, são usados para instalar e carregar os pacotes necessários para este módulo. Esses códigos são comandos para instalar pacotes no R. Um pacote é uma coleção de funções, dados e documentação que ampliam as capacidades do R (\href{https://cran.r-project.org/}{R CRAN}, (\protect\hyperlink{ref-RN2774}{TEAM, 2017})) e \href{https://posit.co/download/rstudio-desktop/}{RStudio} (\protect\hyperlink{ref-RN358}{TEAM, 2022})). No exemplo acima, o pacote \texttt{openxlsx} permite ler e escrever arquivos Excel no R. Para instalar um pacote no R, você precisa usar a função \texttt{install.packages()}.

Depois de instalar um pacote, você precisa carregá-lo na sua sessão R com a função \texttt{library()}. Por exemplo, para carregar o pacote \texttt{openxlsx}, você precisa executar a função \texttt{library(openxlsx)}. Isso irá permitir que você use as funções do pacote na sua sessão R. Você precisa carregar um pacote toda vez que iniciar uma nova sessão R e quiser usar um pacote instalado.

Agora vamos \textbf{definir o diretório de trabalho}. Esse código é usado para obter e definir o diretório de trabalho atual no R. O comando \texttt{getwd()} retorna o caminho do diretório onde o R está lendo e salvando arquivos. O comando \texttt{setwd()} muda esse diretório de trabalho para o caminho especificado entre aspas. No seu caso, você deve ajustar o caminho para o seu próprio diretório de trabalho. \textbf{Lembre de usar a barra ``/'' entre os diretórios. E não a contra-barra ``\textbackslash{}''.}

\begin{Shaded}
\begin{Highlighting}[]
\FunctionTok{getwd}\NormalTok{()}
\FunctionTok{setwd}\NormalTok{(}\StringTok{"C:/Seu/Diretório/De/Trabalho"}\NormalTok{)}
\end{Highlighting}
\end{Shaded}

\hypertarget{importando-a-planilha-5}{%
\section{Importando a planilha}\label{importando-a-planilha-5}}

Note que o símbolo {\texttt{\#}} em programação R significa que o texto que vem depois dele é um comentário e não será executado pelo programa. Isso é útil para explicar o código ou deixar anotações.\\
- Ajuste a primeira linha do código abaixo para refletir {``C:/Seu/Diretório/De/Trabalho/Planilha.xlsx''}.\\
- Ajuste o parâmetro \texttt{sheet\ =\ "Sheet1"} para refletir a aba correta do arquivo .xlsx a ser importado.

\begin{Shaded}
\begin{Highlighting}[]
\CommentTok{\#dir \textless{}{-} getwd() \#criamos um vetor com o diretório  de trbalho }
\CommentTok{\#shell.exec(dir) \#abre o diretorio de trabalho no Windows Explorer}
\NormalTok{ppbio }\OtherTok{\textless{}{-}} \FunctionTok{read.xlsx}\NormalTok{(}\StringTok{"D:/Elvio/OneDrive/Disciplinas/\_EcoNumerica/5.Matrizes/ppbio06{-}peixes.xlsx"}\NormalTok{,}
                   \AttributeTok{rowNames =}\NormalTok{ T, }\AttributeTok{colNames =}\NormalTok{ T,}
                   \AttributeTok{sheet =} \StringTok{"Sheet1"}\NormalTok{)}
\FunctionTok{str}\NormalTok{(ppbio)}
\CommentTok{\#View(ppbio)}
\NormalTok{ppbio[}\DecValTok{1}\SpecialCharTok{:}\DecValTok{5}\NormalTok{,}\DecValTok{1}\SpecialCharTok{:}\DecValTok{5}\NormalTok{] }\CommentTok{\#[1:5,1:5] mostra apenas as linhas e colunas de 1 a 5.}
\end{Highlighting}
\end{Shaded}

\begin{verbatim}
## 'data.frame':    26 obs. of  35 variables:
##  $ ap-davis : num  0 0 0 0 0 0 0 0 0 0 ...
##  $ as-bimac : num  1 99 194 19 23 142 5 46 206 16 ...
##  $ as-fasci : num  0 0 55 0 1 3 1 0 64 0 ...
##  $ ch-bimac : num  0 0 0 0 13 3 0 178 0 0 ...
##  $ ci-ocela : num  0 0 0 0 0 0 40 0 0 13 ...
##  $ ci-orien : num  0 0 5 0 0 69 9 0 25 24 ...
##  $ co-macro : num  0 0 0 0 0 0 0 0 0 0 ...
##  $ co-heter : num  0 0 1 0 0 0 0 0 0 0 ...
##  $ cr-menez : num  0 0 14 0 0 4 0 0 8 0 ...
##  $ cu-lepid : num  0 0 0 0 0 0 0 0 0 0 ...
##  $ cy-gilbe : num  0 0 0 0 0 0 0 0 0 0 ...
##  $ ge-brasi : num  0 0 3 0 0 0 0 0 1 0 ...
##  $ he-margi : num  0 0 0 0 0 1 0 0 0 0 ...
##  $ ho-malab : num  0 0 1 5 0 17 10 2 31 4 ...
##  $ hy-pusar : num  0 0 9 2 0 43 2 0 11 0 ...
##  $ le-melan : num  0 0 0 0 0 0 0 0 0 0 ...
##  $ le-piau  : num  0 0 3 0 0 1 3 0 2 1 ...
##  $ le-taeni : num  0 0 0 0 0 0 0 0 0 0 ...
##  $ mo-costa : num  0 0 0 0 0 0 0 0 0 0 ...
##  $ mo-lepid : num  0 1 39 0 0 1 0 0 0 0 ...
##  $ or-nilot : num  0 2 36 0 0 77 0 0 138 0 ...
##  $ pa-manag : num  0 0 0 0 0 0 0 0 0 0 ...
##  $ pimel-sp : num  0 0 6 0 0 0 0 0 0 0 ...
##  $ po-retic : num  0 0 0 0 0 20 0 0 5 0 ...
##  $ po-vivip : num  0 0 47 15 0 221 32 0 326 10 ...
##  $ pr-brevi : num  9 0 5 0 1 15 5 2 164 0 ...
##  $ ps-rhomb : num  0 0 0 0 0 0 0 0 1 0 ...
##  $ ps-genise: num  0 0 0 0 0 0 0 0 1 0 ...
##  $ se-heter : num  0 0 40 14 4 60 0 0 38 0 ...
##  $ se-piaba : num  0 0 68 0 0 0 0 0 0 0 ...
##  $ se-spilo : num  0 0 0 0 0 0 0 0 1 0 ...
##  $ st-noton : num  0 0 1 0 0 25 0 0 115 0 ...
##  $ sy-marmo : num  0 0 0 0 0 0 1 0 0 0 ...
##  $ te-chalc : num  0 0 0 0 0 0 0 0 0 0 ...
##  $ tr-signa : num  0 0 18 0 0 15 0 0 7 0 ...
##         ap-davis as-bimac as-fasci ch-bimac ci-ocela
## S-A-ZA1        0        1        0        0        0
## S-R-CC1        0       99        0        0        0
## S-R-CT1        0      194       55        0        0
## S-R-CP1        0       19        0        0        0
## S-A-TA1        0       23        1       13        0
\end{verbatim}

Exibindo os dados importados (esses comando são ``case-sensitive'' \texttt{ignore.case(object)}).

\begin{Shaded}
\begin{Highlighting}[]
\CommentTok{\#View(ppbio)}
\FunctionTok{print}\NormalTok{(ppbio[}\DecValTok{1}\SpecialCharTok{:}\DecValTok{8}\NormalTok{,}\DecValTok{1}\SpecialCharTok{:}\DecValTok{8}\NormalTok{])}
\NormalTok{ppbio[}\DecValTok{1}\SpecialCharTok{:}\DecValTok{10}\NormalTok{,}\DecValTok{1}\SpecialCharTok{:}\DecValTok{10}\NormalTok{]}
\FunctionTok{str}\NormalTok{(ppbio)}
\FunctionTok{mode}\NormalTok{(ppbio)}
\FunctionTok{class}\NormalTok{(ppbio)}
\end{Highlighting}
\end{Shaded}

\hypertarget{reset-point-1}{%
\section{Reset point}\label{reset-point-1}}

\begin{Shaded}
\begin{Highlighting}[]
\NormalTok{m\_trab }\OtherTok{\textless{}{-}}\NormalTok{ ppbio}
\CommentTok{\#pat \textless{}{-} "\^{}S"}
\CommentTok{\#m\_trab \textless{}{-} m\_trab[!grepl(pat, rownames(m\_trab)), ] \#exclui quem começa com pat}
\end{Highlighting}
\end{Shaded}

Aqui cria-se um novo objeto do R (\texttt{m\_trab}, ou a matriz de trabalho, para esse momento) que substitui a matriz de dados original, por uma nova matriz que pode ser a matriz relativizada, transformada, transposta, etc. Dessa forma, mantemos a matriz de dados original caso precisemos dela novamente (Veja a Tabela \ref{tab:200arqs}).

\textbf{Revise Seção \ref{abrev} e a Tabela de Abreviações (@ref(tab:200m\_2)) na mesma Seção que resumem os tipos de matrizes e suas abreviações, para os nossos códigos.}

\hypertarget{transpor-a-matriz-para-trabalhar-com-as-espuxe9cies}{%
\section{Transpor a matriz para trabalhar com as espécies}\label{transpor-a-matriz-para-trabalhar-com-as-espuxe9cies}}

Vamos transpor a matriz para trabalharmos com as espécies. A função \texttt{t} transpõe a matriz. Só deve ser usada uma vez, pois se repetida com \texttt{Ctrl+Enter} continua ``girando'' a matriz. As espécies como colunas representam uma matriz comunitária e as espécies como linhas representam uma matriz (comunitária) transposta.

\begin{Shaded}
\begin{Highlighting}[]
\NormalTok{m\_trab }\OtherTok{\textless{}{-}} \FunctionTok{t}\NormalTok{(ppbio)}
\FunctionTok{str}\NormalTok{(m\_trab)}
\CommentTok{\#View(m\_trab)}
\NormalTok{m\_trab}
\FunctionTok{print}\NormalTok{(m\_trab[}\DecValTok{1}\SpecialCharTok{:}\DecValTok{5}\NormalTok{,}\DecValTok{1}\SpecialCharTok{:}\DecValTok{5}\NormalTok{])}
\NormalTok{m\_trab[}\DecValTok{1}\SpecialCharTok{:}\DecValTok{5}\NormalTok{,}\DecValTok{1}\SpecialCharTok{:}\DecValTok{5}\NormalTok{]}
\FunctionTok{str}\NormalTok{(m\_trab)}
\FunctionTok{mode}\NormalTok{(m\_trab)}
\FunctionTok{class}\NormalTok{(m\_trab)}
\end{Highlighting}
\end{Shaded}

\begin{verbatim}
##  num [1:35, 1:26] 0 1 0 0 0 0 0 0 0 0 ...
##  - attr(*, "dimnames")=List of 2
##   ..$ : chr [1:35] "ap-davis" "as-bimac" "as-fasci" "ch-bimac" ...
##   ..$ : chr [1:26] "S-A-ZA1" "S-R-CC1" "S-R-CT1" "S-R-CP1" ...
##           S-A-ZA1 S-R-CC1 S-R-CT1 S-R-CP1 S-A-TA1 S-R-CT2 S-R-CP2 S-A-TA2
## ap-davis        0       0       0       0       0       0       0       0
## as-bimac        1      99     194      19      23     142       5      46
## as-fasci        0       0      55       0       1       3       1       0
## ch-bimac        0       0       0       0      13       3       0     178
## ci-ocela        0       0       0       0       0       0      40       0
## ci-orien        0       0       5       0       0      69       9       0
## co-macro        0       0       0       0       0       0       0       0
## co-heter        0       0       1       0       0       0       0       0
## cr-menez        0       0      14       0       0       4       0       0
## cu-lepid        0       0       0       0       0       0       0       0
## cy-gilbe        0       0       0       0       0       0       0       0
## ge-brasi        0       0       3       0       0       0       0       0
## he-margi        0       0       0       0       0       1       0       0
## ho-malab        0       0       1       5       0      17      10       2
## hy-pusar        0       0       9       2       0      43       2       0
## le-melan        0       0       0       0       0       0       0       0
## le-piau         0       0       3       0       0       1       3       0
## le-taeni        0       0       0       0       0       0       0       0
## mo-costa        0       0       0       0       0       0       0       0
## mo-lepid        0       1      39       0       0       1       0       0
## or-nilot        0       2      36       0       0      77       0       0
## pa-manag        0       0       0       0       0       0       0       0
## pimel-sp        0       0       6       0       0       0       0       0
## po-retic        0       0       0       0       0      20       0       0
## po-vivip        0       0      47      15       0     221      32       0
## pr-brevi        9       0       5       0       1      15       5       2
## ps-rhomb        0       0       0       0       0       0       0       0
## ps-genise       0       0       0       0       0       0       0       0
## se-heter        0       0      40      14       4      60       0       0
## se-piaba        0       0      68       0       0       0       0       0
## se-spilo        0       0       0       0       0       0       0       0
## st-noton        0       0       1       0       0      25       0       0
## sy-marmo        0       0       0       0       0       0       1       0
## te-chalc        0       0       0       0       0       0       0       0
## tr-signa        0       0      18       0       0      15       0       0
##           S-R-CT3 S-R-CP3 S-A-TA3 S-R-CT4 S-R-CP4 S-A-TA4 B-A-MU1 B-R-ET1
## ap-davis        0       0       0       0       0       0       0       0
## as-bimac      206      16     234       0       0     394      12       3
## as-fasci       64       0       7       1       0       0       0       0
## ch-bimac        0       0     238       0       0     273       0       0
## ci-ocela        0      13       0       0      11       0       0       0
## ci-orien       25      24       0       5       6       0       0       0
## co-macro        0       0       2       0       0       0       0       0
## co-heter        0       0       0       0       0       0       0       0
## cr-menez        8       0       0       1       0       1       0       0
## cu-lepid        0       0       0       0       0       0       0       0
## cy-gilbe        0       0       0      50       0       0       0       0
## ge-brasi        1       0       0       3       0       1     190       0
## he-margi        0       0       0       1       0       0       0       0
## ho-malab       31       4      20       4       2       9       0       0
## hy-pusar       11       0       0       3       0       0       0       0
## le-melan        0       0       0       0       0       0       0       0
## le-piau         2       1       0       0       2       2       0       0
## le-taeni        0       0       0       0       0       0       0       0
## mo-costa        0       0       0       0       0       0       0       0
## mo-lepid        0       0       0       0       0       0       0       0
## or-nilot      138       0       0      73       0       1       6       8
## pa-manag        0       0       0       0       0       0       0       1
## pimel-sp        0       0       0       0       0       0       0       0
## po-retic        5       0       0       0       0       0       0      34
## po-vivip      326      10       0      28      80       0       0       0
## pr-brevi      164       0       0      59       0       3       0       0
## ps-rhomb        1       0       0       0       0       0       0       0
## ps-genise       1       0       0       0       0       0       0       0
## se-heter       38       0       0       3       3       0       0       0
## se-piaba        0       0       0       0       0       0       0       0
## se-spilo        1       0       0       0       0       0       0       0
## st-noton      115       0       0      64       0       0       0       0
## sy-marmo        0       0       0       0       0       0       0       0
## te-chalc        0       0       0       0       0       0       0       0
## tr-signa        7       0       0     141       0       0       0       0
##           B-A-GU1 B-R-PC2 B-A-MU2 B-A-GU2 B-R-PC3 B-A-MU3 B-A-GU3 B-R-PC4
## ap-davis        0       5       0       0      22       0       0       0
## as-bimac        2      44      99       0      75     511       6       7
## as-fasci        2       0       0       0       7       0       0      17
## ch-bimac        0       0       0       0       0       0       0       0
## ci-ocela        0       2       0       0       4       0       0       0
## ci-orien        0       0       0       0       0       0       0       0
## co-macro        0       0       0       0       0       0       0       0
## co-heter        0       0       0       0       0       0       0       0
## cr-menez        0       0       0       0       0       0       0       0
## cu-lepid        0       0       0       0      21       0       0       0
## cy-gilbe        0       0       0       0       0       0       0      81
## ge-brasi        7       8      67      23      16     145      32       5
## he-margi        0       0       0       0       0       0       0       0
## ho-malab        0       0       1       0       2       0       0       1
## hy-pusar        0       0       0       0       1       0       0       0
## le-melan        0       2       0       0       0       0       0       0
## le-piau         0       0       0       0       0       0       0       1
## le-taeni        0       1       0       0       0       0       0       0
## mo-costa        0       0       0       0       1       0       0       0
## mo-lepid        0       0       0       0       0       0       0       0
## or-nilot        3       5       1      36      65      11     247       9
## pa-manag       11       0       0     102       0       0     250       0
## pimel-sp        0       0       0       0       0       0       0       0
## po-retic        0       0      10       0       0      46       0       0
## po-vivip        0       0       8       0       0      48       0       0
## pr-brevi        0       9       0       0       6       1       0       0
## ps-rhomb        0       0       0       0       0       0       0       0
## ps-genise       0       0       0       0       0       0       0       0
## se-heter        0      10       0       0      93       0       0      31
## se-piaba        0       0       0       0       0       0       0       0
## se-spilo        0       0       0       0       0       0       0       0
## st-noton        0       0       0       0       0       0       0       0
## sy-marmo        0       0       0       0       0       0       0       0
## te-chalc        0      76       0       0      58       0       0       0
## tr-signa        0      23       0       0       0       0       0       4
##           B-A-MU4 B-A-GU4
## ap-davis        0       0
## as-bimac      235      13
## as-fasci        0       0
## ch-bimac        0       0
## ci-ocela        0       0
## ci-orien        0       0
## co-macro        0       0
## co-heter        0       0
## cr-menez        0       0
## cu-lepid        0       0
## cy-gilbe        0       0
## ge-brasi      509      10
## he-margi        0       0
## ho-malab        0       0
## hy-pusar        0       0
## le-melan        0       0
## le-piau         0       0
## le-taeni        0       0
## mo-costa        0       0
## mo-lepid        0       0
## or-nilot        1     129
## pa-manag        0     190
## pimel-sp        0       0
## po-retic      266       0
## po-vivip      163       0
## pr-brevi        0       0
## ps-rhomb        0       0
## ps-genise       0       0
## se-heter        0       0
## se-piaba        0       0
## se-spilo        0       0
## st-noton        0       0
## sy-marmo        0       0
## te-chalc        0       0
## tr-signa        0       0
##          S-A-ZA1 S-R-CC1 S-R-CT1 S-R-CP1 S-A-TA1
## ap-davis       0       0       0       0       0
## as-bimac       1      99     194      19      23
## as-fasci       0       0      55       0       1
## ch-bimac       0       0       0       0      13
## ci-ocela       0       0       0       0       0
##          S-A-ZA1 S-R-CC1 S-R-CT1 S-R-CP1 S-A-TA1
## ap-davis       0       0       0       0       0
## as-bimac       1      99     194      19      23
## as-fasci       0       0      55       0       1
## ch-bimac       0       0       0       0      13
## ci-ocela       0       0       0       0       0
##  num [1:35, 1:26] 0 1 0 0 0 0 0 0 0 0 ...
##  - attr(*, "dimnames")=List of 2
##   ..$ : chr [1:35] "ap-davis" "as-bimac" "as-fasci" "ch-bimac" ...
##   ..$ : chr [1:26] "S-A-ZA1" "S-R-CC1" "S-R-CT1" "S-R-CP1" ...
## [1] "numeric"
## [1] "matrix" "array"
\end{verbatim}

\hypertarget{informauxe7uxf5es-buxe1sicas-da-matriz}{%
\subsection{Informações básicas da matriz}\label{informauxe7uxf5es-buxe1sicas-da-matriz}}

Agora podemos pedir ao R as informações básicas da matriz de trabalho (\texttt{m\_trab}), como o número de observações ou tamanho do vetor (depende do tipo da matriz), número de observações igual a zero, número de observaçõoes maiores que zero e proporção de zeros na matriz.

\begin{Shaded}
\begin{Highlighting}[]
\FunctionTok{range}\NormalTok{(m\_trab) }\CommentTok{\#menor e maior valores}
\FunctionTok{length}\NormalTok{(m\_trab) }\CommentTok{\#no. de colunas}
\FunctionTok{ncol}\NormalTok{(m\_trab) }\CommentTok{\#no. de N colunas}
\FunctionTok{nrow}\NormalTok{(m\_trab) }\CommentTok{\#no. de M linhas}
\FunctionTok{sum}\NormalTok{(}\FunctionTok{lengths}\NormalTok{(m\_trab)) }\CommentTok{\#soma os nos. de colunas}
\FunctionTok{length}\NormalTok{(}\FunctionTok{as.matrix}\NormalTok{(m\_trab)) }\CommentTok{\#tamanho da matriz m x n}
\FunctionTok{sum}\NormalTok{(m\_trab }\SpecialCharTok{==} \DecValTok{0}\NormalTok{) }\CommentTok{\#número de observações igual a zero}
\FunctionTok{sum}\NormalTok{(m\_trab }\SpecialCharTok{\textgreater{}} \DecValTok{0}\NormalTok{) }\CommentTok{\#número de observações maiores que zero}
\CommentTok{\#calculando a proporção de zeros na matriz}
\NormalTok{zeros }\OtherTok{\textless{}{-}}\NormalTok{ (}\FunctionTok{sum}\NormalTok{(m\_trab }\SpecialCharTok{==} \DecValTok{0}\NormalTok{)}\SpecialCharTok{/}\FunctionTok{length}\NormalTok{(}\FunctionTok{as.matrix}\NormalTok{(m\_trab)))}\SpecialCharTok{*}\DecValTok{100}
\NormalTok{zeros}
\end{Highlighting}
\end{Shaded}

\begin{verbatim}
## [1]   0 511
## [1] 910
## [1] 26
## [1] 35
## [1] 910
## [1] 910
## [1] 716
## [1] 194
## [1] 78.68132
\end{verbatim}

Tabela que resume as informações geradas (Tabela \ref{tab:241tam}).

\begin{verbatim}
##     Comando Resultado
## 1     range   0 - 511
## 2    lenght       910
## 3    n cols        26
## 4  m linhas        35
## 5   Tamanho       910
## 6   Tamanho       910
## 7     Zeros       716
## 8 Nao zeros       194
## 9   % Zeros      78.7
\end{verbatim}

\begin{longtable}[]{@{}ll@{}}
\caption{\label{tab:241tam}Resumo das informações sobre o tamanho da base de dados.}\tabularnewline
\toprule()
Comando & Resultado \\
\midrule()
\endfirsthead
\toprule()
Comando & Resultado \\
\midrule()
\endhead
range & 0 - 511 \\
lenght & 910 \\
n cols & 26 \\
m linhas & 35 \\
Tamanho & 910 \\
Tamanho & 910 \\
Zeros & 716 \\
Nao zeros & 194 \\
\% Zeros & 78.7 \\
\bottomrule()
\end{longtable}

Ou seja, temos uma matriz de tamanho \texttt{m\ x\ n} igual a 35 objetos por 26 atributos, onde 78.68\% dos valores da matriz são iguais a zero!

Agora de conhecimento dessas informações básicas podemos calcular os primeiros descritores da estrutura da comunidade a ser estudada.

\hypertarget{calculando-os-descritores-da-comunidade}{%
\section{Calculando os descritores da comunidade}\label{calculando-os-descritores-da-comunidade}}

Entre outras métricas, calcularemos os seguntes índices:

\begin{enumerate}
\def\labelenumi{\arabic{enumi}.}
\tightlist
\item
  \textbf{Riqueza de Espécies}:\\
\end{enumerate}

\begin{itemize}
\tightlist
\item
  A riqueza de espécies simplesmente se refere ao número total de espécies diferentes em uma comunidade. É uma medida fundamental da diversidade ecológica e reflete a variedade de formas de vida coexistentes em um ecossistema. Comunidades com alta riqueza de espécies têm um grande número de espécies diferentes, enquanto comunidades com baixa riqueza têm menos espécies.
\end{itemize}

\begin{enumerate}
\def\labelenumi{\arabic{enumi}.}
\setcounter{enumi}{1}
\tightlist
\item
  \textbf{Índice de Diversidade de Simpson}:\\
\end{enumerate}

\begin{itemize}
\tightlist
\item
  O índice de diversidade de Simpson (ou índice de Simpson) mede a probabilidade de escolher aleatoriamente duas vezes o mesmo indivíduo de uma comunidade. Quanto mais próximo de 1 for o índice de Simpson, menor é a diversidade, indicando que uma ou algumas espécies dominam a comunidade. Quanto mais próximo de 0 for o índice de Simpson, maior é a diversidade, indicando uma comunidade mais equilibrada.
\end{itemize}

\begin{enumerate}
\def\labelenumi{\arabic{enumi}.}
\setcounter{enumi}{2}
\tightlist
\item
  \textbf{Índice de Diversidade de Shannon-Wiener}:\\
\end{enumerate}

\begin{itemize}
\tightlist
\item
  O índice de Shannon-Wiener (ou índice de Shannon) leva em consideração a riqueza de espécies e a equitabilidade (distribuição uniforme das abundâncias das espécies). Ele mede a incerteza associada à identificação de uma espécie aleatória em uma comunidade. Quanto maior o índice de Shannon, maior é a diversidade, pois indica uma comunidade com várias espécies bem distribuídas em termos de abundância.
\end{itemize}

\begin{enumerate}
\def\labelenumi{\arabic{enumi}.}
\setcounter{enumi}{3}
\tightlist
\item
  \textbf{Equitabilidade}:
\end{enumerate}

\begin{itemize}
\tightlist
\item
  A equitabilidade é uma medida que avalia o quão uniformemente as abundâncias das diferentes espécies estão distribuídas em uma comunidade. Quanto maior a equitabilidade, mais igual é a distribuição das abundâncias, o que indica uma comunidade mais equilibrada.
\end{itemize}

\begin{enumerate}
\def\labelenumi{\arabic{enumi}.}
\setcounter{enumi}{4}
\tightlist
\item
  \textbf{Abundância}:
\end{enumerate}

\begin{itemize}
\tightlist
\item
  A abundância se refere ao número total de indivíduos de uma espécie em uma comunidade. É uma medida simples que indica quantos indivíduos de uma espécie específica estão presentes na comunidade.
\end{itemize}

\begin{enumerate}
\def\labelenumi{\arabic{enumi}.}
\setcounter{enumi}{5}
\tightlist
\item
  \textbf{Abundância Relativa}:
\end{enumerate}

\begin{itemize}
\tightlist
\item
  A abundância relativa é a proporção ou a fração da abundância de uma espécie em relação à abundância total de todas as espécies na comunidade. É uma medida que ajuda a entender a importância relativa de cada espécie na comunidade.
\end{itemize}

\begin{enumerate}
\def\labelenumi{\arabic{enumi}.}
\setcounter{enumi}{6}
\tightlist
\item
  \textbf{Dominância de Espécies}:
\end{enumerate}

\begin{itemize}
\tightlist
\item
  A dominância de espécies se refere à presença de uma ou algumas espécies que têm uma abundância significativamente maior do que as outras na comunidade. Comunidades com alta dominância são frequentemente menos diversas, pois algumas espécies dominantes podem suprimir o crescimento de outras.
\end{itemize}

\hypertarget{variabilidade}{%
\subsection{Variabilidade}\label{variabilidade}}

Primeiro a variabilidade estatística

\begin{Shaded}
\begin{Highlighting}[]
\CommentTok{\#?apply}
\NormalTok{Sum }\OtherTok{\textless{}{-}} \FunctionTok{rowSums}\NormalTok{(m\_trab)}
\CommentTok{\#ou}
\NormalTok{Sum }\OtherTok{\textless{}{-}} \FunctionTok{apply}\NormalTok{(m\_trab,}\DecValTok{1}\NormalTok{,sum)}
\NormalTok{Sum}
\DocumentationTok{\#\# Abundância relativa (\%)}
\NormalTok{RA }\OtherTok{\textless{}{-}}\NormalTok{ (Sum }\SpecialCharTok{/} \FunctionTok{sum}\NormalTok{(Sum)) }\SpecialCharTok{*} \DecValTok{100} \CommentTok{\# percentage}
\DocumentationTok{\#\# Media}
\NormalTok{Mean }\OtherTok{\textless{}{-}} \FunctionTok{rowMeans}\NormalTok{(m\_trab)}
\NormalTok{Mean}
\DocumentationTok{\#\# Ou}
\NormalTok{Mean }\OtherTok{\textless{}{-}} \FunctionTok{apply}\NormalTok{(m\_trab,}\DecValTok{1}\NormalTok{,mean)}
\NormalTok{Mean}
\DocumentationTok{\#\# Desvio padrão}
\NormalTok{DP }\OtherTok{\textless{}{-}} \FunctionTok{apply}\NormalTok{(m\_trab,}\DecValTok{1}\NormalTok{,sd)}
\NormalTok{DP}
\DocumentationTok{\#\# Máximo}
\NormalTok{Max }\OtherTok{\textless{}{-}} \FunctionTok{apply}\NormalTok{(m\_trab,}\DecValTok{1}\NormalTok{,max)}
\NormalTok{Max }
\DocumentationTok{\#\# Mínimo}
\NormalTok{Min }\OtherTok{\textless{}{-}} \FunctionTok{apply}\NormalTok{(m\_trab,}\DecValTok{1}\NormalTok{,min)}
\NormalTok{Min}
\DocumentationTok{\#\# Mínimo não{-}zero}
\NormalTok{MinZ }\OtherTok{\textless{}{-}} \FunctionTok{apply}\NormalTok{(m\_trab, }\DecValTok{1}\NormalTok{, }\ControlFlowTok{function}\NormalTok{(row) \{}
\NormalTok{  non\_zero\_values }\OtherTok{\textless{}{-}}\NormalTok{ row[row }\SpecialCharTok{\textgreater{}} \DecValTok{0}\NormalTok{]  }\CommentTok{\# Filter out zero values}
  \ControlFlowTok{if}\NormalTok{ (}\FunctionTok{length}\NormalTok{(non\_zero\_values) }\SpecialCharTok{==} \DecValTok{0}\NormalTok{) \{}
    \FunctionTok{return}\NormalTok{(}\DecValTok{0}\NormalTok{)  }\CommentTok{\# If all values are zero, return 0}
\NormalTok{  \} }\ControlFlowTok{else}\NormalTok{ \{}
    \FunctionTok{return}\NormalTok{(}\FunctionTok{min}\NormalTok{(non\_zero\_values))  }\CommentTok{\# Return the minimum of non{-}zero values}
\NormalTok{  \}}
\NormalTok{\})}
\NormalTok{MinZ}
\end{Highlighting}
\end{Shaded}

\begin{verbatim}
##  ap-davis  as-bimac  as-fasci  ch-bimac  ci-ocela  ci-orien  co-macro  co-heter 
##        27      2386       158       705        70       143         2         1 
##  cr-menez  cu-lepid  cy-gilbe  ge-brasi  he-margi  ho-malab  hy-pusar  le-melan 
##        28        21       131      1020         2       109        71         2 
##   le-piau  le-taeni  mo-costa  mo-lepid  or-nilot  pa-manag  pimel-sp  po-retic 
##        15         1         1        41       848       554         6       381 
##  po-vivip  pr-brevi  ps-rhomb ps-genise  se-heter  se-piaba  se-spilo  st-noton 
##       978       279         1         1       296        68         1       205 
##  sy-marmo  te-chalc  tr-signa 
##         1       134       208 
##    ap-davis    as-bimac    as-fasci    ch-bimac    ci-ocela    ci-orien 
##  1.03846154 91.76923077  6.07692308 27.11538462  2.69230769  5.50000000 
##    co-macro    co-heter    cr-menez    cu-lepid    cy-gilbe    ge-brasi 
##  0.07692308  0.03846154  1.07692308  0.80769231  5.03846154 39.23076923 
##    he-margi    ho-malab    hy-pusar    le-melan     le-piau    le-taeni 
##  0.07692308  4.19230769  2.73076923  0.07692308  0.57692308  0.03846154 
##    mo-costa    mo-lepid    or-nilot    pa-manag    pimel-sp    po-retic 
##  0.03846154  1.57692308 32.61538462 21.30769231  0.23076923 14.65384615 
##    po-vivip    pr-brevi    ps-rhomb   ps-genise    se-heter    se-piaba 
## 37.61538462 10.73076923  0.03846154  0.03846154 11.38461538  2.61538462 
##    se-spilo    st-noton    sy-marmo    te-chalc    tr-signa 
##  0.03846154  7.88461538  0.03846154  5.15384615  8.00000000 
##    ap-davis    as-bimac    as-fasci    ch-bimac    ci-ocela    ci-orien 
##  1.03846154 91.76923077  6.07692308 27.11538462  2.69230769  5.50000000 
##    co-macro    co-heter    cr-menez    cu-lepid    cy-gilbe    ge-brasi 
##  0.07692308  0.03846154  1.07692308  0.80769231  5.03846154 39.23076923 
##    he-margi    ho-malab    hy-pusar    le-melan     le-piau    le-taeni 
##  0.07692308  4.19230769  2.73076923  0.07692308  0.57692308  0.03846154 
##    mo-costa    mo-lepid    or-nilot    pa-manag    pimel-sp    po-retic 
##  0.03846154  1.57692308 32.61538462 21.30769231  0.23076923 14.65384615 
##    po-vivip    pr-brevi    ps-rhomb   ps-genise    se-heter    se-piaba 
## 37.61538462 10.73076923  0.03846154  0.03846154 11.38461538  2.61538462 
##    se-spilo    st-noton    sy-marmo    te-chalc    tr-signa 
##  0.03846154  7.88461538  0.03846154  5.15384615  8.00000000 
##    ap-davis    as-bimac    as-fasci    ch-bimac    ci-ocela    ci-orien 
##   4.3861671 132.4348316  16.2035134  75.8732242   8.2982853  14.6184815 
##    co-macro    co-heter    cr-menez    cu-lepid    cy-gilbe    ge-brasi 
##   0.3922323   0.1961161   3.1486261   4.1184388  18.3313519 106.3478473 
##    he-margi    ho-malab    hy-pusar    le-melan     le-piau    le-taeni 
##   0.2717465   7.6053625   8.6605205   0.3922323   0.9868364   0.1961161 
##    mo-costa    mo-lepid    or-nilot    pa-manag    pimel-sp    po-retic 
##   0.1961161   7.6376597  59.1452970  62.3928004   1.1766968  52.5242362 
##    po-vivip    pr-brevi    ps-rhomb   ps-genise    se-heter    se-piaba 
##  79.5004790  33.3892889   0.1961161   0.1961161  22.9905666  13.3358972 
##    se-spilo    st-noton    sy-marmo    te-chalc    tr-signa 
##   0.1961161  25.5582893   0.1961161  18.3841068  27.8280434 
##  ap-davis  as-bimac  as-fasci  ch-bimac  ci-ocela  ci-orien  co-macro  co-heter 
##        22       511        64       273        40        69         2         1 
##  cr-menez  cu-lepid  cy-gilbe  ge-brasi  he-margi  ho-malab  hy-pusar  le-melan 
##        14        21        81       509         1        31        43         2 
##   le-piau  le-taeni  mo-costa  mo-lepid  or-nilot  pa-manag  pimel-sp  po-retic 
##         3         1         1        39       247       250         6       266 
##  po-vivip  pr-brevi  ps-rhomb ps-genise  se-heter  se-piaba  se-spilo  st-noton 
##       326       164         1         1        93        68         1       115 
##  sy-marmo  te-chalc  tr-signa 
##         1        76       141 
##  ap-davis  as-bimac  as-fasci  ch-bimac  ci-ocela  ci-orien  co-macro  co-heter 
##         0         0         0         0         0         0         0         0 
##  cr-menez  cu-lepid  cy-gilbe  ge-brasi  he-margi  ho-malab  hy-pusar  le-melan 
##         0         0         0         0         0         0         0         0 
##   le-piau  le-taeni  mo-costa  mo-lepid  or-nilot  pa-manag  pimel-sp  po-retic 
##         0         0         0         0         0         0         0         0 
##  po-vivip  pr-brevi  ps-rhomb ps-genise  se-heter  se-piaba  se-spilo  st-noton 
##         0         0         0         0         0         0         0         0 
##  sy-marmo  te-chalc  tr-signa 
##         0         0         0 
##  ap-davis  as-bimac  as-fasci  ch-bimac  ci-ocela  ci-orien  co-macro  co-heter 
##         5         1         1         3         2         5         2         1 
##  cr-menez  cu-lepid  cy-gilbe  ge-brasi  he-margi  ho-malab  hy-pusar  le-melan 
##         1        21        50         1         1         1         1         2 
##   le-piau  le-taeni  mo-costa  mo-lepid  or-nilot  pa-manag  pimel-sp  po-retic 
##         1         1         1         1         1         1         6         5 
##  po-vivip  pr-brevi  ps-rhomb ps-genise  se-heter  se-piaba  se-spilo  st-noton 
##         8         1         1         1         3        68         1         1 
##  sy-marmo  te-chalc  tr-signa 
##         1        58         4
\end{verbatim}

\hypertarget{riqueza}{%
\subsection{Riqueza}\label{riqueza}}

Atente para o fato de que a riqueza será a frequêcia de ocorrência na matrix transposta. Converte-se primeiro para matriz binária.

\begin{Shaded}
\begin{Highlighting}[]
\NormalTok{m\_pa }\OtherTok{\textless{}{-}}\NormalTok{ m\_trab}
\NormalTok{m\_pa[m\_pa }\SpecialCharTok{!=} \DecValTok{0}\NormalTok{] }\OtherTok{\textless{}{-}} \DecValTok{1}
\FunctionTok{rowSums}\NormalTok{(m\_pa)}
\FunctionTok{library}\NormalTok{(vegan)}
\NormalTok{bin }\OtherTok{\textless{}{-}} \FunctionTok{decostand}\NormalTok{(m\_trab,}\StringTok{"pa"}\NormalTok{)}
\NormalTok{bin[}\DecValTok{1}\SpecialCharTok{:}\DecValTok{10}\NormalTok{, }\DecValTok{1}\SpecialCharTok{:}\DecValTok{10}\NormalTok{]}
\NormalTok{S }\OtherTok{\textless{}{-}} \FunctionTok{apply}\NormalTok{(bin,}\DecValTok{1}\NormalTok{,sum)}
\NormalTok{S}
\CommentTok{\#OU}
\NormalTok{Riqueza }\OtherTok{\textless{}{-}} \FunctionTok{specnumber}\NormalTok{(m\_trab)}
\NormalTok{Riqueza}
\NormalTok{Riqueza\_total }\OtherTok{\textless{}{-}} \FunctionTok{specnumber}\NormalTok{(}\FunctionTok{colSums}\NormalTok{(m\_trab))}
\NormalTok{Riqueza\_total}
\CommentTok{\#OU}
\NormalTok{FO }\OtherTok{\textless{}{-}} \FunctionTok{rowSums}\NormalTok{(m\_trab }\SpecialCharTok{\textgreater{}} \DecValTok{0}\NormalTok{) }\SpecialCharTok{/} \FunctionTok{ncol}\NormalTok{(m\_trab) }\SpecialCharTok{*} \DecValTok{100}
\NormalTok{FO}
\end{Highlighting}
\end{Shaded}

\begin{verbatim}
##  ap-davis  as-bimac  as-fasci  ch-bimac  ci-ocela  ci-orien  co-macro  co-heter 
##         2        23        10         5         5         7         1         1 
##  cr-menez  cu-lepid  cy-gilbe  ge-brasi  he-margi  ho-malab  hy-pusar  le-melan 
##         5         1         2        15         2        14         7         1 
##   le-piau  le-taeni  mo-costa  mo-lepid  or-nilot  pa-manag  pimel-sp  po-retic 
##         8         1         1         3        18         5         1         6 
##  po-vivip  pr-brevi  ps-rhomb ps-genise  se-heter  se-piaba  se-spilo  st-noton 
##        11        12         1         1        10         1         1         4 
##  sy-marmo  te-chalc  tr-signa 
##         1         2         6 
##          S-A-ZA1 S-R-CC1 S-R-CT1 S-R-CP1 S-A-TA1 S-R-CT2 S-R-CP2 S-A-TA2
## ap-davis       0       0       0       0       0       0       0       0
## as-bimac       1       1       1       1       1       1       1       1
## as-fasci       0       0       1       0       1       1       1       0
## ch-bimac       0       0       0       0       1       1       0       1
## ci-ocela       0       0       0       0       0       0       1       0
## ci-orien       0       0       1       0       0       1       1       0
## co-macro       0       0       0       0       0       0       0       0
## co-heter       0       0       1       0       0       0       0       0
## cr-menez       0       0       1       0       0       1       0       0
## cu-lepid       0       0       0       0       0       0       0       0
##          S-R-CT3 S-R-CP3
## ap-davis       0       0
## as-bimac       1       1
## as-fasci       1       0
## ch-bimac       0       0
## ci-ocela       0       1
## ci-orien       1       1
## co-macro       0       0
## co-heter       0       0
## cr-menez       1       0
## cu-lepid       0       0
##  ap-davis  as-bimac  as-fasci  ch-bimac  ci-ocela  ci-orien  co-macro  co-heter 
##         2        23        10         5         5         7         1         1 
##  cr-menez  cu-lepid  cy-gilbe  ge-brasi  he-margi  ho-malab  hy-pusar  le-melan 
##         5         1         2        15         2        14         7         1 
##   le-piau  le-taeni  mo-costa  mo-lepid  or-nilot  pa-manag  pimel-sp  po-retic 
##         8         1         1         3        18         5         1         6 
##  po-vivip  pr-brevi  ps-rhomb ps-genise  se-heter  se-piaba  se-spilo  st-noton 
##        11        12         1         1        10         1         1         4 
##  sy-marmo  te-chalc  tr-signa 
##         1         2         6 
##  ap-davis  as-bimac  as-fasci  ch-bimac  ci-ocela  ci-orien  co-macro  co-heter 
##         2        23        10         5         5         7         1         1 
##  cr-menez  cu-lepid  cy-gilbe  ge-brasi  he-margi  ho-malab  hy-pusar  le-melan 
##         5         1         2        15         2        14         7         1 
##   le-piau  le-taeni  mo-costa  mo-lepid  or-nilot  pa-manag  pimel-sp  po-retic 
##         8         1         1         3        18         5         1         6 
##  po-vivip  pr-brevi  ps-rhomb ps-genise  se-heter  se-piaba  se-spilo  st-noton 
##        11        12         1         1        10         1         1         4 
##  sy-marmo  te-chalc  tr-signa 
##         1         2         6 
## [1] 26
##  ap-davis  as-bimac  as-fasci  ch-bimac  ci-ocela  ci-orien  co-macro  co-heter 
##  7.692308 88.461538 38.461538 19.230769 19.230769 26.923077  3.846154  3.846154 
##  cr-menez  cu-lepid  cy-gilbe  ge-brasi  he-margi  ho-malab  hy-pusar  le-melan 
## 19.230769  3.846154  7.692308 57.692308  7.692308 53.846154 26.923077  3.846154 
##   le-piau  le-taeni  mo-costa  mo-lepid  or-nilot  pa-manag  pimel-sp  po-retic 
## 30.769231  3.846154  3.846154 11.538462 69.230769 19.230769  3.846154 23.076923 
##  po-vivip  pr-brevi  ps-rhomb ps-genise  se-heter  se-piaba  se-spilo  st-noton 
## 42.307692 46.153846  3.846154  3.846154 38.461538  3.846154  3.846154 15.384615 
##  sy-marmo  te-chalc  tr-signa 
##  3.846154  7.692308 23.076923
\end{verbatim}

\hypertarget{uxedndices-de-diversidade}{%
\section{Índices de Diversidade}\label{uxedndices-de-diversidade}}

\hypertarget{shannon}{%
\subsection{Shannon}\label{shannon}}

O índice de diversidade de Shannon ou Shannon--Weaver (or Shannon--Wiener) é uma medida de diversidade que leva em consideração tanto a riqueza de espécies quanto a uniformidade na distribuição dessas espécies (\protect\hyperlink{ref-RN2923}{DIXON, 2003}; \protect\hyperlink{ref-RN2832}{OKSANEN et al., 2020}). Este índice é definido por:

\[
H' = -\sum_{i} p_i\log_{b} p_i
\]
Onde:

\begin{itemize}
\tightlist
\item
  \(H'\) é a entropia de Shannon.
\item
  \(p_i\) é a abundância proporcional da espécie \(i\).
\item
  \(b\) é a base do logaritmo.
\end{itemize}

É mais comum usar-se o logarítmo natural, embora pode-se argumentar que para o logarítmo de base = 2 (o que faz sentido mas nenhuma diferença)

Outra fórmula para calcular o índice de diversidade de Shannon é:

\[
H' = -\sum_{i=1}^{S} \left( \frac{n_i}{N} \times \ln\frac{n_i}{N} \right)
\]

Onde:

\begin{itemize}
\tightlist
\item
  \(H'\) é a entropia de Shannon (ou a diversidade de Shannon).
\item
  \(n_i\) é o número de indivíduos da espécie \(i\).
\item
  \(N\) é o número total de indivíduos na comunidade.
\item
  \(S\) é o número total de espécies na comunidade.
\end{itemize}

Esta fórmula mede a incerteza (ou entropia) na identificação de uma espécie selecionada aleatoriamente da comunidade. Quanto maior o valor de \(H'\), maior a diversidade da comunidade.

\begin{Shaded}
\begin{Highlighting}[]
\NormalTok{H }\OtherTok{\textless{}{-}} \FunctionTok{diversity}\NormalTok{(m\_trab, }\AttributeTok{index =} \StringTok{"shannon"}\NormalTok{)}
\NormalTok{H}
\end{Highlighting}
\end{Shaded}

\begin{verbatim}
##  ap-davis  as-bimac  as-fasci  ch-bimac  ci-ocela  ci-orien  co-macro  co-heter 
## 0.4791656 2.4239775 1.4761308 1.1783625 1.1883832 1.4976803 0.0000000 0.0000000 
##  cr-menez  cu-lepid  cy-gilbe  ge-brasi  he-margi  ho-malab  hy-pusar  le-melan 
## 1.2205076 0.0000000 0.6648803 1.5668772 0.6931472 2.1167591 1.2492728 0.0000000 
##   le-piau  le-taeni  mo-costa  mo-lepid  or-nilot  pa-manag  pimel-sp  po-retic 
## 1.9913464 0.0000000 0.0000000 0.2287207 2.1072815 1.1268689 0.0000000 1.0288585 
##  po-vivip  pr-brevi  ps-rhomb ps-genise  se-heter  se-piaba  se-spilo  st-noton 
## 1.8633952 1.3708151 0.0000000 0.0000000 1.8675926 0.0000000 0.0000000 0.9702899 
##  sy-marmo  te-chalc  tr-signa 
## 0.0000000 0.6840978 1.0985693
\end{verbatim}

\hypertarget{simpson}{%
\subsection{Simpson}\label{simpson}}

O índice de diversidade de Simpson é uma medida de diversidade que leva em consideração a riqueza de espécies e a abundância relativa de cada espécie em uma comunidade. A fórmula para calcular o índice de diversidade de Simpson é:

\[
D = 1 - \sum_{i=1}^{S} \left( \frac{n_i (n_i - 1)}{N (N - 1)} \right)
\]

Onde:

\begin{itemize}
\tightlist
\item
  \(D\) é o índice de diversidade de Simpson.
\item
  \(n_i\) é o número de indivíduos da espécie \(i\).
\item
  \(N\) é o número total de indivíduos na comunidade.
\item
  \(S\) é o número total de espécies na comunidade.
\end{itemize}

Esta fórmula fornece um valor entre 0 e 1, onde 0 indica uma comunidade com apenas uma espécie presente e 1 indica uma comunidade com uma distribuição uniforme de espécies.

Existem duas variantes do índice de Simpson baseado em \(D = \sum p_i^2\) (\protect\hyperlink{ref-RN2923}{DIXON, 2003}; \protect\hyperlink{ref-RN1073}{HURLBERT, 1971}; \protect\hyperlink{ref-RN2832}{OKSANEN et al., 2020}):

\begin{itemize}
\tightlist
\item
  Ao escolhermos \texttt{simpson} o R retorna \(1 - D\), e
\item
  ao escolhermos \texttt{invsimpson} o R retorna \(1/D\) .
\end{itemize}

\begin{Shaded}
\begin{Highlighting}[]
\NormalTok{D }\OtherTok{\textless{}{-}} \FunctionTok{diversity}\NormalTok{(m\_trab, }\StringTok{"simpson"}\NormalTok{)}
\NormalTok{D}
\NormalTok{D[}\FunctionTok{is.na}\NormalTok{(D)] }\OtherTok{\textless{}{-}} \DecValTok{0} \CommentTok{\#substitui NA ou NaN por 0}
\NormalTok{D}
\end{Highlighting}
\end{Shaded}

\begin{verbatim}
##   ap-davis   as-bimac   as-fasci   ch-bimac   ci-ocela   ci-orien   co-macro 
## 0.30178326 0.88451845 0.69860599 0.67197827 0.61020408 0.70027874 0.00000000 
##   co-heter   cr-menez   cu-lepid   cy-gilbe   ge-brasi   he-margi   ho-malab 
## 0.00000000 0.64540816 0.00000000 0.47200047 0.68977124 0.50000000 0.83982830 
##   hy-pusar   le-melan    le-piau   le-taeni   mo-costa   mo-lepid   or-nilot 
## 0.58956556 0.00000000 0.85333333 0.00000000 0.00000000 0.09399167 0.83992302 
##   pa-manag   pimel-sp   po-retic   po-vivip   pr-brevi   ps-rhomb  ps-genise 
## 0.64444343 0.00000000 0.48641164 0.79634160 0.60348659 0.00000000 0.00000000 
##   se-heter   se-piaba   se-spilo   st-noton   sy-marmo   te-chalc   tr-signa 
## 0.81071950 0.00000000 0.00000000 0.57294468 0.00000000 0.49097795 0.51405325 
##   ap-davis   as-bimac   as-fasci   ch-bimac   ci-ocela   ci-orien   co-macro 
## 0.30178326 0.88451845 0.69860599 0.67197827 0.61020408 0.70027874 0.00000000 
##   co-heter   cr-menez   cu-lepid   cy-gilbe   ge-brasi   he-margi   ho-malab 
## 0.00000000 0.64540816 0.00000000 0.47200047 0.68977124 0.50000000 0.83982830 
##   hy-pusar   le-melan    le-piau   le-taeni   mo-costa   mo-lepid   or-nilot 
## 0.58956556 0.00000000 0.85333333 0.00000000 0.00000000 0.09399167 0.83992302 
##   pa-manag   pimel-sp   po-retic   po-vivip   pr-brevi   ps-rhomb  ps-genise 
## 0.64444343 0.00000000 0.48641164 0.79634160 0.60348659 0.00000000 0.00000000 
##   se-heter   se-piaba   se-spilo   st-noton   sy-marmo   te-chalc   tr-signa 
## 0.81071950 0.00000000 0.00000000 0.57294468 0.00000000 0.49097795 0.51405325
\end{verbatim}

\hypertarget{equitabilidade-de-pielou}{%
\subsection{Equitabilidade de Pielou}\label{equitabilidade-de-pielou}}

O índice de equitabilidade de Pielou é uma medida de uniformidade em uma distribuição de espécies (\protect\hyperlink{ref-RN465}{PIELOU, 1975}). Sua fórmula é dada por:

\[
J' = \frac{H'}{\ln(S)}
\]

Onde:

\begin{itemize}
\tightlist
\item
  \(J'\) é o índice de equitabilidade de Pielou.
\item
  \(H'\) é a entropia de Shannon (ou a diversidade de Shannon), que é calculada como \(-\sum_{i=1}^{S} p_i \cdot \ln(p_i)\), onde \(S\) é o número total de espécies e \(p_i\) é a proporção da espécie \(i\).
\item
  \(\ln(S)\) é o logarítmo natural do número total de espécies na comunidade.
\end{itemize}

Esta fórmula fornece um valor entre 0 e 1, onde 0 indica uma distribuição completamente desigual onde uma comunidade apresenta uma única espécie dominante e 1 indica uma distribuição completamente equitativa onde todas as espécies têm a mesma abundância.

\begin{Shaded}
\begin{Highlighting}[]
\NormalTok{E }\OtherTok{\textless{}{-}}\NormalTok{ H}\SpecialCharTok{/}\FunctionTok{log}\NormalTok{(}\FunctionTok{specnumber}\NormalTok{(m\_trab))}
\NormalTok{E}
\NormalTok{E[}\FunctionTok{is.na}\NormalTok{(E)] }\OtherTok{\textless{}{-}} \DecValTok{0} \CommentTok{\#substitui NA ou NaN por 0}
\NormalTok{E}
\end{Highlighting}
\end{Shaded}

\begin{verbatim}
##  ap-davis  as-bimac  as-fasci  ch-bimac  ci-ocela  ci-orien  co-macro  co-heter 
## 0.6912899 0.7730767 0.6410755 0.7321578 0.7383840 0.7696554       NaN       NaN 
##  cr-menez  cu-lepid  cy-gilbe  ge-brasi  he-margi  ho-malab  hy-pusar  le-melan 
## 0.7583440       NaN 0.9592195 0.5785997 1.0000000 0.8020891 0.6419992       NaN 
##   le-piau  le-taeni  mo-costa  mo-lepid  or-nilot  pa-manag  pimel-sp  po-retic 
## 0.9576352       NaN       NaN 0.2081906 0.7290694 0.7001630       NaN 0.5742169 
##  po-vivip  pr-brevi  ps-rhomb ps-genise  se-heter  se-piaba  se-spilo  st-noton 
## 0.7770962 0.5516566       NaN       NaN 0.8110852       NaN       NaN 0.6999162 
##  sy-marmo  te-chalc  tr-signa 
##       NaN 0.9869445 0.6131232 
##  ap-davis  as-bimac  as-fasci  ch-bimac  ci-ocela  ci-orien  co-macro  co-heter 
## 0.6912899 0.7730767 0.6410755 0.7321578 0.7383840 0.7696554 0.0000000 0.0000000 
##  cr-menez  cu-lepid  cy-gilbe  ge-brasi  he-margi  ho-malab  hy-pusar  le-melan 
## 0.7583440 0.0000000 0.9592195 0.5785997 1.0000000 0.8020891 0.6419992 0.0000000 
##   le-piau  le-taeni  mo-costa  mo-lepid  or-nilot  pa-manag  pimel-sp  po-retic 
## 0.9576352 0.0000000 0.0000000 0.2081906 0.7290694 0.7001630 0.0000000 0.5742169 
##  po-vivip  pr-brevi  ps-rhomb ps-genise  se-heter  se-piaba  se-spilo  st-noton 
## 0.7770962 0.5516566 0.0000000 0.0000000 0.8110852 0.0000000 0.0000000 0.6999162 
##  sy-marmo  te-chalc  tr-signa 
## 0.0000000 0.9869445 0.6131232
\end{verbatim}

\hypertarget{assimetria-e-curtose}{%
\subsubsection{Assimetria e curtose}\label{assimetria-e-curtose}}

A assimetria mede a falta de simetria em uma distribuição de frequência. Uma distribuição é simétrica se as duas metades à esquerda e à direita da média são cópias espelhadas uma da outra. Se a distribuição não é simétrica, então é assimétrica. A assimetria pode ser positiva, negativa ou nula (simétrica). Uma assimetria positiva indica que a cauda da distribuição se estende mais para a direita em relação à média, enquanto uma assimetria negativa indica que a cauda da distribuição se estende mais para a esquerda em relação à média.

A curtose descreve o pico ou a ``pontiagudez'' de uma distribuição. Uma distribuição com alta curtose tem uma alta concentração de valores ao redor da média e caudas mais pesadas (ou seja, valores extremos são mais prováveis). Uma distribuição com baixa curtose é mais achatada e dispersa, com caudas mais leves. A curtose pode ser positiva (distribuição leptocúrtica, com alta concentração em torno da média e caudas pesadas), negativa (distribuição platicúrtica, com dispersão alta e caudas mais leves) ou nula (mesocúrtica, similar à distribuição normal).

Essas medidas são úteis para compreender as propriedades e características de diferentes conjuntos de dados e distribuições de frequência. Elas ajudam a compreender a forma e o comportamento dos dados em uma amostra ou população (\protect\hyperlink{ref-RN158}{ZAR, 1999}).

\begin{Shaded}
\begin{Highlighting}[]
\FunctionTok{library}\NormalTok{(moments)}
\NormalTok{Assimetria }\OtherTok{\textless{}{-}} \FunctionTok{apply}\NormalTok{(m\_trab,}\DecValTok{1}\NormalTok{,skewness)}
\NormalTok{Assimetria}
\NormalTok{Curtose }\OtherTok{\textless{}{-}} \FunctionTok{apply}\NormalTok{(m\_trab,}\DecValTok{1}\NormalTok{,kurtosis)}
\NormalTok{Curtose}
\end{Highlighting}
\end{Shaded}

\begin{verbatim}
##  ap-davis  as-bimac  as-fasci  ch-bimac  ci-ocela  ci-orien  co-macro  co-heter 
##  4.469425  1.774738  2.959549  2.544388  3.796856  3.482201  4.800000  4.800000 
##  cr-menez  cu-lepid  cy-gilbe  ge-brasi  he-margi  ho-malab  hy-pusar  le-melan 
##  3.252387  4.800000  3.484076  3.638552  3.175426  2.282120  4.127075  4.800000 
##   le-piau  le-taeni  mo-costa  mo-lepid  or-nilot  pa-manag  pimel-sp  po-retic 
##  1.437653  4.800000  4.800000  4.790304  2.287297  2.867520  4.800000  4.462822 
##  po-vivip  pr-brevi  ps-rhomb ps-genise  se-heter  se-piaba  se-spilo  st-noton 
##  2.544468  4.046241  4.800000  4.800000  2.293389  4.800000  4.800000  3.419722 
##  sy-marmo  te-chalc  tr-signa 
##  4.800000  3.281874  4.442741 
##  ap-davis  as-bimac  as-fasci  ch-bimac  ci-ocela  ci-orien  co-macro  co-heter 
## 21.730097  5.556931 10.273303  7.736436 17.146010 15.065210 24.040000 24.040000 
##  cr-menez  cu-lepid  cy-gilbe  ge-brasi  he-margi  ho-malab  hy-pusar  le-melan 
## 12.820351 24.040000 13.776840 16.059988 11.083333  7.604321 19.499005 24.040000 
##   le-piau  le-taeni  mo-costa  mo-lepid  or-nilot  pa-manag  pimel-sp  po-retic 
##  3.655586 24.040000 24.040000 23.979582  7.972989  9.860393 24.040000 21.825541 
##  po-vivip  pr-brevi  ps-rhomb ps-genise  se-heter  se-piaba  se-spilo  st-noton 
##  8.674340 18.659676 24.040000 24.040000  7.676958 24.040000 24.040000 13.817921 
##  sy-marmo  te-chalc  tr-signa 
## 24.040000 12.021173 21.715543
\end{verbatim}

\hypertarget{tabela-de-descritores}{%
\subsubsection{Tabela de descritores}\label{tabela-de-descritores}}

Muito confuso? Criamos na sequência uma tabela final com todos os descritores da comunidade e da normalidade.

\hypertarget{descritores-da-estrutura-da-comunidade-espuxe9cies}{%
\paragraph{Descritores da estrutura da comunidade: Espécies}\label{descritores-da-estrutura-da-comunidade-espuxe9cies}}

Sum, soma; RA, abundância relativa (\%), mean, média; DP, desvio padrão da média; Max, maior valor; Min, menor valor; S, riqueza (ou frequência de ocorrência na matriz transposta); E, índice de equitabilidade de Pielou (\protect\hyperlink{ref-RN465}{PIELOU, 1975}); H, índice de diversidade de Shannon (\protect\hyperlink{ref-RN136}{LUDWIG; REYNOLDS, 1988}); D, índice de diversidade de Simpson (\protect\hyperlink{ref-RN1073}{HURLBERT, 1971}).

\begin{Shaded}
\begin{Highlighting}[]
\NormalTok{Descritores1 }\OtherTok{\textless{}{-}} \FunctionTok{cbind}\NormalTok{(Sum, RA, Mean, DP, Max, Min, MinZ, FO, S, E, H, D)}
\NormalTok{Descritores1 }\OtherTok{\textless{}{-}} \FunctionTok{as.data.frame}\NormalTok{(Descritores1)}
\NormalTok{Descritores1}
\CommentTok{\#Descritores1 \textless{}{-} Descritores1 \%\textgreater{}\% rownames\_to\_column(var="Espécies") \#da nome a primeira coluna}
\NormalTok{SomaTotalD }\OtherTok{\textless{}{-}} \FunctionTok{apply}\NormalTok{(Descritores1,}\DecValTok{2}\NormalTok{,sum)}
\NormalTok{SomaTotalD}
\NormalTok{MediaTotalD }\OtherTok{\textless{}{-}} \FunctionTok{apply}\NormalTok{(Descritores1,}\DecValTok{2}\NormalTok{,mean)}
\NormalTok{MediaTotalD}
\NormalTok{DPTotalD }\OtherTok{\textless{}{-}} \FunctionTok{apply}\NormalTok{(Descritores1,}\DecValTok{2}\NormalTok{,sd)}
\NormalTok{DPTotalD}
\NormalTok{Descritores2 }\OtherTok{\textless{}{-}} \FunctionTok{cbind}\NormalTok{(SomaTotalD, MediaTotalD, DPTotalD)}
\NormalTok{Descritores2 }\OtherTok{\textless{}{-}} \FunctionTok{as.data.frame}\NormalTok{(Descritores2)}
\NormalTok{Descritores2 }\OtherTok{\textless{}{-}} \FunctionTok{t}\NormalTok{(Descritores2)}
\NormalTok{Descritores2}
\NormalTok{DescritoresFinal }\OtherTok{\textless{}{-}} \FunctionTok{rbind}\NormalTok{(Descritores1, Descritores2)}
\NormalTok{DescritoresFinal}
\NormalTok{DescritoresFinal }\OtherTok{\textless{}{-}} \FunctionTok{round}\NormalTok{ (DescritoresFinal, }\DecValTok{2}\NormalTok{)}
\NormalTok{DescritoresFinal}
\end{Highlighting}
\end{Shaded}

\begin{verbatim}
##            Sum          RA        Mean          DP Max Min MinZ        FO  S
## ap-davis    27  0.30354132  1.03846154   4.3861671  22   0    5  7.692308  2
## as-bimac  2386 26.82405846 91.76923077 132.4348316 511   0    1 88.461538 23
## as-fasci   158  1.77627881  6.07692308  16.2035134  64   0    1 38.461538 10
## ch-bimac   705  7.92580101 27.11538462  75.8732242 273   0    3 19.230769  5
## ci-ocela    70  0.78695897  2.69230769   8.2982853  40   0    2 19.230769  5
## ci-orien   143  1.60764474  5.50000000  14.6184815  69   0    5 26.923077  7
## co-macro     2  0.02248454  0.07692308   0.3922323   2   0    2  3.846154  1
## co-heter     1  0.01124227  0.03846154   0.1961161   1   0    1  3.846154  1
## cr-menez    28  0.31478359  1.07692308   3.1486261  14   0    1 19.230769  5
## cu-lepid    21  0.23608769  0.80769231   4.1184388  21   0   21  3.846154  1
## cy-gilbe   131  1.47273749  5.03846154  18.3313519  81   0   50  7.692308  2
## ge-brasi  1020 11.46711636 39.23076923 106.3478473 509   0    1 57.692308 15
## he-margi     2  0.02248454  0.07692308   0.2717465   1   0    1  7.692308  2
## ho-malab   109  1.22540753  4.19230769   7.6053625  31   0    1 53.846154 14
## hy-pusar    71  0.79820124  2.73076923   8.6605205  43   0    1 26.923077  7
## le-melan     2  0.02248454  0.07692308   0.3922323   2   0    2  3.846154  1
## le-piau     15  0.16863406  0.57692308   0.9868364   3   0    1 30.769231  8
## le-taeni     1  0.01124227  0.03846154   0.1961161   1   0    1  3.846154  1
## mo-costa     1  0.01124227  0.03846154   0.1961161   1   0    1  3.846154  1
## mo-lepid    41  0.46093311  1.57692308   7.6376597  39   0    1 11.538462  3
## or-nilot   848  9.53344576 32.61538462  59.1452970 247   0    1 69.230769 18
## pa-manag   554  6.22821810 21.30769231  62.3928004 250   0    1 19.230769  5
## pimel-sp     6  0.06745363  0.23076923   1.1766968   6   0    6  3.846154  1
## po-retic   381  4.28330523 14.65384615  52.5242362 266   0    5 23.076923  6
## po-vivip   978 10.99494098 37.61538462  79.5004790 326   0    8 42.307692 11
## pr-brevi   279  3.13659359 10.73076923  33.3892889 164   0    1 46.153846 12
## ps-rhomb     1  0.01124227  0.03846154   0.1961161   1   0    1  3.846154  1
## ps-genise    1  0.01124227  0.03846154   0.1961161   1   0    1  3.846154  1
## se-heter   296  3.32771220 11.38461538  22.9905666  93   0    3 38.461538 10
## se-piaba    68  0.76447442  2.61538462  13.3358972  68   0   68  3.846154  1
## se-spilo     1  0.01124227  0.03846154   0.1961161   1   0    1  3.846154  1
## st-noton   205  2.30466554  7.88461538  25.5582893 115   0    1 15.384615  4
## sy-marmo     1  0.01124227  0.03846154   0.1961161   1   0    1  3.846154  1
## te-chalc   134  1.50646431  5.15384615  18.3841068  76   0   58  7.692308  2
## tr-signa   208  2.33839236  8.00000000  27.8280434 141   0    4 23.076923  6
##                   E         H          D
## ap-davis  0.6912899 0.4791656 0.30178326
## as-bimac  0.7730767 2.4239775 0.88451845
## as-fasci  0.6410755 1.4761308 0.69860599
## ch-bimac  0.7321578 1.1783625 0.67197827
## ci-ocela  0.7383840 1.1883832 0.61020408
## ci-orien  0.7696554 1.4976803 0.70027874
## co-macro  0.0000000 0.0000000 0.00000000
## co-heter  0.0000000 0.0000000 0.00000000
## cr-menez  0.7583440 1.2205076 0.64540816
## cu-lepid  0.0000000 0.0000000 0.00000000
## cy-gilbe  0.9592195 0.6648803 0.47200047
## ge-brasi  0.5785997 1.5668772 0.68977124
## he-margi  1.0000000 0.6931472 0.50000000
## ho-malab  0.8020891 2.1167591 0.83982830
## hy-pusar  0.6419992 1.2492728 0.58956556
## le-melan  0.0000000 0.0000000 0.00000000
## le-piau   0.9576352 1.9913464 0.85333333
## le-taeni  0.0000000 0.0000000 0.00000000
## mo-costa  0.0000000 0.0000000 0.00000000
## mo-lepid  0.2081906 0.2287207 0.09399167
## or-nilot  0.7290694 2.1072815 0.83992302
## pa-manag  0.7001630 1.1268689 0.64444343
## pimel-sp  0.0000000 0.0000000 0.00000000
## po-retic  0.5742169 1.0288585 0.48641164
## po-vivip  0.7770962 1.8633952 0.79634160
## pr-brevi  0.5516566 1.3708151 0.60348659
## ps-rhomb  0.0000000 0.0000000 0.00000000
## ps-genise 0.0000000 0.0000000 0.00000000
## se-heter  0.8110852 1.8675926 0.81071950
## se-piaba  0.0000000 0.0000000 0.00000000
## se-spilo  0.0000000 0.0000000 0.00000000
## st-noton  0.6999162 0.9702899 0.57294468
## sy-marmo  0.0000000 0.0000000 0.00000000
## te-chalc  0.9869445 0.6840978 0.49097795
## tr-signa  0.6131232 1.0985693 0.51405325
##        Sum         RA       Mean         DP        Max        Min       MinZ 
## 8895.00000  100.00000  342.11538  807.30587 3484.00000    0.00000  262.00000 
##         FO          S          E          H          D 
##  746.15385  194.00000   16.69499   30.09298   14.31057 
##         Sum          RA        Mean          DP         Max         Min 
## 254.1428571   2.8571429   9.7747253  23.0658821  99.5428571   0.0000000 
##        MinZ          FO           S           E           H           D 
##   7.4857143  21.3186813   5.5428571   0.4769997   0.8597994   0.4088734 
##         Sum          RA        Mean          DP         Max         Min 
## 468.3865386   5.2657284  18.0148669  33.0609233 138.2715450   0.0000000 
##        MinZ          FO           S           E           H           D 
##  16.4573807  21.3807831   5.5590036   0.3757324   0.7784814   0.3354621 
##                   Sum         RA       Mean        DP        Max Min       MinZ
## SomaTotalD  8895.0000 100.000000 342.115385 807.30587 3484.00000   0 262.000000
## MediaTotalD  254.1429   2.857143   9.774725  23.06588   99.54286   0   7.485714
## DPTotalD     468.3865   5.265728  18.014867  33.06092  138.27155   0  16.457381
##                    FO          S          E          H          D
## SomaTotalD  746.15385 194.000000 16.6949878 30.0929803 14.3105692
## MediaTotalD  21.31868   5.542857  0.4769997  0.8597994  0.4088734
## DPTotalD     21.38078   5.559004  0.3757324  0.7784814  0.3354621
##                   Sum           RA         Mean          DP        Max Min
## ap-davis      27.0000   0.30354132   1.03846154   4.3861671   22.00000   0
## as-bimac    2386.0000  26.82405846  91.76923077 132.4348316  511.00000   0
## as-fasci     158.0000   1.77627881   6.07692308  16.2035134   64.00000   0
## ch-bimac     705.0000   7.92580101  27.11538462  75.8732242  273.00000   0
## ci-ocela      70.0000   0.78695897   2.69230769   8.2982853   40.00000   0
## ci-orien     143.0000   1.60764474   5.50000000  14.6184815   69.00000   0
## co-macro       2.0000   0.02248454   0.07692308   0.3922323    2.00000   0
## co-heter       1.0000   0.01124227   0.03846154   0.1961161    1.00000   0
## cr-menez      28.0000   0.31478359   1.07692308   3.1486261   14.00000   0
## cu-lepid      21.0000   0.23608769   0.80769231   4.1184388   21.00000   0
## cy-gilbe     131.0000   1.47273749   5.03846154  18.3313519   81.00000   0
## ge-brasi    1020.0000  11.46711636  39.23076923 106.3478473  509.00000   0
## he-margi       2.0000   0.02248454   0.07692308   0.2717465    1.00000   0
## ho-malab     109.0000   1.22540753   4.19230769   7.6053625   31.00000   0
## hy-pusar      71.0000   0.79820124   2.73076923   8.6605205   43.00000   0
## le-melan       2.0000   0.02248454   0.07692308   0.3922323    2.00000   0
## le-piau       15.0000   0.16863406   0.57692308   0.9868364    3.00000   0
## le-taeni       1.0000   0.01124227   0.03846154   0.1961161    1.00000   0
## mo-costa       1.0000   0.01124227   0.03846154   0.1961161    1.00000   0
## mo-lepid      41.0000   0.46093311   1.57692308   7.6376597   39.00000   0
## or-nilot     848.0000   9.53344576  32.61538462  59.1452970  247.00000   0
## pa-manag     554.0000   6.22821810  21.30769231  62.3928004  250.00000   0
## pimel-sp       6.0000   0.06745363   0.23076923   1.1766968    6.00000   0
## po-retic     381.0000   4.28330523  14.65384615  52.5242362  266.00000   0
## po-vivip     978.0000  10.99494098  37.61538462  79.5004790  326.00000   0
## pr-brevi     279.0000   3.13659359  10.73076923  33.3892889  164.00000   0
## ps-rhomb       1.0000   0.01124227   0.03846154   0.1961161    1.00000   0
## ps-genise      1.0000   0.01124227   0.03846154   0.1961161    1.00000   0
## se-heter     296.0000   3.32771220  11.38461538  22.9905666   93.00000   0
## se-piaba      68.0000   0.76447442   2.61538462  13.3358972   68.00000   0
## se-spilo       1.0000   0.01124227   0.03846154   0.1961161    1.00000   0
## st-noton     205.0000   2.30466554   7.88461538  25.5582893  115.00000   0
## sy-marmo       1.0000   0.01124227   0.03846154   0.1961161    1.00000   0
## te-chalc     134.0000   1.50646431   5.15384615  18.3841068   76.00000   0
## tr-signa     208.0000   2.33839236   8.00000000  27.8280434  141.00000   0
## SomaTotalD  8895.0000 100.00000000 342.11538462 807.3058718 3484.00000   0
## MediaTotalD  254.1429   2.85714286   9.77472527  23.0658821   99.54286   0
## DPTotalD     468.3865   5.26572837  18.01486687  33.0609233  138.27155   0
##                   MinZ         FO          S          E          H           D
## ap-davis      5.000000   7.692308   2.000000  0.6912899  0.4791656  0.30178326
## as-bimac      1.000000  88.461538  23.000000  0.7730767  2.4239775  0.88451845
## as-fasci      1.000000  38.461538  10.000000  0.6410755  1.4761308  0.69860599
## ch-bimac      3.000000  19.230769   5.000000  0.7321578  1.1783625  0.67197827
## ci-ocela      2.000000  19.230769   5.000000  0.7383840  1.1883832  0.61020408
## ci-orien      5.000000  26.923077   7.000000  0.7696554  1.4976803  0.70027874
## co-macro      2.000000   3.846154   1.000000  0.0000000  0.0000000  0.00000000
## co-heter      1.000000   3.846154   1.000000  0.0000000  0.0000000  0.00000000
## cr-menez      1.000000  19.230769   5.000000  0.7583440  1.2205076  0.64540816
## cu-lepid     21.000000   3.846154   1.000000  0.0000000  0.0000000  0.00000000
## cy-gilbe     50.000000   7.692308   2.000000  0.9592195  0.6648803  0.47200047
## ge-brasi      1.000000  57.692308  15.000000  0.5785997  1.5668772  0.68977124
## he-margi      1.000000   7.692308   2.000000  1.0000000  0.6931472  0.50000000
## ho-malab      1.000000  53.846154  14.000000  0.8020891  2.1167591  0.83982830
## hy-pusar      1.000000  26.923077   7.000000  0.6419992  1.2492728  0.58956556
## le-melan      2.000000   3.846154   1.000000  0.0000000  0.0000000  0.00000000
## le-piau       1.000000  30.769231   8.000000  0.9576352  1.9913464  0.85333333
## le-taeni      1.000000   3.846154   1.000000  0.0000000  0.0000000  0.00000000
## mo-costa      1.000000   3.846154   1.000000  0.0000000  0.0000000  0.00000000
## mo-lepid      1.000000  11.538462   3.000000  0.2081906  0.2287207  0.09399167
## or-nilot      1.000000  69.230769  18.000000  0.7290694  2.1072815  0.83992302
## pa-manag      1.000000  19.230769   5.000000  0.7001630  1.1268689  0.64444343
## pimel-sp      6.000000   3.846154   1.000000  0.0000000  0.0000000  0.00000000
## po-retic      5.000000  23.076923   6.000000  0.5742169  1.0288585  0.48641164
## po-vivip      8.000000  42.307692  11.000000  0.7770962  1.8633952  0.79634160
## pr-brevi      1.000000  46.153846  12.000000  0.5516566  1.3708151  0.60348659
## ps-rhomb      1.000000   3.846154   1.000000  0.0000000  0.0000000  0.00000000
## ps-genise     1.000000   3.846154   1.000000  0.0000000  0.0000000  0.00000000
## se-heter      3.000000  38.461538  10.000000  0.8110852  1.8675926  0.81071950
## se-piaba     68.000000   3.846154   1.000000  0.0000000  0.0000000  0.00000000
## se-spilo      1.000000   3.846154   1.000000  0.0000000  0.0000000  0.00000000
## st-noton      1.000000  15.384615   4.000000  0.6999162  0.9702899  0.57294468
## sy-marmo      1.000000   3.846154   1.000000  0.0000000  0.0000000  0.00000000
## te-chalc     58.000000   7.692308   2.000000  0.9869445  0.6840978  0.49097795
## tr-signa      4.000000  23.076923   6.000000  0.6131232  1.0985693  0.51405325
## SomaTotalD  262.000000 746.153846 194.000000 16.6949878 30.0929803 14.31056920
## MediaTotalD   7.485714  21.318681   5.542857  0.4769997  0.8597994  0.40887341
## DPTotalD     16.457381  21.380783   5.559004  0.3757324  0.7784814  0.33546207
##                 Sum     RA   Mean     DP     Max Min   MinZ     FO      S     E
## ap-davis      27.00   0.30   1.04   4.39   22.00   0   5.00   7.69   2.00  0.69
## as-bimac    2386.00  26.82  91.77 132.43  511.00   0   1.00  88.46  23.00  0.77
## as-fasci     158.00   1.78   6.08  16.20   64.00   0   1.00  38.46  10.00  0.64
## ch-bimac     705.00   7.93  27.12  75.87  273.00   0   3.00  19.23   5.00  0.73
## ci-ocela      70.00   0.79   2.69   8.30   40.00   0   2.00  19.23   5.00  0.74
## ci-orien     143.00   1.61   5.50  14.62   69.00   0   5.00  26.92   7.00  0.77
## co-macro       2.00   0.02   0.08   0.39    2.00   0   2.00   3.85   1.00  0.00
## co-heter       1.00   0.01   0.04   0.20    1.00   0   1.00   3.85   1.00  0.00
## cr-menez      28.00   0.31   1.08   3.15   14.00   0   1.00  19.23   5.00  0.76
## cu-lepid      21.00   0.24   0.81   4.12   21.00   0  21.00   3.85   1.00  0.00
## cy-gilbe     131.00   1.47   5.04  18.33   81.00   0  50.00   7.69   2.00  0.96
## ge-brasi    1020.00  11.47  39.23 106.35  509.00   0   1.00  57.69  15.00  0.58
## he-margi       2.00   0.02   0.08   0.27    1.00   0   1.00   7.69   2.00  1.00
## ho-malab     109.00   1.23   4.19   7.61   31.00   0   1.00  53.85  14.00  0.80
## hy-pusar      71.00   0.80   2.73   8.66   43.00   0   1.00  26.92   7.00  0.64
## le-melan       2.00   0.02   0.08   0.39    2.00   0   2.00   3.85   1.00  0.00
## le-piau       15.00   0.17   0.58   0.99    3.00   0   1.00  30.77   8.00  0.96
## le-taeni       1.00   0.01   0.04   0.20    1.00   0   1.00   3.85   1.00  0.00
## mo-costa       1.00   0.01   0.04   0.20    1.00   0   1.00   3.85   1.00  0.00
## mo-lepid      41.00   0.46   1.58   7.64   39.00   0   1.00  11.54   3.00  0.21
## or-nilot     848.00   9.53  32.62  59.15  247.00   0   1.00  69.23  18.00  0.73
## pa-manag     554.00   6.23  21.31  62.39  250.00   0   1.00  19.23   5.00  0.70
## pimel-sp       6.00   0.07   0.23   1.18    6.00   0   6.00   3.85   1.00  0.00
## po-retic     381.00   4.28  14.65  52.52  266.00   0   5.00  23.08   6.00  0.57
## po-vivip     978.00  10.99  37.62  79.50  326.00   0   8.00  42.31  11.00  0.78
## pr-brevi     279.00   3.14  10.73  33.39  164.00   0   1.00  46.15  12.00  0.55
## ps-rhomb       1.00   0.01   0.04   0.20    1.00   0   1.00   3.85   1.00  0.00
## ps-genise      1.00   0.01   0.04   0.20    1.00   0   1.00   3.85   1.00  0.00
## se-heter     296.00   3.33  11.38  22.99   93.00   0   3.00  38.46  10.00  0.81
## se-piaba      68.00   0.76   2.62  13.34   68.00   0  68.00   3.85   1.00  0.00
## se-spilo       1.00   0.01   0.04   0.20    1.00   0   1.00   3.85   1.00  0.00
## st-noton     205.00   2.30   7.88  25.56  115.00   0   1.00  15.38   4.00  0.70
## sy-marmo       1.00   0.01   0.04   0.20    1.00   0   1.00   3.85   1.00  0.00
## te-chalc     134.00   1.51   5.15  18.38   76.00   0  58.00   7.69   2.00  0.99
## tr-signa     208.00   2.34   8.00  27.83  141.00   0   4.00  23.08   6.00  0.61
## SomaTotalD  8895.00 100.00 342.12 807.31 3484.00   0 262.00 746.15 194.00 16.69
## MediaTotalD  254.14   2.86   9.77  23.07   99.54   0   7.49  21.32   5.54  0.48
## DPTotalD     468.39   5.27  18.01  33.06  138.27   0  16.46  21.38   5.56  0.38
##                 H     D
## ap-davis     0.48  0.30
## as-bimac     2.42  0.88
## as-fasci     1.48  0.70
## ch-bimac     1.18  0.67
## ci-ocela     1.19  0.61
## ci-orien     1.50  0.70
## co-macro     0.00  0.00
## co-heter     0.00  0.00
## cr-menez     1.22  0.65
## cu-lepid     0.00  0.00
## cy-gilbe     0.66  0.47
## ge-brasi     1.57  0.69
## he-margi     0.69  0.50
## ho-malab     2.12  0.84
## hy-pusar     1.25  0.59
## le-melan     0.00  0.00
## le-piau      1.99  0.85
## le-taeni     0.00  0.00
## mo-costa     0.00  0.00
## mo-lepid     0.23  0.09
## or-nilot     2.11  0.84
## pa-manag     1.13  0.64
## pimel-sp     0.00  0.00
## po-retic     1.03  0.49
## po-vivip     1.86  0.80
## pr-brevi     1.37  0.60
## ps-rhomb     0.00  0.00
## ps-genise    0.00  0.00
## se-heter     1.87  0.81
## se-piaba     0.00  0.00
## se-spilo     0.00  0.00
## st-noton     0.97  0.57
## sy-marmo     0.00  0.00
## te-chalc     0.68  0.49
## tr-signa     1.10  0.51
## SomaTotalD  30.09 14.31
## MediaTotalD  0.86  0.41
## DPTotalD     0.78  0.34
\end{verbatim}

\begin{Shaded}
\begin{Highlighting}[]
\CommentTok{\#Fazendo uma tabela}
\FunctionTok{library}\NormalTok{(gt)}
\NormalTok{df }\OtherTok{\textless{}{-}}\NormalTok{ DescritoresFinal}
\FunctionTok{ncol}\NormalTok{(df); }\FunctionTok{nrow}\NormalTok{(df) }\CommentTok{\#no. de N colunas x M linhas}
\NormalTok{df }\OtherTok{\textless{}{-}} \FunctionTok{cbind}\NormalTok{(}\AttributeTok{Spp =} \FunctionTok{rownames}\NormalTok{(df), df)}
\FunctionTok{gt}\NormalTok{(df, }\AttributeTok{rowname\_col =} \StringTok{"Espécie"}\NormalTok{, }\AttributeTok{caption =} \StringTok{"Descritores da diversidade por espécie (colunas). Sum, soma; RA, abundância relativa (\%); mean, média; DP, desvio padrão da média; Max, maior valor; Min, menor valor; MinZ, menor valor não zero; FO, frequência de ocorrência (\%); S, riqueza (ou no. de ocorrências, da matriz transposta); E, índice de equitabilidade de Pielou; H, índice de diversidade de Shannon; D, índice de diversidade de Simpson."}\NormalTok{)}
\end{Highlighting}
\end{Shaded}

\begin{verbatim}
## [1] 12
## [1] 38
\end{verbatim}

\begin{longtable}{lrrrrrrrrrrrr}
\toprule
Spp & Sum & RA & Mean & DP & Max & Min & MinZ & FO & S & E & H & D \\ 
\midrule\addlinespace[2.5pt]
ap-davis & 27.00 & 0.30 & 1.04 & 4.39 & 22.00 & 0 & 5.00 & 7.69 & 2.00 & 0.69 & 0.48 & 0.30 \\ 
as-bimac & 2386.00 & 26.82 & 91.77 & 132.43 & 511.00 & 0 & 1.00 & 88.46 & 23.00 & 0.77 & 2.42 & 0.88 \\ 
as-fasci & 158.00 & 1.78 & 6.08 & 16.20 & 64.00 & 0 & 1.00 & 38.46 & 10.00 & 0.64 & 1.48 & 0.70 \\ 
ch-bimac & 705.00 & 7.93 & 27.12 & 75.87 & 273.00 & 0 & 3.00 & 19.23 & 5.00 & 0.73 & 1.18 & 0.67 \\ 
ci-ocela & 70.00 & 0.79 & 2.69 & 8.30 & 40.00 & 0 & 2.00 & 19.23 & 5.00 & 0.74 & 1.19 & 0.61 \\ 
ci-orien & 143.00 & 1.61 & 5.50 & 14.62 & 69.00 & 0 & 5.00 & 26.92 & 7.00 & 0.77 & 1.50 & 0.70 \\ 
co-macro & 2.00 & 0.02 & 0.08 & 0.39 & 2.00 & 0 & 2.00 & 3.85 & 1.00 & 0.00 & 0.00 & 0.00 \\ 
co-heter & 1.00 & 0.01 & 0.04 & 0.20 & 1.00 & 0 & 1.00 & 3.85 & 1.00 & 0.00 & 0.00 & 0.00 \\ 
cr-menez & 28.00 & 0.31 & 1.08 & 3.15 & 14.00 & 0 & 1.00 & 19.23 & 5.00 & 0.76 & 1.22 & 0.65 \\ 
cu-lepid & 21.00 & 0.24 & 0.81 & 4.12 & 21.00 & 0 & 21.00 & 3.85 & 1.00 & 0.00 & 0.00 & 0.00 \\ 
cy-gilbe & 131.00 & 1.47 & 5.04 & 18.33 & 81.00 & 0 & 50.00 & 7.69 & 2.00 & 0.96 & 0.66 & 0.47 \\ 
ge-brasi & 1020.00 & 11.47 & 39.23 & 106.35 & 509.00 & 0 & 1.00 & 57.69 & 15.00 & 0.58 & 1.57 & 0.69 \\ 
he-margi & 2.00 & 0.02 & 0.08 & 0.27 & 1.00 & 0 & 1.00 & 7.69 & 2.00 & 1.00 & 0.69 & 0.50 \\ 
ho-malab & 109.00 & 1.23 & 4.19 & 7.61 & 31.00 & 0 & 1.00 & 53.85 & 14.00 & 0.80 & 2.12 & 0.84 \\ 
hy-pusar & 71.00 & 0.80 & 2.73 & 8.66 & 43.00 & 0 & 1.00 & 26.92 & 7.00 & 0.64 & 1.25 & 0.59 \\ 
le-melan & 2.00 & 0.02 & 0.08 & 0.39 & 2.00 & 0 & 2.00 & 3.85 & 1.00 & 0.00 & 0.00 & 0.00 \\ 
le-piau & 15.00 & 0.17 & 0.58 & 0.99 & 3.00 & 0 & 1.00 & 30.77 & 8.00 & 0.96 & 1.99 & 0.85 \\ 
le-taeni & 1.00 & 0.01 & 0.04 & 0.20 & 1.00 & 0 & 1.00 & 3.85 & 1.00 & 0.00 & 0.00 & 0.00 \\ 
mo-costa & 1.00 & 0.01 & 0.04 & 0.20 & 1.00 & 0 & 1.00 & 3.85 & 1.00 & 0.00 & 0.00 & 0.00 \\ 
mo-lepid & 41.00 & 0.46 & 1.58 & 7.64 & 39.00 & 0 & 1.00 & 11.54 & 3.00 & 0.21 & 0.23 & 0.09 \\ 
or-nilot & 848.00 & 9.53 & 32.62 & 59.15 & 247.00 & 0 & 1.00 & 69.23 & 18.00 & 0.73 & 2.11 & 0.84 \\ 
pa-manag & 554.00 & 6.23 & 21.31 & 62.39 & 250.00 & 0 & 1.00 & 19.23 & 5.00 & 0.70 & 1.13 & 0.64 \\ 
pimel-sp & 6.00 & 0.07 & 0.23 & 1.18 & 6.00 & 0 & 6.00 & 3.85 & 1.00 & 0.00 & 0.00 & 0.00 \\ 
po-retic & 381.00 & 4.28 & 14.65 & 52.52 & 266.00 & 0 & 5.00 & 23.08 & 6.00 & 0.57 & 1.03 & 0.49 \\ 
po-vivip & 978.00 & 10.99 & 37.62 & 79.50 & 326.00 & 0 & 8.00 & 42.31 & 11.00 & 0.78 & 1.86 & 0.80 \\ 
pr-brevi & 279.00 & 3.14 & 10.73 & 33.39 & 164.00 & 0 & 1.00 & 46.15 & 12.00 & 0.55 & 1.37 & 0.60 \\ 
ps-rhomb & 1.00 & 0.01 & 0.04 & 0.20 & 1.00 & 0 & 1.00 & 3.85 & 1.00 & 0.00 & 0.00 & 0.00 \\ 
ps-genise & 1.00 & 0.01 & 0.04 & 0.20 & 1.00 & 0 & 1.00 & 3.85 & 1.00 & 0.00 & 0.00 & 0.00 \\ 
se-heter & 296.00 & 3.33 & 11.38 & 22.99 & 93.00 & 0 & 3.00 & 38.46 & 10.00 & 0.81 & 1.87 & 0.81 \\ 
se-piaba & 68.00 & 0.76 & 2.62 & 13.34 & 68.00 & 0 & 68.00 & 3.85 & 1.00 & 0.00 & 0.00 & 0.00 \\ 
se-spilo & 1.00 & 0.01 & 0.04 & 0.20 & 1.00 & 0 & 1.00 & 3.85 & 1.00 & 0.00 & 0.00 & 0.00 \\ 
st-noton & 205.00 & 2.30 & 7.88 & 25.56 & 115.00 & 0 & 1.00 & 15.38 & 4.00 & 0.70 & 0.97 & 0.57 \\ 
sy-marmo & 1.00 & 0.01 & 0.04 & 0.20 & 1.00 & 0 & 1.00 & 3.85 & 1.00 & 0.00 & 0.00 & 0.00 \\ 
te-chalc & 134.00 & 1.51 & 5.15 & 18.38 & 76.00 & 0 & 58.00 & 7.69 & 2.00 & 0.99 & 0.68 & 0.49 \\ 
tr-signa & 208.00 & 2.34 & 8.00 & 27.83 & 141.00 & 0 & 4.00 & 23.08 & 6.00 & 0.61 & 1.10 & 0.51 \\ 
SomaTotalD & 8895.00 & 100.00 & 342.12 & 807.31 & 3484.00 & 0 & 262.00 & 746.15 & 194.00 & 16.69 & 30.09 & 14.31 \\ 
MediaTotalD & 254.14 & 2.86 & 9.77 & 23.07 & 99.54 & 0 & 7.49 & 21.32 & 5.54 & 0.48 & 0.86 & 0.41 \\ 
DPTotalD & 468.39 & 5.27 & 18.01 & 33.06 & 138.27 & 0 & 16.46 & 21.38 & 5.56 & 0.38 & 0.78 & 0.34 \\ 
\bottomrule
\end{longtable}

\hypertarget{descritores-da-normalidade}{%
\paragraph{Descritores da normalidade}\label{descritores-da-normalidade}}

Agora tabelamos os descritores de normalidade

\begin{Shaded}
\begin{Highlighting}[]
\NormalTok{Normalidade1 }\OtherTok{\textless{}{-}} \FunctionTok{cbind}\NormalTok{(Assimetria, Curtose)}
\NormalTok{Normalidade1 }\OtherTok{\textless{}{-}} \FunctionTok{as.data.frame}\NormalTok{(Normalidade1)}
\NormalTok{Normalidade1}
\NormalTok{SomaTotalN }\OtherTok{\textless{}{-}} \FunctionTok{apply}\NormalTok{(Normalidade1,}\DecValTok{2}\NormalTok{,sum)}
\NormalTok{SomaTotalN}
\NormalTok{MediaTotalN }\OtherTok{\textless{}{-}} \FunctionTok{apply}\NormalTok{(Normalidade1,}\DecValTok{2}\NormalTok{,mean)}
\NormalTok{MediaTotalN}
\NormalTok{DPTotalN }\OtherTok{\textless{}{-}} \FunctionTok{apply}\NormalTok{(Normalidade1,}\DecValTok{2}\NormalTok{,sd)}
\NormalTok{DPTotalN}
\NormalTok{Normalidade2}\OtherTok{\textless{}{-}}\FunctionTok{cbind}\NormalTok{(SomaTotalN, MediaTotalN, DPTotalN)}
\NormalTok{Normalidade2}\OtherTok{\textless{}{-}}\FunctionTok{as.data.frame}\NormalTok{(Normalidade2)}
\NormalTok{Normalidade2 }\OtherTok{\textless{}{-}} \FunctionTok{t}\NormalTok{(Normalidade2) }\CommentTok{\#"t" transpoe a matriz}
\NormalTok{Normalidade2}
\NormalTok{NormalidadeFinal }\OtherTok{\textless{}{-}} \FunctionTok{rbind}\NormalTok{(Normalidade1, Normalidade2)}
\NormalTok{NormalidadeFinal}
\NormalTok{NormalidadeFinal }\OtherTok{\textless{}{-}} \FunctionTok{round}\NormalTok{(NormalidadeFinal, }\DecValTok{2}\NormalTok{)}
\NormalTok{NormalidadeFinal}
\end{Highlighting}
\end{Shaded}

\begin{verbatim}
##           Assimetria   Curtose
## ap-davis    4.469425 21.730097
## as-bimac    1.774738  5.556931
## as-fasci    2.959549 10.273303
## ch-bimac    2.544388  7.736436
## ci-ocela    3.796856 17.146010
## ci-orien    3.482201 15.065210
## co-macro    4.800000 24.040000
## co-heter    4.800000 24.040000
## cr-menez    3.252387 12.820351
## cu-lepid    4.800000 24.040000
## cy-gilbe    3.484076 13.776840
## ge-brasi    3.638552 16.059988
## he-margi    3.175426 11.083333
## ho-malab    2.282120  7.604321
## hy-pusar    4.127075 19.499005
## le-melan    4.800000 24.040000
## le-piau     1.437653  3.655586
## le-taeni    4.800000 24.040000
## mo-costa    4.800000 24.040000
## mo-lepid    4.790304 23.979582
## or-nilot    2.287297  7.972989
## pa-manag    2.867520  9.860393
## pimel-sp    4.800000 24.040000
## po-retic    4.462822 21.825541
## po-vivip    2.544468  8.674340
## pr-brevi    4.046241 18.659676
## ps-rhomb    4.800000 24.040000
## ps-genise   4.800000 24.040000
## se-heter    2.293389  7.676958
## se-piaba    4.800000 24.040000
## se-spilo    4.800000 24.040000
## st-noton    3.419722 13.817921
## sy-marmo    4.800000 24.040000
## te-chalc    3.281874 12.021173
## tr-signa    4.442741 21.715543
## Assimetria    Curtose 
##   132.4608   596.6915 
## Assimetria    Curtose 
##   3.784595  17.048329 
## Assimetria    Curtose 
##   1.043802   6.944634 
##             Assimetria    Curtose
## SomaTotalN  132.460824 596.691528
## MediaTotalN   3.784595  17.048329
## DPTotalN      1.043802   6.944634
##             Assimetria    Curtose
## ap-davis      4.469425  21.730097
## as-bimac      1.774738   5.556931
## as-fasci      2.959549  10.273303
## ch-bimac      2.544388   7.736436
## ci-ocela      3.796856  17.146010
## ci-orien      3.482201  15.065210
## co-macro      4.800000  24.040000
## co-heter      4.800000  24.040000
## cr-menez      3.252387  12.820351
## cu-lepid      4.800000  24.040000
## cy-gilbe      3.484076  13.776840
## ge-brasi      3.638552  16.059988
## he-margi      3.175426  11.083333
## ho-malab      2.282120   7.604321
## hy-pusar      4.127075  19.499005
## le-melan      4.800000  24.040000
## le-piau       1.437653   3.655586
## le-taeni      4.800000  24.040000
## mo-costa      4.800000  24.040000
## mo-lepid      4.790304  23.979582
## or-nilot      2.287297   7.972989
## pa-manag      2.867520   9.860393
## pimel-sp      4.800000  24.040000
## po-retic      4.462822  21.825541
## po-vivip      2.544468   8.674340
## pr-brevi      4.046241  18.659676
## ps-rhomb      4.800000  24.040000
## ps-genise     4.800000  24.040000
## se-heter      2.293389   7.676958
## se-piaba      4.800000  24.040000
## se-spilo      4.800000  24.040000
## st-noton      3.419722  13.817921
## sy-marmo      4.800000  24.040000
## te-chalc      3.281874  12.021173
## tr-signa      4.442741  21.715543
## SomaTotalN  132.460824 596.691528
## MediaTotalN   3.784595  17.048329
## DPTotalN      1.043802   6.944634
##             Assimetria Curtose
## ap-davis          4.47   21.73
## as-bimac          1.77    5.56
## as-fasci          2.96   10.27
## ch-bimac          2.54    7.74
## ci-ocela          3.80   17.15
## ci-orien          3.48   15.07
## co-macro          4.80   24.04
## co-heter          4.80   24.04
## cr-menez          3.25   12.82
## cu-lepid          4.80   24.04
## cy-gilbe          3.48   13.78
## ge-brasi          3.64   16.06
## he-margi          3.18   11.08
## ho-malab          2.28    7.60
## hy-pusar          4.13   19.50
## le-melan          4.80   24.04
## le-piau           1.44    3.66
## le-taeni          4.80   24.04
## mo-costa          4.80   24.04
## mo-lepid          4.79   23.98
## or-nilot          2.29    7.97
## pa-manag          2.87    9.86
## pimel-sp          4.80   24.04
## po-retic          4.46   21.83
## po-vivip          2.54    8.67
## pr-brevi          4.05   18.66
## ps-rhomb          4.80   24.04
## ps-genise         4.80   24.04
## se-heter          2.29    7.68
## se-piaba          4.80   24.04
## se-spilo          4.80   24.04
## st-noton          3.42   13.82
## sy-marmo          4.80   24.04
## te-chalc          3.28   12.02
## tr-signa          4.44   21.72
## SomaTotalN      132.46  596.69
## MediaTotalN       3.78   17.05
## DPTotalN          1.04    6.94
\end{verbatim}

A função \texttt{fix(nome\ da\ matriz)} dá acesso ao grid da matriz criada para manipulação dos dados numéricos.

\begin{Shaded}
\begin{Highlighting}[]
\CommentTok{\#Fazendo uma tabela}
\NormalTok{nf }\OtherTok{\textless{}{-}}\NormalTok{ NormalidadeFinal}
\FunctionTok{ncol}\NormalTok{(nf); }\FunctionTok{nrow}\NormalTok{(nf) }\CommentTok{\#no. de N colunas x M linhas}
\NormalTok{nf }\OtherTok{\textless{}{-}} \FunctionTok{cbind}\NormalTok{(}\AttributeTok{Spp =} \FunctionTok{rownames}\NormalTok{(nf), nf)}
\FunctionTok{gt}\NormalTok{(nf, }\AttributeTok{rowname\_col =} \StringTok{"Espécie"}\NormalTok{, }\AttributeTok{caption =} \StringTok{"Descritores da normalidade por espécie (coluna)"}\NormalTok{)}
\end{Highlighting}
\end{Shaded}

\begin{verbatim}
## [1] 2
## [1] 38
\end{verbatim}

\begin{longtable}{lrr}
\toprule
Spp & Assimetria & Curtose \\ 
\midrule\addlinespace[2.5pt]
ap-davis & 4.47 & 21.73 \\ 
as-bimac & 1.77 & 5.56 \\ 
as-fasci & 2.96 & 10.27 \\ 
ch-bimac & 2.54 & 7.74 \\ 
ci-ocela & 3.80 & 17.15 \\ 
ci-orien & 3.48 & 15.07 \\ 
co-macro & 4.80 & 24.04 \\ 
co-heter & 4.80 & 24.04 \\ 
cr-menez & 3.25 & 12.82 \\ 
cu-lepid & 4.80 & 24.04 \\ 
cy-gilbe & 3.48 & 13.78 \\ 
ge-brasi & 3.64 & 16.06 \\ 
he-margi & 3.18 & 11.08 \\ 
ho-malab & 2.28 & 7.60 \\ 
hy-pusar & 4.13 & 19.50 \\ 
le-melan & 4.80 & 24.04 \\ 
le-piau & 1.44 & 3.66 \\ 
le-taeni & 4.80 & 24.04 \\ 
mo-costa & 4.80 & 24.04 \\ 
mo-lepid & 4.79 & 23.98 \\ 
or-nilot & 2.29 & 7.97 \\ 
pa-manag & 2.87 & 9.86 \\ 
pimel-sp & 4.80 & 24.04 \\ 
po-retic & 4.46 & 21.83 \\ 
po-vivip & 2.54 & 8.67 \\ 
pr-brevi & 4.05 & 18.66 \\ 
ps-rhomb & 4.80 & 24.04 \\ 
ps-genise & 4.80 & 24.04 \\ 
se-heter & 2.29 & 7.68 \\ 
se-piaba & 4.80 & 24.04 \\ 
se-spilo & 4.80 & 24.04 \\ 
st-noton & 3.42 & 13.82 \\ 
sy-marmo & 4.80 & 24.04 \\ 
te-chalc & 3.28 & 12.02 \\ 
tr-signa & 4.44 & 21.72 \\ 
SomaTotalN & 132.46 & 596.69 \\ 
MediaTotalN & 3.78 & 17.05 \\ 
DPTotalN & 1.04 & 6.94 \\ 
\bottomrule
\end{longtable}

Nos \protect\hyperlink{descruas}{Apêndices} você pode ver o script para os mesmos descritores para as Unidades Amostrais (UA´s)

\hypertarget{lidando-com-cuxe9lulas-vazias}{%
\subsubsection{Lidando com células vazias}\label{lidando-com-cuxe9lulas-vazias}}

Em R, \texttt{NaN} (Not a Number) e \texttt{NA} (Not Available) são valores especiais que representam dados ausentes ou valores inválidos em um vetor, matriz, data frame ou outra estrutura de dados.

\begin{itemize}
\item
  \textbf{NaN (Not a Number)}: É usado para representar resultados inválidos em operações matemáticas, como a divisão por zero ou a operação de raiz quadrada de um número negativo.
\item
  \textbf{NA (Not Available)}: É usado para indicar dados ausentes. Pode ser usado em contextos onde o valor real está ausente ou desconhecido.
\end{itemize}

Em muitas operações, o \texttt{NA} é tratado como um valor especial que propagará em outras operações. Isso significa que se uma operação é realizada com um ou mais valores \texttt{NA}, o resultado geralmente será \texttt{NA}. Por outro lado, \texttt{NaN} é um valor específico que indica um resultado matematicamente indefinido ou inválido.

Por exemplo, ao realizar operações em um vetor que contém \texttt{NA}, o resultado será \texttt{NA} para qualquer operação que envolva um valor \texttt{NA}:

\begin{Shaded}
\begin{Highlighting}[]
\NormalTok{x }\OtherTok{\textless{}{-}} \FunctionTok{c}\NormalTok{(}\DecValTok{1}\NormalTok{, }\DecValTok{2}\NormalTok{, }\ConstantTok{NA}\NormalTok{, }\DecValTok{4}\NormalTok{)}
\FunctionTok{mean}\NormalTok{(x) }\CommentTok{\# Resultado será NA porque há um valor NA no vetor}
\end{Highlighting}
\end{Shaded}

\begin{verbatim}
## [1] NA
\end{verbatim}

Por outro lado, \texttt{NaN} resulta de operações matemáticas inválidas:

\begin{Shaded}
\begin{Highlighting}[]
\DecValTok{0}\SpecialCharTok{/}\DecValTok{0} \CommentTok{\# Resultado será NaN, pois a divisão por zero é indefinida}
\FunctionTok{sqrt}\NormalTok{(}\SpecialCharTok{{-}}\DecValTok{1}\NormalTok{) }\CommentTok{\# Resultado será NaN, pois não há raiz quadrada real de um número negativo}
\end{Highlighting}
\end{Shaded}

\begin{verbatim}
## Warning in sqrt(-1): NaNs produzidos
\end{verbatim}

\begin{verbatim}
## [1] NaN
## [1] NaN
\end{verbatim}

Em resumo, enquanto \texttt{NA} indica dados ausentes ou não disponíveis, \texttt{NaN} indica resultados de operações matematicamente inválidos.

\hypertarget{salvando-as-tabelas-criadas-em-.txt-direto-no-diretuxf3rio-de-trabalho}{%
\paragraph{Salvando as tabelas criadas em .txt direto no diretório de trabalho}\label{salvando-as-tabelas-criadas-em-.txt-direto-no-diretuxf3rio-de-trabalho}}

\begin{Shaded}
\begin{Highlighting}[]
\FunctionTok{write.table}\NormalTok{(}\FunctionTok{data.frame}\NormalTok{(}\StringTok{"Spp"}\OtherTok{=}\FunctionTok{rownames}\NormalTok{(DescritoresFinal),}
\NormalTok{                       DescritoresFinal),}
            \StringTok{"DescritoresSPP.txt"}\NormalTok{,}
            \AttributeTok{row.names=}\ConstantTok{FALSE}\NormalTok{,}
            \AttributeTok{sep=}\StringTok{"}\SpecialCharTok{\textbackslash{}t}\StringTok{"}\NormalTok{)}
\FunctionTok{write.table}\NormalTok{(}\FunctionTok{data.frame}\NormalTok{(}\StringTok{"Spp"}\OtherTok{=}\FunctionTok{rownames}\NormalTok{(NormalidadeFinal),}
\NormalTok{                       NormalidadeFinal),}
            \StringTok{"NormalidadeSPP.txt"}\NormalTok{,}
            \AttributeTok{row.names=}\ConstantTok{FALSE}\NormalTok{,}
            \AttributeTok{sep=}\StringTok{"}\SpecialCharTok{\textbackslash{}t}\StringTok{"}\NormalTok{)}
\end{Highlighting}
\end{Shaded}

Agora é necessário voltar à matriz comunitária (\textbf{antes de ter sido transposta}), antes de continuarmos a análise, porque as análises anteriores também precisam ser refeitas com a matriz não transposta. A seguir confere-se se está sendo usada a matriz comunitária com as espécies nas colunas.

Para ver o código dos descritores da estrutura da comunidade para as Unidades Amostrais (UA's), matriz não transposta, veja \protect\hyperlink{descruas}{Descritores da estrutura da comunidade: UA's} nos apêndices.

\begin{Shaded}
\begin{Highlighting}[]
\NormalTok{m\_trab}
\end{Highlighting}
\end{Shaded}

\begin{verbatim}
##           S-A-ZA1 S-R-CC1 S-R-CT1 S-R-CP1 S-A-TA1 S-R-CT2 S-R-CP2 S-A-TA2
## ap-davis        0       0       0       0       0       0       0       0
## as-bimac        1      99     194      19      23     142       5      46
## as-fasci        0       0      55       0       1       3       1       0
## ch-bimac        0       0       0       0      13       3       0     178
## ci-ocela        0       0       0       0       0       0      40       0
## ci-orien        0       0       5       0       0      69       9       0
## co-macro        0       0       0       0       0       0       0       0
## co-heter        0       0       1       0       0       0       0       0
## cr-menez        0       0      14       0       0       4       0       0
## cu-lepid        0       0       0       0       0       0       0       0
## cy-gilbe        0       0       0       0       0       0       0       0
## ge-brasi        0       0       3       0       0       0       0       0
## he-margi        0       0       0       0       0       1       0       0
## ho-malab        0       0       1       5       0      17      10       2
## hy-pusar        0       0       9       2       0      43       2       0
## le-melan        0       0       0       0       0       0       0       0
## le-piau         0       0       3       0       0       1       3       0
## le-taeni        0       0       0       0       0       0       0       0
## mo-costa        0       0       0       0       0       0       0       0
## mo-lepid        0       1      39       0       0       1       0       0
## or-nilot        0       2      36       0       0      77       0       0
## pa-manag        0       0       0       0       0       0       0       0
## pimel-sp        0       0       6       0       0       0       0       0
## po-retic        0       0       0       0       0      20       0       0
## po-vivip        0       0      47      15       0     221      32       0
## pr-brevi        9       0       5       0       1      15       5       2
## ps-rhomb        0       0       0       0       0       0       0       0
## ps-genise       0       0       0       0       0       0       0       0
## se-heter        0       0      40      14       4      60       0       0
## se-piaba        0       0      68       0       0       0       0       0
## se-spilo        0       0       0       0       0       0       0       0
## st-noton        0       0       1       0       0      25       0       0
## sy-marmo        0       0       0       0       0       0       1       0
## te-chalc        0       0       0       0       0       0       0       0
## tr-signa        0       0      18       0       0      15       0       0
##           S-R-CT3 S-R-CP3 S-A-TA3 S-R-CT4 S-R-CP4 S-A-TA4 B-A-MU1 B-R-ET1
## ap-davis        0       0       0       0       0       0       0       0
## as-bimac      206      16     234       0       0     394      12       3
## as-fasci       64       0       7       1       0       0       0       0
## ch-bimac        0       0     238       0       0     273       0       0
## ci-ocela        0      13       0       0      11       0       0       0
## ci-orien       25      24       0       5       6       0       0       0
## co-macro        0       0       2       0       0       0       0       0
## co-heter        0       0       0       0       0       0       0       0
## cr-menez        8       0       0       1       0       1       0       0
## cu-lepid        0       0       0       0       0       0       0       0
## cy-gilbe        0       0       0      50       0       0       0       0
## ge-brasi        1       0       0       3       0       1     190       0
## he-margi        0       0       0       1       0       0       0       0
## ho-malab       31       4      20       4       2       9       0       0
## hy-pusar       11       0       0       3       0       0       0       0
## le-melan        0       0       0       0       0       0       0       0
## le-piau         2       1       0       0       2       2       0       0
## le-taeni        0       0       0       0       0       0       0       0
## mo-costa        0       0       0       0       0       0       0       0
## mo-lepid        0       0       0       0       0       0       0       0
## or-nilot      138       0       0      73       0       1       6       8
## pa-manag        0       0       0       0       0       0       0       1
## pimel-sp        0       0       0       0       0       0       0       0
## po-retic        5       0       0       0       0       0       0      34
## po-vivip      326      10       0      28      80       0       0       0
## pr-brevi      164       0       0      59       0       3       0       0
## ps-rhomb        1       0       0       0       0       0       0       0
## ps-genise       1       0       0       0       0       0       0       0
## se-heter       38       0       0       3       3       0       0       0
## se-piaba        0       0       0       0       0       0       0       0
## se-spilo        1       0       0       0       0       0       0       0
## st-noton      115       0       0      64       0       0       0       0
## sy-marmo        0       0       0       0       0       0       0       0
## te-chalc        0       0       0       0       0       0       0       0
## tr-signa        7       0       0     141       0       0       0       0
##           B-A-GU1 B-R-PC2 B-A-MU2 B-A-GU2 B-R-PC3 B-A-MU3 B-A-GU3 B-R-PC4
## ap-davis        0       5       0       0      22       0       0       0
## as-bimac        2      44      99       0      75     511       6       7
## as-fasci        2       0       0       0       7       0       0      17
## ch-bimac        0       0       0       0       0       0       0       0
## ci-ocela        0       2       0       0       4       0       0       0
## ci-orien        0       0       0       0       0       0       0       0
## co-macro        0       0       0       0       0       0       0       0
## co-heter        0       0       0       0       0       0       0       0
## cr-menez        0       0       0       0       0       0       0       0
## cu-lepid        0       0       0       0      21       0       0       0
## cy-gilbe        0       0       0       0       0       0       0      81
## ge-brasi        7       8      67      23      16     145      32       5
## he-margi        0       0       0       0       0       0       0       0
## ho-malab        0       0       1       0       2       0       0       1
## hy-pusar        0       0       0       0       1       0       0       0
## le-melan        0       2       0       0       0       0       0       0
## le-piau         0       0       0       0       0       0       0       1
## le-taeni        0       1       0       0       0       0       0       0
## mo-costa        0       0       0       0       1       0       0       0
## mo-lepid        0       0       0       0       0       0       0       0
## or-nilot        3       5       1      36      65      11     247       9
## pa-manag       11       0       0     102       0       0     250       0
## pimel-sp        0       0       0       0       0       0       0       0
## po-retic        0       0      10       0       0      46       0       0
## po-vivip        0       0       8       0       0      48       0       0
## pr-brevi        0       9       0       0       6       1       0       0
## ps-rhomb        0       0       0       0       0       0       0       0
## ps-genise       0       0       0       0       0       0       0       0
## se-heter        0      10       0       0      93       0       0      31
## se-piaba        0       0       0       0       0       0       0       0
## se-spilo        0       0       0       0       0       0       0       0
## st-noton        0       0       0       0       0       0       0       0
## sy-marmo        0       0       0       0       0       0       0       0
## te-chalc        0      76       0       0      58       0       0       0
## tr-signa        0      23       0       0       0       0       0       4
##           B-A-MU4 B-A-GU4
## ap-davis        0       0
## as-bimac      235      13
## as-fasci        0       0
## ch-bimac        0       0
## ci-ocela        0       0
## ci-orien        0       0
## co-macro        0       0
## co-heter        0       0
## cr-menez        0       0
## cu-lepid        0       0
## cy-gilbe        0       0
## ge-brasi      509      10
## he-margi        0       0
## ho-malab        0       0
## hy-pusar        0       0
## le-melan        0       0
## le-piau         0       0
## le-taeni        0       0
## mo-costa        0       0
## mo-lepid        0       0
## or-nilot        1     129
## pa-manag        0     190
## pimel-sp        0       0
## po-retic      266       0
## po-vivip      163       0
## pr-brevi        0       0
## ps-rhomb        0       0
## ps-genise       0       0
## se-heter        0       0
## se-piaba        0       0
## se-spilo        0       0
## st-noton        0       0
## sy-marmo        0       0
## te-chalc        0       0
## tr-signa        0       0
\end{verbatim}

\begin{Shaded}
\begin{Highlighting}[]
\NormalTok{m\_trab }\OtherTok{\textless{}{-}}\NormalTok{ ppbio}
\end{Highlighting}
\end{Shaded}

\leavevmode\vadjust pre{\hypertarget{hello}{}}%
Note que em uma matriz comunitária (onde os atributos são as espécies), a matriz transposta vai conter as unidades amostrais como colunas e espécies como linhas. Não tendo a mesma interpretação de matriz transposta em matemática.

Agora faremos uma série de gráficos para a matriz comunitária.

\hypertarget{gruxe1ficos-descritivos}{%
\section{Gráficos descritivos}\label{gruxe1ficos-descritivos}}

\hypertarget{suxe9rie-de-hill-e-pefil-da-diversidade}{%
\subsection{Série de Hill e pefil da diversidade}\label{suxe9rie-de-hill-e-pefil-da-diversidade}}

Pode-se utilizar uma série de Hill ao invés dos índices específicos de diversidade. Assim, quanto maior o valor de \texttt{q} (definido em \texttt{scales}), maior será o peso para a equabilidade. Quanto mais próximo de zero, maior o peso para riqueza. Quando \texttt{q\ =\ 0}, o valor de Hill é igual a riqueza de espécies.

\begin{Shaded}
\begin{Highlighting}[]
\NormalTok{Hill }\OtherTok{\textless{}{-}} \FunctionTok{renyi}\NormalTok{(m\_trab,}
              \AttributeTok{scales =} \FunctionTok{c}\NormalTok{(}\DecValTok{0}\SpecialCharTok{:}\DecValTok{5}\NormalTok{),}
              \AttributeTok{hill =} \ConstantTok{TRUE}\NormalTok{)}
\NormalTok{Hill}
\end{Highlighting}
\end{Shaded}

\begin{verbatim}
##          0        1        2        3        4        5
## S-A-ZA1  2 1.384145 1.219512 1.170411 1.150768 1.140762
## S-R-CC1  3 1.163476 1.060983 1.045792 1.040607 1.038021
## S-R-CT1 18 8.433432 5.644824 4.479923 3.925407 3.628965
## S-R-CP1  5 4.089106 3.729963 3.562264 3.464767 3.397679
## S-A-TA1  5 2.988203 2.463687 2.265900 2.160258 2.093130
## S-R-CT2 17 8.120284 5.912807 4.995084 4.517360 4.233135
## S-R-CP2 10 5.475674 4.064111 3.571020 3.347621 3.223289
## S-A-TA2  4 1.820866 1.537624 1.437328 1.389021 1.362287
## S-R-CT3 18 7.820420 6.079509 5.349788 4.920580 4.635690
## S-R-CP3  6 4.641083 4.135957 3.852830 3.659934 3.518793
## S-A-TA3  5 2.507947 2.244026 2.186542 2.165167 2.154376
## S-R-CT4 14 6.554517 5.259698 4.665939 4.284640 4.026809
## S-R-CP4  6 2.359044 1.645269 1.479932 1.418625 1.388109
## S-A-TA4  8 2.249743 2.035378 1.981428 1.947091 1.920271
## B-A-MU1  3 1.418412 1.192503 1.145257 1.128263 1.119791
## B-R-ET1  4 2.200997 1.720325 1.562987 1.494812 1.458881
## B-A-GU1  5 3.959865 3.342246 3.016650 2.834369 2.720943
## B-R-PC2 11 5.630175 4.005266 3.427912 3.151799 2.991159
## B-A-MU2  6 2.863745 2.393193 2.248679 2.175626 2.127807
## B-A-GU2  3 2.464739 2.119634 1.930295 1.826854 1.766551
## B-R-PC3 13 7.253371 5.945359 5.447786 5.182436 5.009929
## B-A-MU3  6 2.710687 2.025352 1.799132 1.699897 1.647087
## B-A-GU3  4 2.537469 2.297723 2.232385 2.206183 2.192731
## B-R-PC4  9 4.374404 3.048096 2.586659 2.377604 2.263942
## B-A-MU4  5 3.672499 3.348321 3.103452 2.925311 2.799120
## B-A-GU4  4 2.514025 2.206452 2.107190 2.053368 2.015850
\end{verbatim}

Podemos mapear isso em um gráfico de perfil de diversidade

\begin{Shaded}
\begin{Highlighting}[]
\FunctionTok{library}\NormalTok{(ggplot2)}
\FunctionTok{library}\NormalTok{(tidyr)}
\FunctionTok{library}\NormalTok{(tidyverse)}
\NormalTok{grafico1 }\OtherTok{\textless{}{-}}\NormalTok{ Hill }\SpecialCharTok{\%\textgreater{}\%}  
  \FunctionTok{rownames\_to\_column}\NormalTok{() }\SpecialCharTok{\%\textgreater{}\%} 
  \FunctionTok{pivot\_longer}\NormalTok{(}\SpecialCharTok{{-}}\NormalTok{rowname) }\SpecialCharTok{\%\textgreater{}\%} 
  \FunctionTok{mutate}\NormalTok{(}\AttributeTok{name =} \FunctionTok{factor}\NormalTok{(name, name[}\DecValTok{1}\SpecialCharTok{:}\FunctionTok{length}\NormalTok{(Hill)])) }\SpecialCharTok{\%\textgreater{}\%} 
  \FunctionTok{ggplot}\NormalTok{(}\FunctionTok{aes}\NormalTok{(}\AttributeTok{x =}\NormalTok{ name, }\AttributeTok{y =}\NormalTok{ value, }\AttributeTok{group =}\NormalTok{ rowname,}
             \AttributeTok{col =}\NormalTok{ rowname)) }\SpecialCharTok{+}
  \FunctionTok{geom\_point}\NormalTok{(}\AttributeTok{size =} \DecValTok{2}\NormalTok{) }\SpecialCharTok{+}
  \FunctionTok{geom\_line}\NormalTok{(}\AttributeTok{size =} \DecValTok{1}\NormalTok{) }\SpecialCharTok{+}
  \FunctionTok{xlab}\NormalTok{(}\StringTok{"Parâmetro de ordem de diversidade (q)"}\NormalTok{) }\SpecialCharTok{+}
  \FunctionTok{ylab}\NormalTok{(}\StringTok{"Diversidade"}\NormalTok{) }\SpecialCharTok{+}
  \FunctionTok{labs}\NormalTok{(}\AttributeTok{col =} \StringTok{"Locais"}\NormalTok{) }\SpecialCharTok{+}
  \FunctionTok{theme\_bw}\NormalTok{() }\SpecialCharTok{+}
  \FunctionTok{theme}\NormalTok{(}\AttributeTok{text =} \FunctionTok{element\_text}\NormalTok{(}\AttributeTok{size =} \DecValTok{10}\NormalTok{)) }\CommentTok{\#ajustar a fonte caso nao caiba no output html }
\end{Highlighting}
\end{Shaded}

\begin{verbatim}
## Warning: Using `size` aesthetic for lines was deprecated in ggplot2 3.4.0.
## i Please use `linewidth` instead.
## This warning is displayed once every 8 hours.
## Call `lifecycle::last_lifecycle_warnings()` to see where this warning was
## generated.
\end{verbatim}

\begin{Shaded}
\begin{Highlighting}[]
\NormalTok{grafico1}
\FunctionTok{ggsave}\NormalTok{(grafico1, }\AttributeTok{dpi =} \DecValTok{300}\NormalTok{, }\AttributeTok{filename =} \StringTok{"fig{-}hill.png"}\NormalTok{)}
\end{Highlighting}
\end{Shaded}

\begin{verbatim}
## Saving 6.5 x 4.5 in image
\end{verbatim}

\begin{figure}
\centering
\includegraphics{fundamentos_de_ecologia_numerica_files/figure-latex/241hill-1.pdf}
\caption{\label{fig:241hill}Mapeamento dos pontos de coleta em um gráfico de perfil de diversidade.}
\end{figure}

\hypertarget{distribuiuxe7uxe3o-de-abunduxe2ncia}{%
\section{Distribuição de abundância}\label{distribuiuxe7uxe3o-de-abunduxe2ncia}}

Uma maneira de observar a diversidade de espécies é usando um gráfico de distribuição de abundâncias de uma comunidade, como mostrado a seguir. Note que aqui utiliza-se as abundâncias totais das espécies (Figura \ref{fig:241abun}), mas é possível fazer por linha, basta substituir o objeto \texttt{abund} pela abundância de uma linha. Como na linha marcada com o \texttt{\#}.

\begin{Shaded}
\begin{Highlighting}[]
\NormalTok{abund }\OtherTok{\textless{}{-}} \FunctionTok{colSums}\NormalTok{(m\_trab)}
\end{Highlighting}
\end{Shaded}

Na primeira linha do código abaixo, retira-se o \texttt{\#} do início da linha para rodar a distribuição de abundância apenas para o local (linha) 1 \texttt{comunidade{[}1,{]}}. Pode-se mudar o número para o local de interesse.

O código a seguir produzirá o gráfico para o local escolhido.

\begin{Shaded}
\begin{Highlighting}[]
\CommentTok{\#abund \textless{}{-} m\_trab[1, ] \#escolhe a primeira linha para a distribuição de abundância}
\NormalTok{df }\OtherTok{\textless{}{-}} \FunctionTok{data.frame}\NormalTok{(}\AttributeTok{sp =} \FunctionTok{colnames}\NormalTok{(m\_trab), }
                 \AttributeTok{abun =}\NormalTok{ abund) }
\NormalTok{grafico2 }\OtherTok{\textless{}{-}} \FunctionTok{ggplot}\NormalTok{(df, }\FunctionTok{aes}\NormalTok{(}\FunctionTok{fct\_reorder}\NormalTok{(sp, }\SpecialCharTok{{-}}\NormalTok{abun),}
\NormalTok{                           abun, }\AttributeTok{group =} \DecValTok{1}\NormalTok{)) }\SpecialCharTok{+}
  \FunctionTok{geom\_col}\NormalTok{() }\SpecialCharTok{+}
  \FunctionTok{geom\_line}\NormalTok{(}\AttributeTok{col =} \StringTok{"red"}\NormalTok{, }\AttributeTok{linetype =} \StringTok{"dashed"}\NormalTok{) }\SpecialCharTok{+} 
  \FunctionTok{geom\_point}\NormalTok{(}\AttributeTok{col =} \StringTok{"red"}\NormalTok{) }\SpecialCharTok{+}
  \FunctionTok{xlab}\NormalTok{(}\StringTok{"Espécies"}\NormalTok{) }\SpecialCharTok{+}
  \FunctionTok{ylab}\NormalTok{(}\StringTok{"Abundância"}\NormalTok{) }\SpecialCharTok{+}
  \FunctionTok{theme\_bw}\NormalTok{() }\SpecialCharTok{+}
  \FunctionTok{theme}\NormalTok{(}\AttributeTok{axis.text.x =} \FunctionTok{element\_text}\NormalTok{(}
    \AttributeTok{angle =} \DecValTok{45}\NormalTok{,}
    \AttributeTok{hjust =} \DecValTok{1}\NormalTok{,}
    \AttributeTok{face =} \StringTok{"italic"}\NormalTok{))}
\NormalTok{grafico2}
\FunctionTok{ggsave}\NormalTok{(grafico2, }\AttributeTok{dpi =} \DecValTok{300}\NormalTok{, }\AttributeTok{filename =} \StringTok{"fig{-}abun.png"}\NormalTok{)}
\end{Highlighting}
\end{Shaded}

\begin{verbatim}
## Saving 6.5 x 4.5 in image
\end{verbatim}

\begin{figure}
\centering
\includegraphics{fundamentos_de_ecologia_numerica_files/figure-latex/241abun-1.pdf}
\caption{\label{fig:241abun}Distribuição da densidade de indivíduos.}
\end{figure}

\hypertarget{curva-de-rarefauxe7uxe3o}{%
\section{Curva de rarefação}\label{curva-de-rarefauxe7uxe3o}}

O código abaixo indica como usar o pacote \href{https://youtu.be/4yUXG28QPLM}{iNEXT} para extrapolar e interpolar curvas de rarefação visando comparar a riqueza de diferentes locais (linhas, por isso se trabalha com a matriz transposta). A curva de raferação vai ser construida para as UA's demarcadas em \texttt{rarefa\ \textless{}-\ t(m\_trab{[}c("S-R-CT2","S-R-CP2","S-A-TA2","B-A-MU2"),{]})}

Pode-se, alternativamente, construrir a curva de raferação para as 8 linhas (UA's) com maior somatório dos valores das espécies (linhas marcadas com o símbolo {\texttt{\#}}).

\begin{Shaded}
\begin{Highlighting}[]
\NormalTok{m\_trab\_r }\OtherTok{\textless{}{-}} \FunctionTok{mutate}\NormalTok{(m\_trab, }\FunctionTok{across}\NormalTok{(}\FunctionTok{everything}\NormalTok{(), ceiling)) }\CommentTok{\#arredonda a matriz original para números inteiros}
\FunctionTok{colSums}\NormalTok{(}\FunctionTok{t}\NormalTok{(m\_trab\_r))}
\FunctionTok{colnames}\NormalTok{(}\FunctionTok{t}\NormalTok{(m\_trab\_r))}
\NormalTok{rarefa }\OtherTok{\textless{}{-}} \FunctionTok{t}\NormalTok{(m\_trab\_r[}\FunctionTok{c}\NormalTok{(}\StringTok{"S{-}R{-}CT2"}\NormalTok{,}\StringTok{"S{-}R{-}CP2"}\NormalTok{,}\StringTok{"S{-}A{-}TA2"}\NormalTok{,}\StringTok{"B{-}A{-}MU2"}\NormalTok{),]) }\CommentTok{\#curva de rarefação para os sítios especificados}
\FunctionTok{class}\NormalTok{(m\_trab)}
\DocumentationTok{\#\# Curva de rarefação para todos os sítios}
\CommentTok{\#rarefa \textless{}{-} t(m\_trab\_r)}
\DocumentationTok{\#\# Curva de rarefação para as 8 UA\textquotesingle{}s com maior soma }
\CommentTok{\#m\_trab\_r \textless{}{-} as.data.frame(t(m\_trab\_r)) \#transpõe a matriz}
\CommentTok{\#col\_sums \textless{}{-} colSums(m\_trab\_r)}
\CommentTok{\#largest\_columns \textless{}{-} names(sort(col\_sums, decreasing = TRUE)[1:8])}
\CommentTok{\#rarefa \textless{}{-} m\_trab\_r[largest\_columns] \#curva de rarefação para as 8 UA\textquotesingle{}s com maior soma}
\end{Highlighting}
\end{Shaded}

\begin{verbatim}
## S-A-ZA1 S-R-CC1 S-R-CT1 S-R-CP1 S-A-TA1 S-R-CT2 S-R-CP2 S-A-TA2 S-R-CT3 S-R-CP3 
##      10     102     545      55      42     717     108     228    1144      68 
## S-A-TA3 S-R-CT4 S-R-CP4 S-A-TA4 B-A-MU1 B-R-ET1 B-A-GU1 B-R-PC2 B-A-MU2 B-A-GU2 
##     501     436     104     684     208      46      25     185     186     161 
## B-R-PC3 B-A-MU3 B-A-GU3 B-R-PC4 B-A-MU4 B-A-GU4 
##     371     762     535     156    1174     342 
##  [1] "S-A-ZA1" "S-R-CC1" "S-R-CT1" "S-R-CP1" "S-A-TA1" "S-R-CT2" "S-R-CP2"
##  [8] "S-A-TA2" "S-R-CT3" "S-R-CP3" "S-A-TA3" "S-R-CT4" "S-R-CP4" "S-A-TA4"
## [15] "B-A-MU1" "B-R-ET1" "B-A-GU1" "B-R-PC2" "B-A-MU2" "B-A-GU2" "B-R-PC3"
## [22] "B-A-MU3" "B-A-GU3" "B-R-PC4" "B-A-MU4" "B-A-GU4"
## [1] "data.frame"
\end{verbatim}

\leavevmode\vadjust pre{\hypertarget{hello}{}}%
A curva de rarefação vai ser construida para os sítios: ``S-R-CT2'',``S-R-CP2'',``S-A-TA2'',``B-A-MU2''

No código abaixo o valor de \texttt{q} pode ser mudado para \texttt{q=1} para comparar a diversidade de Shannon e para \texttt{q=2} para Simpson (Figura (\ref{fig:241rare}). A função \texttt{endpoint=} define o tamanho da amostra para o qual se deseja extrapolar a curva de rarefação. Se \texttt{endpoint=NULL}, a extrapolação será feita para o dobro do valor amostrado.

\begin{Shaded}
\begin{Highlighting}[]
\FunctionTok{library}\NormalTok{(iNEXT)}
\NormalTok{out }\OtherTok{\textless{}{-}} \FunctionTok{iNEXT}\NormalTok{(rarefa, }\AttributeTok{q =} \DecValTok{0}\NormalTok{,}
             \AttributeTok{datatype =} \StringTok{"abundance"}\NormalTok{,}
             \AttributeTok{size =} \ConstantTok{NULL}\NormalTok{,}
             \AttributeTok{endpoint =} \DecValTok{2000}\NormalTok{, }\CommentTok{\#define o comprimento de eixo x}
             \AttributeTok{knots =} \DecValTok{40}\NormalTok{,}
             \AttributeTok{se =} \ConstantTok{TRUE}\NormalTok{,}
             \AttributeTok{conf =} \FloatTok{0.95}\NormalTok{,}
             \AttributeTok{nboot =} \DecValTok{50}\NormalTok{)}
\NormalTok{grafico3 }\OtherTok{\textless{}{-}} \FunctionTok{ggiNEXT}\NormalTok{(out, }\AttributeTok{type =} \DecValTok{1}\NormalTok{, }\AttributeTok{facet.var=}\StringTok{"None"}\NormalTok{) }\SpecialCharTok{+}
  \FunctionTok{theme\_bw}\NormalTok{() }\SpecialCharTok{+}
  \FunctionTok{labs}\NormalTok{(}\AttributeTok{fill =} \StringTok{"Áreas"}\NormalTok{) }\SpecialCharTok{+}
  \FunctionTok{xlab}\NormalTok{(}\StringTok{"Número de indivíduos"}\NormalTok{) }\SpecialCharTok{+} 
  \FunctionTok{ylab}\NormalTok{(}\StringTok{"Riqueza de espécies"}\NormalTok{) }\SpecialCharTok{+}
  \FunctionTok{theme}\NormalTok{(}\AttributeTok{legend.title=}\FunctionTok{element\_blank}\NormalTok{()) }\CommentTok{\#ver como fica com facet.var="Assemblage"}
\NormalTok{grafico3}
\FunctionTok{ggsave}\NormalTok{(grafico3, }\AttributeTok{dpi =} \DecValTok{300}\NormalTok{, }\AttributeTok{filename =} \StringTok{"fig{-}rare1.png"}\NormalTok{)}
\end{Highlighting}
\end{Shaded}

\begin{verbatim}
## Saving 6.5 x 4.5 in image
\end{verbatim}

\begin{figure}
\centering
\includegraphics{fundamentos_de_ecologia_numerica_files/figure-latex/241rare-1.pdf}
\caption{\label{fig:241rare}Curva de rafefação para sítios específicos. \texttt{q=0}, riqueza; \texttt{q=1}, diversidade de Shannon; \texttt{q=2}, diversidade de Simpson; \texttt{q=5}, equitabilidade.}
\end{figure}

A linha sólida representa a interpolação do número de espécies observadas, e a linha tracejada mostra uma extrapolação do que seria esperado dado um aumento no número de indivíduos coletados. A área mais clara representa o intervalo de confiança de 95\%.

\hypertarget{interpretando-a-curva-de-rarefauxe7uxe3o}{%
\subsection{Interpretando a curva de rarefação}\label{interpretando-a-curva-de-rarefauxe7uxe3o}}

A curva de rarefação é uma ferramenta gráfica comumente usada em ecologia para estimar e visualizar a riqueza de espécies em uma comunidade, especialmente quando se têm dados de amostragem incompletos. Ela ajuda a responder à pergunta: ``Quantas espécies diferentes podemos esperar encontrar em uma comunidade com base nas amostras que coletamos até agora?''

Aqui está uma explicação passo a passo sobre como interpretar uma curva de rarefação:

\begin{enumerate}
\def\labelenumi{\arabic{enumi}.}
\tightlist
\item
  \textbf{Eixo X (Número de Indivíduos Amostrados)}:

  \begin{itemize}
  \tightlist
  \item
    O eixo horizontal (eixo X) representa o número de indivíduos ou unidades amostrados da comunidade. Isso pode ser o número de indivíduos observados, o número de amostras coletadas ou qualquer outra unidade de amostragem que seja relevante para o estudo.
  \end{itemize}
\item
  \textbf{Eixo Y (Riqueza de Espécies)}:

  \begin{itemize}
  \tightlist
  \item
    O eixo vertical (eixo Y) representa a riqueza de espécies, ou seja, o número total de espécies diferentes observadas ou estimadas nas amostras.
  \end{itemize}
\item
  \textbf{Pontos na Curva}:

  \begin{itemize}
  \tightlist
  \item
    Cada ponto na curva de rarefação representa a riqueza de espécies estimada com base no número de indivíduos amostrados até o momento.
  \item
    À medida que você aumenta o número de indivíduos amostrados (ou seja, move-se para a direita ao longo do eixo X), a riqueza de espécies estimada também aumenta (ou seja, move-se para cima ao longo do eixo Y).
  \end{itemize}
\item
  \textbf{Inclinação da Curva}:

  \begin{itemize}
  \tightlist
  \item
    A inclinação da curva de rarefação é importante. Uma curva que sobe rapidamente indica que novas espécies estão sendo encontradas à medida que mais indivíduos são amostrados. Isso sugere que a comunidade é rica em espécies, e ainda há muitas espécies não observadas.
  \item
    Uma curva que sobe lentamente sugere que a maioria das espécies já foi observada, e a riqueza de espécies está se estabilizando. Isso indica uma comunidade menos diversa ou uma amostragem mais completa.
  \end{itemize}
\item
  \textbf{Assíntota}:

  \begin{itemize}
  \tightlist
  \item
    A assíntota é o ponto em que a curva de rarefação começa a nivelar-se, e a adição de mais indivíduos à amostra tem um impacto mínimo na riqueza de espécies estimada. A assíntota é uma estimativa da riqueza de espécies máxima que pode ser alcançada com a amostragem disponível.
  \end{itemize}
\item
  \textbf{Interpretação}:

  \begin{itemize}
  \tightlist
  \item
    A interpretação da curva de rarefação depende do contexto. Se a curva ainda estiver subindo acentuadamente no ponto em que você parou de amostrar, isso sugere que a amostragem está incompleta, e mais espécies provavelmente serão encontradas com mais esforço de amostragem.
  \item
    Se a curva estiver nivelada e próxima de uma assíntota, isso sugere que a amostragem foi mais completa, e você pode ter uma estimativa confiável da riqueza de espécies na comunidade.
  \end{itemize}
\item
  \textbf{Estimativas da Riqueza}:

  \begin{itemize}
  \tightlist
  \item
    A curva de rarefação pode ser usada para fazer estimativas da riqueza de espécies com base nas amostras coletadas. No entanto, lembre-se de que essas são estimativas e estão sujeitas a variações amostrais. É comum calcular intervalos de confiança ao redor dessas estimativas.
  \end{itemize}
\end{enumerate}

Em resumo, a curva de rarefação é uma ferramenta valiosa para estimar a riqueza de espécies em uma comunidade com base em amostras coletadas. A interpretação depende da inclinação da curva, da presença de uma assíntota e do contexto do estudo. É importante lembrar que a amostragem mais completa geralmente resulta em estimativas mais confiáveis da riqueza de espécies.

\hypertarget{curva-de-acumulauxe7uxe3o-de-espuxe9cies}{%
\section{Curva de acumulação de espécies}\label{curva-de-acumulauxe7uxe3o-de-espuxe9cies}}

\begin{Shaded}
\begin{Highlighting}[]
\NormalTok{acumula }\OtherTok{\textless{}{-}} \FunctionTok{specaccum}\NormalTok{(m\_trab,}
                     \AttributeTok{method =} \StringTok{"random"}\NormalTok{)}
\NormalTok{acumula}
\end{Highlighting}
\end{Shaded}

\begin{verbatim}
## Species Accumulation Curve
## Accumulation method: random, with 100 permutations
## Call: specaccum(comm = m_trab, method = "random") 
## 
##                                                                               
## Sites    1.0000  2.00000  3.00000  4.00000  5.00000  6.00000  7.00000  8.00000
## Richness 7.4400 12.32000 15.48000 17.67000 19.28000 20.81000 21.93000 23.07000
## sd       4.6326  5.39525  4.97972  4.95852  4.57724  4.35471  4.00317  3.75043
##                                                                                
## Sites     9.00000 10.00000 11.00000 12.00000 13.00000 14.00000 15.00000 16.0000
## Richness 24.22000 25.25000 25.98000 26.60000 27.56000 28.43000 29.01000 29.7400
## sd        3.45411  3.36763  3.25353  3.14947  2.95871  2.82577  2.81947  2.7289
##                                                                               
## Sites    17.0000 18.00000 19.00000 20.00000 21.00000 22.0000 23.00000 24.00000
## Richness 30.3500 30.87000 31.39000 32.10000 32.57000 33.0000 33.50000 34.08000
## sd        2.5361  2.54913  2.40326  2.20422  2.12372  1.8532  1.52753  1.16063
##                     
## Sites    25.00000 26
## Richness 34.46000 35
## sd        0.99919  0
\end{verbatim}

\hypertarget{plot-locais}{%
\subsection{Plot locais}\label{plot-locais}}

Plotamos uma curva de acumulação de espécies simples.

\begin{Shaded}
\begin{Highlighting}[]
\FunctionTok{plot}\NormalTok{(acumula)}
\end{Highlighting}
\end{Shaded}

\includegraphics{fundamentos_de_ecologia_numerica_files/figure-latex/unnamed-chunk-136-1.pdf}

Uma curva de acumulação de espécies é uma ferramenta gráfica usada para entender como a riqueza de espécies em uma comunidade aumenta à medida que mais amostras são coletadas ou observações são feitas. Ela é especialmente útil em estudos de biodiversidade e ecologia. Aqui está uma explicação sobre como interpretar uma curva de acumulação de espécies:

\begin{enumerate}
\def\labelenumi{\arabic{enumi}.}
\tightlist
\item
  \textbf{Eixo X (Número de Amostras ou Unidades de Observação)}:

  \begin{itemize}
  \tightlist
  \item
    O eixo horizontal (eixo X) representa o número de amostras coletadas, unidades de observação ou esforço amostral. Isso pode ser o número de áreas amostradas, pontos de observação, horas de coleta de dados ou qualquer unidade relevante para o estudo.
  \end{itemize}
\item
  \textbf{Eixo Y (Riqueza de Espécies Acumulada)}:

  \begin{itemize}
  \tightlist
  \item
    O eixo vertical (eixo Y) representa a riqueza de espécies acumulada, ou seja, o número total de espécies diferentes observadas ou registradas até o momento.
  \end{itemize}
\item
  \textbf{Pontos na Curva}:

  \begin{itemize}
  \tightlist
  \item
    Cada ponto na curva de acumulação de espécies representa a riqueza de espécies acumulada com base no número de amostras coletadas ou observações feitas até aquele ponto.
  \item
    À medida que você aumenta o número de amostras (ou seja, move-se para a direita ao longo do eixo X), a riqueza de espécies acumulada também aumenta (ou seja, move-se para cima ao longo do eixo Y).
  \end{itemize}
\item
  \textbf{Inclinação da Curva}:

  \begin{itemize}
  \tightlist
  \item
    A inclinação da curva de acumulação de espécies é importante. Uma curva que sobe rapidamente indica que novas espécies estão sendo encontradas à medida que mais amostras são coletadas. Isso sugere que a comunidade é rica em espécies e que ainda há muitas espécies não observadas.
  \item
    Uma curva que sobe lentamente sugere que a maioria das espécies já foi observada, e a riqueza de espécies está se estabilizando. Isso indica uma comunidade menos diversa ou uma amostragem mais completa.
  \end{itemize}
\item
  \textbf{Assíntota}:

  \begin{itemize}
  \tightlist
  \item
    A assíntota é o ponto onde a curva de acumulação de espécies começa a nivelar-se, e a adição de mais amostras tem um impacto mínimo na riqueza de espécies acumulada. A assíntota é uma estimativa da riqueza de espécies máxima que pode ser alcançada com o esforço amostral disponível.
  \end{itemize}
\item
  \textbf{Interpretação}:

  \begin{itemize}
  \tightlist
  \item
    A interpretação da curva de acumulação de espécies depende do contexto. Se a curva ainda estiver subindo acentuadamente no ponto em que você parou de amostrar, isso sugere que a amostragem está incompleta, e mais espécies provavelmente serão encontradas com mais esforço amostral.
  \item
    Se a curva estiver nivelada e próxima de uma assíntota, isso sugere que a amostragem foi mais completa, e você pode ter uma estimativa confiável da riqueza de espécies na comunidade.
  \end{itemize}
\item
  \textbf{Estimativas da Riqueza}:

  \begin{itemize}
  \tightlist
  \item
    A curva de acumulação de espécies pode ser usada para fazer estimativas da riqueza de espécies com base nas amostras coletadas. No entanto, lembre-se de que essas são estimativas e estão sujeitas a variações amostrais. É comum calcular intervalos de confiança ao redor dessas estimativas.
  \end{itemize}
\end{enumerate}

Em resumo, a curva de acumulação de espécies é uma ferramenta valiosa para entender como a riqueza de espécies aumenta com o esforço amostral em uma comunidade. A interpretação depende da inclinação da curva, da presença de uma assíntota e do contexto do estudo. Ela ajuda a responder à pergunta: ``Quantas espécies diferentes podemos esperar encontrar em uma comunidade com base nas amostras que coletamos até agora?''

Agora podemos plotar uma curva de acumulação de espécies melhorada.

\begin{Shaded}
\begin{Highlighting}[]
\NormalTok{plot\_data }\OtherTok{\textless{}{-}} \FunctionTok{data.frame}\NormalTok{(}\StringTok{"UAs"} \OtherTok{=} \FunctionTok{c}\NormalTok{(}\DecValTok{0}\NormalTok{, acumula}\SpecialCharTok{$}\NormalTok{sites),}
                        \StringTok{"Riqueza"} \OtherTok{=} \FunctionTok{c}\NormalTok{(}\DecValTok{0}\NormalTok{, acumula}\SpecialCharTok{$}\NormalTok{richness),}
                        \StringTok{"lower"} \OtherTok{=} \FunctionTok{c}\NormalTok{(}\DecValTok{0}\NormalTok{, acumula}\SpecialCharTok{$}\NormalTok{richness }\SpecialCharTok{{-}}\NormalTok{ acumula}\SpecialCharTok{$}\NormalTok{sd),}
                        \StringTok{"upper"} \OtherTok{=} \FunctionTok{c}\NormalTok{(}\DecValTok{0}\NormalTok{, acumula}\SpecialCharTok{$}\NormalTok{richness }\SpecialCharTok{+}\NormalTok{ acumula}\SpecialCharTok{$}\NormalTok{sd))}
\NormalTok{gLocais }\OtherTok{\textless{}{-}} \FunctionTok{ggplot}\NormalTok{(plot\_data, }\FunctionTok{aes}\NormalTok{(}\AttributeTok{x =}\NormalTok{ UAs, }\AttributeTok{y =}\NormalTok{ Riqueza)) }\SpecialCharTok{+}
  \FunctionTok{geom\_point}\NormalTok{(}\AttributeTok{color =} \StringTok{"blue"}\NormalTok{, }\AttributeTok{size =} \DecValTok{4}\NormalTok{) }\SpecialCharTok{+}
  \FunctionTok{geom\_line}\NormalTok{(}\AttributeTok{color =} \StringTok{"blue"}\NormalTok{, }\AttributeTok{lwd =} \DecValTok{2}\NormalTok{) }\SpecialCharTok{+}
  \FunctionTok{geom\_ribbon}\NormalTok{(}\FunctionTok{aes}\NormalTok{(}\AttributeTok{ymin =}\NormalTok{ lower, }\AttributeTok{ymax =}\NormalTok{ upper), }
              \AttributeTok{linetype=}\DecValTok{2}\NormalTok{, }\AttributeTok{alpha=}\FloatTok{0.3}\NormalTok{, }\AttributeTok{fill =} \StringTok{"yellow"}\NormalTok{) }\SpecialCharTok{+}
  \FunctionTok{ylab}\NormalTok{(}\StringTok{"Riqueza acumulada"}\NormalTok{) }\SpecialCharTok{+}
  \FunctionTok{theme\_classic}\NormalTok{() }\SpecialCharTok{+}
  \FunctionTok{theme}\NormalTok{(}\AttributeTok{text =} \FunctionTok{element\_text}\NormalTok{(}\AttributeTok{size =} \DecValTok{16}\NormalTok{))}
\NormalTok{gLocais}
\FunctionTok{ggsave}\NormalTok{(gLocais, }\AttributeTok{dpi =} \DecValTok{300}\NormalTok{, }\AttributeTok{filename =} \StringTok{"fig{-}acum.png"}\NormalTok{)}
\end{Highlighting}
\end{Shaded}

\begin{verbatim}
## Saving 6.5 x 4.5 in image
\end{verbatim}

\includegraphics{fundamentos_de_ecologia_numerica_files/figure-latex/unnamed-chunk-137-1.pdf}

\hypertarget{plot-indivuxedduos}{%
\section{Plot indivíduos}\label{plot-indivuxedduos}}

\begin{Shaded}
\begin{Highlighting}[]
\NormalTok{plot\_data }\OtherTok{\textless{}{-}} \FunctionTok{data.frame}\NormalTok{(}\StringTok{"Individuos"} \OtherTok{=} \FunctionTok{c}\NormalTok{(}\DecValTok{0}\NormalTok{, acumula}\SpecialCharTok{$}\NormalTok{individuals),}
                        \StringTok{"Riqueza"} \OtherTok{=} \FunctionTok{c}\NormalTok{(}\DecValTok{0}\NormalTok{, acumula}\SpecialCharTok{$}\NormalTok{richness),}
                        \StringTok{"lower"} \OtherTok{=} \FunctionTok{c}\NormalTok{(}\DecValTok{0}\NormalTok{, acumula}\SpecialCharTok{$}\NormalTok{richness }\SpecialCharTok{{-}}\NormalTok{ acumula}\SpecialCharTok{$}\NormalTok{sd),}
                        \StringTok{"upper"} \OtherTok{=} \FunctionTok{c}\NormalTok{(}\DecValTok{0}\NormalTok{, acumula}\SpecialCharTok{$}\NormalTok{richness }\SpecialCharTok{+}\NormalTok{ acumula}\SpecialCharTok{$}\NormalTok{sd))}
\NormalTok{gInd }\OtherTok{\textless{}{-}} \FunctionTok{ggplot}\NormalTok{(plot\_data, }\FunctionTok{aes}\NormalTok{(}\AttributeTok{x =}\NormalTok{ Individuos, }\AttributeTok{y =}\NormalTok{ Riqueza)) }\SpecialCharTok{+}
  \FunctionTok{geom\_point}\NormalTok{(}\AttributeTok{color =} \StringTok{"blue"}\NormalTok{, }\AttributeTok{size =} \DecValTok{4}\NormalTok{) }\SpecialCharTok{+}
  \FunctionTok{geom\_line}\NormalTok{(}\AttributeTok{color =} \StringTok{"blue"}\NormalTok{, }\AttributeTok{lwd =} \DecValTok{2}\NormalTok{) }\SpecialCharTok{+}
  \FunctionTok{geom\_ribbon}\NormalTok{(}\FunctionTok{aes}\NormalTok{(}\AttributeTok{ymin =}\NormalTok{ lower, }\AttributeTok{ymax =}\NormalTok{ upper), }
              \AttributeTok{linetype=}\DecValTok{2}\NormalTok{, }\AttributeTok{alpha=}\FloatTok{0.3}\NormalTok{, }\AttributeTok{fill =} \StringTok{"yellow"}\NormalTok{) }\SpecialCharTok{+}
  \FunctionTok{ylab}\NormalTok{(}\StringTok{"Riqueza acumulada"}\NormalTok{) }\SpecialCharTok{+}
  \FunctionTok{theme\_classic}\NormalTok{() }\SpecialCharTok{+}
  \FunctionTok{theme}\NormalTok{(}\AttributeTok{text =} \FunctionTok{element\_text}\NormalTok{(}\AttributeTok{size =} \DecValTok{16}\NormalTok{))}
\NormalTok{gInd}
\end{Highlighting}
\end{Shaded}

\includegraphics{fundamentos_de_ecologia_numerica_files/figure-latex/unnamed-chunk-138-1.pdf}

\hypertarget{apuxeandices-8}{%
\section*{Apêndices}\label{apuxeandices-8}}
\addcontentsline{toc}{section}{Apêndices}

\hypertarget{criando-uma-tabela-de-espuxe9cies}{%
\subsection*{Criando uma tabela de espécies}\label{criando-uma-tabela-de-espuxe9cies}}
\addcontentsline{toc}{subsection}{Criando uma tabela de espécies}

Nessa etapa precisaremos da tabela de grupos. Selecionamos a aba \texttt{peixes}, porque nela constam os agrupamentos \emph{a-priori} para os dados de peixes.

\begin{Shaded}
\begin{Highlighting}[]
\FunctionTok{library}\NormalTok{(openxlsx)}
\NormalTok{t\_grps }\OtherTok{\textless{}{-}} \FunctionTok{read.xlsx}\NormalTok{(}\StringTok{"D:/Elvio/OneDrive/Disciplinas/\_EcoNumerica/5.Matrizes/ppbio06{-}grupos.xlsx"}\NormalTok{,}
                   \AttributeTok{rowNames =}\NormalTok{ T, }\AttributeTok{colNames =}\NormalTok{ T,}
                   \AttributeTok{sheet =} \StringTok{"peixes"}\NormalTok{)}
\FunctionTok{str}\NormalTok{(t\_grps)}
\CommentTok{\#View(ppbio)}
\NormalTok{t\_grps[}\DecValTok{1}\SpecialCharTok{:}\DecValTok{4}\NormalTok{,}\DecValTok{1}\SpecialCharTok{:}\DecValTok{4}\NormalTok{] }\CommentTok{\#[1:4,1:4] mostra apenas as linhas e colunas de 1 a 4.}
\end{Highlighting}
\end{Shaded}

\begin{verbatim}
## 'data.frame':    26 obs. of  4 variables:
##  $ area    : chr  "Serido" "Serido" "Serido" "Serido" ...
##  $ ambiente: chr  "acude" "rio" "rio" "rio" ...
##  $ UA      : chr  "ZA" "CC" "CT" "CP" ...
##  $ coleta  : num  1 1 1 1 1 2 2 2 3 3 ...
##           area ambiente UA coleta
## S-A-ZA1 Serido    acude ZA      1
## S-R-CC1 Serido      rio CC      1
## S-R-CT1 Serido      rio CT      1
## S-R-CP1 Serido      rio CP      1
\end{verbatim}

\begin{Shaded}
\begin{Highlighting}[]
\FunctionTok{library}\NormalTok{(dplyr)}
\FunctionTok{library}\NormalTok{(tidyr)}
\NormalTok{m\_trab }\OtherTok{\textless{}{-}}\NormalTok{ m\_trab }\SpecialCharTok{\%\textgreater{}\%}
  \FunctionTok{rename\_with}\NormalTok{(}\SpecialCharTok{\textasciitilde{}} \FunctionTok{gsub}\NormalTok{(}\StringTok{"{-}"}\NormalTok{, }\StringTok{"."}\NormalTok{, .)) }\CommentTok{\#apenas troquei o hífen pelo ponto.}
\NormalTok{m }\OtherTok{\textless{}{-}}\NormalTok{ m\_trab }\SpecialCharTok{\%\textgreater{}\%}
  \FunctionTok{group\_by}\NormalTok{(}\AttributeTok{Area =}\NormalTok{ t\_grps}\SpecialCharTok{$}\NormalTok{area,}
           \AttributeTok{Habitat =}\NormalTok{ t\_grps}\SpecialCharTok{$}\NormalTok{ambiente,}
           \AttributeTok{Ponto =}\NormalTok{ t\_grps}\SpecialCharTok{$}\NormalTok{UA) }\SpecialCharTok{\%\textgreater{}\%}
  \FunctionTok{summarise}\NormalTok{(}\FunctionTok{across}\NormalTok{(}\FunctionTok{where}\NormalTok{(is.numeric),}
                   \FunctionTok{list}\NormalTok{(}\AttributeTok{mean =}\NormalTok{ mean, }\AttributeTok{sd =}\NormalTok{ sd)),}
            \AttributeTok{.groups =} \StringTok{\textquotesingle{}drop\textquotesingle{}}\NormalTok{) }\SpecialCharTok{\%\textgreater{}\%}
  \FunctionTok{pivot\_longer}\NormalTok{(}
    \AttributeTok{cols =} \SpecialCharTok{{-}}\FunctionTok{c}\NormalTok{(Area, Habitat, Ponto),}
    \AttributeTok{names\_to =} \FunctionTok{c}\NormalTok{(}\StringTok{"Variable"}\NormalTok{, }\StringTok{".value"}\NormalTok{),}
    \AttributeTok{names\_sep =} \StringTok{"\_"}
\NormalTok{  )}
\NormalTok{m}
\NormalTok{m }\OtherTok{\textless{}{-}} \FunctionTok{as.data.frame}\NormalTok{(m)}
\NormalTok{m\_wide }\OtherTok{\textless{}{-}}\NormalTok{ m }\SpecialCharTok{\%\textgreater{}\%}
  \FunctionTok{mutate}\NormalTok{(}\AttributeTok{stat\_string =} \FunctionTok{ifelse}\NormalTok{(}
\NormalTok{    Variable }\SpecialCharTok{==} \StringTok{"nome\_da\_variavel"}\NormalTok{,  }\CommentTok{\#escolhe uma variável pra ter 3 casas decimais}
    \FunctionTok{paste0}\NormalTok{(}\FunctionTok{round}\NormalTok{(mean, }\DecValTok{3}\NormalTok{), }\StringTok{"("}\NormalTok{, }\FunctionTok{round}\NormalTok{(sd, }\DecValTok{3}\NormalTok{), }\StringTok{")"}\NormalTok{),}
    \FunctionTok{paste0}\NormalTok{(}\FunctionTok{round}\NormalTok{(mean, }\DecValTok{1}\NormalTok{), }\StringTok{"("}\NormalTok{, }\FunctionTok{round}\NormalTok{(sd, }\DecValTok{1}\NormalTok{), }\StringTok{")"}\NormalTok{)}
\NormalTok{  )) }\SpecialCharTok{\%\textgreater{}\%}
  \FunctionTok{unite}\NormalTok{(}\StringTok{"Location"}\NormalTok{, Area, Habitat, Ponto, }\AttributeTok{sep =} \StringTok{"\_"}\NormalTok{) }\SpecialCharTok{\%\textgreater{}\%}
\NormalTok{  dplyr}\SpecialCharTok{::}\FunctionTok{select}\NormalTok{(Variable, Location, stat\_string) }\SpecialCharTok{\%\textgreater{}\%} \CommentTok{\#dplyr dá conflito com MASS}
  \FunctionTok{pivot\_wider}\NormalTok{(}\AttributeTok{names\_from =}\NormalTok{ Location, }\AttributeTok{values\_from =}\NormalTok{ stat\_string)}

\NormalTok{m\_wide}
\NormalTok{m\_wide }\OtherTok{\textless{}{-}} \FunctionTok{as.data.frame}\NormalTok{(m\_wide)}
\NormalTok{m\_wide}

\CommentTok{\#Exportando dados para Excel{-}{-}{-}{-}}
\FunctionTok{library}\NormalTok{(openxlsx)}
\FunctionTok{write.xlsx}\NormalTok{(m\_wide, }\AttributeTok{file =} \StringTok{"tabela da comunidade.xlsx"}\NormalTok{, }\AttributeTok{rowNames =} \ConstantTok{FALSE}\NormalTok{)}
\NormalTok{wb }\OtherTok{\textless{}{-}} \FunctionTok{loadWorkbook}\NormalTok{(}\StringTok{"tabela da comunidade.xlsx"}\NormalTok{)}
\FunctionTok{writeData}\NormalTok{(wb, }\AttributeTok{sheet =} \StringTok{"Sheet 1"}\NormalTok{, }\AttributeTok{x =}\NormalTok{ m\_wide)}
\FunctionTok{saveWorkbook}\NormalTok{(wb, }\StringTok{"tabela da comunidade.xlsx"}\NormalTok{, }\AttributeTok{overwrite =} \ConstantTok{TRUE}\NormalTok{)}

\CommentTok{\#Escolher sumário de uma variavel{-}{-}{-}{-}}
\NormalTok{m}
\NormalTok{var }\OtherTok{\textless{}{-}} \StringTok{"as.bimac"}
\NormalTok{m[m}\SpecialCharTok{$}\NormalTok{Variable }\SpecialCharTok{==}\NormalTok{ var, }\StringTok{"mean"}\NormalTok{] }\CommentTok{\#cada valor de var}
\FunctionTok{summary}\NormalTok{(m[m}\SpecialCharTok{$}\NormalTok{Variable }\SpecialCharTok{==}\NormalTok{ var, }\StringTok{"mean"}\NormalTok{]) }\CommentTok{\#sumário dos valores de var}
\end{Highlighting}
\end{Shaded}

\begin{verbatim}
## # A tibble: 315 x 6
##    Area   Habitat Ponto Variable  mean    sd
##    <chr>  <chr>   <chr> <chr>    <dbl> <dbl>
##  1 Buique acude   GU    ap.davis  0     0   
##  2 Buique acude   GU    as.bimac  5.25  5.74
##  3 Buique acude   GU    as.fasci  0.5   1   
##  4 Buique acude   GU    ch.bimac  0     0   
##  5 Buique acude   GU    ci.ocela  0     0   
##  6 Buique acude   GU    ci.orien  0     0   
##  7 Buique acude   GU    co.macro  0     0   
##  8 Buique acude   GU    co.heter  0     0   
##  9 Buique acude   GU    cr.menez  0     0   
## 10 Buique acude   GU    cu.lepid  0     0   
## # i 305 more rows
## # A tibble: 35 x 10
##    Variable Buique_acude_GU Buique_acude_MU Buique_rio_ET Buique_rio_PC
##    <chr>    <chr>           <chr>           <chr>         <chr>        
##  1 ap.davis 0(0)            0(0)            0(NA)         9(11.5)      
##  2 as.bimac 5.2(5.7)        214.2(218.1)    3(NA)         42(34)       
##  3 as.fasci 0.5(1)          0(0)            0(NA)         8(8.5)       
##  4 ch.bimac 0(0)            0(0)            0(NA)         0(0)         
##  5 ci.ocela 0(0)            0(0)            0(NA)         2(2)         
##  6 ci.orien 0(0)            0(0)            0(NA)         0(0)         
##  7 co.macro 0(0)            0(0)            0(NA)         0(0)         
##  8 co.heter 0(0)            0(0)            0(NA)         0(0)         
##  9 cr.menez 0(0)            0(0)            0(NA)         0(0)         
## 10 cu.lepid 0(0)            0(0)            0(NA)         7(12.1)      
## # i 25 more rows
## # i 5 more variables: Serido_acude_TA <chr>, Serido_acude_ZA <chr>,
## #   Serido_rio_CC <chr>, Serido_rio_CP <chr>, Serido_rio_CT <chr>
##     Variable Buique_acude_GU Buique_acude_MU Buique_rio_ET Buique_rio_PC
## 1   ap.davis            0(0)            0(0)         0(NA)       9(11.5)
## 2   as.bimac        5.2(5.7)    214.2(218.1)         3(NA)        42(34)
## 3   as.fasci          0.5(1)            0(0)         0(NA)        8(8.5)
## 4   ch.bimac            0(0)            0(0)         0(NA)          0(0)
## 5   ci.ocela            0(0)            0(0)         0(NA)          2(2)
## 6   ci.orien            0(0)            0(0)         0(NA)          0(0)
## 7   co.macro            0(0)            0(0)         0(NA)          0(0)
## 8   co.heter            0(0)            0(0)         0(NA)          0(0)
## 9   cr.menez            0(0)            0(0)         0(NA)          0(0)
## 10  cu.lepid            0(0)            0(0)         0(NA)       7(12.1)
## 11  cy.gilbe            0(0)            0(0)         0(NA)      27(46.8)
## 12  ge.brasi        18(11.6)    227.8(194.3)         0(NA)      9.7(5.7)
## 13  he.margi            0(0)            0(0)         0(NA)          0(0)
## 14  ho.malab            0(0)        0.2(0.5)         0(NA)          1(1)
## 15  hy.pusar            0(0)            0(0)         0(NA)      0.3(0.6)
## 16  le.melan            0(0)            0(0)         0(NA)      0.7(1.2)
## 17   le.piau            0(0)            0(0)         0(NA)      0.3(0.6)
## 18  le.taeni            0(0)            0(0)         0(NA)      0.3(0.6)
## 19  mo.costa            0(0)            0(0)         0(NA)      0.3(0.6)
## 20  mo.lepid            0(0)            0(0)         0(NA)          0(0)
## 21  or.nilot    103.8(109.4)        4.8(4.8)         8(NA)    26.3(33.5)
## 22  pa.manag    138.2(104.4)            0(0)         1(NA)          0(0)
## 23  pimel.sp            0(0)            0(0)         0(NA)          0(0)
## 24  po.retic            0(0)     80.5(125.2)        34(NA)          0(0)
## 25  po.vivip            0(0)      54.8(75.2)         0(NA)          0(0)
## 26  pr.brevi            0(0)        0.2(0.5)         0(NA)        5(4.6)
## 27  ps.rhomb            0(0)            0(0)         0(NA)          0(0)
## 28 ps.genise            0(0)            0(0)         0(NA)          0(0)
## 29  se.heter            0(0)            0(0)         0(NA)    44.7(43.2)
## 30  se.piaba            0(0)            0(0)         0(NA)          0(0)
## 31  se.spilo            0(0)            0(0)         0(NA)          0(0)
## 32  st.noton            0(0)            0(0)         0(NA)          0(0)
## 33  sy.marmo            0(0)            0(0)         0(NA)          0(0)
## 34  te.chalc            0(0)            0(0)         0(NA)    44.7(39.7)
## 35  tr.signa            0(0)            0(0)         0(NA)       9(12.3)
##    Serido_acude_TA Serido_acude_ZA Serido_rio_CC Serido_rio_CP Serido_rio_CT
## 1             0(0)           0(NA)         0(NA)          0(0)          0(0)
## 2     174.2(174.3)           1(NA)        99(NA)         10(9)   135.5(94.5)
## 3           2(3.4)           0(NA)         0(NA)      0.2(0.5)    30.8(33.4)
## 4     175.5(115.2)           0(NA)         0(NA)          0(0)      0.8(1.5)
## 5             0(0)           0(NA)         0(NA)        16(17)          0(0)
## 6             0(0)           0(NA)         0(NA)     9.8(10.2)      26(30.2)
## 7           0.5(1)           0(NA)         0(NA)          0(0)          0(0)
## 8             0(0)           0(NA)         0(NA)          0(0)      0.2(0.5)
## 9         0.2(0.5)           0(NA)         0(NA)          0(0)      6.8(5.6)
## 10            0(0)           0(NA)         0(NA)          0(0)          0(0)
## 11            0(0)           0(NA)         0(NA)          0(0)      12.5(25)
## 12        0.2(0.5)           0(NA)         0(NA)          0(0)      1.8(1.5)
## 13            0(0)           0(NA)         0(NA)          0(0)      0.5(0.6)
## 14          7.8(9)           0(NA)         0(NA)      5.2(3.4)    13.2(13.7)
## 15            0(0)           0(NA)         0(NA)        1(1.2)      16.5(18)
## 16            0(0)           0(NA)         0(NA)          0(0)          0(0)
## 17          0.5(1)           0(NA)         0(NA)      1.5(1.3)      1.5(1.3)
## 18            0(0)           0(NA)         0(NA)          0(0)          0(0)
## 19            0(0)           0(NA)         0(NA)          0(0)          0(0)
## 20            0(0)           0(NA)         1(NA)          0(0)      10(19.3)
## 21        0.2(0.5)           0(NA)         2(NA)          0(0)      81(42.2)
## 22            0(0)           0(NA)         0(NA)          0(0)          0(0)
## 23            0(0)           0(NA)         0(NA)          0(0)        1.5(3)
## 24            0(0)           0(NA)         0(NA)          0(0)      6.2(9.5)
## 25            0(0)           0(NA)         0(NA)    34.2(31.9)    155.5(143)
## 26        1.5(1.3)           9(NA)         0(NA)      1.2(2.5)    60.8(72.7)
## 27            0(0)           0(NA)         0(NA)          0(0)      0.2(0.5)
## 28            0(0)           0(NA)         0(NA)          0(0)      0.2(0.5)
## 29            1(2)           0(NA)         0(NA)      4.2(6.7)    35.2(23.7)
## 30            0(0)           0(NA)         0(NA)          0(0)        17(34)
## 31            0(0)           0(NA)         0(NA)          0(0)      0.2(0.5)
## 32            0(0)           0(NA)         0(NA)          0(0)    51.2(49.8)
## 33            0(0)           0(NA)         0(NA)      0.2(0.5)          0(0)
## 34            0(0)           0(NA)         0(NA)          0(0)          0(0)
## 35            0(0)           0(NA)         0(NA)          0(0)      45.2(64)
##       Area Habitat Ponto  Variable        mean          sd
## 1   Buique   acude    GU  ap.davis   0.0000000   0.0000000
## 2   Buique   acude    GU  as.bimac   5.2500000   5.7373048
## 3   Buique   acude    GU  as.fasci   0.5000000   1.0000000
## 4   Buique   acude    GU  ch.bimac   0.0000000   0.0000000
## 5   Buique   acude    GU  ci.ocela   0.0000000   0.0000000
## 6   Buique   acude    GU  ci.orien   0.0000000   0.0000000
## 7   Buique   acude    GU  co.macro   0.0000000   0.0000000
## 8   Buique   acude    GU  co.heter   0.0000000   0.0000000
## 9   Buique   acude    GU  cr.menez   0.0000000   0.0000000
## 10  Buique   acude    GU  cu.lepid   0.0000000   0.0000000
## 11  Buique   acude    GU  cy.gilbe   0.0000000   0.0000000
## 12  Buique   acude    GU  ge.brasi  18.0000000  11.6332856
## 13  Buique   acude    GU  he.margi   0.0000000   0.0000000
## 14  Buique   acude    GU  ho.malab   0.0000000   0.0000000
## 15  Buique   acude    GU  hy.pusar   0.0000000   0.0000000
## 16  Buique   acude    GU  le.melan   0.0000000   0.0000000
## 17  Buique   acude    GU   le.piau   0.0000000   0.0000000
## 18  Buique   acude    GU  le.taeni   0.0000000   0.0000000
## 19  Buique   acude    GU  mo.costa   0.0000000   0.0000000
## 20  Buique   acude    GU  mo.lepid   0.0000000   0.0000000
## 21  Buique   acude    GU  or.nilot 103.7500000 109.3903561
## 22  Buique   acude    GU  pa.manag 138.2500000 104.3595547
## 23  Buique   acude    GU  pimel.sp   0.0000000   0.0000000
## 24  Buique   acude    GU  po.retic   0.0000000   0.0000000
## 25  Buique   acude    GU  po.vivip   0.0000000   0.0000000
## 26  Buique   acude    GU  pr.brevi   0.0000000   0.0000000
## 27  Buique   acude    GU  ps.rhomb   0.0000000   0.0000000
## 28  Buique   acude    GU ps.genise   0.0000000   0.0000000
## 29  Buique   acude    GU  se.heter   0.0000000   0.0000000
## 30  Buique   acude    GU  se.piaba   0.0000000   0.0000000
## 31  Buique   acude    GU  se.spilo   0.0000000   0.0000000
## 32  Buique   acude    GU  st.noton   0.0000000   0.0000000
## 33  Buique   acude    GU  sy.marmo   0.0000000   0.0000000
## 34  Buique   acude    GU  te.chalc   0.0000000   0.0000000
## 35  Buique   acude    GU  tr.signa   0.0000000   0.0000000
## 36  Buique   acude    MU  ap.davis   0.0000000   0.0000000
## 37  Buique   acude    MU  as.bimac 214.2500000 218.0815979
## 38  Buique   acude    MU  as.fasci   0.0000000   0.0000000
## 39  Buique   acude    MU  ch.bimac   0.0000000   0.0000000
## 40  Buique   acude    MU  ci.ocela   0.0000000   0.0000000
## 41  Buique   acude    MU  ci.orien   0.0000000   0.0000000
## 42  Buique   acude    MU  co.macro   0.0000000   0.0000000
## 43  Buique   acude    MU  co.heter   0.0000000   0.0000000
## 44  Buique   acude    MU  cr.menez   0.0000000   0.0000000
## 45  Buique   acude    MU  cu.lepid   0.0000000   0.0000000
## 46  Buique   acude    MU  cy.gilbe   0.0000000   0.0000000
## 47  Buique   acude    MU  ge.brasi 227.7500000 194.2633522
## 48  Buique   acude    MU  he.margi   0.0000000   0.0000000
## 49  Buique   acude    MU  ho.malab   0.2500000   0.5000000
## 50  Buique   acude    MU  hy.pusar   0.0000000   0.0000000
## 51  Buique   acude    MU  le.melan   0.0000000   0.0000000
## 52  Buique   acude    MU   le.piau   0.0000000   0.0000000
## 53  Buique   acude    MU  le.taeni   0.0000000   0.0000000
## 54  Buique   acude    MU  mo.costa   0.0000000   0.0000000
## 55  Buique   acude    MU  mo.lepid   0.0000000   0.0000000
## 56  Buique   acude    MU  or.nilot   4.7500000   4.7871355
## 57  Buique   acude    MU  pa.manag   0.0000000   0.0000000
## 58  Buique   acude    MU  pimel.sp   0.0000000   0.0000000
## 59  Buique   acude    MU  po.retic  80.5000000 125.2344468
## 60  Buique   acude    MU  po.vivip  54.7500000  75.1592753
## 61  Buique   acude    MU  pr.brevi   0.2500000   0.5000000
## 62  Buique   acude    MU  ps.rhomb   0.0000000   0.0000000
## 63  Buique   acude    MU ps.genise   0.0000000   0.0000000
## 64  Buique   acude    MU  se.heter   0.0000000   0.0000000
## 65  Buique   acude    MU  se.piaba   0.0000000   0.0000000
## 66  Buique   acude    MU  se.spilo   0.0000000   0.0000000
## 67  Buique   acude    MU  st.noton   0.0000000   0.0000000
## 68  Buique   acude    MU  sy.marmo   0.0000000   0.0000000
## 69  Buique   acude    MU  te.chalc   0.0000000   0.0000000
## 70  Buique   acude    MU  tr.signa   0.0000000   0.0000000
## 71  Buique     rio    ET  ap.davis   0.0000000          NA
## 72  Buique     rio    ET  as.bimac   3.0000000          NA
## 73  Buique     rio    ET  as.fasci   0.0000000          NA
## 74  Buique     rio    ET  ch.bimac   0.0000000          NA
## 75  Buique     rio    ET  ci.ocela   0.0000000          NA
## 76  Buique     rio    ET  ci.orien   0.0000000          NA
## 77  Buique     rio    ET  co.macro   0.0000000          NA
## 78  Buique     rio    ET  co.heter   0.0000000          NA
## 79  Buique     rio    ET  cr.menez   0.0000000          NA
## 80  Buique     rio    ET  cu.lepid   0.0000000          NA
## 81  Buique     rio    ET  cy.gilbe   0.0000000          NA
## 82  Buique     rio    ET  ge.brasi   0.0000000          NA
## 83  Buique     rio    ET  he.margi   0.0000000          NA
## 84  Buique     rio    ET  ho.malab   0.0000000          NA
## 85  Buique     rio    ET  hy.pusar   0.0000000          NA
## 86  Buique     rio    ET  le.melan   0.0000000          NA
## 87  Buique     rio    ET   le.piau   0.0000000          NA
## 88  Buique     rio    ET  le.taeni   0.0000000          NA
## 89  Buique     rio    ET  mo.costa   0.0000000          NA
## 90  Buique     rio    ET  mo.lepid   0.0000000          NA
## 91  Buique     rio    ET  or.nilot   8.0000000          NA
## 92  Buique     rio    ET  pa.manag   1.0000000          NA
## 93  Buique     rio    ET  pimel.sp   0.0000000          NA
## 94  Buique     rio    ET  po.retic  34.0000000          NA
## 95  Buique     rio    ET  po.vivip   0.0000000          NA
## 96  Buique     rio    ET  pr.brevi   0.0000000          NA
## 97  Buique     rio    ET  ps.rhomb   0.0000000          NA
## 98  Buique     rio    ET ps.genise   0.0000000          NA
## 99  Buique     rio    ET  se.heter   0.0000000          NA
## 100 Buique     rio    ET  se.piaba   0.0000000          NA
## 101 Buique     rio    ET  se.spilo   0.0000000          NA
## 102 Buique     rio    ET  st.noton   0.0000000          NA
## 103 Buique     rio    ET  sy.marmo   0.0000000          NA
## 104 Buique     rio    ET  te.chalc   0.0000000          NA
## 105 Buique     rio    ET  tr.signa   0.0000000          NA
## 106 Buique     rio    PC  ap.davis   9.0000000  11.5325626
## 107 Buique     rio    PC  as.bimac  42.0000000  34.0440891
## 108 Buique     rio    PC  as.fasci   8.0000000   8.5440037
## 109 Buique     rio    PC  ch.bimac   0.0000000   0.0000000
## 110 Buique     rio    PC  ci.ocela   2.0000000   2.0000000
## 111 Buique     rio    PC  ci.orien   0.0000000   0.0000000
## 112 Buique     rio    PC  co.macro   0.0000000   0.0000000
## 113 Buique     rio    PC  co.heter   0.0000000   0.0000000
## 114 Buique     rio    PC  cr.menez   0.0000000   0.0000000
## 115 Buique     rio    PC  cu.lepid   7.0000000  12.1243557
## 116 Buique     rio    PC  cy.gilbe  27.0000000  46.7653718
## 117 Buique     rio    PC  ge.brasi   9.6666667   5.6862407
## 118 Buique     rio    PC  he.margi   0.0000000   0.0000000
## 119 Buique     rio    PC  ho.malab   1.0000000   1.0000000
## 120 Buique     rio    PC  hy.pusar   0.3333333   0.5773503
## 121 Buique     rio    PC  le.melan   0.6666667   1.1547005
## 122 Buique     rio    PC   le.piau   0.3333333   0.5773503
## 123 Buique     rio    PC  le.taeni   0.3333333   0.5773503
## 124 Buique     rio    PC  mo.costa   0.3333333   0.5773503
## 125 Buique     rio    PC  mo.lepid   0.0000000   0.0000000
## 126 Buique     rio    PC  or.nilot  26.3333333  33.5459883
## 127 Buique     rio    PC  pa.manag   0.0000000   0.0000000
## 128 Buique     rio    PC  pimel.sp   0.0000000   0.0000000
## 129 Buique     rio    PC  po.retic   0.0000000   0.0000000
## 130 Buique     rio    PC  po.vivip   0.0000000   0.0000000
## 131 Buique     rio    PC  pr.brevi   5.0000000   4.5825757
## 132 Buique     rio    PC  ps.rhomb   0.0000000   0.0000000
## 133 Buique     rio    PC ps.genise   0.0000000   0.0000000
## 134 Buique     rio    PC  se.heter  44.6666667  43.1547603
## 135 Buique     rio    PC  se.piaba   0.0000000   0.0000000
## 136 Buique     rio    PC  se.spilo   0.0000000   0.0000000
## 137 Buique     rio    PC  st.noton   0.0000000   0.0000000
## 138 Buique     rio    PC  sy.marmo   0.0000000   0.0000000
## 139 Buique     rio    PC  te.chalc  44.6666667  39.7156560
## 140 Buique     rio    PC  tr.signa   9.0000000  12.2882057
## 141 Serido   acude    TA  ap.davis   0.0000000   0.0000000
## 142 Serido   acude    TA  as.bimac 174.2500000 174.3413797
## 143 Serido   acude    TA  as.fasci   2.0000000   3.3665016
## 144 Serido   acude    TA  ch.bimac 175.5000000 115.2171862
## 145 Serido   acude    TA  ci.ocela   0.0000000   0.0000000
## 146 Serido   acude    TA  ci.orien   0.0000000   0.0000000
## 147 Serido   acude    TA  co.macro   0.5000000   1.0000000
## 148 Serido   acude    TA  co.heter   0.0000000   0.0000000
## 149 Serido   acude    TA  cr.menez   0.2500000   0.5000000
## 150 Serido   acude    TA  cu.lepid   0.0000000   0.0000000
## 151 Serido   acude    TA  cy.gilbe   0.0000000   0.0000000
## 152 Serido   acude    TA  ge.brasi   0.2500000   0.5000000
## 153 Serido   acude    TA  he.margi   0.0000000   0.0000000
## 154 Serido   acude    TA  ho.malab   7.7500000   9.0323493
## 155 Serido   acude    TA  hy.pusar   0.0000000   0.0000000
## 156 Serido   acude    TA  le.melan   0.0000000   0.0000000
## 157 Serido   acude    TA   le.piau   0.5000000   1.0000000
## 158 Serido   acude    TA  le.taeni   0.0000000   0.0000000
## 159 Serido   acude    TA  mo.costa   0.0000000   0.0000000
## 160 Serido   acude    TA  mo.lepid   0.0000000   0.0000000
## 161 Serido   acude    TA  or.nilot   0.2500000   0.5000000
## 162 Serido   acude    TA  pa.manag   0.0000000   0.0000000
## 163 Serido   acude    TA  pimel.sp   0.0000000   0.0000000
## 164 Serido   acude    TA  po.retic   0.0000000   0.0000000
## 165 Serido   acude    TA  po.vivip   0.0000000   0.0000000
## 166 Serido   acude    TA  pr.brevi   1.5000000   1.2909944
## 167 Serido   acude    TA  ps.rhomb   0.0000000   0.0000000
## 168 Serido   acude    TA ps.genise   0.0000000   0.0000000
## 169 Serido   acude    TA  se.heter   1.0000000   2.0000000
## 170 Serido   acude    TA  se.piaba   0.0000000   0.0000000
## 171 Serido   acude    TA  se.spilo   0.0000000   0.0000000
## 172 Serido   acude    TA  st.noton   0.0000000   0.0000000
## 173 Serido   acude    TA  sy.marmo   0.0000000   0.0000000
## 174 Serido   acude    TA  te.chalc   0.0000000   0.0000000
## 175 Serido   acude    TA  tr.signa   0.0000000   0.0000000
## 176 Serido   acude    ZA  ap.davis   0.0000000          NA
## 177 Serido   acude    ZA  as.bimac   1.0000000          NA
## 178 Serido   acude    ZA  as.fasci   0.0000000          NA
## 179 Serido   acude    ZA  ch.bimac   0.0000000          NA
## 180 Serido   acude    ZA  ci.ocela   0.0000000          NA
## 181 Serido   acude    ZA  ci.orien   0.0000000          NA
## 182 Serido   acude    ZA  co.macro   0.0000000          NA
## 183 Serido   acude    ZA  co.heter   0.0000000          NA
## 184 Serido   acude    ZA  cr.menez   0.0000000          NA
## 185 Serido   acude    ZA  cu.lepid   0.0000000          NA
## 186 Serido   acude    ZA  cy.gilbe   0.0000000          NA
## 187 Serido   acude    ZA  ge.brasi   0.0000000          NA
## 188 Serido   acude    ZA  he.margi   0.0000000          NA
## 189 Serido   acude    ZA  ho.malab   0.0000000          NA
## 190 Serido   acude    ZA  hy.pusar   0.0000000          NA
## 191 Serido   acude    ZA  le.melan   0.0000000          NA
## 192 Serido   acude    ZA   le.piau   0.0000000          NA
## 193 Serido   acude    ZA  le.taeni   0.0000000          NA
## 194 Serido   acude    ZA  mo.costa   0.0000000          NA
## 195 Serido   acude    ZA  mo.lepid   0.0000000          NA
## 196 Serido   acude    ZA  or.nilot   0.0000000          NA
## 197 Serido   acude    ZA  pa.manag   0.0000000          NA
## 198 Serido   acude    ZA  pimel.sp   0.0000000          NA
## 199 Serido   acude    ZA  po.retic   0.0000000          NA
## 200 Serido   acude    ZA  po.vivip   0.0000000          NA
## 201 Serido   acude    ZA  pr.brevi   9.0000000          NA
## 202 Serido   acude    ZA  ps.rhomb   0.0000000          NA
## 203 Serido   acude    ZA ps.genise   0.0000000          NA
## 204 Serido   acude    ZA  se.heter   0.0000000          NA
## 205 Serido   acude    ZA  se.piaba   0.0000000          NA
## 206 Serido   acude    ZA  se.spilo   0.0000000          NA
## 207 Serido   acude    ZA  st.noton   0.0000000          NA
## 208 Serido   acude    ZA  sy.marmo   0.0000000          NA
## 209 Serido   acude    ZA  te.chalc   0.0000000          NA
## 210 Serido   acude    ZA  tr.signa   0.0000000          NA
## 211 Serido     rio    CC  ap.davis   0.0000000          NA
## 212 Serido     rio    CC  as.bimac  99.0000000          NA
## 213 Serido     rio    CC  as.fasci   0.0000000          NA
## 214 Serido     rio    CC  ch.bimac   0.0000000          NA
## 215 Serido     rio    CC  ci.ocela   0.0000000          NA
## 216 Serido     rio    CC  ci.orien   0.0000000          NA
## 217 Serido     rio    CC  co.macro   0.0000000          NA
## 218 Serido     rio    CC  co.heter   0.0000000          NA
## 219 Serido     rio    CC  cr.menez   0.0000000          NA
## 220 Serido     rio    CC  cu.lepid   0.0000000          NA
## 221 Serido     rio    CC  cy.gilbe   0.0000000          NA
## 222 Serido     rio    CC  ge.brasi   0.0000000          NA
## 223 Serido     rio    CC  he.margi   0.0000000          NA
## 224 Serido     rio    CC  ho.malab   0.0000000          NA
## 225 Serido     rio    CC  hy.pusar   0.0000000          NA
## 226 Serido     rio    CC  le.melan   0.0000000          NA
## 227 Serido     rio    CC   le.piau   0.0000000          NA
## 228 Serido     rio    CC  le.taeni   0.0000000          NA
## 229 Serido     rio    CC  mo.costa   0.0000000          NA
## 230 Serido     rio    CC  mo.lepid   1.0000000          NA
## 231 Serido     rio    CC  or.nilot   2.0000000          NA
## 232 Serido     rio    CC  pa.manag   0.0000000          NA
## 233 Serido     rio    CC  pimel.sp   0.0000000          NA
## 234 Serido     rio    CC  po.retic   0.0000000          NA
## 235 Serido     rio    CC  po.vivip   0.0000000          NA
## 236 Serido     rio    CC  pr.brevi   0.0000000          NA
## 237 Serido     rio    CC  ps.rhomb   0.0000000          NA
## 238 Serido     rio    CC ps.genise   0.0000000          NA
## 239 Serido     rio    CC  se.heter   0.0000000          NA
## 240 Serido     rio    CC  se.piaba   0.0000000          NA
## 241 Serido     rio    CC  se.spilo   0.0000000          NA
## 242 Serido     rio    CC  st.noton   0.0000000          NA
## 243 Serido     rio    CC  sy.marmo   0.0000000          NA
## 244 Serido     rio    CC  te.chalc   0.0000000          NA
## 245 Serido     rio    CC  tr.signa   0.0000000          NA
## 246 Serido     rio    CP  ap.davis   0.0000000   0.0000000
## 247 Serido     rio    CP  as.bimac  10.0000000   8.9814624
## 248 Serido     rio    CP  as.fasci   0.2500000   0.5000000
## 249 Serido     rio    CP  ch.bimac   0.0000000   0.0000000
## 250 Serido     rio    CP  ci.ocela  16.0000000  16.9901932
## 251 Serido     rio    CP  ci.orien   9.7500000  10.2102889
## 252 Serido     rio    CP  co.macro   0.0000000   0.0000000
## 253 Serido     rio    CP  co.heter   0.0000000   0.0000000
## 254 Serido     rio    CP  cr.menez   0.0000000   0.0000000
## 255 Serido     rio    CP  cu.lepid   0.0000000   0.0000000
## 256 Serido     rio    CP  cy.gilbe   0.0000000   0.0000000
## 257 Serido     rio    CP  ge.brasi   0.0000000   0.0000000
## 258 Serido     rio    CP  he.margi   0.0000000   0.0000000
## 259 Serido     rio    CP  ho.malab   5.2500000   3.4034296
## 260 Serido     rio    CP  hy.pusar   1.0000000   1.1547005
## 261 Serido     rio    CP  le.melan   0.0000000   0.0000000
## 262 Serido     rio    CP   le.piau   1.5000000   1.2909944
## 263 Serido     rio    CP  le.taeni   0.0000000   0.0000000
## 264 Serido     rio    CP  mo.costa   0.0000000   0.0000000
## 265 Serido     rio    CP  mo.lepid   0.0000000   0.0000000
## 266 Serido     rio    CP  or.nilot   0.0000000   0.0000000
## 267 Serido     rio    CP  pa.manag   0.0000000   0.0000000
## 268 Serido     rio    CP  pimel.sp   0.0000000   0.0000000
## 269 Serido     rio    CP  po.retic   0.0000000   0.0000000
## 270 Serido     rio    CP  po.vivip  34.2500000  31.9204741
## 271 Serido     rio    CP  pr.brevi   1.2500000   2.5000000
## 272 Serido     rio    CP  ps.rhomb   0.0000000   0.0000000
## 273 Serido     rio    CP ps.genise   0.0000000   0.0000000
## 274 Serido     rio    CP  se.heter   4.2500000   6.6520673
## 275 Serido     rio    CP  se.piaba   0.0000000   0.0000000
## 276 Serido     rio    CP  se.spilo   0.0000000   0.0000000
## 277 Serido     rio    CP  st.noton   0.0000000   0.0000000
## 278 Serido     rio    CP  sy.marmo   0.2500000   0.5000000
## 279 Serido     rio    CP  te.chalc   0.0000000   0.0000000
## 280 Serido     rio    CP  tr.signa   0.0000000   0.0000000
## 281 Serido     rio    CT  ap.davis   0.0000000   0.0000000
## 282 Serido     rio    CT  as.bimac 135.5000000  94.5074953
## 283 Serido     rio    CT  as.fasci  30.7500000  33.4103277
## 284 Serido     rio    CT  ch.bimac   0.7500000   1.5000000
## 285 Serido     rio    CT  ci.ocela   0.0000000   0.0000000
## 286 Serido     rio    CT  ci.orien  26.0000000  30.1772541
## 287 Serido     rio    CT  co.macro   0.0000000   0.0000000
## 288 Serido     rio    CT  co.heter   0.2500000   0.5000000
## 289 Serido     rio    CT  cr.menez   6.7500000   5.6199051
## 290 Serido     rio    CT  cu.lepid   0.0000000   0.0000000
## 291 Serido     rio    CT  cy.gilbe  12.5000000  25.0000000
## 292 Serido     rio    CT  ge.brasi   1.7500000   1.5000000
## 293 Serido     rio    CT  he.margi   0.5000000   0.5773503
## 294 Serido     rio    CT  ho.malab  13.2500000  13.7204227
## 295 Serido     rio    CT  hy.pusar  16.5000000  17.9907384
## 296 Serido     rio    CT  le.melan   0.0000000   0.0000000
## 297 Serido     rio    CT   le.piau   1.5000000   1.2909944
## 298 Serido     rio    CT  le.taeni   0.0000000   0.0000000
## 299 Serido     rio    CT  mo.costa   0.0000000   0.0000000
## 300 Serido     rio    CT  mo.lepid  10.0000000  19.3390796
## 301 Serido     rio    CT  or.nilot  81.0000000  42.2453153
## 302 Serido     rio    CT  pa.manag   0.0000000   0.0000000
## 303 Serido     rio    CT  pimel.sp   1.5000000   3.0000000
## 304 Serido     rio    CT  po.retic   6.2500000   9.4648472
## 305 Serido     rio    CT  po.vivip 155.5000000 143.0489427
## 306 Serido     rio    CT  pr.brevi  60.7500000  72.7203548
## 307 Serido     rio    CT  ps.rhomb   0.2500000   0.5000000
## 308 Serido     rio    CT ps.genise   0.2500000   0.5000000
## 309 Serido     rio    CT  se.heter  35.2500000  23.6836793
## 310 Serido     rio    CT  se.piaba  17.0000000  34.0000000
## 311 Serido     rio    CT  se.spilo   0.2500000   0.5000000
## 312 Serido     rio    CT  st.noton  51.2500000  49.8021084
## 313 Serido     rio    CT  sy.marmo   0.0000000   0.0000000
## 314 Serido     rio    CT  te.chalc   0.0000000   0.0000000
## 315 Serido     rio    CT  tr.signa  45.2500000  64.0019531
## [1]   5.25 214.25   3.00  42.00 174.25   1.00  99.00  10.00 135.50
##    Min. 1st Qu.  Median    Mean 3rd Qu.    Max. 
##    1.00    5.25   42.00   76.03  135.50  214.25
\end{verbatim}

\hypertarget{descruas}{%
\subsection*{Descritores da estrutura da comunidade: UA's}\label{descruas}}
\addcontentsline{toc}{subsection}{Descritores da estrutura da comunidade: UA's}

Sum, soma; mean, média; DP, desvio padrão da média; Max, maior valor; Min, menor valor; S, riqueza (ou frequência de ocorrência na matriz transposta); E, índice de equitabilidade de Pielou (\protect\hyperlink{ref-RN465}{PIELOU, 1975}); H, índice de diversidade de Shannon (\protect\hyperlink{ref-RN136}{LUDWIG; REYNOLDS, 1988}); D, índice de diversidade de Simpson (\protect\hyperlink{ref-RN1073}{HURLBERT, 1971}).

\begin{Shaded}
\begin{Highlighting}[]
\CommentTok{\#Matriz transposta}
\NormalTok{m\_trab }\OtherTok{\textless{}{-}} \FunctionTok{t}\NormalTok{(m\_trab) }\CommentTok{\#transpõe a matriz de volta para comunitaria}
\NormalTok{Sum }\OtherTok{\textless{}{-}} \FunctionTok{rowSums}\NormalTok{(m\_trab)}
\NormalTok{Sum }\OtherTok{\textless{}{-}} \FunctionTok{apply}\NormalTok{(m\_trab,}\DecValTok{1}\NormalTok{,sum)}
\NormalTok{Mean }\OtherTok{\textless{}{-}} \FunctionTok{rowMeans}\NormalTok{(m\_trab)}
\NormalTok{Mean }\OtherTok{\textless{}{-}} \FunctionTok{apply}\NormalTok{(m\_trab,}\DecValTok{1}\NormalTok{,mean)}
\NormalTok{DP }\OtherTok{\textless{}{-}} \FunctionTok{apply}\NormalTok{(m\_trab,}\DecValTok{1}\NormalTok{,sd)}
\NormalTok{Max }\OtherTok{\textless{}{-}} \FunctionTok{apply}\NormalTok{(m\_trab,}\DecValTok{1}\NormalTok{,max)}
\NormalTok{Min }\OtherTok{\textless{}{-}} \FunctionTok{apply}\NormalTok{(m\_trab,}\DecValTok{1}\NormalTok{,min)}
\CommentTok{\#library(vegan)}
\NormalTok{bin }\OtherTok{\textless{}{-}} \FunctionTok{decostand}\NormalTok{(m\_trab,}\StringTok{"pa"}\NormalTok{)}
\NormalTok{S }\OtherTok{\textless{}{-}} \FunctionTok{apply}\NormalTok{(bin,}\DecValTok{1}\NormalTok{,sum)}
\NormalTok{Riqueza }\OtherTok{\textless{}{-}} \FunctionTok{specnumber}\NormalTok{(m\_trab)}
\NormalTok{Riqueza\_total }\OtherTok{\textless{}{-}} \FunctionTok{specnumber}\NormalTok{(}\FunctionTok{colSums}\NormalTok{(m\_trab))}
\NormalTok{H }\OtherTok{\textless{}{-}} \FunctionTok{diversity}\NormalTok{(m\_trab)}
\NormalTok{D }\OtherTok{\textless{}{-}} \FunctionTok{diversity}\NormalTok{(m\_trab, }\StringTok{"simpson"}\NormalTok{)}
\NormalTok{D[}\FunctionTok{is.na}\NormalTok{(D)] }\OtherTok{\textless{}{-}} \DecValTok{0} \CommentTok{\#substitui NA ou NaN por 0}
\NormalTok{E }\OtherTok{\textless{}{-}}\NormalTok{ H}\SpecialCharTok{/}\FunctionTok{log}\NormalTok{(}\FunctionTok{specnumber}\NormalTok{(m\_trab))}
\NormalTok{E[}\FunctionTok{is.na}\NormalTok{(E)] }\OtherTok{\textless{}{-}} \DecValTok{0} \CommentTok{\#substitui NA ou NaN por 0}
\CommentTok{\#library(moments)}
\NormalTok{Assimetria }\OtherTok{\textless{}{-}} \FunctionTok{apply}\NormalTok{(m\_trab,}\DecValTok{1}\NormalTok{,skewness)}
\NormalTok{Curtose }\OtherTok{\textless{}{-}} \FunctionTok{apply}\NormalTok{(m\_trab,}\DecValTok{1}\NormalTok{,kurtosis)}
\NormalTok{m\_trab }\OtherTok{\textless{}{-}} \FunctionTok{t}\NormalTok{(m\_trab) }\CommentTok{\#traz de volta}
\NormalTok{Descritores1 }\OtherTok{\textless{}{-}} \FunctionTok{cbind}\NormalTok{(Sum, Mean, DP, Max, Min, S, E, H, D)}
\NormalTok{Descritores1 }\OtherTok{\textless{}{-}} \FunctionTok{as.data.frame}\NormalTok{(Descritores1)}
\NormalTok{Descritores1}
\CommentTok{\#Descritores1 \textless{}{-} Descritores1 \%\textgreater{}\% rownames\_to\_column(var="UA´s") \#da nome a primeira coluna}
\NormalTok{SomaTotalD }\OtherTok{\textless{}{-}} \FunctionTok{apply}\NormalTok{(Descritores1,}\DecValTok{2}\NormalTok{,sum)}
\NormalTok{SomaTotalD}
\NormalTok{MediaTotalD }\OtherTok{\textless{}{-}} \FunctionTok{apply}\NormalTok{(Descritores1,}\DecValTok{2}\NormalTok{,mean)}
\NormalTok{MediaTotalD}
\NormalTok{DPTotalD }\OtherTok{\textless{}{-}} \FunctionTok{apply}\NormalTok{(Descritores1,}\DecValTok{2}\NormalTok{,sd)}
\NormalTok{DPTotalD}
\NormalTok{Descritores2 }\OtherTok{\textless{}{-}} \FunctionTok{cbind}\NormalTok{(SomaTotalD, MediaTotalD, DPTotalD)}
\NormalTok{Descritores2 }\OtherTok{\textless{}{-}} \FunctionTok{as.data.frame}\NormalTok{(Descritores2)}
\NormalTok{Descritores2 }\OtherTok{\textless{}{-}} \FunctionTok{t}\NormalTok{(Descritores2)}
\NormalTok{Descritores2}
\NormalTok{DescritoresFinal }\OtherTok{\textless{}{-}} \FunctionTok{rbind}\NormalTok{(Descritores1, Descritores2)}
\NormalTok{DescritoresFinal}
\NormalTok{DescritoresFinal }\OtherTok{\textless{}{-}} \FunctionTok{round}\NormalTok{ (DescritoresFinal, }\DecValTok{2}\NormalTok{)}
\CommentTok{\#Fazendo uma tabela}
\CommentTok{\#library(gt)}
\NormalTok{df }\OtherTok{\textless{}{-}}\NormalTok{ DescritoresFinal}
\FunctionTok{ncol}\NormalTok{(df); }\FunctionTok{nrow}\NormalTok{(df) }\CommentTok{\#no. de N colunas x M linhas}
\NormalTok{df }\OtherTok{\textless{}{-}} \FunctionTok{cbind}\NormalTok{(}\AttributeTok{UA =} \FunctionTok{rownames}\NormalTok{(df), df)}
\FunctionTok{gt}\NormalTok{(df, }\AttributeTok{caption =} \StringTok{"Descritores da diversidade por Unidade Amostral (UA). Sum, soma; mean, média; DP, desvio padrão da média; Max, maior valor; Min, menor valor; S, riqueza (ou frequência de ocorrência na matriz transposta); E, índice de equitabilidade de Pielou; H, índice de diversidade de Shannon; D, índice de diversidade de Simpson."}\NormalTok{)}
\end{Highlighting}
\end{Shaded}

\hypertarget{descritores-da-normalidade-1}{%
\subsection*{Descritores da normalidade}\label{descritores-da-normalidade-1}}
\addcontentsline{toc}{subsection}{Descritores da normalidade}

\begin{Shaded}
\begin{Highlighting}[]
\NormalTok{Normalidade1 }\OtherTok{\textless{}{-}} \FunctionTok{cbind}\NormalTok{(Assimetria, Curtose)}
\NormalTok{Normalidade1 }\OtherTok{\textless{}{-}} \FunctionTok{as.data.frame}\NormalTok{(Normalidade1)}
\NormalTok{Normalidade1 }\OtherTok{\textless{}{-}} \FunctionTok{na.omit}\NormalTok{(Normalidade1) }\CommentTok{\#remove NA e NaN}
\NormalTok{Normalidade1}
\NormalTok{SomaTotalN }\OtherTok{\textless{}{-}} \FunctionTok{apply}\NormalTok{(Normalidade1,}\DecValTok{2}\NormalTok{,sum)}
\NormalTok{SomaTotalN}
\NormalTok{MediaTotalN }\OtherTok{\textless{}{-}} \FunctionTok{apply}\NormalTok{(Normalidade1,}\DecValTok{2}\NormalTok{,mean)}
\NormalTok{MediaTotalN}
\NormalTok{DPTotalN }\OtherTok{\textless{}{-}} \FunctionTok{apply}\NormalTok{(Normalidade1,}\DecValTok{2}\NormalTok{,sd)}
\NormalTok{DPTotalN}
\NormalTok{Normalidade2}\OtherTok{\textless{}{-}}\FunctionTok{cbind}\NormalTok{(SomaTotalN, MediaTotalN, DPTotalN)}
\NormalTok{Normalidade2}\OtherTok{\textless{}{-}}\FunctionTok{as.data.frame}\NormalTok{(Normalidade2)}
\NormalTok{Normalidade2 }\OtherTok{\textless{}{-}} \FunctionTok{t}\NormalTok{(Normalidade2) }\CommentTok{\#"t" transpoe a matriz}
\NormalTok{Normalidade2}
\NormalTok{NormalidadeFinal }\OtherTok{\textless{}{-}} \FunctionTok{rbind}\NormalTok{(Normalidade1, Normalidade2)}
\NormalTok{NormalidadeFinal}
\NormalTok{NormalidadeFinal }\OtherTok{\textless{}{-}} \FunctionTok{round}\NormalTok{(NormalidadeFinal, }\DecValTok{2}\NormalTok{)}
\CommentTok{\#Fazendo uma tabela}
\NormalTok{nf }\OtherTok{\textless{}{-}}\NormalTok{ NormalidadeFinal}
\FunctionTok{ncol}\NormalTok{(nf); }\FunctionTok{nrow}\NormalTok{(nf) }\CommentTok{\#no. de N colunas x M linhas}
\NormalTok{nf }\OtherTok{\textless{}{-}} \FunctionTok{cbind}\NormalTok{(}\AttributeTok{UA =} \FunctionTok{rownames}\NormalTok{(nf), nf)}
\FunctionTok{gt}\NormalTok{(nf, }\AttributeTok{caption =} \StringTok{"Descritores da normalidade por Unidade Amostral (UA)"}\NormalTok{)}
\end{Highlighting}
\end{Shaded}

A função \texttt{fix(nome\ da\ matriz)} dá acesso ao grid da matriz criada para manipulação dos dados numéricos.

\hypertarget{salvando-as-tabelas-criadas-em-txt-direto-no-diretuxf3rio-de-trabalho}{%
\subsection*{Salvando as tabelas criadas em txt direto no diretório de trabalho}\label{salvando-as-tabelas-criadas-em-txt-direto-no-diretuxf3rio-de-trabalho}}
\addcontentsline{toc}{subsection}{Salvando as tabelas criadas em txt direto no diretório de trabalho}

\begin{Shaded}
\begin{Highlighting}[]
\FunctionTok{write.table}\NormalTok{(}\FunctionTok{data.frame}\NormalTok{(}\StringTok{"Spp"}\OtherTok{=}\FunctionTok{rownames}\NormalTok{(DescritoresFinal),}
\NormalTok{                       DescritoresFinal),}
            \StringTok{"DescritoresUA.csv"}\NormalTok{,}
            \AttributeTok{row.names=}\ConstantTok{FALSE}\NormalTok{,}
            \AttributeTok{sep=}\StringTok{"}\SpecialCharTok{\textbackslash{}t}\StringTok{"}\NormalTok{)}
\FunctionTok{write.table}\NormalTok{(}\FunctionTok{data.frame}\NormalTok{(}\StringTok{"Spp"}\OtherTok{=}\FunctionTok{rownames}\NormalTok{(NormalidadeFinal),}
\NormalTok{                       NormalidadeFinal),}
            \StringTok{"NormalidadeUA.txt"}\NormalTok{,}
            \AttributeTok{row.names=}\ConstantTok{FALSE}\NormalTok{,}
            \AttributeTok{sep=}\StringTok{"}\SpecialCharTok{\textbackslash{}t}\StringTok{"}\NormalTok{)}
\end{Highlighting}
\end{Shaded}

\hypertarget{sites-consultados-6}{%
\section*{Sites consultados}\label{sites-consultados-6}}
\addcontentsline{toc}{section}{Sites consultados}

\hypertarget{script-limpo-5}{%
\section*{Script limpo}\label{script-limpo-5}}
\addcontentsline{toc}{section}{Script limpo}

Aqui apresento o scrip na íntegra sem os textos ou outros comentários. Você pode copiar e colar no R para executa-lo. Lembre de remover os {\texttt{\#}} ou {\texttt{\#\#}} caso necessite executar essas linhas.

\begin{Shaded}
\begin{Highlighting}[]
\DocumentationTok{\#\# dev.off() \#apaga os graficos, se houver algum}
\DocumentationTok{\#\# rm(list=ls(all=TRUE)) \#limpa a memória}
\DocumentationTok{\#\# cat("\textbackslash{}014") \#limpa o console}
\DocumentationTok{\#\# install.packages("vegan")}
\DocumentationTok{\#\# install.packages("moments")}
\DocumentationTok{\#\# install.packages("ggplot2")}
\DocumentationTok{\#\# install.packages("dplyr")}
\DocumentationTok{\#\# install.packages("tidyr")}
\DocumentationTok{\#\# install.packages("tibble")}
\DocumentationTok{\#\# install.packages("tidyverse") \#atente para alguma msg de erro qdo executar essa linha}
\DocumentationTok{\#\# install.packages("forcats")}
\DocumentationTok{\#\# install.packages("iNEXT")}
\DocumentationTok{\#\# install.packages("openxlsx")}
\DocumentationTok{\#\# install.packages("gt")}
\DocumentationTok{\#\# library(tibble); library(tidyverse); library(forcats); library(openxlsx); library(Rcpp)}
\DocumentationTok{\#\# getwd()}
\DocumentationTok{\#\# setwd("C:/Seu/Diretório/De/Trabalho")}
\CommentTok{\#dir \textless{}{-} getwd() \#criamos um vetor com o diretório  de trbalho }
\CommentTok{\#shell.exec(dir) \#abre o diretorio de trabalho no Windows Explorer}
\NormalTok{ppbio }\OtherTok{\textless{}{-}} \FunctionTok{read.xlsx}\NormalTok{(}\StringTok{"D:/Elvio/OneDrive/Disciplinas/\_EcoNumerica/5.Matrizes/ppbio06.xlsx"}\NormalTok{,}
                   \AttributeTok{rowNames =}\NormalTok{ T, }\AttributeTok{colNames =}\NormalTok{ T,}
                   \AttributeTok{sheet =} \StringTok{"Sheet1"}\NormalTok{)}
\FunctionTok{str}\NormalTok{(ppbio)}
\CommentTok{\#View(ppbio)}
\NormalTok{ppbio[}\DecValTok{1}\SpecialCharTok{:}\DecValTok{5}\NormalTok{,}\DecValTok{1}\SpecialCharTok{:}\DecValTok{5}\NormalTok{] }\CommentTok{\#[1:5,1:5] mostra apenas as linhas e colunas de 1 a 5.}
\DocumentationTok{\#\# \#View(ppbio)}
\DocumentationTok{\#\# print(ppbio[1:8,1:8])}
\DocumentationTok{\#\# ppbio[1:10,1:10]}
\DocumentationTok{\#\# str(ppbio)}
\DocumentationTok{\#\# mode(ppbio)}
\DocumentationTok{\#\# class(ppbio)}
\NormalTok{m\_trab }\OtherTok{\textless{}{-}}\NormalTok{ ppbio}
\NormalTok{m\_trab }\OtherTok{\textless{}{-}} \FunctionTok{t}\NormalTok{(ppbio)}
\FunctionTok{str}\NormalTok{(m\_trab)}
\CommentTok{\#View(m\_trab)}
\NormalTok{m\_trab}
\FunctionTok{print}\NormalTok{(m\_trab[}\DecValTok{1}\SpecialCharTok{:}\DecValTok{5}\NormalTok{,}\DecValTok{1}\SpecialCharTok{:}\DecValTok{5}\NormalTok{])}
\NormalTok{m\_trab[}\DecValTok{1}\SpecialCharTok{:}\DecValTok{5}\NormalTok{,}\DecValTok{1}\SpecialCharTok{:}\DecValTok{5}\NormalTok{]}
\FunctionTok{str}\NormalTok{(m\_trab)}
\FunctionTok{mode}\NormalTok{(m\_trab)}
\FunctionTok{class}\NormalTok{(m\_trab)}
\FunctionTok{range}\NormalTok{(m\_trab) }\CommentTok{\#menor e maior valores}
\FunctionTok{length}\NormalTok{(m\_trab) }\CommentTok{\#no. de colunas}
\FunctionTok{ncol}\NormalTok{(m\_trab) }\CommentTok{\#no. de N colunas}
\FunctionTok{nrow}\NormalTok{(m\_trab) }\CommentTok{\#no. de M linhas}
\FunctionTok{sum}\NormalTok{(}\FunctionTok{lengths}\NormalTok{(m\_trab)) }\CommentTok{\#soma os nos. de colunas}
\FunctionTok{length}\NormalTok{(}\FunctionTok{as.matrix}\NormalTok{(m\_trab)) }\CommentTok{\#tamanho da matriz m x n}
\FunctionTok{sum}\NormalTok{(m\_trab }\SpecialCharTok{==} \DecValTok{0}\NormalTok{) }\CommentTok{\#número de observações igual a zero}
\FunctionTok{sum}\NormalTok{(m\_trab }\SpecialCharTok{\textgreater{}} \DecValTok{0}\NormalTok{) }\CommentTok{\#número de observações maiores que zero}
\CommentTok{\#calculando a proporção de zeros na matriz}
\NormalTok{zeros }\OtherTok{\textless{}{-}}\NormalTok{ (}\FunctionTok{sum}\NormalTok{(m\_trab }\SpecialCharTok{==} \DecValTok{0}\NormalTok{)}\SpecialCharTok{/}\FunctionTok{length}\NormalTok{(}\FunctionTok{as.matrix}\NormalTok{(m\_trab)))}\SpecialCharTok{*}\DecValTok{100}
\NormalTok{zeros}
\NormalTok{tamanho }\OtherTok{\textless{}{-}} \FunctionTok{data.frame}\NormalTok{(}
\NormalTok{  Função }\OtherTok{=} \FunctionTok{c}\NormalTok{(}\StringTok{"range"}\NormalTok{, }\StringTok{"lenght"}\NormalTok{, }\StringTok{"n cols"}\NormalTok{, }\StringTok{"m linhas"}\NormalTok{, }\StringTok{"Tamanho"}\NormalTok{, }\StringTok{"Tamanho"}\NormalTok{,}
             \StringTok{"Zeros"}\NormalTok{, }\StringTok{"Nao zeros"}\NormalTok{, }\StringTok{"\% Zeros"}\NormalTok{),}
  \AttributeTok{Resultado =} \FunctionTok{c}\NormalTok{(}\FunctionTok{paste}\NormalTok{(}\FunctionTok{range}\NormalTok{(m\_trab), }\AttributeTok{collapse =} \StringTok{" {-} "}\NormalTok{),}
                \FunctionTok{length}\NormalTok{(m\_trab), }\FunctionTok{ncol}\NormalTok{(m\_trab),}
                \FunctionTok{nrow}\NormalTok{(m\_trab), }\FunctionTok{sum}\NormalTok{(}\FunctionTok{lengths}\NormalTok{(m\_trab)),}
                \FunctionTok{length}\NormalTok{(}\FunctionTok{as.matrix}\NormalTok{(m\_trab)), }\FunctionTok{sum}\NormalTok{(m\_trab }\SpecialCharTok{==} \DecValTok{0}\NormalTok{),}
                \FunctionTok{sum}\NormalTok{(m\_trab }\SpecialCharTok{\textgreater{}} \DecValTok{0}\NormalTok{), }\FunctionTok{round}\NormalTok{(zeros, }\DecValTok{1}\NormalTok{)))}
\NormalTok{tamanho}
\NormalTok{knitr}\SpecialCharTok{::}\FunctionTok{kable}\NormalTok{(tamanho, }\AttributeTok{format =} \StringTok{"markdown"}\NormalTok{, }\AttributeTok{caption =} \StringTok{"Resumo das informações sobre o tamanho da base de dados."}\NormalTok{)}
\CommentTok{\#?apply}
\NormalTok{Sum }\OtherTok{\textless{}{-}} \FunctionTok{rowSums}\NormalTok{(m\_trab)}
\CommentTok{\#ou}
\NormalTok{Sum }\OtherTok{\textless{}{-}} \FunctionTok{apply}\NormalTok{(m\_trab,}\DecValTok{1}\NormalTok{,sum)}
\NormalTok{Sum}
\DocumentationTok{\#\# Media}
\NormalTok{Mean }\OtherTok{\textless{}{-}} \FunctionTok{rowMeans}\NormalTok{(m\_trab)}
\NormalTok{Mean}
\DocumentationTok{\#\# Ou}
\NormalTok{Mean }\OtherTok{\textless{}{-}} \FunctionTok{apply}\NormalTok{(m\_trab,}\DecValTok{1}\NormalTok{,mean)}
\NormalTok{Mean}
\DocumentationTok{\#\# Desvio padrão}
\NormalTok{DP }\OtherTok{\textless{}{-}} \FunctionTok{apply}\NormalTok{(m\_trab,}\DecValTok{1}\NormalTok{,sd)}
\NormalTok{DP}
\DocumentationTok{\#\# Máximo}
\NormalTok{Max }\OtherTok{\textless{}{-}} \FunctionTok{apply}\NormalTok{(m\_trab,}\DecValTok{1}\NormalTok{,max)}
\NormalTok{Max }
\DocumentationTok{\#\# Mínimo}
\NormalTok{Min }\OtherTok{\textless{}{-}} \FunctionTok{apply}\NormalTok{(m\_trab,}\DecValTok{1}\NormalTok{,min)}
\NormalTok{Min}
\FunctionTok{library}\NormalTok{(vegan)}
\NormalTok{bin }\OtherTok{\textless{}{-}} \FunctionTok{decostand}\NormalTok{(m\_trab,}\StringTok{"pa"}\NormalTok{)}
\NormalTok{bin[}\DecValTok{1}\SpecialCharTok{:}\DecValTok{10}\NormalTok{, }\DecValTok{1}\SpecialCharTok{:}\DecValTok{10}\NormalTok{]}
\NormalTok{S }\OtherTok{\textless{}{-}} \FunctionTok{apply}\NormalTok{(bin,}\DecValTok{1}\NormalTok{,sum)}
\NormalTok{S}
\CommentTok{\#OU}
\NormalTok{Riqueza }\OtherTok{\textless{}{-}} \FunctionTok{specnumber}\NormalTok{(m\_trab)}
\NormalTok{Riqueza}
\NormalTok{Riqueza\_total }\OtherTok{\textless{}{-}} \FunctionTok{specnumber}\NormalTok{(}\FunctionTok{colSums}\NormalTok{(m\_trab))}
\NormalTok{Riqueza\_total}
\NormalTok{H }\OtherTok{\textless{}{-}} \FunctionTok{diversity}\NormalTok{(m\_trab, }\AttributeTok{index =} \StringTok{"shannon"}\NormalTok{)}
\NormalTok{H}
\NormalTok{D }\OtherTok{\textless{}{-}} \FunctionTok{diversity}\NormalTok{(m\_trab, }\StringTok{"simpson"}\NormalTok{)}
\NormalTok{D}
\NormalTok{D[}\FunctionTok{is.na}\NormalTok{(D)] }\OtherTok{\textless{}{-}} \DecValTok{0} \CommentTok{\#substitui NA ou NaN por 0}
\NormalTok{D}
\NormalTok{E }\OtherTok{\textless{}{-}}\NormalTok{ H}\SpecialCharTok{/}\FunctionTok{log}\NormalTok{(}\FunctionTok{specnumber}\NormalTok{(m\_trab))}
\NormalTok{E}
\NormalTok{E[}\FunctionTok{is.na}\NormalTok{(E)] }\OtherTok{\textless{}{-}} \DecValTok{0} \CommentTok{\#substitui NA ou NaN por 0}
\NormalTok{E}
\FunctionTok{library}\NormalTok{(moments)}
\NormalTok{Assimetria }\OtherTok{\textless{}{-}} \FunctionTok{apply}\NormalTok{(m\_trab,}\DecValTok{1}\NormalTok{,skewness)}
\NormalTok{Assimetria}
\NormalTok{Curtose }\OtherTok{\textless{}{-}} \FunctionTok{apply}\NormalTok{(m\_trab,}\DecValTok{1}\NormalTok{,kurtosis)}
\NormalTok{Curtose}
\NormalTok{Descritores1 }\OtherTok{\textless{}{-}} \FunctionTok{cbind}\NormalTok{(Sum, Mean, DP, Max, Min, S, E, H, D)}
\NormalTok{Descritores1 }\OtherTok{\textless{}{-}} \FunctionTok{as.data.frame}\NormalTok{(Descritores1)}
\NormalTok{Descritores1}
\CommentTok{\#Descritores1 \textless{}{-} Descritores1 \%\textgreater{}\% rownames\_to\_column(var="Espécies") \#da nome a primeira coluna}
\NormalTok{SomaTotalD }\OtherTok{\textless{}{-}} \FunctionTok{apply}\NormalTok{(Descritores1,}\DecValTok{2}\NormalTok{,sum)}
\NormalTok{SomaTotalD}
\NormalTok{MediaTotalD }\OtherTok{\textless{}{-}} \FunctionTok{apply}\NormalTok{(Descritores1,}\DecValTok{2}\NormalTok{,mean)}
\NormalTok{MediaTotalD}
\NormalTok{DPTotalD }\OtherTok{\textless{}{-}} \FunctionTok{apply}\NormalTok{(Descritores1,}\DecValTok{2}\NormalTok{,sd)}
\NormalTok{DPTotalD}
\NormalTok{Descritores2 }\OtherTok{\textless{}{-}} \FunctionTok{cbind}\NormalTok{(SomaTotalD, MediaTotalD, DPTotalD)}
\NormalTok{Descritores2 }\OtherTok{\textless{}{-}} \FunctionTok{as.data.frame}\NormalTok{(Descritores2)}
\NormalTok{Descritores2 }\OtherTok{\textless{}{-}} \FunctionTok{t}\NormalTok{(Descritores2)}
\NormalTok{Descritores2}
\NormalTok{DescritoresFinal }\OtherTok{\textless{}{-}} \FunctionTok{rbind}\NormalTok{(Descritores1, Descritores2)}
\NormalTok{DescritoresFinal}
\NormalTok{DescritoresFinal }\OtherTok{\textless{}{-}} \FunctionTok{round}\NormalTok{ (DescritoresFinal, }\DecValTok{2}\NormalTok{)}
\NormalTok{DescritoresFinal}
\CommentTok{\#Fazendo uma tabela}
\FunctionTok{library}\NormalTok{(gt)}
\NormalTok{df }\OtherTok{\textless{}{-}}\NormalTok{ DescritoresFinal}
\FunctionTok{ncol}\NormalTok{(df); }\FunctionTok{nrow}\NormalTok{(df) }\CommentTok{\#no. de N colunas x M linhas}
\NormalTok{df }\OtherTok{\textless{}{-}} \FunctionTok{cbind}\NormalTok{(}\AttributeTok{Spp =} \FunctionTok{rownames}\NormalTok{(df), df)}
\FunctionTok{gt}\NormalTok{(df, }\AttributeTok{rowname\_col =} \StringTok{"Espécie"}\NormalTok{, }\AttributeTok{caption =} \StringTok{"Descritores da diversidade por espécie (colunas). Sum, soma; mean, média; DP, desvio padrão da média; Max, maior valor; Min, menor valor; S, riqueza (ou frequência de ocorrência na matriz transposta); E, índice de equitabilidade de Pielou; H, índice de diversidade de Shannon; D, índice de diversidade de Simpson."}\NormalTok{)}
\NormalTok{Normalidade1 }\OtherTok{\textless{}{-}} \FunctionTok{cbind}\NormalTok{(Assimetria, Curtose)}
\NormalTok{Normalidade1 }\OtherTok{\textless{}{-}} \FunctionTok{as.data.frame}\NormalTok{(Normalidade1)}
\NormalTok{Normalidade1}
\NormalTok{SomaTotalN }\OtherTok{\textless{}{-}} \FunctionTok{apply}\NormalTok{(Normalidade1,}\DecValTok{2}\NormalTok{,sum)}
\NormalTok{SomaTotalN}
\NormalTok{MediaTotalN }\OtherTok{\textless{}{-}} \FunctionTok{apply}\NormalTok{(Normalidade1,}\DecValTok{2}\NormalTok{,mean)}
\NormalTok{MediaTotalN}
\NormalTok{DPTotalN }\OtherTok{\textless{}{-}} \FunctionTok{apply}\NormalTok{(Normalidade1,}\DecValTok{2}\NormalTok{,sd)}
\NormalTok{DPTotalN}
\NormalTok{Normalidade2}\OtherTok{\textless{}{-}}\FunctionTok{cbind}\NormalTok{(SomaTotalN, MediaTotalN, DPTotalN)}
\NormalTok{Normalidade2}\OtherTok{\textless{}{-}}\FunctionTok{as.data.frame}\NormalTok{(Normalidade2)}
\NormalTok{Normalidade2 }\OtherTok{\textless{}{-}} \FunctionTok{t}\NormalTok{(Normalidade2) }\CommentTok{\#"t" transpoe a matriz}
\NormalTok{Normalidade2}
\NormalTok{NormalidadeFinal }\OtherTok{\textless{}{-}} \FunctionTok{rbind}\NormalTok{(Normalidade1, Normalidade2)}
\NormalTok{NormalidadeFinal}
\NormalTok{NormalidadeFinal }\OtherTok{\textless{}{-}} \FunctionTok{round}\NormalTok{(NormalidadeFinal, }\DecValTok{2}\NormalTok{)}
\NormalTok{NormalidadeFinal}
\CommentTok{\#Fazendo uma tabela}
\NormalTok{nf }\OtherTok{\textless{}{-}}\NormalTok{ NormalidadeFinal}
\FunctionTok{ncol}\NormalTok{(nf); }\FunctionTok{nrow}\NormalTok{(nf) }\CommentTok{\#no. de N colunas x M linhas}
\NormalTok{nf }\OtherTok{\textless{}{-}} \FunctionTok{cbind}\NormalTok{(}\AttributeTok{Spp =} \FunctionTok{rownames}\NormalTok{(nf), nf)}
\FunctionTok{gt}\NormalTok{(nf, }\AttributeTok{rowname\_col =} \StringTok{"Espécie"}\NormalTok{, }\AttributeTok{caption =} \StringTok{"Descritores da normalidade por espécie (coluna)"}\NormalTok{)}
\FunctionTok{write.table}\NormalTok{(}\FunctionTok{data.frame}\NormalTok{(}\StringTok{"Spp"}\OtherTok{=}\FunctionTok{rownames}\NormalTok{(DescritoresFinal),}
\NormalTok{                       DescritoresFinal),}
            \StringTok{"DescritoresSPP.txt"}\NormalTok{,}
            \AttributeTok{row.names=}\ConstantTok{FALSE}\NormalTok{,}
            \AttributeTok{sep=}\StringTok{"}\SpecialCharTok{\textbackslash{}t}\StringTok{"}\NormalTok{)}
\FunctionTok{write.table}\NormalTok{(}\FunctionTok{data.frame}\NormalTok{(}\StringTok{"Spp"}\OtherTok{=}\FunctionTok{rownames}\NormalTok{(NormalidadeFinal),}
\NormalTok{                       NormalidadeFinal),}
            \StringTok{"NormalidadeSPP.txt"}\NormalTok{,}
            \AttributeTok{row.names=}\ConstantTok{FALSE}\NormalTok{,}
            \AttributeTok{sep=}\StringTok{"}\SpecialCharTok{\textbackslash{}t}\StringTok{"}\NormalTok{)}
\NormalTok{m\_trab }\OtherTok{\textless{}{-}}\NormalTok{ ppbio}
\NormalTok{m\_trab}
\NormalTok{Hill }\OtherTok{\textless{}{-}} \FunctionTok{renyi}\NormalTok{(m\_trab,}
              \AttributeTok{scales =} \FunctionTok{c}\NormalTok{(}\DecValTok{0}\SpecialCharTok{:}\DecValTok{5}\NormalTok{),}
              \AttributeTok{hill =} \ConstantTok{TRUE}\NormalTok{)}
\NormalTok{Hill}
\FunctionTok{library}\NormalTok{(ggplot2)}
\FunctionTok{library}\NormalTok{(tidyr)}
\FunctionTok{library}\NormalTok{(tidyverse)}
\NormalTok{grafico1 }\OtherTok{\textless{}{-}}\NormalTok{ Hill }\SpecialCharTok{\%\textgreater{}\%}  
  \FunctionTok{rownames\_to\_column}\NormalTok{() }\SpecialCharTok{\%\textgreater{}\%} 
  \FunctionTok{pivot\_longer}\NormalTok{(}\SpecialCharTok{{-}}\NormalTok{rowname) }\SpecialCharTok{\%\textgreater{}\%} 
  \FunctionTok{mutate}\NormalTok{(}\AttributeTok{name =} \FunctionTok{factor}\NormalTok{(name, name[}\DecValTok{1}\SpecialCharTok{:}\FunctionTok{length}\NormalTok{(Hill)])) }\SpecialCharTok{\%\textgreater{}\%} 
  \FunctionTok{ggplot}\NormalTok{(}\FunctionTok{aes}\NormalTok{(}\AttributeTok{x =}\NormalTok{ name, }\AttributeTok{y =}\NormalTok{ value, }\AttributeTok{group =}\NormalTok{ rowname,}
             \AttributeTok{col =}\NormalTok{ rowname)) }\SpecialCharTok{+}
  \FunctionTok{geom\_point}\NormalTok{(}\AttributeTok{size =} \DecValTok{2}\NormalTok{) }\SpecialCharTok{+}
  \FunctionTok{geom\_line}\NormalTok{(}\AttributeTok{size =} \DecValTok{1}\NormalTok{) }\SpecialCharTok{+}
  \FunctionTok{xlab}\NormalTok{(}\StringTok{"Parâmetro de ordem de diversidade (q)"}\NormalTok{) }\SpecialCharTok{+}
  \FunctionTok{ylab}\NormalTok{(}\StringTok{"Diversidade"}\NormalTok{) }\SpecialCharTok{+}
  \FunctionTok{labs}\NormalTok{(}\AttributeTok{col =} \StringTok{"Locais"}\NormalTok{) }\SpecialCharTok{+}
  \FunctionTok{theme\_bw}\NormalTok{() }\SpecialCharTok{+}
  \FunctionTok{theme}\NormalTok{(}\AttributeTok{text =} \FunctionTok{element\_text}\NormalTok{(}\AttributeTok{size =} \DecValTok{5}\NormalTok{)) }\CommentTok{\#ajustar a fonte caso nao caiba no output html }
\NormalTok{grafico1}
\FunctionTok{ggsave}\NormalTok{(grafico1, }\AttributeTok{dpi =} \DecValTok{300}\NormalTok{, }\AttributeTok{filename =} \StringTok{"fig{-}hill.png"}\NormalTok{)}
\NormalTok{abund }\OtherTok{\textless{}{-}} \FunctionTok{colSums}\NormalTok{(m\_trab)}
\CommentTok{\#abund \textless{}{-} m\_trab[1, ] \#escolhe a primeira linha para a distribuição de abundância}
\NormalTok{df }\OtherTok{\textless{}{-}} \FunctionTok{data.frame}\NormalTok{(}\AttributeTok{sp =} \FunctionTok{colnames}\NormalTok{(m\_trab), }
                 \AttributeTok{abun =}\NormalTok{ abund) }
\NormalTok{grafico2 }\OtherTok{\textless{}{-}} \FunctionTok{ggplot}\NormalTok{(df, }\FunctionTok{aes}\NormalTok{(}\FunctionTok{fct\_reorder}\NormalTok{(sp, }\SpecialCharTok{{-}}\NormalTok{abun),}
\NormalTok{                           abun, }\AttributeTok{group =} \DecValTok{1}\NormalTok{)) }\SpecialCharTok{+}
  \FunctionTok{geom\_col}\NormalTok{() }\SpecialCharTok{+}
  \FunctionTok{geom\_line}\NormalTok{(}\AttributeTok{col =} \StringTok{"red"}\NormalTok{, }\AttributeTok{linetype =} \StringTok{"dashed"}\NormalTok{) }\SpecialCharTok{+} 
  \FunctionTok{geom\_point}\NormalTok{(}\AttributeTok{col =} \StringTok{"red"}\NormalTok{) }\SpecialCharTok{+}
  \FunctionTok{xlab}\NormalTok{(}\StringTok{"Espécies"}\NormalTok{) }\SpecialCharTok{+}
  \FunctionTok{ylab}\NormalTok{(}\StringTok{"Abundância"}\NormalTok{) }\SpecialCharTok{+}
  \FunctionTok{theme\_bw}\NormalTok{() }\SpecialCharTok{+}
  \FunctionTok{theme}\NormalTok{(}\AttributeTok{axis.text.x =} \FunctionTok{element\_text}\NormalTok{(}
    \AttributeTok{angle =} \DecValTok{45}\NormalTok{,}
    \AttributeTok{hjust =} \DecValTok{1}\NormalTok{,}
    \AttributeTok{face =} \StringTok{"italic"}\NormalTok{))}
\NormalTok{grafico2}
\FunctionTok{ggsave}\NormalTok{(grafico2, }\AttributeTok{dpi =} \DecValTok{300}\NormalTok{, }\AttributeTok{filename =} \StringTok{"fig{-}abun.png"}\NormalTok{)}
\NormalTok{m\_trab\_r }\OtherTok{\textless{}{-}} \FunctionTok{mutate}\NormalTok{(m\_trab, }\FunctionTok{across}\NormalTok{(}\FunctionTok{everything}\NormalTok{(), ceiling)) }\CommentTok{\#arredonda a matriz original para números inteiros}
\FunctionTok{colnames}\NormalTok{(}\FunctionTok{t}\NormalTok{(m\_trab\_r))}
\NormalTok{rarefa }\OtherTok{\textless{}{-}} \FunctionTok{t}\NormalTok{(m\_trab\_r[}\FunctionTok{c}\NormalTok{(}\StringTok{"S{-}R{-}CT2"}\NormalTok{,}\StringTok{"S{-}R{-}CP2"}\NormalTok{,}\StringTok{"S{-}A{-}TA2"}\NormalTok{,}\StringTok{"B{-}A{-}MU2"}\NormalTok{),]) }\CommentTok{\#curva de rarefação para os sítios especificados}
\FunctionTok{class}\NormalTok{(m\_trab)}
\DocumentationTok{\#\# Curva de rarefação para todos os sítios}
\CommentTok{\#rarefa \textless{}{-} t(m\_trab\_r)}
\DocumentationTok{\#\# Curva de rarefação para as 8 UA\textquotesingle{}s com maior soma }
\CommentTok{\#m\_trab\_r \textless{}{-} as.data.frame(t(m\_trab\_r)) \#transpõe a matriz}
\CommentTok{\#col\_sums \textless{}{-} colSums(m\_trab\_r)}
\CommentTok{\#largest\_columns \textless{}{-} names(sort(col\_sums, decreasing = TRUE)[1:8])}
\CommentTok{\#rarefa \textless{}{-} m\_trab\_r[largest\_columns] \#curva de rarefação para as 8 UA\textquotesingle{}s com maior soma}
\FunctionTok{library}\NormalTok{(iNEXT)}
\NormalTok{out }\OtherTok{\textless{}{-}} \FunctionTok{iNEXT}\NormalTok{(rarefa, }\AttributeTok{q =} \DecValTok{0}\NormalTok{,}
             \AttributeTok{datatype =} \StringTok{"abundance"}\NormalTok{,}
             \AttributeTok{size =} \ConstantTok{NULL}\NormalTok{,}
             \AttributeTok{endpoint =} \DecValTok{1500}\NormalTok{, }\CommentTok{\#define o comprimento de eixo x}
             \AttributeTok{knots =} \DecValTok{40}\NormalTok{,}
             \AttributeTok{se =} \ConstantTok{TRUE}\NormalTok{,}
             \AttributeTok{conf =} \FloatTok{0.95}\NormalTok{,}
             \AttributeTok{nboot =} \DecValTok{50}\NormalTok{)}
\NormalTok{grafico3 }\OtherTok{\textless{}{-}} \FunctionTok{ggiNEXT}\NormalTok{(out, }\AttributeTok{type =} \DecValTok{1}\NormalTok{, }\AttributeTok{facet.var=}\StringTok{"None"}\NormalTok{) }\SpecialCharTok{+}
  \FunctionTok{theme\_bw}\NormalTok{() }\SpecialCharTok{+}
  \FunctionTok{labs}\NormalTok{(}\AttributeTok{fill =} \StringTok{"Áreas"}\NormalTok{) }\SpecialCharTok{+}
  \FunctionTok{xlab}\NormalTok{(}\StringTok{"Número de indivíduos"}\NormalTok{) }\SpecialCharTok{+} 
  \FunctionTok{ylab}\NormalTok{(}\StringTok{"Riqueza de espécies"}\NormalTok{) }\SpecialCharTok{+}
  \FunctionTok{theme}\NormalTok{(}\AttributeTok{legend.title=}\FunctionTok{element\_blank}\NormalTok{()) }\CommentTok{\#ver como fica com facet.var="Assemblage"}
\NormalTok{grafico3}
\FunctionTok{ggsave}\NormalTok{(grafico3, }\AttributeTok{dpi =} \DecValTok{300}\NormalTok{, }\AttributeTok{filename =} \StringTok{"fig{-}rare1.png"}\NormalTok{)}
\NormalTok{acumula }\OtherTok{\textless{}{-}} \FunctionTok{specaccum}\NormalTok{(m\_trab,}
                     \AttributeTok{method =} \StringTok{"random"}\NormalTok{)}
\NormalTok{acumula}
\FunctionTok{plot}\NormalTok{(acumula)}
\NormalTok{plot\_data }\OtherTok{\textless{}{-}} \FunctionTok{data.frame}\NormalTok{(}\StringTok{"UAs"} \OtherTok{=} \FunctionTok{c}\NormalTok{(}\DecValTok{0}\NormalTok{, acumula}\SpecialCharTok{$}\NormalTok{sites),}
                        \StringTok{"Riqueza"} \OtherTok{=} \FunctionTok{c}\NormalTok{(}\DecValTok{0}\NormalTok{, acumula}\SpecialCharTok{$}\NormalTok{richness),}
                        \StringTok{"lower"} \OtherTok{=} \FunctionTok{c}\NormalTok{(}\DecValTok{0}\NormalTok{, acumula}\SpecialCharTok{$}\NormalTok{richness }\SpecialCharTok{{-}}\NormalTok{ acumula}\SpecialCharTok{$}\NormalTok{sd),}
                        \StringTok{"upper"} \OtherTok{=} \FunctionTok{c}\NormalTok{(}\DecValTok{0}\NormalTok{, acumula}\SpecialCharTok{$}\NormalTok{richness }\SpecialCharTok{+}\NormalTok{ acumula}\SpecialCharTok{$}\NormalTok{sd))}
\NormalTok{gLocais }\OtherTok{\textless{}{-}} \FunctionTok{ggplot}\NormalTok{(plot\_data, }\FunctionTok{aes}\NormalTok{(}\AttributeTok{x =}\NormalTok{ UAs, }\AttributeTok{y =}\NormalTok{ Riqueza)) }\SpecialCharTok{+}
  \FunctionTok{geom\_point}\NormalTok{(}\AttributeTok{color =} \StringTok{"blue"}\NormalTok{, }\AttributeTok{size =} \DecValTok{4}\NormalTok{) }\SpecialCharTok{+}
  \FunctionTok{geom\_line}\NormalTok{(}\AttributeTok{color =} \StringTok{"blue"}\NormalTok{, }\AttributeTok{lwd =} \DecValTok{2}\NormalTok{) }\SpecialCharTok{+}
  \FunctionTok{geom\_ribbon}\NormalTok{(}\FunctionTok{aes}\NormalTok{(}\AttributeTok{ymin =}\NormalTok{ lower, }\AttributeTok{ymax =}\NormalTok{ upper), }
              \AttributeTok{linetype=}\DecValTok{2}\NormalTok{, }\AttributeTok{alpha=}\FloatTok{0.3}\NormalTok{, }\AttributeTok{fill =} \StringTok{"yellow"}\NormalTok{) }\SpecialCharTok{+}
  \FunctionTok{ylab}\NormalTok{(}\StringTok{"Riqueza acumulada"}\NormalTok{) }\SpecialCharTok{+}
  \FunctionTok{theme\_classic}\NormalTok{() }\SpecialCharTok{+}
  \FunctionTok{theme}\NormalTok{(}\AttributeTok{text =} \FunctionTok{element\_text}\NormalTok{(}\AttributeTok{size =} \DecValTok{16}\NormalTok{))}
\NormalTok{gLocais}
\NormalTok{plot\_data }\OtherTok{\textless{}{-}} \FunctionTok{data.frame}\NormalTok{(}\StringTok{"Individuos"} \OtherTok{=} \FunctionTok{c}\NormalTok{(}\DecValTok{0}\NormalTok{, acumula}\SpecialCharTok{$}\NormalTok{individuals),}
                        \StringTok{"Riqueza"} \OtherTok{=} \FunctionTok{c}\NormalTok{(}\DecValTok{0}\NormalTok{, acumula}\SpecialCharTok{$}\NormalTok{richness),}
                        \StringTok{"lower"} \OtherTok{=} \FunctionTok{c}\NormalTok{(}\DecValTok{0}\NormalTok{, acumula}\SpecialCharTok{$}\NormalTok{richness }\SpecialCharTok{{-}}\NormalTok{ acumula}\SpecialCharTok{$}\NormalTok{sd),}
                        \StringTok{"upper"} \OtherTok{=} \FunctionTok{c}\NormalTok{(}\DecValTok{0}\NormalTok{, acumula}\SpecialCharTok{$}\NormalTok{richness }\SpecialCharTok{+}\NormalTok{ acumula}\SpecialCharTok{$}\NormalTok{sd))}
\NormalTok{gInd }\OtherTok{\textless{}{-}} \FunctionTok{ggplot}\NormalTok{(plot\_data, }\FunctionTok{aes}\NormalTok{(}\AttributeTok{x =}\NormalTok{ Individuos, }\AttributeTok{y =}\NormalTok{ Riqueza)) }\SpecialCharTok{+}
  \FunctionTok{geom\_point}\NormalTok{(}\AttributeTok{color =} \StringTok{"blue"}\NormalTok{, }\AttributeTok{size =} \DecValTok{4}\NormalTok{) }\SpecialCharTok{+}
  \FunctionTok{geom\_line}\NormalTok{(}\AttributeTok{color =} \StringTok{"blue"}\NormalTok{, }\AttributeTok{lwd =} \DecValTok{2}\NormalTok{) }\SpecialCharTok{+}
  \FunctionTok{geom\_ribbon}\NormalTok{(}\FunctionTok{aes}\NormalTok{(}\AttributeTok{ymin =}\NormalTok{ lower, }\AttributeTok{ymax =}\NormalTok{ upper), }
              \AttributeTok{linetype=}\DecValTok{2}\NormalTok{, }\AttributeTok{alpha=}\FloatTok{0.3}\NormalTok{, }\AttributeTok{fill =} \StringTok{"yellow"}\NormalTok{) }\SpecialCharTok{+}
  \FunctionTok{ylab}\NormalTok{(}\StringTok{"Riqueza acumulada"}\NormalTok{) }\SpecialCharTok{+}
  \FunctionTok{theme\_classic}\NormalTok{() }\SpecialCharTok{+}
  \FunctionTok{theme}\NormalTok{(}\AttributeTok{text =} \FunctionTok{element\_text}\NormalTok{(}\AttributeTok{size =} \DecValTok{16}\NormalTok{))}
\NormalTok{gInd}
\CommentTok{\#Matriz transposta}
\NormalTok{m\_trab }\OtherTok{\textless{}{-}} \FunctionTok{t}\NormalTok{(m\_trab) }\CommentTok{\#transpõe a matriz}
\NormalTok{Sum }\OtherTok{\textless{}{-}} \FunctionTok{rowSums}\NormalTok{(m\_trab)}
\NormalTok{Sum }\OtherTok{\textless{}{-}} \FunctionTok{apply}\NormalTok{(m\_trab,}\DecValTok{1}\NormalTok{,sum)}
\NormalTok{Mean }\OtherTok{\textless{}{-}} \FunctionTok{rowMeans}\NormalTok{(m\_trab)}
\NormalTok{Mean }\OtherTok{\textless{}{-}} \FunctionTok{apply}\NormalTok{(m\_trab,}\DecValTok{1}\NormalTok{,mean)}
\NormalTok{DP }\OtherTok{\textless{}{-}} \FunctionTok{apply}\NormalTok{(m\_trab,}\DecValTok{1}\NormalTok{,sd)}
\NormalTok{Max }\OtherTok{\textless{}{-}} \FunctionTok{apply}\NormalTok{(m\_trab,}\DecValTok{1}\NormalTok{,max)}
\NormalTok{Min }\OtherTok{\textless{}{-}} \FunctionTok{apply}\NormalTok{(m\_trab,}\DecValTok{1}\NormalTok{,min)}
\CommentTok{\#library(vegan)}
\NormalTok{bin }\OtherTok{\textless{}{-}} \FunctionTok{decostand}\NormalTok{(m\_trab,}\StringTok{"pa"}\NormalTok{)}
\NormalTok{S }\OtherTok{\textless{}{-}} \FunctionTok{apply}\NormalTok{(bin,}\DecValTok{1}\NormalTok{,sum)}
\NormalTok{Riqueza }\OtherTok{\textless{}{-}} \FunctionTok{specnumber}\NormalTok{(m\_trab)}
\NormalTok{Riqueza\_total }\OtherTok{\textless{}{-}} \FunctionTok{specnumber}\NormalTok{(}\FunctionTok{colSums}\NormalTok{(m\_trab))}
\NormalTok{H }\OtherTok{\textless{}{-}} \FunctionTok{diversity}\NormalTok{(m\_trab)}
\NormalTok{D }\OtherTok{\textless{}{-}} \FunctionTok{diversity}\NormalTok{(m\_trab, }\StringTok{"simpson"}\NormalTok{)}
\NormalTok{D[}\FunctionTok{is.na}\NormalTok{(D)] }\OtherTok{\textless{}{-}} \DecValTok{0} \CommentTok{\#substitui NA ou NaN por 0}
\NormalTok{E }\OtherTok{\textless{}{-}}\NormalTok{ H}\SpecialCharTok{/}\FunctionTok{log}\NormalTok{(}\FunctionTok{specnumber}\NormalTok{(m\_trab))}
\NormalTok{E[}\FunctionTok{is.na}\NormalTok{(E)] }\OtherTok{\textless{}{-}} \DecValTok{0} \CommentTok{\#substitui NA ou NaN por 0}
\CommentTok{\#library(moments)}
\NormalTok{Assimetria }\OtherTok{\textless{}{-}} \FunctionTok{apply}\NormalTok{(m\_trab,}\DecValTok{1}\NormalTok{,skewness)}
\NormalTok{Curtose }\OtherTok{\textless{}{-}} \FunctionTok{apply}\NormalTok{(m\_trab,}\DecValTok{1}\NormalTok{,kurtosis)}
\NormalTok{m\_trab }\OtherTok{\textless{}{-}} \FunctionTok{t}\NormalTok{(m\_trab) }\CommentTok{\#traz de volta}
\NormalTok{Descritores1 }\OtherTok{\textless{}{-}} \FunctionTok{cbind}\NormalTok{(Sum, Mean, DP, Max, Min, S, E, H, D)}
\NormalTok{Descritores1 }\OtherTok{\textless{}{-}} \FunctionTok{as.data.frame}\NormalTok{(Descritores1)}
\NormalTok{Descritores1}
\CommentTok{\#Descritores1 \textless{}{-} Descritores1 \%\textgreater{}\% rownames\_to\_column(var="Espécies") \#da nome a primeira coluna}
\NormalTok{SomaTotalD }\OtherTok{\textless{}{-}} \FunctionTok{apply}\NormalTok{(Descritores1,}\DecValTok{2}\NormalTok{,sum)}
\NormalTok{SomaTotalD}
\NormalTok{MediaTotalD }\OtherTok{\textless{}{-}} \FunctionTok{apply}\NormalTok{(Descritores1,}\DecValTok{2}\NormalTok{,mean)}
\NormalTok{MediaTotalD}
\NormalTok{DPTotalD }\OtherTok{\textless{}{-}} \FunctionTok{apply}\NormalTok{(Descritores1,}\DecValTok{2}\NormalTok{,sd)}
\NormalTok{DPTotalD}
\NormalTok{Descritores2 }\OtherTok{\textless{}{-}} \FunctionTok{cbind}\NormalTok{(SomaTotalD, MediaTotalD, DPTotalD)}
\NormalTok{Descritores2 }\OtherTok{\textless{}{-}} \FunctionTok{as.data.frame}\NormalTok{(Descritores2)}
\NormalTok{Descritores2 }\OtherTok{\textless{}{-}} \FunctionTok{t}\NormalTok{(Descritores2)}
\NormalTok{Descritores2}
\NormalTok{DescritoresFinal }\OtherTok{\textless{}{-}} \FunctionTok{rbind}\NormalTok{(Descritores1, Descritores2)}
\NormalTok{DescritoresFinal}
\NormalTok{DescritoresFinal }\OtherTok{\textless{}{-}} \FunctionTok{round}\NormalTok{ (DescritoresFinal, }\DecValTok{2}\NormalTok{)}
\CommentTok{\#Fazendo uma tabela}
\CommentTok{\#library(gt)}
\NormalTok{df }\OtherTok{\textless{}{-}}\NormalTok{ DescritoresFinal}
\FunctionTok{ncol}\NormalTok{(df); }\FunctionTok{nrow}\NormalTok{(df) }\CommentTok{\#no. de N colunas x M linhas}
\NormalTok{df }\OtherTok{\textless{}{-}} \FunctionTok{cbind}\NormalTok{(}\AttributeTok{UA =} \FunctionTok{rownames}\NormalTok{(df), df)}
\FunctionTok{gt}\NormalTok{(df, }\AttributeTok{caption =} \StringTok{"Descritores da diversidade por Unidade Amostral (UA). Sum, soma; mean, média; DP, desvio padrão da média; Max, maior valor; Min, menor valor; S, riqueza (ou frequência de ocorrência na matriz transposta); E, índice de equitabilidade de Pielou; H, índice de diversidade de Shannon; D, índice de diversidade de Simpson."}\NormalTok{)}
\NormalTok{Normalidade1 }\OtherTok{\textless{}{-}} \FunctionTok{cbind}\NormalTok{(Assimetria, Curtose)}
\NormalTok{Normalidade1 }\OtherTok{\textless{}{-}} \FunctionTok{as.data.frame}\NormalTok{(Normalidade1)}
\NormalTok{Normalidade1 }\OtherTok{\textless{}{-}} \FunctionTok{na.omit}\NormalTok{(Normalidade1) }\CommentTok{\#remove NA e NaN}
\NormalTok{Normalidade1}
\NormalTok{SomaTotalN }\OtherTok{\textless{}{-}} \FunctionTok{apply}\NormalTok{(Normalidade1,}\DecValTok{2}\NormalTok{,sum)}
\NormalTok{SomaTotalN}
\NormalTok{MediaTotalN }\OtherTok{\textless{}{-}} \FunctionTok{apply}\NormalTok{(Normalidade1,}\DecValTok{2}\NormalTok{,mean)}
\NormalTok{MediaTotalN}
\NormalTok{DPTotalN }\OtherTok{\textless{}{-}} \FunctionTok{apply}\NormalTok{(Normalidade1,}\DecValTok{2}\NormalTok{,sd)}
\NormalTok{DPTotalN}
\NormalTok{Normalidade2}\OtherTok{\textless{}{-}}\FunctionTok{cbind}\NormalTok{(SomaTotalN, MediaTotalN, DPTotalN)}
\NormalTok{Normalidade2}\OtherTok{\textless{}{-}}\FunctionTok{as.data.frame}\NormalTok{(Normalidade2)}
\NormalTok{Normalidade2 }\OtherTok{\textless{}{-}} \FunctionTok{t}\NormalTok{(Normalidade2) }\CommentTok{\#"t" transpoe a matriz}
\NormalTok{Normalidade2}
\NormalTok{NormalidadeFinal }\OtherTok{\textless{}{-}} \FunctionTok{rbind}\NormalTok{(Normalidade1, Normalidade2)}
\NormalTok{NormalidadeFinal}
\NormalTok{NormalidadeFinal }\OtherTok{\textless{}{-}} \FunctionTok{round}\NormalTok{(NormalidadeFinal, }\DecValTok{2}\NormalTok{)}
\CommentTok{\#Fazendo uma tabela}
\NormalTok{nf }\OtherTok{\textless{}{-}}\NormalTok{ NormalidadeFinal}
\FunctionTok{ncol}\NormalTok{(nf); }\FunctionTok{nrow}\NormalTok{(nf) }\CommentTok{\#no. de N colunas x M linhas}
\NormalTok{nf }\OtherTok{\textless{}{-}} \FunctionTok{cbind}\NormalTok{(}\AttributeTok{UA =} \FunctionTok{rownames}\NormalTok{(nf), nf)}
\FunctionTok{gt}\NormalTok{(nf, }\AttributeTok{caption =} \StringTok{"Descritores da normalidade por Unidade Amostral (UA)"}\NormalTok{)}
\FunctionTok{write.table}\NormalTok{(}\FunctionTok{data.frame}\NormalTok{(}\StringTok{"Spp"}\OtherTok{=}\FunctionTok{rownames}\NormalTok{(DescritoresFinal),}
\NormalTok{                       DescritoresFinal),}
            \StringTok{"DescritoresUA.csv"}\NormalTok{,}
            \AttributeTok{row.names=}\ConstantTok{FALSE}\NormalTok{,}
            \AttributeTok{sep=}\StringTok{"}\SpecialCharTok{\textbackslash{}t}\StringTok{"}\NormalTok{)}
\FunctionTok{write.table}\NormalTok{(}\FunctionTok{data.frame}\NormalTok{(}\StringTok{"Spp"}\OtherTok{=}\FunctionTok{rownames}\NormalTok{(NormalidadeFinal),}
\NormalTok{                       NormalidadeFinal),}
            \StringTok{"NormalidadeUA.txt"}\NormalTok{,}
            \AttributeTok{row.names=}\ConstantTok{FALSE}\NormalTok{,}
            \AttributeTok{sep=}\StringTok{"}\SpecialCharTok{\textbackslash{}t}\StringTok{"}\NormalTok{)}
\end{Highlighting}
\end{Shaded}

\hypertarget{referuxeancias-9}{%
\section*{Referências}\label{referuxeancias-9}}
\addcontentsline{toc}{section}{Referências}

\hypertarget{partdiv}{%
\chapter{R Modulo 4.2 - Decomposição da diversidade}\label{partdiv}}

\hypertarget{resumo-9}{%
\subsection*{RESUMO}\label{resumo-9}}
\addcontentsline{toc}{subsection}{RESUMO}

Na ecologia de comunidades, a diversidade de espécies é uma medida importante para entender a complexidade e a estrutura de uma comunidade de organismos. Vamos explorar algumas das métricas comuns usadas para avaliar a diversidade de espécies.

\hypertarget{apresentauxe7uxe3o-9}{%
\subsection*{Apresentação}\label{apresentauxe7uxe3o-9}}
\addcontentsline{toc}{subsection}{Apresentação}

Na ecologia de comunidades, a diversidade de espécies é uma medida importante para entender a complexidade e a estrutura de uma comunidade de organismos. Vamos explorar algumas das métricas comuns usadas para avaliar a diversidade de espécies. Essas métricas são usadas para avaliar a diversidade e a estrutura de comunidades ecológicas. Elas podem fornecer informações importantes sobre como as diferentes espécies interagem em um ecossistema e como a diversidade de espécies pode ser afetada por mudanças ambientais ou distúrbios (\protect\hyperlink{ref-RN357}{MAGURRAN, 1988}).

\hypertarget{sobre-os-dados-4}{%
\section{Sobre os dados}\label{sobre-os-dados-4}}

Usaremos para esse tutorial dois conjuntos de dados. A \textbf{Matriz comunitária} (\href{https://docs.google.com/spreadsheets/d/1Olyew6L3SCFeSXWJC54Ak3MKmgHD3sTK/edit?usp=drive_link\&ouid=101766125969589673518\&rtpof=true\&sd=true}{ppbio06c-peixes.xlsx}) de dados coletados no Programa de Pesquisa em Biodiversidade - PPBio (Veja \href{https://ppbio.inpa.gov.br/Sobre}{Programa de Pesquisa em Biodiversidade -- PPBio}). Esses são dados de espécies de peixes distribuidas em diversas unidades amostrais (UA's ou sítios). Essa é a \textbf{matriz bruta de dados}, porque os valores ainda não foram ajustados para os valores de Captura Por Unidade de Esforço (CPUE), nem foram relativizados ou transformados.

Além disso usaremos a \textbf{tabela de agrupamentos} (\href{https://docs.google.com/spreadsheets/d/1H52eBBxZ1k-6Xgrse87etq6mq3mMUTZi/edit?usp=drive_link\&ouid=101766125969589673518\&rtpof=true\&sd=true}{ppbio06-grupos})

\textbf{Revise as informações sobre as bases de dados no capítulo
\protect\hyperlink{bases}{Bases de Dados}. A matriz de dados para esse Módulo pode ser baixada na seção
\protect\hyperlink{arqs}{Arquivos disponíveis}.}

\textbf{Revise o módulo anterior \protect\hyperlink{estrcom}{Estrutura da Comunidade}}

\hypertarget{organizauxe7uxe3o-buxe1sica-6}{%
\section{Organização básica}\label{organizauxe7uxe3o-buxe1sica-6}}

\hypertarget{apuxeandices-9}{%
\section*{Apêndices}\label{apuxeandices-9}}
\addcontentsline{toc}{section}{Apêndices}

\hypertarget{sites-consultados-7}{%
\subsection*{Sites consultados}\label{sites-consultados-7}}
\addcontentsline{toc}{subsection}{Sites consultados}

\href{www.link.com}{Primeira coluna para nomes das linhas}

\hypertarget{script-limpo-6}{%
\subsection*{Script limpo}\label{script-limpo-6}}
\addcontentsline{toc}{subsection}{Script limpo}

Aqui apresento o scrip na íntegra sem os textos ou outros comentários. Você pode copiar e colar no R para executa-lo. Lembre de remover os {\texttt{\#}} ou {\texttt{\#\#}} caso necessite executar essas linhas.

\hypertarget{referuxeancias-10}{%
\section*{Referências}\label{referuxeancias-10}}
\addcontentsline{toc}{section}{Referências}

\hypertarget{estrhab}{%
\chapter{R Modulo 4.2 - Estrutura do habitat}\label{estrhab}}

\hypertarget{resumo-10}{%
\subsection*{RESUMO}\label{resumo-10}}
\addcontentsline{toc}{subsection}{RESUMO}

\hypertarget{apresentauxe7uxe3o-10}{%
\subsection*{Apresentação}\label{apresentauxe7uxe3o-10}}
\addcontentsline{toc}{subsection}{Apresentação}

\hypertarget{sobre-os-dados-5}{%
\section{Sobre os dados}\label{sobre-os-dados-5}}

Usaremos para esse tutorial dois conjuntos de dados. A \textbf{Matriz ambiental} (\href{https://docs.google.com/spreadsheets/d/1V_QxcorksaSOa40Uzj1z2jy02WgKeFTx/edit?usp=drive_link\&ouid=101766125969589673518\&rtpof=true\&sd=true}{ppbio06p-amb.xlsx}) de dados coletados no Programa de Pesquisa em Biodiversidade - PPBio (Veja \href{https://ppbio.inpa.gov.br/Sobre}{Programa de Pesquisa em Biodiversidade -- PPBio}). Esses são dados de \textbf{variáveis ambientais} distribuidas em diversas unidades amostrais (UA's ou sítios). Essa é a \textbf{matriz bruta de dados}, porque os valores ainda não foram transformados (\protect\hyperlink{ref-RN67}{SOKAL; ROHLF, 1995}). Note que as variáveis foram medidas em diferentes unidades (cm, m, °C, mg/L, etc.), com uma alta amplitude de variação, sugerido uso de matriz transformada e/ou reescalada.
Além disso usaremos a \textbf{tabela de agrupamentos} (\href{https://docs.google.com/spreadsheets/d/1H52eBBxZ1k-6Xgrse87etq6mq3mMUTZi/edit?usp=drive_link\&ouid=101766125969589673518\&rtpof=true\&sd=true}{ppbio06-grupos}).

\textbf{Revise as informações sobre as bases de dados no Capítulo \ref{bases}. As dases de dados para esse Módulo podem ser baixadam na Seção \ref{arqs}.}

\hypertarget{organizauxe7uxe3o-buxe1sica-7}{%
\section{Organização básica}\label{organizauxe7uxe3o-buxe1sica-7}}

\begin{Shaded}
\begin{Highlighting}[]
\FunctionTok{dev.off}\NormalTok{() }\CommentTok{\#apaga os graficos, se houver algum}
\FunctionTok{rm}\NormalTok{(}\AttributeTok{list=}\FunctionTok{ls}\NormalTok{(}\AttributeTok{all=}\ConstantTok{TRUE}\NormalTok{)) }\CommentTok{\#limpa a memória}
\FunctionTok{cat}\NormalTok{(}\StringTok{"}\SpecialCharTok{\textbackslash{}014}\StringTok{"}\NormalTok{) }\CommentTok{\#limpa o console }
\end{Highlighting}
\end{Shaded}

\hypertarget{instalando-os-pacotes-necessuxe1rios-para-esse-muxf3dulo.}{%
\subsection{Instalando os pacotes necessários para esse módulo.}\label{instalando-os-pacotes-necessuxe1rios-para-esse-muxf3dulo.}}

Se você ja fez o módulo anterior, não vai precisar instalar nenhum pacote novo. Os pacotes necessários serão carregados ao longo desse módulo.

\begin{Shaded}
\begin{Highlighting}[]
\FunctionTok{install.packages}\NormalTok{(}\StringTok{"psych"}\NormalTok{)}
\FunctionTok{install.packages}\NormalTok{(}\StringTok{"corrplot"}\NormalTok{)}
\end{Highlighting}
\end{Shaded}

Agora vamos \textbf{definir o diretório de trabalho}. Esse código é usado para obter e definir o diretório de trabalho atual no R. O comando \texttt{getwd()} retorna o caminho do diretório onde o R está lendo e salvando arquivos. O comando \texttt{setwd()} muda esse diretório de trabalho para o caminho especificado entre aspas. No seu caso, você deve ajustar o caminho para o seu próprio diretório de trabalho. \textbf{Lembre de usar a barra ``/'' entre os diretórios. E não a contra-barra ``\textbackslash{}''.}

\begin{Shaded}
\begin{Highlighting}[]
\FunctionTok{getwd}\NormalTok{()}
\FunctionTok{setwd}\NormalTok{(}\StringTok{"C:/Seu/Diretório/De/Trabalho"}\NormalTok{)}
\end{Highlighting}
\end{Shaded}

\hypertarget{importando-a-planilha-6}{%
\section{Importando a planilha}\label{importando-a-planilha-6}}

Note que o símbolo {\texttt{\#}} em programação R significa que o texto que vem depois dele é um comentário e não será executado pelo programa. Isso é útil para explicar o código ou deixar anotações.\\
- Ajuste a primeira linha do código abaixo para refletir {``C:/Seu/Diretório/De/Trabalho/Planilha.xlsx''}.\\
- Ajuste o parâmetro \texttt{sheet\ =\ "Sheet1"} para refletir a aba correta do arquivo .xlsx a ser importado. Vamos trabalhar com os dados do ano 1.

\begin{Shaded}
\begin{Highlighting}[]
\CommentTok{\#dir \textless{}{-} getwd() \#criamos um vetor com o diretório  de trbalho }
\CommentTok{\#shell.exec(dir) \#abre o diretorio de trabalho no Windows Explorer}
\FunctionTok{library}\NormalTok{(openxlsx)}
\NormalTok{ppbio }\OtherTok{\textless{}{-}} \FunctionTok{read.xlsx}\NormalTok{(}\StringTok{"D:/Elvio/OneDrive/Disciplinas/\_EcoNumerica/5.Matrizes/ppbio06p{-}amb.xlsx"}\NormalTok{,}
                   \AttributeTok{rowNames =}\NormalTok{ T, }\AttributeTok{colNames =}\NormalTok{ T,}
                   \AttributeTok{sheet =} \StringTok{"ano1"}\NormalTok{)}
\FunctionTok{str}\NormalTok{(ppbio)}
\CommentTok{\#View(ppbio)}
\NormalTok{ppbio[}\DecValTok{1}\SpecialCharTok{:}\DecValTok{5}\NormalTok{,}\DecValTok{1}\SpecialCharTok{:}\DecValTok{5}\NormalTok{] }\CommentTok{\#[1:5,1:5] mostra apenas as linhas e colunas de 1 a 5.}
\end{Highlighting}
\end{Shaded}

\begin{verbatim}
## 'data.frame':    23 obs. of  30 variables:
##  $ h.macroph   : num  8.33 0 5.83 0 0 ...
##  $ h.grass     : num  20 23.3 0 20 0 ...
##  $ h.subveg    : num  10 3 16.7 0 0 ...
##  $ h.overhveg  : num  0 26.7 33.3 0 33.3 ...
##  $ h.litter    : num  0.667 2.333 23.667 0 1 ...
##  $ h.filalgae  : num  0 3.33 25 0 0 ...
##  $ h.attalgae  : num  12 0.667 3.333 1 0 ...
##  $ h.roots     : num  0 3.33 0 0 5 ...
##  $ h.lrgdeb    : num  1.67 1.67 0 0 3.33 ...
##  $ h.smldeb    : num  1.67 2 10 0 7 ...
##  $ s.mud       : num  65 16.667 81.667 40 0.667 ...
##  $ s.sand      : num  30 70 8.33 40 87.67 ...
##  $ s.smlgrav   : num  0 5 1.67 0 3.33 ...
##  $ s.lrggrav   : num  0 5 0 0 3.33 ...
##  $ s.cobbles   : num  1.67 3.33 0 20 5 ...
##  $ s.rocks     : num  0 0 0 0 0 ...
##  $ s.bedrock   : num  3.33 0 8.33 0 0 ...
##  $ m.elev      : num  226 169 270 226 169 270 226 169 270 226 ...
##  $ m.river     : num  163 83 110 163 83 ...
##  $ m.stream    : num  163.2 13.6 37.1 163.2 13.6 ...
##  $ m.distsource: num  84.45 6.15 8.85 84.45 6.15 ...
##  $ m.distmouth : num  217 186 255 217 186 ...
##  $ m.maxslope  : num  30 60 60 30 60 60 30 60 60 30 ...
##  $ m.maxdepth  : num  106 60 154 105 68 118 110 79 109 74 ...
##  $ m.habdepth  : num  81.3 22.7 54.7 67 32.7 ...
##  $ m.width     : num  19.6 17.2 102 16.1 18.5 ...
##  $ a.veloc     : num  0.167 0.159 0.1 0.125 0 ...
##  $ a.temp      : num  32.9 35.2 34 32 29 ...
##  $ a.do        : num  6.51 6.86 4.82 5.38 3.02 ...
##  $ a.transp    : num  46 26 61 44 33 ...
##         h.macroph  h.grass h.subveg h.overhveg   h.litter
## S-R-CT1  8.333333 20.00000 10.00000    0.00000  0.6666667
## S-R-CP1  0.000000 23.33333  3.00000   26.66667  2.3333333
## S-A-TA1  5.833333  0.00000 16.66667   33.33333 23.6666660
## S-R-CT2  0.000000 20.00000  0.00000    0.00000  0.0000000
## S-R-CP2  0.000000  0.00000  0.00000   33.33333  1.0000000
\end{verbatim}

Exibindo os dados importados (esses comando são ``case-sensitive'' \texttt{ignore.case(object)}).

\begin{Shaded}
\begin{Highlighting}[]
\CommentTok{\#View(ppbio)}
\FunctionTok{print}\NormalTok{(ppbio[}\DecValTok{1}\SpecialCharTok{:}\DecValTok{8}\NormalTok{,}\DecValTok{1}\SpecialCharTok{:}\DecValTok{8}\NormalTok{])}
\NormalTok{ppbio[}\DecValTok{1}\SpecialCharTok{:}\DecValTok{10}\NormalTok{,}\DecValTok{1}\SpecialCharTok{:}\DecValTok{10}\NormalTok{]}
\FunctionTok{str}\NormalTok{(ppbio)}
\FunctionTok{mode}\NormalTok{(ppbio)}
\FunctionTok{class}\NormalTok{(ppbio)}
\end{Highlighting}
\end{Shaded}

\hypertarget{reset-point-2}{%
\section{Reset point}\label{reset-point-2}}

\begin{Shaded}
\begin{Highlighting}[]
\NormalTok{m\_amb }\OtherTok{\textless{}{-}}\NormalTok{ ppbio}
\CommentTok{\#pat \textless{}{-} "\^{}S"}
\CommentTok{\#m\_amb \textless{}{-} m\_amb[!grepl(pat, rownames(m\_amb)), ] \#exclui quem começa com pat}
\end{Highlighting}
\end{Shaded}

Aqui cria-se um novo objeto do R (\texttt{m\_trab}, ou a matriz de trabalho, para esse momento) que substitui a matriz de dados original, por uma nova matriz que pode ser a matriz relativizada, transformada, transposta, etc. Dessa forma, mantemos a matriz de dados original caso precisemos dela novamente (Veja a Tabela \ref{tab:200arqs}).

\textbf{Revise Seção \ref{abrev} e a Tabela de Abreviações (@ref(tab:200m\_2)) na mesma Seção que resumem os tipos de matrizes e suas abreviações, para os nossos códigos.}

\hypertarget{correlograma-e-remouxe7uxe3o-de-variuxe1veis-redundantes-ou-desnecessuxe1rias}{%
\section{Correlograma e remoção de variáveis redundantes ou desnecessárias}\label{correlograma-e-remouxe7uxe3o-de-variuxe1veis-redundantes-ou-desnecessuxe1rias}}

\begin{Shaded}
\begin{Highlighting}[]
\FunctionTok{library}\NormalTok{(psych)}
\end{Highlighting}
\end{Shaded}

\begin{verbatim}
## 
## Attaching package: 'psych'
\end{verbatim}

\begin{verbatim}
## The following object is masked from 'package:mdatools':
## 
##     pca
\end{verbatim}

\begin{verbatim}
## The following object is masked from 'package:car':
## 
##     logit
\end{verbatim}

\begin{verbatim}
## The following objects are masked from 'package:ggplot2':
## 
##     %+%, alpha
\end{verbatim}

\begin{Shaded}
\begin{Highlighting}[]
\FunctionTok{colnames}\NormalTok{(m\_amb)}
\FunctionTok{pairs.panels}\NormalTok{(m\_amb[,}\DecValTok{1}\SpecialCharTok{:}\DecValTok{10}\NormalTok{],}
             \AttributeTok{method =} \StringTok{"pearson"}\NormalTok{, }\CommentTok{\# correlation method}
             \AttributeTok{scale =} \ConstantTok{FALSE}\NormalTok{, }\AttributeTok{lm =} \ConstantTok{FALSE}\NormalTok{,}
             \AttributeTok{hist.col =} \StringTok{"\#00AFBB"}\NormalTok{, }\AttributeTok{pch =} \DecValTok{19}\NormalTok{,}
             \AttributeTok{density =} \ConstantTok{TRUE}\NormalTok{,  }\CommentTok{\# show density plots}
             \AttributeTok{ellipses =} \ConstantTok{TRUE}\NormalTok{, }\CommentTok{\# show correlation ellipses}
             \AttributeTok{alpha =} \FloatTok{0.5}\NormalTok{)}

\NormalTok{cor }\OtherTok{\textless{}{-}} \FunctionTok{cor}\NormalTok{(m\_amb)}
\NormalTok{cor}
\FunctionTok{library}\NormalTok{(corrplot)}
\end{Highlighting}
\end{Shaded}

\begin{verbatim}
## corrplot 0.92 loaded
\end{verbatim}

\begin{Shaded}
\begin{Highlighting}[]
\FunctionTok{corrplot}\NormalTok{(cor, }\AttributeTok{method =} \StringTok{"circle"}\NormalTok{)}
\CommentTok{\#win.print()}
\CommentTok{\#corrplot(cor, method = "circle")}
\CommentTok{\#dev.off()}
\end{Highlighting}
\end{Shaded}

\begin{verbatim}
##  [1] "h.macroph"    "h.grass"      "h.subveg"     "h.overhveg"   "h.litter"    
##  [6] "h.filalgae"   "h.attalgae"   "h.roots"      "h.lrgdeb"     "h.smldeb"    
## [11] "s.mud"        "s.sand"       "s.smlgrav"    "s.lrggrav"    "s.cobbles"   
## [16] "s.rocks"      "s.bedrock"    "m.elev"       "m.river"      "m.stream"    
## [21] "m.distsource" "m.distmouth"  "m.maxslope"   "m.maxdepth"   "m.habdepth"  
## [26] "m.width"      "a.veloc"      "a.temp"       "a.do"         "a.transp"    
##                h.macroph      h.grass    h.subveg    h.overhveg     h.litter
## h.macroph     1.00000000 -0.246531504 -0.13690655 -0.0460990189 -0.062637491
## h.grass      -0.24653150  1.000000000  0.48407509 -0.1460764735 -0.001721976
## h.subveg     -0.13690655  0.484075094  1.00000000  0.0635287907  0.399197150
## h.overhveg   -0.04609902 -0.146076474  0.06352879  1.0000000000  0.603643875
## h.litter     -0.06263749 -0.001721976  0.39919715  0.6036438749  1.000000000
## h.filalgae    0.30739321 -0.225737505 -0.06840003  0.3151725645  0.391954687
## h.attalgae   -0.20061743 -0.203856535 -0.15804953 -0.0588612264 -0.074417928
## h.roots      -0.17114523 -0.042846121 -0.12422422  0.7224571129 -0.020670464
## h.lrgdeb      0.03975689  0.517038783 -0.13497524 -0.0002521598 -0.076622164
## h.smldeb     -0.11282416  0.430029457  0.32450367  0.3420275576  0.524639005
## s.mud        -0.02825772 -0.213876066  0.09356431 -0.3318047182  0.081241055
## s.sand        0.10165279  0.290788117 -0.02468805  0.2908755666 -0.108025446
## s.smlgrav     0.02303766 -0.230252599 -0.26591389  0.1532179364  0.008573628
## s.lrggrav     0.05597825 -0.088160779 -0.19524294  0.1761326511 -0.054082053
## s.cobbles    -0.24512665 -0.007205913 -0.15483540  0.0694742877 -0.093430622
## s.rocks      -0.23075335 -0.259155527 -0.21954489 -0.0873692368 -0.121239165
## s.bedrock    -0.03652910 -0.083573123  0.30828355  0.5073656218  0.900953133
## m.elev       -0.08310889  0.403591279  0.35058982 -0.4003114855 -0.131742426
## m.river      -0.23257092  0.335464812  0.24926832 -0.5955683647 -0.242615414
## m.stream     -0.06608660 -0.004596811 -0.07497728 -0.2227767752 -0.096058730
## m.distsource -0.16898179  0.005784151 -0.10079544 -0.2397842969 -0.143925113
## m.distmouth  -0.09461959  0.058821726 -0.01701650 -0.2886438017  0.009362746
## m.maxslope    0.06917689 -0.432093408 -0.28679549  0.2396730909  0.112055941
## m.maxdepth    0.03186630  0.402241943  0.19851159 -0.0016801722  0.489997569
## m.habdepth   -0.18039247 -0.045055633 -0.13643528  0.0703614430  0.206858486
## m.width       0.10609087  0.339661628  0.37489017 -0.2460756693 -0.036647042
## a.veloc      -0.13503225  0.204580859  0.11506415  0.3775858154  0.314867909
## a.temp        0.18273224  0.093689794  0.10708693  0.3872777253  0.376177481
## a.do          0.26259801 -0.202794752 -0.60025144 -0.2351348072 -0.186841357
## a.transp      0.30879962 -0.007818165  0.06381010 -0.0405424177  0.149062804
##                h.filalgae    h.attalgae     h.roots      h.lrgdeb      h.smldeb
## h.macroph     0.307393212 -0.2006174303 -0.17114523  0.0397568923 -0.1128241586
## h.grass      -0.225737505 -0.2038565351 -0.04284612  0.5170387832  0.4300294573
## h.subveg     -0.068400029 -0.1580495292 -0.12422422 -0.1349752429  0.3245036701
## h.overhveg    0.315172564 -0.0588612264  0.72245711 -0.0002521598  0.3420275576
## h.litter      0.391954687 -0.0744179283 -0.02067046 -0.0766221642  0.5246390055
## h.filalgae    1.000000000  0.1650218328 -0.12170792  0.0330671467  0.0566631854
## h.attalgae    0.165021833  1.0000000000 -0.15141029 -0.1110887477 -0.0007226332
## h.roots      -0.121707915 -0.1514102880  1.00000000  0.0359061688  0.1278778876
## h.lrgdeb      0.033067147 -0.1110887477  0.03590617  1.0000000000  0.5126858957
## h.smldeb      0.056663185 -0.0007226332  0.12787789  0.5126858957  1.0000000000
## s.mud        -0.203493831  0.0966629147 -0.45742597 -0.2661041300 -0.1358265852
## s.sand        0.131463557 -0.2065073854  0.42226823  0.3631216979  0.2116433147
## s.smlgrav     0.056355761 -0.2466165417  0.37504581 -0.2178131916 -0.1938969370
## s.lrggrav     0.071026543 -0.2019284184  0.45157180 -0.1064395977 -0.1131217094
## s.cobbles     0.193135679  0.4075488652  0.08813464 -0.1218105205 -0.2307397229
## s.rocks       0.144067566  0.5595435077 -0.15324139 -0.1450923302 -0.0896567456
## s.bedrock     0.338537041 -0.0146391847 -0.09492929 -0.0956520825  0.3503440506
## m.elev       -0.203448440  0.0716487680 -0.36640440  0.2656332571  0.1727891903
## m.river      -0.405669260 -0.0195694902 -0.51853032  0.1309548871 -0.0189887633
## m.stream     -0.185417996 -0.0598126521 -0.21720079 -0.1720191090 -0.1810175388
## m.distsource -0.254533090 -0.0572352576 -0.18429493 -0.1776252995 -0.2083808690
## m.distmouth  -0.217780469 -0.2330343386 -0.33511609  0.1038897654  0.0767920775
## m.maxslope    0.145284072 -0.1601113205  0.18673984 -0.1907894213 -0.1808970343
## m.maxdepth    0.114574709 -0.2474674119 -0.35179352  0.5122182964  0.6060301921
## m.habdepth    0.002882856  0.0164745166 -0.09468290  0.0036984721  0.0212264396
## m.width      -0.011447418  0.0797573366 -0.28844323  0.2562191774  0.1713597968
## a.veloc      -0.002302649 -0.0605797247  0.20772262 -0.1007706522 -0.0883913083
## a.temp       -0.047385170 -0.3528557182  0.21266182 -0.0633115802  0.0495255921
## a.do          0.075932799  0.2107159895 -0.19965987 -0.0527554510 -0.2305667359
## a.transp      0.486443594  0.1991293428 -0.26299621  0.2727840805  0.2437019444
##                    s.mud      s.sand    s.smlgrav   s.lrggrav    s.cobbles
## h.macroph    -0.02825772  0.10165279  0.023037659  0.05597825 -0.245126650
## h.grass      -0.21387607  0.29078812 -0.230252599 -0.08816078 -0.007205913
## h.subveg      0.09356431 -0.02468805 -0.265913890 -0.19524294 -0.154835404
## h.overhveg   -0.33180472  0.29087557  0.153217936  0.17613265  0.069474288
## h.litter      0.08124106 -0.10802545  0.008573628 -0.05408205 -0.093430622
## h.filalgae   -0.20349383  0.13146356  0.056355761  0.07102654  0.193135679
## h.attalgae    0.09666291 -0.20650739 -0.246616542 -0.20192842  0.407548865
## h.roots      -0.45742597  0.42226823  0.375045814  0.45157180  0.088134635
## h.lrgdeb     -0.26610413  0.36312170 -0.217813192 -0.10643960 -0.121810520
## h.smldeb     -0.13582659  0.21164331 -0.193896937 -0.11312171 -0.230739723
## s.mud         1.00000000 -0.95732831 -0.196136235 -0.20727196 -0.436951580
## s.sand       -0.95732831  1.00000000  0.077601964  0.06113988  0.208220232
## s.smlgrav    -0.19613624  0.07760196  1.000000000  0.91794933 -0.018580075
## s.lrggrav    -0.20727196  0.06113988  0.917949325  1.00000000  0.092736437
## s.cobbles    -0.43695158  0.20822023 -0.018580075  0.09273644  1.000000000
## s.rocks      -0.05450604 -0.08712170 -0.290187186 -0.20917141  0.471919983
## s.bedrock     0.20788751 -0.23158686 -0.068825165 -0.12957657 -0.117937719
## m.elev        0.46647761 -0.31506556 -0.363536747 -0.35844427 -0.418891510
## m.river       0.52267706 -0.35870643 -0.487625688 -0.54998626 -0.404162728
## m.stream      0.25913400 -0.29280195 -0.351718673 -0.23344861  0.162336709
## m.distsource  0.20798419 -0.24660914 -0.335764568 -0.23558448  0.197820570
## m.distmouth  -0.12783195  0.26298990 -0.055199753 -0.33221276 -0.343006017
## m.maxslope   -0.48822579  0.47051421  0.480231912  0.19989479  0.003285850
## m.maxdepth   -0.08984828  0.15690152 -0.233610098 -0.23813743 -0.181918712
## m.habdepth   -0.29918013  0.24152725 -0.124634583 -0.20123580  0.329637349
## m.width       0.49913277 -0.37568825 -0.310246711 -0.23740245 -0.381602616
## a.veloc      -0.07881376  0.02561095 -0.011480476  0.04477510  0.201826380
## a.temp       -0.20642510  0.16196759  0.223976700  0.27319038  0.054683155
## a.do          0.29936318 -0.31184728  0.228246923  0.14674375 -0.188261015
## a.transp     -0.07274786  0.08367750 -0.090575822 -0.07405382  0.010898759
##                   s.rocks   s.bedrock      m.elev     m.river     m.stream
## h.macroph    -0.230753352 -0.03652910 -0.08310889 -0.23257092 -0.066086599
## h.grass      -0.259155527 -0.08357312  0.40359128  0.33546481 -0.004596811
## h.subveg     -0.219544891  0.30828355  0.35058982  0.24926832 -0.074977285
## h.overhveg   -0.087369237  0.50736562 -0.40031149 -0.59556836 -0.222776775
## h.litter     -0.121239165  0.90095313 -0.13174243 -0.24261541 -0.096058730
## h.filalgae    0.144067566  0.33853704 -0.20344844 -0.40566926 -0.185417996
## h.attalgae    0.559543508 -0.01463918  0.07164877 -0.01956949 -0.059812652
## h.roots      -0.153241391 -0.09492929 -0.36640440 -0.51853032 -0.217200787
## h.lrgdeb     -0.145092330 -0.09565208  0.26563326  0.13095489 -0.172019109
## h.smldeb     -0.089656746  0.35034405  0.17278919 -0.01898876 -0.181017539
## s.mud        -0.054506037  0.20788751  0.46647761  0.52267706  0.259134004
## s.sand       -0.087121705 -0.23158686 -0.31506556 -0.35870643 -0.292801947
## s.smlgrav    -0.290187186 -0.06882517 -0.36353675 -0.48762569 -0.351718673
## s.lrggrav    -0.209171407 -0.12957657 -0.35844427 -0.54998626 -0.233448612
## s.cobbles     0.471919983 -0.11793772 -0.41889151 -0.40416273  0.162336709
## s.rocks       1.000000000 -0.12799218 -0.21907348 -0.13545807  0.215737335
## s.bedrock    -0.127992185  1.00000000 -0.19742578 -0.20071090  0.131806173
## m.elev       -0.219073482 -0.19742578  1.00000000  0.85251041 -0.351025053
## m.river      -0.135458068 -0.20071090  0.85251041  1.00000000  0.017040248
## m.stream      0.215737335  0.13180617 -0.35102505  0.01704025  1.000000000
## m.distsource  0.248398309  0.08398842 -0.36628911  0.04500187  0.987553629
## m.distmouth  -0.128592531 -0.03694612  0.21088433  0.43005863 -0.285305822
## m.maxslope   -0.063334534  0.04208746 -0.45271956 -0.37494422 -0.445502469
## m.maxdepth   -0.078095404  0.47564528  0.08022982  0.12595144  0.094085376
## m.habdepth    0.133210844  0.41315901 -0.52848500 -0.21000817  0.395997655
## m.width      -0.267413674 -0.09980488  0.92959448  0.66160089 -0.341120113
## a.veloc      -0.203924152  0.50163597 -0.39264584 -0.27533927  0.410516289
## a.temp       -0.221454717  0.41538130 -0.60522317 -0.50743529  0.318565681
## a.do          0.004835464 -0.06655234 -0.04245875 -0.05281226  0.104875559
## a.transp     -0.016360295  0.14038232  0.16444871 -0.08762683 -0.339047044
##              m.distsource  m.distmouth   m.maxslope   m.maxdepth   m.habdepth
## h.macroph    -0.168981791 -0.094619591  0.069176887  0.031866297 -0.180392470
## h.grass       0.005784151  0.058821726 -0.432093408  0.402241943 -0.045055633
## h.subveg     -0.100795436 -0.017016500 -0.286795491  0.198511591 -0.136435275
## h.overhveg   -0.239784297 -0.288643802  0.239673091 -0.001680172  0.070361443
## h.litter     -0.143925113  0.009362746  0.112055941  0.489997569  0.206858486
## h.filalgae   -0.254533090 -0.217780469  0.145284072  0.114574709  0.002882856
## h.attalgae   -0.057235258 -0.233034339 -0.160111320 -0.247467412  0.016474517
## h.roots      -0.184294932 -0.335116093  0.186739840 -0.351793522 -0.094682904
## h.lrgdeb     -0.177625299  0.103889765 -0.190789421  0.512218296  0.003698472
## h.smldeb     -0.208380869  0.076792078 -0.180897034  0.606030192  0.021226440
## s.mud         0.207984189 -0.127831955 -0.488225793 -0.089848279 -0.299180125
## s.sand       -0.246609145  0.262989896  0.470514210  0.156901516  0.241527247
## s.smlgrav    -0.335764568 -0.055199753  0.480231912 -0.233610098 -0.124634583
## s.lrggrav    -0.235584477 -0.332212758  0.199894791 -0.238137430 -0.201235803
## s.cobbles     0.197820570 -0.343006017  0.003285850 -0.181918712  0.329637349
## s.rocks       0.248398309 -0.128592531 -0.063334534 -0.078095404  0.133210844
## s.bedrock     0.083988424 -0.036946118  0.042087459  0.475645280  0.413159010
## m.elev       -0.366289105  0.210884333 -0.452719558  0.080229817 -0.528485000
## m.river       0.045001873  0.430058635 -0.374944223  0.125951443 -0.210008174
## m.stream      0.987553629 -0.285305822 -0.445502469  0.094085376  0.395997655
## m.distsource  1.000000000 -0.208886217 -0.380287859  0.070018120  0.454126171
## m.distmouth  -0.208886217  1.000000000  0.575655522  0.409794260  0.329568908
## m.maxslope   -0.380287859  0.575655522  1.000000000 -0.033910389  0.326367540
## m.maxdepth    0.070018120  0.409794260 -0.033910389  1.000000000  0.425648040
## m.habdepth    0.454126171  0.329568908  0.326367540  0.425648040  1.000000000
## m.width      -0.403406954 -0.091476026 -0.569731109  0.010919605 -0.663930169
## a.veloc       0.409162331 -0.251728578 -0.087772483  0.109911454  0.516555276
## a.temp        0.303332037 -0.107191864  0.190578794  0.204458211  0.390454228
## a.do          0.054863239 -0.066103944 -0.039194058 -0.068299977 -0.098869325
## a.transp     -0.402578165  0.084213818  0.008475289  0.389325169  0.109987158
##                   m.width      a.veloc      a.temp         a.do     a.transp
## h.macroph     0.106090873 -0.135032245  0.18273224  0.262598008  0.308799625
## h.grass       0.339661628  0.204580859  0.09368979 -0.202794752 -0.007818165
## h.subveg      0.374890169  0.115064150  0.10708693 -0.600251445  0.063810096
## h.overhveg   -0.246075669  0.377585815  0.38727773 -0.235134807 -0.040542418
## h.litter     -0.036647042  0.314867909  0.37617748 -0.186841357  0.149062804
## h.filalgae   -0.011447418 -0.002302649 -0.04738517  0.075932799  0.486443594
## h.attalgae    0.079757337 -0.060579725 -0.35285572  0.210715990  0.199129343
## h.roots      -0.288443233  0.207722616  0.21266182 -0.199659873 -0.262996213
## h.lrgdeb      0.256219177 -0.100770652 -0.06331158 -0.052755451  0.272784080
## h.smldeb      0.171359797 -0.088391308  0.04952559 -0.230566736  0.243701944
## s.mud         0.499132769 -0.078813761 -0.20642510  0.299363179 -0.072747855
## s.sand       -0.375688250  0.025610951  0.16196759 -0.311847278  0.083677500
## s.smlgrav    -0.310246711 -0.011480476  0.22397670  0.228246923 -0.090575822
## s.lrggrav    -0.237402453  0.044775100  0.27319038  0.146743753 -0.074053824
## s.cobbles    -0.381602616  0.201826380  0.05468315 -0.188261015  0.010898759
## s.rocks      -0.267413674 -0.203924152 -0.22145472  0.004835464 -0.016360295
## s.bedrock    -0.099804876  0.501635966  0.41538130 -0.066552340  0.140382315
## m.elev        0.929594477 -0.392645839 -0.60522317 -0.042458750  0.164448713
## m.river       0.661600887 -0.275339274 -0.50743529 -0.052812259 -0.087626835
## m.stream     -0.341120113  0.410516289  0.31856568  0.104875559 -0.339047044
## m.distsource -0.403406954  0.409162331  0.30333204  0.054863239 -0.402578165
## m.distmouth  -0.091476026 -0.251728578 -0.10719186 -0.066103944  0.084213818
## m.maxslope   -0.569731109 -0.087772483  0.19057879 -0.039194058  0.008475289
## m.maxdepth    0.010919605  0.109911454  0.20445821 -0.068299977  0.389325169
## m.habdepth   -0.663930169  0.516555276  0.39045423 -0.098869325  0.109987158
## m.width       1.000000000 -0.326799203 -0.51641177 -0.002086238  0.232399977
## a.veloc      -0.326799203  1.000000000  0.64196801  0.063666386 -0.111321669
## a.temp       -0.516411771  0.641968008  1.00000000 -0.072530252 -0.241797672
## a.do         -0.002086238  0.063666386 -0.07253025  1.000000000  0.109896052
## a.transp      0.232399977 -0.111321669 -0.24179767  0.109896052  1.000000000
\end{verbatim}

\includegraphics{fundamentos_de_ecologia_numerica_files/figure-latex/unnamed-chunk-153-1.pdf} \includegraphics{fundamentos_de_ecologia_numerica_files/figure-latex/unnamed-chunk-153-2.pdf}

\hypertarget{deletando-variuxe1veis-colineares}{%
\section{Deletando variáveis colineares}\label{deletando-variuxe1veis-colineares}}

Primeiro vamos listar as variáveis com correlação maior que 70\% ou r \textgreater{} 0.7.

\begin{Shaded}
\begin{Highlighting}[]
\CommentTok{\# Get upper triangle (to avoid duplicate pairs)}
\NormalTok{r }\OtherTok{\textless{}{-}} \FloatTok{0.7}
\NormalTok{cor\_pairs }\OtherTok{\textless{}{-}} \FunctionTok{which}\NormalTok{(}\FunctionTok{abs}\NormalTok{(cor) }\SpecialCharTok{\textgreater{}}\NormalTok{ r }\SpecialCharTok{\&} \FunctionTok{abs}\NormalTok{(cor) }\SpecialCharTok{\textless{}} \DecValTok{1}\NormalTok{, }\AttributeTok{arr.ind =} \ConstantTok{TRUE}\NormalTok{)}

\CommentTok{\# Extract unique variable pairs}
\NormalTok{results }\OtherTok{\textless{}{-}} \FunctionTok{data.frame}\NormalTok{(}
  \AttributeTok{var1 =} \FunctionTok{rownames}\NormalTok{(cor)[cor\_pairs[, }\DecValTok{1}\NormalTok{]],}
  \AttributeTok{var2 =} \FunctionTok{colnames}\NormalTok{(cor)[cor\_pairs[, }\DecValTok{2}\NormalTok{]],}
  \AttributeTok{correlation =}\NormalTok{ cor[cor\_pairs]}
\NormalTok{)}

\CommentTok{\# Remove duplicates (e.g. A{-}B and B{-}A)}
\NormalTok{results }\OtherTok{\textless{}{-}}\NormalTok{ results[results}\SpecialCharTok{$}\NormalTok{var1 }\SpecialCharTok{\textless{}}\NormalTok{ results}\SpecialCharTok{$}\NormalTok{var2, ]}

\CommentTok{\# Sort by absolute correlation (descending)}
\NormalTok{results }\OtherTok{\textless{}{-}}\NormalTok{ results[}\FunctionTok{order}\NormalTok{(}\SpecialCharTok{{-}}\FunctionTok{abs}\NormalTok{(results}\SpecialCharTok{$}\NormalTok{correlation)), ]}
\NormalTok{results}
\end{Highlighting}
\end{Shaded}

\begin{verbatim}
##            var1      var2 correlation
## 12 m.distsource  m.stream   0.9875536
## 5         s.mud    s.sand  -0.9573283
## 14       m.elev   m.width   0.9295945
## 6     s.lrggrav s.smlgrav   0.9179493
## 8      h.litter s.bedrock   0.9009531
## 11       m.elev   m.river   0.8525104
## 3    h.overhveg   h.roots   0.7224571
\end{verbatim}

Agora que temos as variáveis correlacionadas, prosseguimos para deleta-las.

\begin{Shaded}
\begin{Highlighting}[]
\FunctionTok{colnames}\NormalTok{(m\_amb)}
\NormalTok{del\_cols }\OtherTok{\textless{}{-}} \FunctionTok{c}\NormalTok{(}\StringTok{"m.distsource"}\NormalTok{, }\StringTok{"m.elev"}\NormalTok{) }\CommentTok{\#exemplo, "g.river\_length","g.altitude"}
\NormalTok{m\_amb\_part }\OtherTok{\textless{}{-}}\NormalTok{ m\_amb[, }\SpecialCharTok{!}\NormalTok{(}\FunctionTok{colnames}\NormalTok{(m\_amb) }\SpecialCharTok{\%in\%}\NormalTok{ del\_cols)]}
\end{Highlighting}
\end{Shaded}

\begin{verbatim}
##  [1] "h.macroph"    "h.grass"      "h.subveg"     "h.overhveg"   "h.litter"    
##  [6] "h.filalgae"   "h.attalgae"   "h.roots"      "h.lrgdeb"     "h.smldeb"    
## [11] "s.mud"        "s.sand"       "s.smlgrav"    "s.lrggrav"    "s.cobbles"   
## [16] "s.rocks"      "s.bedrock"    "m.elev"       "m.river"      "m.stream"    
## [21] "m.distsource" "m.distmouth"  "m.maxslope"   "m.maxdepth"   "m.habdepth"  
## [26] "m.width"      "a.veloc"      "a.temp"       "a.do"         "a.transp"
\end{verbatim}

\hypertarget{somando-variuxe1veis-redundantes}{%
\section{Somando variáveis redundantes}\label{somando-variuxe1veis-redundantes}}

Algumas dessa variáveis são redundantes, vamos soma-las.

\begin{Shaded}
\begin{Highlighting}[]
\NormalTok{m\_amb\_part}\SpecialCharTok{$}\NormalTok{s.gravel }\OtherTok{\textless{}{-}}\NormalTok{ m\_amb\_part}\SpecialCharTok{$}\NormalTok{s.smlgrav }\SpecialCharTok{+}\NormalTok{ m\_amb\_part}\SpecialCharTok{$}\NormalTok{s.lrggrav }\SpecialCharTok{+}\NormalTok{ m\_amb\_part}\SpecialCharTok{$}\NormalTok{s.cobbles}
\NormalTok{m\_amb\_part }\OtherTok{\textless{}{-}}\NormalTok{ m\_amb\_part[, }\SpecialCharTok{!}\NormalTok{(}\FunctionTok{colnames}\NormalTok{(m\_amb\_part)}
                             \SpecialCharTok{\%in\%} \FunctionTok{c}\NormalTok{(}\StringTok{"s.smlgrav"}\NormalTok{, }\StringTok{"s.lrggrav"}\NormalTok{, }\StringTok{"s.cobbles"}\NormalTok{))]}
\NormalTok{m\_amb\_part}\SpecialCharTok{$}\NormalTok{s.rock }\OtherTok{\textless{}{-}}\NormalTok{ m\_amb\_part}\SpecialCharTok{$}\NormalTok{s.rocks }\SpecialCharTok{+}\NormalTok{ m\_amb\_part}\SpecialCharTok{$}\NormalTok{s.bedrock}
\NormalTok{m\_amb\_part }\OtherTok{\textless{}{-}}\NormalTok{ m\_amb\_part[, }\SpecialCharTok{!}\NormalTok{(}\FunctionTok{colnames}\NormalTok{(m\_amb\_part)}
                             \SpecialCharTok{\%in\%} \FunctionTok{c}\NormalTok{(}\StringTok{"s.rocks"}\NormalTok{, }\StringTok{"s.bedrock"}\NormalTok{))]}
\NormalTok{m\_amb\_part}\SpecialCharTok{$}\NormalTok{h.algae }\OtherTok{\textless{}{-}}\NormalTok{ m\_amb\_part}\SpecialCharTok{$}\NormalTok{h.filalgae }\SpecialCharTok{+}\NormalTok{ m\_amb\_part}\SpecialCharTok{$}\NormalTok{h.attalgae}
\NormalTok{m\_amb\_part }\OtherTok{\textless{}{-}}\NormalTok{ m\_amb\_part[, }\SpecialCharTok{!}\NormalTok{(}\FunctionTok{colnames}\NormalTok{(m\_amb\_part)}
                             \SpecialCharTok{\%in\%} \FunctionTok{c}\NormalTok{(}\StringTok{"h.filalgae"}\NormalTok{, }\StringTok{"h.attalgae"}\NormalTok{))]}
\NormalTok{m\_amb\_part}\SpecialCharTok{$}\NormalTok{h.debris }\OtherTok{\textless{}{-}}\NormalTok{ m\_amb\_part}\SpecialCharTok{$}\NormalTok{h.smldeb }\SpecialCharTok{+}\NormalTok{ m\_amb\_part}\SpecialCharTok{$}\NormalTok{h.lrgdeb}
\NormalTok{m\_amb\_part }\OtherTok{\textless{}{-}}\NormalTok{ m\_amb\_part[, }\SpecialCharTok{!}\NormalTok{(}\FunctionTok{colnames}\NormalTok{(m\_amb\_part)}
                             \SpecialCharTok{\%in\%} \FunctionTok{c}\NormalTok{(}\StringTok{"h.smldeb"}\NormalTok{, }\StringTok{"h.lrgdeb"}\NormalTok{))]}

\FunctionTok{colnames}\NormalTok{(m\_amb\_part)}
\NormalTok{m\_amb\_part}
\end{Highlighting}
\end{Shaded}

\begin{verbatim}
##  [1] "h.macroph"   "h.grass"     "h.subveg"    "h.overhveg"  "h.litter"   
##  [6] "h.roots"     "s.mud"       "s.sand"      "m.river"     "m.stream"   
## [11] "m.distmouth" "m.maxslope"  "m.maxdepth"  "m.habdepth"  "m.width"    
## [16] "a.veloc"     "a.temp"      "a.do"        "a.transp"    "s.gravel"   
## [21] "s.rock"      "h.algae"     "h.debris"   
##         h.macroph    h.grass  h.subveg h.overhveg   h.litter  h.roots
## S-R-CT1  8.333333 20.0000000 10.000000   0.000000  0.6666667 0.000000
## S-R-CP1  0.000000 23.3333340  3.000000  26.666666  2.3333333 3.333333
## S-A-TA1  5.833333  0.0000000 16.666666  33.333332 23.6666660 0.000000
## S-R-CT2  0.000000 20.0000000  0.000000   0.000000  0.0000000 0.000000
## S-R-CP2  0.000000  0.0000000  0.000000  33.333332  1.0000000 5.000000
## S-A-TA2 54.833332  0.3333333  0.000000   8.333333  1.0000000 0.000000
## S-R-CT3  0.000000  0.1111111  0.000000   0.000000  0.0000000 0.000000
## S-R-CP3  0.000000  0.0000000  0.000000   0.000000  1.5000000 1.000000
## S-A-TA3 44.629631  8.1111107  0.000000   0.000000  0.6666667 0.000000
## S-R-CT4  0.000000  4.1666665  0.000000   0.000000  0.8333333 0.000000
## S-R-CP4  0.000000  0.0000000  0.000000   8.333333  1.0000000 0.000000
## S-A-TA4 37.309525  0.0000000  0.000000   0.000000  0.5714286 0.000000
## B-A-MU1  0.000000 50.0000000  0.000000   0.000000  1.0000000 0.000000
## B-A-GU1 46.666668  0.0000000  3.333333   3.333333  0.0000000 0.000000
## B-R-PC2  0.000000  2.0000000  0.000000   0.000000  0.4000000 0.000000
## B-A-MU2  0.000000 54.0000000 36.599998   0.000000  5.0000000 0.000000
## B-A-GU2  2.600000  7.5000000 26.000000   0.400000  0.0000000 0.000000
## B-R-PC3  0.000000  0.0000000  0.000000   0.000000  0.3333333 0.000000
## B-A-MU3  0.000000 28.3333340  8.333333   0.000000  1.0000000 0.000000
## B-A-GU3  0.000000 13.3333330  0.000000   0.000000  0.0000000 0.000000
## B-R-PC4  0.000000  0.0000000  0.000000   0.000000  0.5000000 0.000000
## B-A-MU4  0.000000  5.5555553  0.000000   0.000000  0.4444444 0.000000
## B-A-GU4  2.083333  0.0000000  0.000000   0.000000  0.0000000 0.000000
##              s.mud    s.sand  m.river  m.stream m.distmouth m.maxslope
## S-R-CT1 65.0000000 30.000000 163.2000 163.20000    217.1500         30
## S-R-CP1 16.6666660 70.000000  83.0500  13.55000    185.5500         60
## S-A-TA1 81.6666641  8.333333 110.2000  37.10000    254.5500         60
## S-R-CT2 40.0000000 40.000000 163.2000 163.20000    217.1500         30
## S-R-CP2  0.6666667 87.666664  83.0500  13.55000    185.5500         60
## S-A-TA2  5.0000000 95.000000 110.2000  37.10000    254.5500         60
## S-R-CT3 65.5555573 23.111111 163.2000 163.20000    217.1500         30
## S-R-CP3 48.7500000 22.500000  83.0500  13.55000    185.5500         60
## S-A-TA3 46.6666679 40.000000 110.2000  37.10000    254.5500         60
## S-R-CT4 95.0000000  1.833333 163.2000 163.20000    217.1500         30
## S-R-CP4  5.0000000 60.000000  83.0500  13.55000    185.5500         60
## S-A-TA4 59.1428566 38.714287 110.2000  37.10000    254.5500         60
## B-A-MU1 33.3333321 65.000000 214.0200  19.83333    290.8767         30
## B-A-GU1 96.6666641  3.333333 212.6667  32.33333    203.8333         30
## B-R-PC2 39.0000000 56.000000 196.8333  13.75000    401.6667         90
## B-A-MU2 20.6000004 77.000000 214.0200  19.83333    290.8767         30
## B-A-GU2 98.0000000  2.000000 212.6667  32.33333    203.8333         30
## B-R-PC3 33.6666679 63.000000 196.8333  13.75000    401.6667         90
## B-A-MU3 65.0000000 35.000000 214.0200  19.83333    290.8767         30
## B-A-GU3 87.7777786  6.666667 212.6667  32.33333    203.8333         30
## B-R-PC4 48.8750000 47.875000 196.8333  13.75000    401.6667         90
## B-A-MU4 91.7777786  3.222222 214.0200  19.83333    290.8767         30
## B-A-GU4 95.1666641  3.416667 212.6667  32.33333    203.8333         30
##         m.maxdepth m.habdepth m.width   a.veloc   a.temp     a.do a.transp
## S-R-CT1        106  81.333333   19.64 0.1666667 32.90000 6.510000 46.00000
## S-R-CP1         60  22.666667   17.24 0.1591512 35.20000 6.863333 26.00000
## S-A-TA1        154  54.666667  102.00 0.1000000 34.00000 4.820000 61.00000
## S-R-CT2        105  67.000000   16.10 0.1250000 32.00000 5.375000 44.00000
## S-R-CP2         68  32.666667   18.47 0.0000000 29.00000 3.015000 33.00000
## S-A-TA2        118  37.666667  100.00 0.0000000 29.00000 5.000000 90.00000
## S-R-CT3        110  32.333333    6.20 0.0000000 28.26667 6.000000 17.33333
## S-R-CP3         79  32.666667   15.10 0.0000000 29.66667 5.000000 50.33333
## S-A-TA3        109  22.333333   88.00 0.0000000 34.00000 9.000000 51.66667
## S-R-CT4         74  32.333333    5.40 0.0000000 32.60000 6.000000 16.00000
## S-R-CP4         64  45.333333   10.70 0.0000000 27.60000 4.900000 60.00000
## S-A-TA4         87  30.333333   72.20 0.0000000 29.53333 9.433333 67.00000
## B-A-MU1        152  32.666667  270.00 0.0000000 29.83333 5.666667 48.00000
## B-A-GU1         60   8.166667  330.00 0.0000000 29.23333 5.136667 25.66667
## B-R-PC2        110  49.333333   29.60 0.0000000 29.00000 5.635000 50.00000
## B-A-MU2        115  22.000000  247.63 0.0000000 29.00000 1.815000 43.00000
## B-A-GU2         69   7.000000  321.00 0.0000000 29.00000 1.850000 55.00000
## B-R-PC3         80  50.000000   27.30 0.0000000 29.00000 5.000000 32.33333
## B-A-MU3        117  25.666667  234.53 0.0000000 26.00000 5.700000 63.00000
## B-A-GU3         68   6.833333  314.20 0.0000000 24.00000 8.800000 51.66667
## B-R-PC4         95  52.833333   20.00 0.0000000 28.85000 5.100000 30.00000
## B-A-MU4        112  41.333333  239.00 0.0000000 25.95000 7.300000 89.00000
## B-A-GU4         63   4.666667  289.50 0.0000000 24.70000 8.750000 36.00000
##           s.gravel    s.rock    h.algae   h.debris
## S-R-CT1  1.6666666  3.333333 12.0000000  3.3333333
## S-R-CP1 13.3333333  0.000000  3.9999999  3.6666666
## S-A-TA1  1.6666666  8.333333 28.3333333 10.0000000
## S-R-CT2 20.0000000  0.000000  1.0000000  0.0000000
## S-R-CP2 11.6666665  0.000000  0.0000000 10.3333333
## S-A-TA2  0.0000000  0.000000 33.3333321 13.6666665
## S-R-CT3  3.0000000  8.333333  0.0000000  2.2222223
## S-R-CP3 28.7500000  0.000000  6.7500000  3.6666667
## S-A-TA3 13.3333330  0.000000  9.0000000  1.6666666
## S-R-CT4  0.3333333  2.833333  7.5000000  5.0000000
## S-R-CP4 25.0000000 10.000000 75.0000000  1.0000000
## S-A-TA4  2.1428571  0.000000  0.0000000  3.2857144
## B-A-MU1  1.6666666  0.000000  1.0000000 30.0000000
## B-A-GU1  0.0000000  0.000000  0.0000000  0.0000000
## B-R-PC2  4.0000000  1.000000  0.4000000  0.4000000
## B-A-MU2  2.4000001  0.000000  0.0000000 10.0000000
## B-A-GU2  0.0000000  0.000000  1.0000000  0.0000000
## B-R-PC3  3.3333333  0.000000  0.0000000  0.4444444
## B-A-MU3  0.0000000  0.000000  0.3333333 11.6666665
## B-A-GU3  5.5555556  0.000000 25.0000000  0.0000000
## B-R-PC4  2.0000000  1.250000  0.0000000  2.5000000
## B-A-MU4  0.0000000  5.000000 30.0000000 10.4444447
## B-A-GU4  1.4166667  0.000000 43.3333333  3.7500000
\end{verbatim}

Salvando a matriz final particionada de dados ambientais.

\begin{Shaded}
\begin{Highlighting}[]
\FunctionTok{write.table}\NormalTok{(m\_amb\_part, }\StringTok{"m\_amb\_part.csv"}\NormalTok{,}
            \AttributeTok{sep =} \StringTok{";"}\NormalTok{, }\AttributeTok{dec =} \StringTok{"."}\NormalTok{, }\CommentTok{\#"\textbackslash{}t",}
            \AttributeTok{row.names =} \ConstantTok{TRUE}\NormalTok{,}
            \AttributeTok{quote =} \ConstantTok{TRUE}\NormalTok{,}
            \AttributeTok{append =} \ConstantTok{FALSE}\NormalTok{)}
\NormalTok{m\_amb\_part }\OtherTok{\textless{}{-}} \FunctionTok{read.csv}\NormalTok{(}\StringTok{"m\_amb\_part.csv"}\NormalTok{,}
                  \AttributeTok{sep =} \StringTok{";"}\NormalTok{, }\AttributeTok{dec =} \StringTok{"."}\NormalTok{,}
                  \AttributeTok{row.names =} \DecValTok{1}\NormalTok{,}
                  \AttributeTok{header =} \ConstantTok{TRUE}\NormalTok{,}
                  \AttributeTok{na.strings =} \ConstantTok{NA}\NormalTok{)}
\end{Highlighting}
\end{Shaded}

\hypertarget{tabela-de-dados-ambientais}{%
\section{Tabela de dados ambientais}\label{tabela-de-dados-ambientais}}

Nessa etapa precisaremos da tabela de grupos. Selecionamos a aba \texttt{ambientalp}, porque nela constam os agrupamentos \emph{a-priori} para os dados ambientais.

\begin{Shaded}
\begin{Highlighting}[]
\FunctionTok{library}\NormalTok{(openxlsx)}
\NormalTok{t\_grps }\OtherTok{\textless{}{-}} \FunctionTok{read.xlsx}\NormalTok{(}\StringTok{"D:/Elvio/OneDrive/Disciplinas/\_EcoNumerica/5.Matrizes/ppbio06{-}grupos.xlsx"}\NormalTok{,}
                   \AttributeTok{rowNames =}\NormalTok{ T, }\AttributeTok{colNames =}\NormalTok{ T,}
                   \AttributeTok{sheet =} \StringTok{"ambientalp"}\NormalTok{)}
\FunctionTok{str}\NormalTok{(t\_grps)}
\CommentTok{\#View(ppbio)}
\NormalTok{t\_grps[}\DecValTok{1}\SpecialCharTok{:}\DecValTok{4}\NormalTok{,}\DecValTok{1}\SpecialCharTok{:}\DecValTok{4}\NormalTok{] }\CommentTok{\#[1:4,1:4] mostra apenas as linhas e colunas de 1 a 4.}
\end{Highlighting}
\end{Shaded}

\begin{verbatim}
## 'data.frame':    23 obs. of  4 variables:
##  $ area    : chr  "Serido" "Serido" "Serido" "Serido" ...
##  $ ambiente: chr  "rio" "rio" "acude" "rio" ...
##  $ UA      : chr  "CT" "CP" "TA" "CT" ...
##  $ coleta  : num  1 1 1 2 2 2 3 3 3 4 ...
##           area ambiente UA coleta
## S-R-CT1 Serido      rio CT      1
## S-R-CP1 Serido      rio CP      1
## S-A-TA1 Serido    acude TA      1
## S-R-CT2 Serido      rio CT      2
\end{verbatim}

\begin{Shaded}
\begin{Highlighting}[]
\FunctionTok{library}\NormalTok{(dplyr)}
\FunctionTok{library}\NormalTok{(tidyr)}
\NormalTok{m\_trab }\OtherTok{\textless{}{-}}\NormalTok{ m\_amb\_part }\SpecialCharTok{\%\textgreater{}\%}
  \FunctionTok{rename\_with}\NormalTok{(}\SpecialCharTok{\textasciitilde{}} \FunctionTok{gsub}\NormalTok{(}\StringTok{"\_"}\NormalTok{, }\StringTok{"."}\NormalTok{, .)) }\CommentTok{\#apenas troquei o underscore pelo ponto. }

\NormalTok{m }\OtherTok{\textless{}{-}}\NormalTok{ m\_trab }\SpecialCharTok{\%\textgreater{}\%}
  \FunctionTok{group\_by}\NormalTok{(}\AttributeTok{Area =}\NormalTok{ t\_grps}\SpecialCharTok{$}\NormalTok{area,}
           \AttributeTok{Habitat =}\NormalTok{ t\_grps}\SpecialCharTok{$}\NormalTok{ambiente,}
           \AttributeTok{Ponto =}\NormalTok{ t\_grps}\SpecialCharTok{$}\NormalTok{UA) }\SpecialCharTok{\%\textgreater{}\%}
  \FunctionTok{summarise}\NormalTok{(}\FunctionTok{across}\NormalTok{(}\FunctionTok{where}\NormalTok{(is.numeric),}
                   \FunctionTok{list}\NormalTok{(}\AttributeTok{mean =}\NormalTok{ mean, }\AttributeTok{min =}\NormalTok{ min, }\AttributeTok{max =}\NormalTok{ max)),}
            \AttributeTok{.groups =} \StringTok{\textquotesingle{}drop\textquotesingle{}}\NormalTok{) }\SpecialCharTok{\%\textgreater{}\%}
  \FunctionTok{pivot\_longer}\NormalTok{(}
    \AttributeTok{cols =} \SpecialCharTok{{-}}\FunctionTok{c}\NormalTok{(Area, Habitat, Ponto),}
    \AttributeTok{names\_to =} \FunctionTok{c}\NormalTok{(}\StringTok{"Variable"}\NormalTok{, }\StringTok{".value"}\NormalTok{),}
    \AttributeTok{names\_sep =} \StringTok{"\_"}
\NormalTok{  )}

\NormalTok{m }\OtherTok{\textless{}{-}} \FunctionTok{as.data.frame}\NormalTok{(m)}
\NormalTok{m\_wide }\OtherTok{\textless{}{-}}\NormalTok{ m }\SpecialCharTok{\%\textgreater{}\%}
  \FunctionTok{mutate}\NormalTok{(}\AttributeTok{stat\_string =} \FunctionTok{ifelse}\NormalTok{(Variable }\SpecialCharTok{==} \FunctionTok{c}\NormalTok{(}\StringTok{"fq.w.vel"}\NormalTok{),}
                              \FunctionTok{paste0}\NormalTok{(}\FunctionTok{round}\NormalTok{(mean, }\DecValTok{3}\NormalTok{), }\StringTok{"("}\NormalTok{, }\FunctionTok{round}\NormalTok{(min, }\DecValTok{3}\NormalTok{), }\StringTok{"{-}"}\NormalTok{, }\FunctionTok{round}\NormalTok{(max, }\DecValTok{3}\NormalTok{), }\StringTok{")"}\NormalTok{),}
                              \FunctionTok{paste0}\NormalTok{(}\FunctionTok{round}\NormalTok{(mean, }\DecValTok{1}\NormalTok{), }\StringTok{"("}\NormalTok{, }\FunctionTok{round}\NormalTok{(min, }\DecValTok{1}\NormalTok{), }\StringTok{"{-}"}\NormalTok{, }\FunctionTok{round}\NormalTok{(max, }\DecValTok{1}\NormalTok{), }\StringTok{")"}\NormalTok{))) }\SpecialCharTok{\%\textgreater{}\%}
  \FunctionTok{unite}\NormalTok{(}\StringTok{"Location"}\NormalTok{, Area, Habitat, Ponto, }\AttributeTok{sep =} \StringTok{"\_"}\NormalTok{) }\SpecialCharTok{\%\textgreater{}\%}
\NormalTok{  dplyr}\SpecialCharTok{::}\FunctionTok{select}\NormalTok{(Variable, Location, stat\_string) }\SpecialCharTok{\%\textgreater{}\%} \CommentTok{\# find("select"), dplyr::select evita conflito entre package:MASS  package:dplyr}
  \FunctionTok{pivot\_wider}\NormalTok{(}\AttributeTok{names\_from =}\NormalTok{ Location, }\AttributeTok{values\_from =}\NormalTok{ stat\_string)}

\NormalTok{m\_wide}
\NormalTok{m\_wide }\OtherTok{\textless{}{-}} \FunctionTok{as.data.frame}\NormalTok{(m\_wide)}
\NormalTok{m\_wide}

\CommentTok{\#Exportando dados para Excel{-}{-}{-}{-}}
\FunctionTok{library}\NormalTok{(openxlsx)}
\FunctionTok{write.xlsx}\NormalTok{(m\_wide, }\AttributeTok{file =} \StringTok{"tabela de habitat.xlsx"}\NormalTok{, }\AttributeTok{rowNames =} \ConstantTok{FALSE}\NormalTok{)}
\NormalTok{wb }\OtherTok{\textless{}{-}} \FunctionTok{loadWorkbook}\NormalTok{(}\StringTok{"tabela de habitat.xlsx"}\NormalTok{)}
\FunctionTok{writeData}\NormalTok{(wb, }\AttributeTok{sheet =} \StringTok{"Sheet 1"}\NormalTok{, }\AttributeTok{x =}\NormalTok{ m\_wide)}
\FunctionTok{saveWorkbook}\NormalTok{(wb, }\StringTok{"tabela de habitat.xlsx"}\NormalTok{, }\AttributeTok{overwrite =} \ConstantTok{TRUE}\NormalTok{)}

\CommentTok{\#Escolher sumário de uma variavel{-}{-}{-}{-}}
\NormalTok{m}
\NormalTok{var }\OtherTok{\textless{}{-}} \StringTok{"h.roots"}
\NormalTok{m[m}\SpecialCharTok{$}\NormalTok{Variable }\SpecialCharTok{==}\NormalTok{ var, }\StringTok{"mean"}\NormalTok{] }\CommentTok{\#cada valor de var}
\FunctionTok{summary}\NormalTok{(m[m}\SpecialCharTok{$}\NormalTok{Variable }\SpecialCharTok{==}\NormalTok{ var, }\StringTok{"mean"}\NormalTok{]) }\CommentTok{\#sumário dos valores de var}

\CommentTok{\#Escolher sumário de um grupo de variáveis do df m{-}{-}{-}{-}}
\NormalTok{vars }\OtherTok{\textless{}{-}} \FunctionTok{unique}\NormalTok{(}\FunctionTok{grep}\NormalTok{(}\StringTok{"\^{}h}\SpecialCharTok{\textbackslash{}\textbackslash{}}\StringTok{."}\NormalTok{, m}\SpecialCharTok{$}\NormalTok{Variable, }\AttributeTok{value =} \ConstantTok{TRUE}\NormalTok{))}
\NormalTok{summaries }\OtherTok{\textless{}{-}} \FunctionTok{list}\NormalTok{() }\CommentTok{\#criam uma lista vazia para guardar os sumários}
\CommentTok{\# Loop para cada variável do grupo e guarda em summaries}
\ControlFlowTok{for}\NormalTok{ (var }\ControlFlowTok{in}\NormalTok{ vars) \{}
\NormalTok{  summaries[[var]] }\OtherTok{\textless{}{-}} \FunctionTok{summary}\NormalTok{(m[m}\SpecialCharTok{$}\NormalTok{Variable }\SpecialCharTok{==}\NormalTok{ var, }\StringTok{"mean"}\NormalTok{])}
\NormalTok{\}}
\CommentTok{\# var is a temporary variable used in the for loop to iterate through}
\CommentTok{\# each variable name that starts with "h."}

\NormalTok{summaries}
\NormalTok{summary\_table }\OtherTok{\textless{}{-}} \FunctionTok{do.call}\NormalTok{(rbind, }\FunctionTok{lapply}\NormalTok{(summaries, as.data.frame.list))}
\FunctionTok{round}\NormalTok{(summary\_table, }\DecValTok{2}\NormalTok{)}
\CommentTok{\#sink(file = "summary\_h.txt", split = TRUE)}
\FunctionTok{round}\NormalTok{(summary\_table[}\FunctionTok{order}\NormalTok{(summary\_table}\SpecialCharTok{$}\NormalTok{Mean, }\AttributeTok{decreasing =} \ConstantTok{FALSE}\NormalTok{), ], }\DecValTok{2}\NormalTok{)}
\CommentTok{\#sink()}

\CommentTok{\#Tabela limpa}
\NormalTok{summary\_table }\OtherTok{\textless{}{-}} \FunctionTok{cbind}\NormalTok{(}\AttributeTok{Variable =} \FunctionTok{rownames}\NormalTok{(summary\_table), summary\_table)}
\FunctionTok{rownames}\NormalTok{(summary\_table) }\OtherTok{\textless{}{-}} \ConstantTok{NULL}
\FunctionTok{colnames}\NormalTok{(summary\_table) }\OtherTok{\textless{}{-}} \FunctionTok{c}\NormalTok{(}\StringTok{"Variable"}\NormalTok{, }\StringTok{"Min"}\NormalTok{, }\StringTok{"Q1"}\NormalTok{, }\StringTok{"Median"}\NormalTok{, }\StringTok{"Mean"}\NormalTok{, }\StringTok{"Q3"}\NormalTok{, }\StringTok{"Max"}\NormalTok{)}
\NormalTok{summary\_table}
\end{Highlighting}
\end{Shaded}

\hypertarget{apuxeandices-10}{%
\section*{Apêndices}\label{apuxeandices-10}}
\addcontentsline{toc}{section}{Apêndices}

\hypertarget{sites-consultados-8}{%
\section*{Sites consultados}\label{sites-consultados-8}}
\addcontentsline{toc}{section}{Sites consultados}

\hypertarget{script-limpo-7}{%
\section*{Script limpo}\label{script-limpo-7}}
\addcontentsline{toc}{section}{Script limpo}

Aqui apresento o scrip na íntegra sem os textos ou outros comentários. Você pode copiar e colar no R para executa-lo. Lembre de remover os {\texttt{\#}} ou {\texttt{\#\#}} caso necessite executar essas linhas.

\hypertarget{referuxeancias-11}{%
\section*{Referências}\label{referuxeancias-11}}
\addcontentsline{toc}{section}{Referências}

\hypertarget{reltrns}{%
\chapter{R Modulo 5 - Relativizações e transformações}\label{reltrns}}

\hypertarget{resumo-11}{%
\subsection*{RESUMO}\label{resumo-11}}
\addcontentsline{toc}{subsection}{RESUMO}

Transformações e relativizações de dados são técnicas essenciais na análise de dados em Ecologia Numérica, permitindo comparar diferentes variáveis que estão em escalas diferentes ou que possuem unidades distintas.

\hypertarget{apresentauxe7uxe3o-11}{%
\subsection*{Apresentação}\label{apresentauxe7uxe3o-11}}
\addcontentsline{toc}{subsection}{Apresentação}

O R é uma ferramenta poderosa para realização dessas operações, com diversos pacotes disponíveis para essa finalidade. Neste tutorial, iremos explorar algumas das principais funções do pacote vegan, que permite realizar transformações e relativizações de dados de forma simples e eficiente. Além disso, também utilizaremos outras bibliotecas importantes para reorganização e visualozação de dados. O tutorial irá abordar desde as operações básicas, como filtragem e seleção de dados, até as transformações mais complexas, como a normalização de dados e a relativização por medidas de biomassa ou área. Com exemplos práticos e ilustrações gráficas, o objetivo é permitir que os usuários do R possam aplicar essas técnicas em suas próprias análises e estudos de Ecologia Numérica. Ao final do tutorial, espera-se que o usuário esteja apto a realizar transformações e relativizações de dados em suas próprias análises, aumentando a qualidade e a precisão dos resultados obtidos.

\hypertarget{organizauxe7uxe3o-buxe1sica-8}{%
\section{Organização básica}\label{organizauxe7uxe3o-buxe1sica-8}}

\begin{Shaded}
\begin{Highlighting}[]
\FunctionTok{dev.off}\NormalTok{() }\CommentTok{\#apaga os graficos, se houver algum}
\FunctionTok{rm}\NormalTok{(}\AttributeTok{list=}\FunctionTok{ls}\NormalTok{(}\AttributeTok{all=}\ConstantTok{TRUE}\NormalTok{)) }\CommentTok{\#limpa a memória}
\FunctionTok{cat}\NormalTok{(}\StringTok{"}\SpecialCharTok{\textbackslash{}014}\StringTok{"}\NormalTok{) }\CommentTok{\#limpa o console }
\end{Highlighting}
\end{Shaded}

Instalando os pacotes necessários para esse módulo

\begin{Shaded}
\begin{Highlighting}[]
\FunctionTok{install.packages}\NormalTok{(}\StringTok{"openxlsx"}\NormalTok{) }\CommentTok{\#importa arquivos do excel}
\FunctionTok{install.packages}\NormalTok{(}\StringTok{"moments"}\NormalTok{) }\CommentTok{\#calcula assimetria e curtose dos dados}
\FunctionTok{install.packages}\NormalTok{(}\StringTok{"vegan"}\NormalTok{) }\CommentTok{\#estatisticas para ecologia de comunidades}
\end{Highlighting}
\end{Shaded}

Agora vamos \textbf{definir o diretório de trabalho}. Esse código é usado para obter e definir o diretório de trabalho atual no R. O comando \texttt{getwd()} retorna o caminho do diretório onde o R está lendo e salvando arquivos. O comando \texttt{setwd()} muda esse diretório de trabalho para o caminho especificado entre aspas. No seu caso, você deve ajustar o caminho para o seu próprio diretório de trabalho. \textbf{Lembre de usar a barra ``/'' entre os diretórios. E não a contra-barra ``\textbackslash{}''.}

Definindo o diretório de trabalho e installando os pacotes necessários:

\begin{Shaded}
\begin{Highlighting}[]
\FunctionTok{getwd}\NormalTok{()}
\FunctionTok{setwd}\NormalTok{(}\StringTok{"C:/Seu/Diretório/De/Trabalho"}\NormalTok{)}
\end{Highlighting}
\end{Shaded}

\hypertarget{importando-a-planilha-7}{%
\subsection{Importando a planilha}\label{importando-a-planilha-7}}

Note que o símbolo {\texttt{\#}} em programação R significa que o texto que vem depois dele é um comentário e não será executado pelo programa. Isso é útil para explicar o código ou deixar anotações.\\
- Ajuste a primeira linha do código abaixo para refletir {``C:/Seu/Diretório/De/Trabalho/Planilha.xlsx''}.\\
- Ajuste o parâmetro \texttt{sheet\ =\ "Sheet1"} para refletir a aba correta do arquivo .xlsx a ser importado.

\begin{Shaded}
\begin{Highlighting}[]
\FunctionTok{library}\NormalTok{(openxlsx)}
\CommentTok{\#dir \textless{}{-} getwd() \#criamos um vetor com o diretório  de trbalho }
\CommentTok{\#shell.exec(dir) \#abre o diretorio de trabalho no Windows Explorer}
\NormalTok{m\_bruta }\OtherTok{\textless{}{-}} \FunctionTok{read.xlsx}\NormalTok{(}\StringTok{"D:/Elvio/OneDrive/Disciplinas/\_EcoNumerica/5.Matrizes/ppbio06{-}peixes.xlsx"}\NormalTok{,}
                   \AttributeTok{rowNames =}\NormalTok{ T, }\AttributeTok{colNames =}\NormalTok{ T,}
                   \AttributeTok{sheet =} \StringTok{"Sheet1"}\NormalTok{)}
\FunctionTok{str}\NormalTok{(m\_bruta)}
\CommentTok{\#View(m\_bruta)}
\NormalTok{m\_bruta[}\DecValTok{1}\SpecialCharTok{:}\DecValTok{5}\NormalTok{,}\DecValTok{1}\SpecialCharTok{:}\DecValTok{5}\NormalTok{] }\CommentTok{\#[1:5,1:5] mostra apenas as linhas e colunas de 1 a 5.}
\end{Highlighting}
\end{Shaded}

\begin{verbatim}
## 'data.frame':    26 obs. of  35 variables:
##  $ ap-davis : num  0 0 0 0 0 0 0 0 0 0 ...
##  $ as-bimac : num  1 99 194 19 23 142 5 46 206 16 ...
##  $ as-fasci : num  0 0 55 0 1 3 1 0 64 0 ...
##  $ ch-bimac : num  0 0 0 0 13 3 0 178 0 0 ...
##  $ ci-ocela : num  0 0 0 0 0 0 40 0 0 13 ...
##  $ ci-orien : num  0 0 5 0 0 69 9 0 25 24 ...
##  $ co-macro : num  0 0 0 0 0 0 0 0 0 0 ...
##  $ co-heter : num  0 0 1 0 0 0 0 0 0 0 ...
##  $ cr-menez : num  0 0 14 0 0 4 0 0 8 0 ...
##  $ cu-lepid : num  0 0 0 0 0 0 0 0 0 0 ...
##  $ cy-gilbe : num  0 0 0 0 0 0 0 0 0 0 ...
##  $ ge-brasi : num  0 0 3 0 0 0 0 0 1 0 ...
##  $ he-margi : num  0 0 0 0 0 1 0 0 0 0 ...
##  $ ho-malab : num  0 0 1 5 0 17 10 2 31 4 ...
##  $ hy-pusar : num  0 0 9 2 0 43 2 0 11 0 ...
##  $ le-melan : num  0 0 0 0 0 0 0 0 0 0 ...
##  $ le-piau  : num  0 0 3 0 0 1 3 0 2 1 ...
##  $ le-taeni : num  0 0 0 0 0 0 0 0 0 0 ...
##  $ mo-costa : num  0 0 0 0 0 0 0 0 0 0 ...
##  $ mo-lepid : num  0 1 39 0 0 1 0 0 0 0 ...
##  $ or-nilot : num  0 2 36 0 0 77 0 0 138 0 ...
##  $ pa-manag : num  0 0 0 0 0 0 0 0 0 0 ...
##  $ pimel-sp : num  0 0 6 0 0 0 0 0 0 0 ...
##  $ po-retic : num  0 0 0 0 0 20 0 0 5 0 ...
##  $ po-vivip : num  0 0 47 15 0 221 32 0 326 10 ...
##  $ pr-brevi : num  9 0 5 0 1 15 5 2 164 0 ...
##  $ ps-rhomb : num  0 0 0 0 0 0 0 0 1 0 ...
##  $ ps-genise: num  0 0 0 0 0 0 0 0 1 0 ...
##  $ se-heter : num  0 0 40 14 4 60 0 0 38 0 ...
##  $ se-piaba : num  0 0 68 0 0 0 0 0 0 0 ...
##  $ se-spilo : num  0 0 0 0 0 0 0 0 1 0 ...
##  $ st-noton : num  0 0 1 0 0 25 0 0 115 0 ...
##  $ sy-marmo : num  0 0 0 0 0 0 1 0 0 0 ...
##  $ te-chalc : num  0 0 0 0 0 0 0 0 0 0 ...
##  $ tr-signa : num  0 0 18 0 0 15 0 0 7 0 ...
##         ap-davis as-bimac as-fasci ch-bimac ci-ocela
## S-A-ZA1        0        1        0        0        0
## S-R-CC1        0       99        0        0        0
## S-R-CT1        0      194       55        0        0
## S-R-CP1        0       19        0        0        0
## S-A-TA1        0       23        1       13        0
\end{verbatim}

Exibindo os dados importados (esses comando são ``case-sensitive'' \texttt{ignore.case(object)}).

\begin{Shaded}
\begin{Highlighting}[]
\CommentTok{\#View(m\_bruta)}
\FunctionTok{print}\NormalTok{(m\_bruta[}\DecValTok{1}\SpecialCharTok{:}\DecValTok{8}\NormalTok{,}\DecValTok{1}\SpecialCharTok{:}\DecValTok{8}\NormalTok{])}
\NormalTok{m\_bruta[}\DecValTok{1}\SpecialCharTok{:}\DecValTok{10}\NormalTok{,}\DecValTok{1}\SpecialCharTok{:}\DecValTok{10}\NormalTok{]}
\FunctionTok{str}\NormalTok{(m\_bruta)}
\FunctionTok{mode}\NormalTok{(m\_bruta)}
\FunctionTok{class}\NormalTok{(m\_bruta)}
\end{Highlighting}
\end{Shaded}

Podemos exibir a planilha depois de ter sido importada para o ambiente R/RStudio usando as funções \texttt{View()}, \texttt{print()} ou \texttt{head()}. Note que essas funções são case-sensitive. A função \texttt{ignore.case()} é uma função do pacote \texttt{stringr} que modifica um padrão para que ele não considere o caso das letras nas correspondências. Por exemplo, se você quiser encontrar todas as ocorrências da letra ``a'' em um vetor de caracteres, independente de ser ``A'' ou ``a'', você pode usar essa função.

\begin{Shaded}
\begin{Highlighting}[]
\CommentTok{\#View(m\_bruta)}
\FunctionTok{print}\NormalTok{(m\_bruta)}
\FunctionTok{head}\NormalTok{(m\_bruta)}
\end{Highlighting}
\end{Shaded}

A função \texttt{head()} no RStudio é uma forma de ver as primeiras \texttt{(n=6)} linhas de um objeto, como um vetor, uma matriz, um data frame ou uma lista. Ela é útil para ter uma ideia do conteúdo e da estrutura do objeto.

Também podemos explorar as características da planilha usando as funções \texttt{str()}, \texttt{mode()}, \texttt{class()} e \texttt{length()}. O número de observações ou tamanho do vetor depende do tipo de dados, se eles são uma matrix ou um data.frame.

\hypertarget{abreviauxe7uxf5es}{%
\subsubsection*{Abreviações}\label{abreviauxe7uxf5es}}
\addcontentsline{toc}{subsubsection}{Abreviações}

No interesse de sistematizar o código R das várias matrizes que são comumente usadas em uma AMD, a tabela \ref{tab:5tblm-2}, a seguir, resume seus tipos e abreviações.

\begin{verbatim}
## Warning in table_info$align_vector[column] <-
## unlist(lapply(table_info$align_vector_origin[column], : número de itens para
## para substituir não é um múltiplo do comprimento do substituto
\end{verbatim}

\begin{table}

\caption{\label{tab:5tblm-2}Nomenclatura das matrizes em AMD em relação aos atributos das colunas}
\centering
\begin{tabular}[t]{>{\raggedright\arraybackslash}p{2cm}>{\raggedright\arraybackslash}p{5cm}>{\raggedright\arraybackslash}p{2cm}}
\toprule
Nome & Atributos (colunas) & Abreviação no R\\
\midrule
Matriz comunitaria & Os atributos são táxons ou OTU's (Unidades Taxonômicas Operacionais) (ex. espécies, gêneros, morfotipos) & m\_com\\
Matriz ambiental & Os atributos são dados ambientais e variáveis  físicas e químicas (ex. pH, condutividade, temperatura) & m\_amb\\
Matriz de habitat & Os atributos são elementos da estrutura do habitat (ex. macróficas, algas, pedras, lama, etc) & m\_hab\\
Matriz bruta & Os atributos ainda não receberam nenhum tipo de tratamento estatísco (valores brutos, como coletados) & m\_bruta\\
Matriz transposta & Os atributos foram transpostos para as linhas & m\_t\\
\addlinespace
Matriz relativizada & Os atributos foram relativizados por um critério de tamanho ou de variação (ex. dividir os valores de cada coluna pela soma) & m\_rel, m\_relcol, m\_rellin\\
Matriz transformada & Foi aplicado um operador matemático a todos os atributos (ex. raiz quadrada, log) & m\_trns, m\_log10, m\_asrq\\
Matriz de distâncias & Matriz de m x m similaridades ou de distâncias (ex. Euclidiana, Manhattan, Bray-Curtis, etc) & m\_dists, m\_euclid, m\_bray\\
Matriz de trabalho & Qualquer matriz que seja o foco da análise atual (ex. comunitária, relativizada, etc) & m\_trab\\
Matriz particionada & Foram removidas linhas ou colunas (ex. linhas que são outliers e espécies zeradas) & m\_part\\
\addlinespace
Base de dados & Arquivo do Excel planilhado a partir de dados de campo ou de laboratório. Será manejada e particionada no R, para criar a Matriz bruta & ppbio06.xlsx, zoorebio.xlsx, bentos06.xlsx\\
\bottomrule
\end{tabular}
\end{table}

\hypertarget{tamanho-da-matriz}{%
\section{Tamanho da matriz}\label{tamanho-da-matriz}}

Podemos agora calcular o número e a proporção de zeros na matriz usando as funções \texttt{sum()} e \texttt{length()} (Você pode pesquisar o que faz a função \texttt{length()} usando o comando \texttt{?length}).

\begin{Shaded}
\begin{Highlighting}[]
\FunctionTok{range}\NormalTok{(m\_bruta) }\CommentTok{\#menor e maior valores}
\FunctionTok{length}\NormalTok{(m\_bruta) }\CommentTok{\#no. de colunas}
\FunctionTok{ncol}\NormalTok{(m\_bruta) }\CommentTok{\#no. de N colunas}
\FunctionTok{nrow}\NormalTok{(m\_bruta) }\CommentTok{\#no. de M linhas}
\FunctionTok{sum}\NormalTok{(}\FunctionTok{lengths}\NormalTok{(m\_bruta)) }\CommentTok{\#soma os nos. de colunas}
\FunctionTok{length}\NormalTok{(}\FunctionTok{as.matrix}\NormalTok{(m\_bruta)) }\CommentTok{\#tamanho da matriz m x n}
\FunctionTok{sum}\NormalTok{(m\_bruta }\SpecialCharTok{==} \DecValTok{0}\NormalTok{) }\CommentTok{\# número de observações igual a zero}
\FunctionTok{sum}\NormalTok{(m\_bruta }\SpecialCharTok{\textgreater{}} \DecValTok{0}\NormalTok{) }\CommentTok{\# número de observações maiores que zero}
\NormalTok{zeros }\OtherTok{\textless{}{-}}\NormalTok{ (}\FunctionTok{sum}\NormalTok{(m\_bruta }\SpecialCharTok{==} \DecValTok{0}\NormalTok{)}\SpecialCharTok{/}\FunctionTok{length}\NormalTok{(}\FunctionTok{as.matrix}\NormalTok{(m\_bruta)))}\SpecialCharTok{*}\DecValTok{100} \CommentTok{\# proporção de zeros na matriz}
\NormalTok{zeros}
\end{Highlighting}
\end{Shaded}

\begin{verbatim}
## [1]   0 511
## [1] 35
## [1] 35
## [1] 26
## [1] 910
## [1] 910
## [1] 716
## [1] 194
## [1] 78.68132
\end{verbatim}

Tabela que resume as informações geradas (\ref{tab:tbl5-tam}).

\begin{Shaded}
\begin{Highlighting}[]
\NormalTok{tamanho }\OtherTok{\textless{}{-}} \FunctionTok{data.frame}\NormalTok{(}
  \AttributeTok{Comando =} \FunctionTok{c}\NormalTok{(}\StringTok{"range"}\NormalTok{, }\StringTok{"lenght"}\NormalTok{, }\StringTok{"m cols"}\NormalTok{, }\StringTok{"n linhas"}\NormalTok{, }\StringTok{"Tamanho"}\NormalTok{, }\StringTok{"Tamanho"}\NormalTok{,}
             \StringTok{"Zeros"}\NormalTok{, }\StringTok{"Nao zeros"}\NormalTok{, }\StringTok{"\% Zeros"}\NormalTok{),}
  \AttributeTok{Resultado =} \FunctionTok{c}\NormalTok{(}\FunctionTok{paste}\NormalTok{(}\FunctionTok{range}\NormalTok{(m\_bruta), }\AttributeTok{collapse =} \StringTok{" {-} "}\NormalTok{), }\FunctionTok{length}\NormalTok{(m\_bruta), }\FunctionTok{ncol}\NormalTok{(m\_bruta),}
                \FunctionTok{nrow}\NormalTok{(m\_bruta), }\FunctionTok{sum}\NormalTok{(}\FunctionTok{lengths}\NormalTok{(m\_bruta)), }\FunctionTok{length}\NormalTok{(}\FunctionTok{as.matrix}\NormalTok{(m\_bruta)), }\FunctionTok{sum}\NormalTok{(m\_bruta }\SpecialCharTok{==} \DecValTok{0}\NormalTok{),}
                \FunctionTok{sum}\NormalTok{(m\_bruta }\SpecialCharTok{\textgreater{}} \DecValTok{0}\NormalTok{), }\FunctionTok{round}\NormalTok{(zeros, }\DecValTok{1}\NormalTok{)))}
\NormalTok{tamanho}
\end{Highlighting}
\end{Shaded}

\begin{verbatim}
##     Comando Resultado
## 1     range   0 - 511
## 2    lenght        35
## 3    m cols        35
## 4  n linhas        26
## 5   Tamanho       910
## 6   Tamanho       910
## 7     Zeros       716
## 8 Nao zeros       194
## 9   % Zeros      78.7
\end{verbatim}

\begin{table}

\caption{\label{tab:tbl5-tam}Resumo das informações geradas.}
\centering
\begin{tabular}[t]{l|l}
\hline
Comando & Resultado\\
\hline
range & 0 - 511\\
\hline
lenght & 35\\
\hline
m cols & 35\\
\hline
n linhas & 26\\
\hline
Tamanho & 910\\
\hline
Tamanho & 910\\
\hline
Zeros & 716\\
\hline
Nao zeros & 194\\
\hline
\% Zeros & 78.7\\
\hline
\end{tabular}
\end{table}

Ou seja, temos uma matriz de tamanho \texttt{m\ x\ n} igual a 26 objetos por 35 atributos, onde 78.68\% dos valores da matriz são iguais a zero!

\hypertarget{reset-point-1-1}{%
\section{Reset point 1}\label{reset-point-1-1}}

\begin{Shaded}
\begin{Highlighting}[]
\NormalTok{m\_trab }\OtherTok{\textless{}{-}}\NormalTok{ m\_bruta}
\end{Highlighting}
\end{Shaded}

Aqui cria-se um novo objeto do R (\texttt{m\_trab}, ou a matriz de trabalho, para esse momento) que substitui a matriz de dados original, por uma nova matriz que pode ser a matriz relativizada, transformada, transposta, etc. Dessa forma, mantemos a matriz de dados original caso precisemos dela novamente (Veja a Tabela Tabela @ref(tab:5tblm\_2)).

\hypertarget{cuxe1lculo-da-matriz-de-distuxe2ncias}{%
\section{Cálculo da matriz de distâncias}\label{cuxe1lculo-da-matriz-de-distuxe2ncias}}

Agora vamos calcular a matriz de distâncias euclidiana (LEGENDRE; LEGENDRE (\protect\hyperlink{ref-RN1278}{1998}), BORCARD; GILLET; LEGENDRE (\protect\hyperlink{ref-RN1336}{2018})) entre as UA´s usando a função \texttt{dist()} (pesquise o que faz essa função usando o comando \texttt{?dist}).

\begin{Shaded}
\begin{Highlighting}[]
\NormalTok{m\_dists }\OtherTok{\textless{}{-}} \FunctionTok{dist}\NormalTok{(m\_trab, }\AttributeTok{method =} \StringTok{"euclidian"}\NormalTok{, }\AttributeTok{diag =} \ConstantTok{TRUE}\NormalTok{, }\AttributeTok{upper =} \ConstantTok{FALSE}\NormalTok{)}
\end{Highlighting}
\end{Shaded}

Pronto, calculamos a matriz de distâncias (euclidiana). Agora podemos visualizar a matriz.

\begin{Shaded}
\begin{Highlighting}[]
\CommentTok{\#m\_dists}
\FunctionTok{str}\NormalTok{(m\_dists)}
\FunctionTok{mode}\NormalTok{(m\_dists)}
\FunctionTok{class}\NormalTok{(m\_dists)}
\FunctionTok{length}\NormalTok{(}\FunctionTok{as.matrix}\NormalTok{(m\_dists))}
\FunctionTok{as.matrix}\NormalTok{(m\_dists)[}\DecValTok{1}\SpecialCharTok{:}\DecValTok{10}\NormalTok{, }\DecValTok{1}\SpecialCharTok{:}\DecValTok{10}\NormalTok{]}
\end{Highlighting}
\end{Shaded}

\begin{verbatim}
##  'dist' num [1:325] 98.4 228.5 29.2 27.1 294.1 ...
##  - attr(*, "Size")= int 26
##  - attr(*, "Labels")= chr [1:26] "S-A-ZA1" "S-R-CC1" "S-R-CT1" "S-R-CP1" ...
##  - attr(*, "Diag")= logi TRUE
##  - attr(*, "Upper")= logi FALSE
##  - attr(*, "method")= chr "euclidean"
##  - attr(*, "call")= language dist(x = m_trab, method = "euclidian", diag = TRUE, upper = FALSE)
## [1] "numeric"
## [1] "dist"
## [1] 676
##           S-A-ZA1   S-R-CC1  S-R-CT1   S-R-CP1   S-A-TA1  S-R-CT2   S-R-CP2
## S-A-ZA1   0.00000  98.43780 228.5235  29.24038  27.09243 294.0629  53.40412
## S-R-CC1  98.43780   0.00000 154.2433  82.79493  77.25283 261.3905 108.10180
## S-R-CT1 228.52352 154.24331   0.0000 208.52338 209.69263 224.9133 224.09150
## S-R-CP1  29.24038  82.79493 208.5234   0.00000  23.25941 271.4922  49.22398
## S-A-TA1  27.09243  77.25283 209.6926  23.25941   0.00000 283.7869  57.82733
## S-R-CT2 294.06292 261.39051 224.9133 271.49217 283.78689   0.0000 268.95167
## S-R-CP2  53.40412 108.10180 224.0915  49.22398  57.82733 268.9517   0.00000
## S-A-TA2 183.74439 185.75791 261.8301 181.24845 166.66133 326.3327 190.15257
## S-R-CT3 460.42155 428.64437 366.6429 443.90877 453.59674 238.3799 438.54532
## S-R-CP3  34.17601  88.06816 215.2557  31.32092  33.13608 275.0436  40.37326
##          S-A-TA2  S-R-CT3   S-R-CP3
## S-A-ZA1 183.7444 460.4215  34.17601
## S-R-CC1 185.7579 428.6444  88.06816
## S-R-CT1 261.8301 366.6429 215.25566
## S-R-CP1 181.2484 443.9088  31.32092
## S-A-TA1 166.6613 453.5967  33.13608
## S-R-CT2 326.3327 238.3799 275.04363
## S-R-CP2 190.1526 438.5453  40.37326
## S-A-TA2   0.0000 478.9008 182.86060
## S-R-CT3 478.9008   0.0000 449.24826
## S-R-CP3 182.8606 449.2483   0.00000
\end{verbatim}

Uma matriz do tipo \texttt{dist} no R é um objeto que armazena as distâncias entre as linhas de uma matriz ou um data frame. Ela é criada pela função \texttt{dist()}, que calcula as distâncias usando diferentes medidas, como ``euclidean'', ``manhattan'', ``canberra'', ``binary'' ou ``minkowski'' (\protect\hyperlink{ref-RN419}{HORTON; KLEINMAN, 2015}).

\begin{Shaded}
\begin{Highlighting}[]
\FunctionTok{range}\NormalTok{(m\_trab)}
\FunctionTok{range}\NormalTok{(m\_dists)}
\FunctionTok{min}\NormalTok{(m\_dists)}
\FunctionTok{max}\NormalTok{(m\_dists)}
\FunctionTok{mean}\NormalTok{(m\_dists) }\CommentTok{\#CENTROIDE!! ou Grand mean}
\FunctionTok{sd}\NormalTok{(m\_dists)  }\CommentTok{\#standard deviation}
\NormalTok{centroide }\OtherTok{\textless{}{-}} \FunctionTok{mean}\NormalTok{(m\_dists)}
\NormalTok{centroide}
\end{Highlighting}
\end{Shaded}

\begin{verbatim}
## [1]   0 511
## [1]  16.27882 707.32666
## [1] 16.27882
## [1] 707.3267
## [1] 285.9043
## [1] 172.9708
## [1] 285.9043
\end{verbatim}

A função \texttt{mean()} calcula a média de todos os valores da matriz de distâncias, ou seja, a média multivariada, que é o centróide. Nesse caso o \textbf{centroide assume o valor de 285.9}. Usamos agora a fórmula \texttt{m*(m-1)/2}, onde \texttt{m} é o no. de objetos sendo comparados, para calcular quantas distâcias temos na nossa matriz.

\begin{Shaded}
\begin{Highlighting}[]
\FunctionTok{length}\NormalTok{(m\_dists)}
\NormalTok{m }\OtherTok{\textless{}{-}} \FunctionTok{nrow}\NormalTok{(}\FunctionTok{as.matrix}\NormalTok{(m\_dists))}
\NormalTok{m}
\NormalTok{m}\SpecialCharTok{*}\NormalTok{(m}\DecValTok{{-}1}\NormalTok{)}\SpecialCharTok{/}\DecValTok{2}
\FunctionTok{summary}\NormalTok{(m\_dists)}
\end{Highlighting}
\end{Shaded}

\begin{verbatim}
## [1] 325
## [1] 26
## [1] 325
##    Min. 1st Qu.  Median    Mean 3rd Qu.    Max. 
##   16.28  142.70  242.82  285.90  423.64  707.33
\end{verbatim}

Temos então que m é 26 objetos (ou linhas), e portanto, a matriz de distâncias tem 325 valores. Fazemos agora um breve sumário do que foi calculado até agora com base na matriz de distâncias.

\begin{Shaded}
\begin{Highlighting}[]
\NormalTok{Sumario1 }\OtherTok{\textless{}{-}} \FunctionTok{cbind}\NormalTok{(}\FunctionTok{min}\NormalTok{(m\_dists),}
                  \FunctionTok{max}\NormalTok{(m\_dists),}
                  \FunctionTok{sd}\NormalTok{(m\_dists),}
                  \FunctionTok{mean}\NormalTok{(m\_dists),}
                  \FunctionTok{length}\NormalTok{(m\_dists))}
\FunctionTok{colnames}\NormalTok{(Sumario1) }\OtherTok{\textless{}{-}} \FunctionTok{c}\NormalTok{(}\StringTok{"Minimo"}\NormalTok{, }\StringTok{"Maximo"}\NormalTok{, }\StringTok{"Desv.Padr"}\NormalTok{, }\StringTok{"Media"}\NormalTok{, }\StringTok{"m(m{-}1)/2"}\NormalTok{)}
\FunctionTok{rownames}\NormalTok{(Sumario1) }\OtherTok{\textless{}{-}}\NormalTok{ (}\StringTok{"Valores"}\NormalTok{)}
\NormalTok{Sumario1}
\end{Highlighting}
\end{Shaded}

\begin{verbatim}
##           Minimo   Maximo Desv.Padr    Media m(m-1)/2
## Valores 16.27882 707.3267  172.9708 285.9043      325
\end{verbatim}

A matriz \texttt{m\_dists} está como uma classe \texttt{dist}, mas podemos transformá-la em uma classe \texttt{matrix} usando a função \texttt{as.matrix()}.

\begin{Shaded}
\begin{Highlighting}[]
\NormalTok{dist }\OtherTok{\textless{}{-}}\NormalTok{ (}\FunctionTok{as.matrix}\NormalTok{(m\_dists))}
\CommentTok{\#View(dist)}
\FunctionTok{str}\NormalTok{(dist)}
\FunctionTok{mode}\NormalTok{(dist)}
\FunctionTok{class}\NormalTok{(dist)}
\CommentTok{\#head(dist)}
\NormalTok{dist[}\DecValTok{1}\SpecialCharTok{:}\DecValTok{5}\NormalTok{, }\DecValTok{1}\SpecialCharTok{:}\DecValTok{5}\NormalTok{] }\CommentTok{\#mostra as 5 primeiras linhas e colunas da matriz}
\end{Highlighting}
\end{Shaded}

\begin{verbatim}
##  num [1:26, 1:26] 0 98.4 228.5 29.2 27.1 ...
##  - attr(*, "dimnames")=List of 2
##   ..$ : chr [1:26] "S-A-ZA1" "S-R-CC1" "S-R-CT1" "S-R-CP1" ...
##   ..$ : chr [1:26] "S-A-ZA1" "S-R-CC1" "S-R-CT1" "S-R-CP1" ...
## [1] "numeric"
## [1] "matrix" "array" 
##           S-A-ZA1   S-R-CC1  S-R-CT1   S-R-CP1   S-A-TA1
## S-A-ZA1   0.00000  98.43780 228.5235  29.24038  27.09243
## S-R-CC1  98.43780   0.00000 154.2433  82.79493  77.25283
## S-R-CT1 228.52352 154.24331   0.0000 208.52338 209.69263
## S-R-CP1  29.24038  82.79493 208.5234   0.00000  23.25941
## S-A-TA1  27.09243  77.25283 209.6926  23.25941   0.00000
\end{verbatim}

Visualizando a matriz de distâncias, observamos que ela é uma matriz quadrada que contém as distâncias entre cada par de elementos do conjuto de dados. A matriz de distâncias terá dimensão m x m, e cada elemento da matriz será a distância entre cada par de observações ou objetos. A matriz de distâncias é simétrica, pois a distância entre i e j é igual à distância entre j e i. A matriz de distâncias também tem diagonal zero, pois a distância entre uma observação e ela mesma é zero.

\hypertarget{distribuiuxe7uxe3o-de-frequuxeancia-da-matriz-de-distuxe2ncias}{%
\section{Distribuição de frequência da matriz de distâncias}\label{distribuiuxe7uxe3o-de-frequuxeancia-da-matriz-de-distuxe2ncias}}

Por fim, vamos produzir uma distribuição de frequência da matriz de distâncias, mas antes vamos explorar alguns parâmetros básicos da matriz de distâncias, como o intervalo, a média e o desvio padrão. Para isso, usamos as funções \texttt{range()}, \texttt{mean()} e \texttt{sd()} do R.

\hypertarget{calculando-alguns-paruxe2metros-buxe1sicos}{%
\subsection{Calculando alguns parâmetros básicos}\label{calculando-alguns-paruxe2metros-buxe1sicos}}

\begin{Shaded}
\begin{Highlighting}[]
\FunctionTok{range}\NormalTok{(m\_dists) }\CommentTok{\# intervalo dos valores da matriz}
\FunctionTok{mean}\NormalTok{(m\_dists) }\CommentTok{\# média dos valores da matriz}
\FunctionTok{sd}\NormalTok{(m\_dists) }\CommentTok{\# desvio padrão dos valores da matriz}
\end{Highlighting}
\end{Shaded}

\begin{verbatim}
## [1]  16.27882 707.32666
## [1] 285.9043
## [1] 172.9708
\end{verbatim}

Esses comandos mostram que o \texttt{menor,\ maior} valores na matriz de distância foram 16.28, 707.33, enquanto que a média foi de 285.9 com um desvio padrão de 172.97.

\leavevmode\vadjust pre{\hypertarget{hello}{}}%
{IMPORTANTE}
Lembre que no R e RStudio, a vírgula é usada como separador de decimal e o ponto é apenas indicador de milhar. Por exemplo, 2.500,50 lê-se dois mil e quinhentos e 50 décimos.

Também podemos calcular a assimetria e a curtose da matriz de distâncias, que são medidas que indicam o grau de desvio da normalidade da distribuição. Para isso, precisamos carregar o pacote \texttt{moments}, que contém as funções \texttt{skewness()} (assimetria) e \texttt{kurtosis()} (curtose).

\hypertarget{calculando-a-assimetria-e-a-curtose}{%
\subsection{Calculando a assimetria e a curtose}\label{calculando-a-assimetria-e-a-curtose}}

\begin{Shaded}
\begin{Highlighting}[]
\FunctionTok{library}\NormalTok{(moments)}
\NormalTok{sk }\OtherTok{\textless{}{-}} \FunctionTok{skewness}\NormalTok{(}\FunctionTok{as.matrix}\NormalTok{(m\_dists)) }\CommentTok{\#calcula a assimetria da matriz}
\NormalTok{ku }\OtherTok{\textless{}{-}} \FunctionTok{kurtosis}\NormalTok{(}\FunctionTok{as.matrix}\NormalTok{(m\_dists)) }\CommentTok{\#calcula a curtose da matriz}
\NormalTok{sku }\OtherTok{\textless{}{-}} \FunctionTok{cbind}\NormalTok{(sk,ku) }\CommentTok{\# junta os dois valores em um vetor}
\FunctionTok{colnames}\NormalTok{(sku) }\OtherTok{\textless{}{-}} \FunctionTok{c}\NormalTok{(}\StringTok{"assimetria"}\NormalTok{, }\StringTok{"curtose"}\NormalTok{) }\CommentTok{\# nomeia as colunas do vetor}
\NormalTok{sku[}\DecValTok{1}\SpecialCharTok{:}\DecValTok{10}\NormalTok{, ] }\CommentTok{\# mostra as primeiras 10 linhas desse vetor}
\FunctionTok{summary}\NormalTok{(sku) }\CommentTok{\# mostra um resumo estatístico do vetor}
\end{Highlighting}
\end{Shaded}

\begin{verbatim}
##          assimetria   curtose
## S-A-ZA1  1.03617037  3.101029
## S-R-CC1  1.21511849  3.852134
## S-R-CT1  0.91278496  6.263889
## S-R-CP1  1.06839482  3.177206
## S-A-TA1  1.07960546  3.220821
## S-R-CT2  0.05290597  6.265832
## S-R-CP2  1.08571040  3.241712
## S-A-TA2  1.25111833  4.838956
## S-R-CT3 -2.35918827 10.736848
## S-R-CP3  1.06511174  3.178170
##    assimetria         curtose      
##  Min.   :-3.9660   Min.   : 2.946  
##  1st Qu.: 0.1891   1st Qu.: 3.226  
##  Median : 1.0531   Median : 3.793  
##  Mean   : 0.3625   Mean   : 5.248  
##  3rd Qu.: 1.1593   3rd Qu.: 5.659  
##  Max.   : 1.2826   Max.   :18.799
\end{verbatim}

\hypertarget{distribuiuxe7uxe3o-de-frequuxeancias-da-matriz-de-distuxe2ncias}{%
\section{Distribuição de frequências da matriz de distâncias}\label{distribuiuxe7uxe3o-de-frequuxeancias-da-matriz-de-distuxe2ncias}}

Agora que temos uma ideia dos parâmetros básicos da matriz de distâncias, vamos entender sua distribuição de frequência usando um histograma e um boxplot. O histograma mostra a frequência relativa de cada intervalo de valores da matriz, enquanto o boxplot mostra a mediana, os quartis e os valores extremos da matriz.

Para fazer os gráficos, usamos as funções \texttt{hist()} e \texttt{boxplot.default()} do R. Também usamos a função \texttt{curve()} para sobrepor uma curva de normalidade ao histograma, e a função par para ajustar os parâmetros gráficos.

Nas linhas \texttt{xlim\ =\ (),\ \#\ limites\ dos\ eixos} do código abaixo, os limites dos eixos do gráfico a ser criado foram definidos entre 0 e 707.3266572. No seu caso, você deve ajustar os valores de \texttt{xlim} para que eles sejam os limites dos valores da sua matriz, definidos pela linha \texttt{range(dist)} \emph{(primeira linha do código abaixo)}.

Em resumo, o código abaixo plota um histograma e um boxplot da distribuição de dados armazenada na matriz de distâncias. Também é adicionada uma curva normal teórica ao histograma utilizando a média e o desvio padrão dos dados (Figura \ref{fig:5dists}. As funções \texttt{floor(min(m\_dists))} e \texttt{ceiling(max(m\_dists))} definem os limites dos graficos pelo valores mínimo e máximo do objeto.

\begin{Shaded}
\begin{Highlighting}[]
\FunctionTok{range}\NormalTok{(m\_dists)}
\FunctionTok{par}\NormalTok{(}\AttributeTok{mfrow=}\FunctionTok{c}\NormalTok{(}\DecValTok{2}\NormalTok{,}\DecValTok{1}\NormalTok{)) }\CommentTok{\# esse comando faz os gráficos aparecerem um acima do outro}
\FunctionTok{hist}\NormalTok{(m\_dists,}
     \AttributeTok{breaks =} \DecValTok{20}\NormalTok{, }\CommentTok{\# determina o número de colunas do histograma}
     \AttributeTok{xlim =} \FunctionTok{range}\NormalTok{(}\FunctionTok{floor}\NormalTok{(}\FunctionTok{min}\NormalTok{(m\_dists)), }\FunctionTok{ceiling}\NormalTok{(}\FunctionTok{max}\NormalTok{(m\_dists))), }\CommentTok{\# limites dos eixos}
     \AttributeTok{xlab =} \StringTok{"Distr. de Frequências"}\NormalTok{,}
     \AttributeTok{freq =} \ConstantTok{FALSE}\NormalTok{, }\CommentTok{\# mostra as frequências relativas em vez das absolutas}
     \AttributeTok{main =} \StringTok{"Histograma e boxplot da matriz de distâncias"}\NormalTok{)}
\FunctionTok{curve}\NormalTok{(}\FunctionTok{dnorm}\NormalTok{(x, }\AttributeTok{mean=}\FunctionTok{mean}\NormalTok{(m\_dists), }\AttributeTok{sd=}\FunctionTok{sd}\NormalTok{(m\_dists)), }\AttributeTok{add=}\ConstantTok{TRUE}\NormalTok{) }\CommentTok{\# sobrepõe a curva de normalidade}
\FunctionTok{boxplot.default}\NormalTok{(m\_dists, }\AttributeTok{horizontal =} \ConstantTok{TRUE}\NormalTok{, }\AttributeTok{frame =} \ConstantTok{FALSE}\NormalTok{,}
                \AttributeTok{xlab=}\StringTok{"Distr. de Frequências"}\NormalTok{,}
                \AttributeTok{ylim=}\FunctionTok{c}\NormalTok{(}\FunctionTok{floor}\NormalTok{(}\FunctionTok{min}\NormalTok{(m\_dists)), }\FunctionTok{ceiling}\NormalTok{(}\FunctionTok{max}\NormalTok{(m\_dists)))) }\CommentTok{\# limites dos eixos}
\end{Highlighting}
\end{Shaded}

\begin{figure}
\centering
\includegraphics{fundamentos_de_ecologia_numerica_files/figure-latex/5dists-1.pdf}
\caption{\label{fig:5dists}Distribuições de frequências da matriz de distâncias}
\end{figure}

\begin{verbatim}
## [1]  16.27882 707.32666
\end{verbatim}

\leavevmode\vadjust pre{\hypertarget{hello}{}}%
{IMPORTANTE}
{[}Atenção:{]} Se você recebeu a mensagem de erro {``Error in plot.new() : figure margins too large''}, aumente o tamanho da janela do gráfico e execute o bloco de comandos novamente.

O comando abaixo apaga os gráficos.

\begin{Shaded}
\begin{Highlighting}[]
\FunctionTok{dev.off}\NormalTok{()}
\end{Highlighting}
\end{Shaded}

\hypertarget{relativizando-e-transformando-a-base-de-dados}{%
\section{Relativizando e transformando a base de dados}\label{relativizando-e-transformando-a-base-de-dados}}

Na sequencia de códigos a seguir o R realiza relativizações e transformações em dados biológicos (SOKAL; ROHLF (\protect\hyperlink{ref-RN67}{1995})). Vamos ver passo a passo o que cada seção do código faz. Os códigos para as demais principais relativizações estão \protect\hyperlink{relativizauxe7uxf5es}{AQUI}

\begin{Shaded}
\begin{Highlighting}[]
\FunctionTok{library}\NormalTok{(vegan)}
\end{Highlighting}
\end{Shaded}

Essa linha carrega o pacote \texttt{vegan} no ambiente R, permitindo o uso de suas funções. O pacote \texttt{vegan} é um pacote do R para análise de ecologia numérica e comunidades biológicas. Ele fornece uma ampla variedade de funções para análise de dados ecológicos, como análise de diversidade, ordenação de espécies, análise de similaridade, entre outras.

A função \texttt{decostand()} é usada para relativizar os dados armazenados na matriz de trabalho (\texttt{m\_trab}). O parâmetro \texttt{method} especifica o método de relativização, neste caso \emph{``total''} (relativização pelo total de cada coluna). Outros métodos disponíveis são \emph{max, normalize, range e rankm} (@ref(tab:(5tbl-rel)). O parâmetro \texttt{MARGIN} especifica em qual dimensão os cálculos devem ser feitos \emph{(1 para linhas, 2 para colunas)}. Neste caso, \texttt{MARGIN=2} (colunas). A variável \texttt{m\_relcol} armazena o resultado da relativização. Sokal e Rohlf (SOKAL; ROHLF (\protect\hyperlink{ref-RN286}{1987})) tratam dos diversos tipos de relativizações e transformações em dados ecológicos.

\hypertarget{relativizauxe7uxe3o-pelo-total}{%
\subsection{Relativização pelo total}\label{relativizauxe7uxe3o-pelo-total}}

\begin{Shaded}
\begin{Highlighting}[]
\NormalTok{m\_relcol }\OtherTok{\textless{}{-}} \FunctionTok{decostand}\NormalTok{(m\_trab,}
                         \AttributeTok{method=}\StringTok{"total"}\NormalTok{,}
                         \AttributeTok{MARGIN =} \DecValTok{2}\NormalTok{)}
\CommentTok{\#View(m\_relcol)}
\NormalTok{m\_relcol}
\end{Highlighting}
\end{Shaded}

\begin{verbatim}
##          ap-davis     as-bimac    as-fasci    ch-bimac   ci-ocela   ci-orien
## S-A-ZA1 0.0000000 0.0004191115 0.000000000 0.000000000 0.00000000 0.00000000
## S-R-CC1 0.0000000 0.0414920369 0.000000000 0.000000000 0.00000000 0.00000000
## S-R-CT1 0.0000000 0.0813076278 0.348101266 0.000000000 0.00000000 0.03496503
## S-R-CP1 0.0000000 0.0079631182 0.000000000 0.000000000 0.00000000 0.00000000
## S-A-TA1 0.0000000 0.0096395641 0.006329114 0.018439716 0.00000000 0.00000000
## S-R-CT2 0.0000000 0.0595138307 0.018987342 0.004255319 0.00000000 0.48251748
## S-R-CP2 0.0000000 0.0020955574 0.006329114 0.000000000 0.57142857 0.06293706
## S-A-TA2 0.0000000 0.0192791282 0.000000000 0.252482270 0.00000000 0.00000000
## S-R-CT3 0.0000000 0.0863369656 0.405063291 0.000000000 0.00000000 0.17482517
## S-R-CP3 0.0000000 0.0067057837 0.000000000 0.000000000 0.18571429 0.16783217
## S-A-TA3 0.0000000 0.0980720872 0.044303797 0.337588652 0.00000000 0.00000000
## S-R-CT4 0.0000000 0.0000000000 0.006329114 0.000000000 0.00000000 0.03496503
## S-R-CP4 0.0000000 0.0000000000 0.000000000 0.000000000 0.15714286 0.04195804
## S-A-TA4 0.0000000 0.1651299246 0.000000000 0.387234043 0.00000000 0.00000000
## B-A-MU1 0.0000000 0.0050293378 0.000000000 0.000000000 0.00000000 0.00000000
## B-R-ET1 0.0000000 0.0012573345 0.000000000 0.000000000 0.00000000 0.00000000
## B-A-GU1 0.0000000 0.0008382230 0.012658228 0.000000000 0.00000000 0.00000000
## B-R-PC2 0.1851852 0.0184409053 0.000000000 0.000000000 0.02857143 0.00000000
## B-A-MU2 0.0000000 0.0414920369 0.000000000 0.000000000 0.00000000 0.00000000
## B-A-GU2 0.0000000 0.0000000000 0.000000000 0.000000000 0.00000000 0.00000000
## B-R-PC3 0.8148148 0.0314333613 0.044303797 0.000000000 0.05714286 0.00000000
## B-A-MU3 0.0000000 0.2141659681 0.000000000 0.000000000 0.00000000 0.00000000
## B-A-GU3 0.0000000 0.0025146689 0.000000000 0.000000000 0.00000000 0.00000000
## B-R-PC4 0.0000000 0.0029337804 0.107594937 0.000000000 0.00000000 0.00000000
## B-A-MU4 0.0000000 0.0984911987 0.000000000 0.000000000 0.00000000 0.00000000
## B-A-GU4 0.0000000 0.0054484493 0.000000000 0.000000000 0.00000000 0.00000000
##         co-macro co-heter   cr-menez cu-lepid  cy-gilbe     ge-brasi he-margi
## S-A-ZA1        0        0 0.00000000        0 0.0000000 0.0000000000      0.0
## S-R-CC1        0        0 0.00000000        0 0.0000000 0.0000000000      0.0
## S-R-CT1        0        1 0.50000000        0 0.0000000 0.0029411765      0.0
## S-R-CP1        0        0 0.00000000        0 0.0000000 0.0000000000      0.0
## S-A-TA1        0        0 0.00000000        0 0.0000000 0.0000000000      0.0
## S-R-CT2        0        0 0.14285714        0 0.0000000 0.0000000000      0.5
## S-R-CP2        0        0 0.00000000        0 0.0000000 0.0000000000      0.0
## S-A-TA2        0        0 0.00000000        0 0.0000000 0.0000000000      0.0
## S-R-CT3        0        0 0.28571429        0 0.0000000 0.0009803922      0.0
## S-R-CP3        0        0 0.00000000        0 0.0000000 0.0000000000      0.0
## S-A-TA3        1        0 0.00000000        0 0.0000000 0.0000000000      0.0
## S-R-CT4        0        0 0.03571429        0 0.3816794 0.0029411765      0.5
## S-R-CP4        0        0 0.00000000        0 0.0000000 0.0000000000      0.0
## S-A-TA4        0        0 0.03571429        0 0.0000000 0.0009803922      0.0
## B-A-MU1        0        0 0.00000000        0 0.0000000 0.1862745098      0.0
## B-R-ET1        0        0 0.00000000        0 0.0000000 0.0000000000      0.0
## B-A-GU1        0        0 0.00000000        0 0.0000000 0.0068627451      0.0
## B-R-PC2        0        0 0.00000000        0 0.0000000 0.0078431373      0.0
## B-A-MU2        0        0 0.00000000        0 0.0000000 0.0656862745      0.0
## B-A-GU2        0        0 0.00000000        0 0.0000000 0.0225490196      0.0
## B-R-PC3        0        0 0.00000000        1 0.0000000 0.0156862745      0.0
## B-A-MU3        0        0 0.00000000        0 0.0000000 0.1421568627      0.0
## B-A-GU3        0        0 0.00000000        0 0.0000000 0.0313725490      0.0
## B-R-PC4        0        0 0.00000000        0 0.6183206 0.0049019608      0.0
## B-A-MU4        0        0 0.00000000        0 0.0000000 0.4990196078      0.0
## B-A-GU4        0        0 0.00000000        0 0.0000000 0.0098039216      0.0
##            ho-malab   hy-pusar le-melan    le-piau le-taeni mo-costa   mo-lepid
## S-A-ZA1 0.000000000 0.00000000        0 0.00000000        0        0 0.00000000
## S-R-CC1 0.000000000 0.00000000        0 0.00000000        0        0 0.02439024
## S-R-CT1 0.009174312 0.12676056        0 0.20000000        0        0 0.95121951
## S-R-CP1 0.045871560 0.02816901        0 0.00000000        0        0 0.00000000
## S-A-TA1 0.000000000 0.00000000        0 0.00000000        0        0 0.00000000
## S-R-CT2 0.155963303 0.60563380        0 0.06666667        0        0 0.02439024
## S-R-CP2 0.091743119 0.02816901        0 0.20000000        0        0 0.00000000
## S-A-TA2 0.018348624 0.00000000        0 0.00000000        0        0 0.00000000
## S-R-CT3 0.284403670 0.15492958        0 0.13333333        0        0 0.00000000
## S-R-CP3 0.036697248 0.00000000        0 0.06666667        0        0 0.00000000
## S-A-TA3 0.183486239 0.00000000        0 0.00000000        0        0 0.00000000
## S-R-CT4 0.036697248 0.04225352        0 0.00000000        0        0 0.00000000
## S-R-CP4 0.018348624 0.00000000        0 0.13333333        0        0 0.00000000
## S-A-TA4 0.082568807 0.00000000        0 0.13333333        0        0 0.00000000
## B-A-MU1 0.000000000 0.00000000        0 0.00000000        0        0 0.00000000
## B-R-ET1 0.000000000 0.00000000        0 0.00000000        0        0 0.00000000
## B-A-GU1 0.000000000 0.00000000        0 0.00000000        0        0 0.00000000
## B-R-PC2 0.000000000 0.00000000        1 0.00000000        1        0 0.00000000
## B-A-MU2 0.009174312 0.00000000        0 0.00000000        0        0 0.00000000
## B-A-GU2 0.000000000 0.00000000        0 0.00000000        0        0 0.00000000
## B-R-PC3 0.018348624 0.01408451        0 0.00000000        0        1 0.00000000
## B-A-MU3 0.000000000 0.00000000        0 0.00000000        0        0 0.00000000
## B-A-GU3 0.000000000 0.00000000        0 0.00000000        0        0 0.00000000
## B-R-PC4 0.009174312 0.00000000        0 0.06666667        0        0 0.00000000
## B-A-MU4 0.000000000 0.00000000        0 0.00000000        0        0 0.00000000
## B-A-GU4 0.000000000 0.00000000        0 0.00000000        0        0 0.00000000
##            or-nilot    pa-manag pimel-sp   po-retic    po-vivip    pr-brevi
## S-A-ZA1 0.000000000 0.000000000        0 0.00000000 0.000000000 0.032258065
## S-R-CC1 0.002358491 0.000000000        0 0.00000000 0.000000000 0.000000000
## S-R-CT1 0.042452830 0.000000000        1 0.00000000 0.048057260 0.017921147
## S-R-CP1 0.000000000 0.000000000        0 0.00000000 0.015337423 0.000000000
## S-A-TA1 0.000000000 0.000000000        0 0.00000000 0.000000000 0.003584229
## S-R-CT2 0.090801887 0.000000000        0 0.05249344 0.225971370 0.053763441
## S-R-CP2 0.000000000 0.000000000        0 0.00000000 0.032719836 0.017921147
## S-A-TA2 0.000000000 0.000000000        0 0.00000000 0.000000000 0.007168459
## S-R-CT3 0.162735849 0.000000000        0 0.01312336 0.333333333 0.587813620
## S-R-CP3 0.000000000 0.000000000        0 0.00000000 0.010224949 0.000000000
## S-A-TA3 0.000000000 0.000000000        0 0.00000000 0.000000000 0.000000000
## S-R-CT4 0.086084906 0.000000000        0 0.00000000 0.028629857 0.211469534
## S-R-CP4 0.000000000 0.000000000        0 0.00000000 0.081799591 0.000000000
## S-A-TA4 0.001179245 0.000000000        0 0.00000000 0.000000000 0.010752688
## B-A-MU1 0.007075472 0.000000000        0 0.00000000 0.000000000 0.000000000
## B-R-ET1 0.009433962 0.001805054        0 0.08923885 0.000000000 0.000000000
## B-A-GU1 0.003537736 0.019855596        0 0.00000000 0.000000000 0.000000000
## B-R-PC2 0.005896226 0.000000000        0 0.00000000 0.000000000 0.032258065
## B-A-MU2 0.001179245 0.000000000        0 0.02624672 0.008179959 0.000000000
## B-A-GU2 0.042452830 0.184115523        0 0.00000000 0.000000000 0.000000000
## B-R-PC3 0.076650943 0.000000000        0 0.00000000 0.000000000 0.021505376
## B-A-MU3 0.012971698 0.000000000        0 0.12073491 0.049079755 0.003584229
## B-A-GU3 0.291273585 0.451263538        0 0.00000000 0.000000000 0.000000000
## B-R-PC4 0.010613208 0.000000000        0 0.00000000 0.000000000 0.000000000
## B-A-MU4 0.001179245 0.000000000        0 0.69816273 0.166666667 0.000000000
## B-A-GU4 0.152122642 0.342960289        0 0.00000000 0.000000000 0.000000000
##         ps-rhomb ps-genise   se-heter se-piaba se-spilo    st-noton sy-marmo
## S-A-ZA1        0         0 0.00000000        0        0 0.000000000        0
## S-R-CC1        0         0 0.00000000        0        0 0.000000000        0
## S-R-CT1        0         0 0.13513514        1        0 0.004878049        0
## S-R-CP1        0         0 0.04729730        0        0 0.000000000        0
## S-A-TA1        0         0 0.01351351        0        0 0.000000000        0
## S-R-CT2        0         0 0.20270270        0        0 0.121951220        0
## S-R-CP2        0         0 0.00000000        0        0 0.000000000        1
## S-A-TA2        0         0 0.00000000        0        0 0.000000000        0
## S-R-CT3        1         1 0.12837838        0        1 0.560975610        0
## S-R-CP3        0         0 0.00000000        0        0 0.000000000        0
## S-A-TA3        0         0 0.00000000        0        0 0.000000000        0
## S-R-CT4        0         0 0.01013514        0        0 0.312195122        0
## S-R-CP4        0         0 0.01013514        0        0 0.000000000        0
## S-A-TA4        0         0 0.00000000        0        0 0.000000000        0
## B-A-MU1        0         0 0.00000000        0        0 0.000000000        0
## B-R-ET1        0         0 0.00000000        0        0 0.000000000        0
## B-A-GU1        0         0 0.00000000        0        0 0.000000000        0
## B-R-PC2        0         0 0.03378378        0        0 0.000000000        0
## B-A-MU2        0         0 0.00000000        0        0 0.000000000        0
## B-A-GU2        0         0 0.00000000        0        0 0.000000000        0
## B-R-PC3        0         0 0.31418919        0        0 0.000000000        0
## B-A-MU3        0         0 0.00000000        0        0 0.000000000        0
## B-A-GU3        0         0 0.00000000        0        0 0.000000000        0
## B-R-PC4        0         0 0.10472973        0        0 0.000000000        0
## B-A-MU4        0         0 0.00000000        0        0 0.000000000        0
## B-A-GU4        0         0 0.00000000        0        0 0.000000000        0
##          te-chalc   tr-signa
## S-A-ZA1 0.0000000 0.00000000
## S-R-CC1 0.0000000 0.00000000
## S-R-CT1 0.0000000 0.08653846
## S-R-CP1 0.0000000 0.00000000
## S-A-TA1 0.0000000 0.00000000
## S-R-CT2 0.0000000 0.07211538
## S-R-CP2 0.0000000 0.00000000
## S-A-TA2 0.0000000 0.00000000
## S-R-CT3 0.0000000 0.03365385
## S-R-CP3 0.0000000 0.00000000
## S-A-TA3 0.0000000 0.00000000
## S-R-CT4 0.0000000 0.67788462
## S-R-CP4 0.0000000 0.00000000
## S-A-TA4 0.0000000 0.00000000
## B-A-MU1 0.0000000 0.00000000
## B-R-ET1 0.0000000 0.00000000
## B-A-GU1 0.0000000 0.00000000
## B-R-PC2 0.5671642 0.11057692
## B-A-MU2 0.0000000 0.00000000
## B-A-GU2 0.0000000 0.00000000
## B-R-PC3 0.4328358 0.00000000
## B-A-MU3 0.0000000 0.00000000
## B-A-GU3 0.0000000 0.00000000
## B-R-PC4 0.0000000 0.01923077
## B-A-MU4 0.0000000 0.00000000
## B-A-GU4 0.0000000 0.00000000
\end{verbatim}

\begin{Shaded}
\begin{Highlighting}[]
\CommentTok{\#colSums(m\_relcol)}
\CommentTok{\#range(m\_relcol)}
\end{Highlighting}
\end{Shaded}

\hypertarget{transformauxe7uxe3o-pelo-arcoseno-da-raiz-quadrada}{%
\subsection{Transformação pelo arcoseno da raiz quadrada}\label{transformauxe7uxe3o-pelo-arcoseno-da-raiz-quadrada}}

A seguir é realizada a transformação pelo arcoseno da raiz quadrada, apropriada para ados em proporção como os ados previamente relativisados. Para cada transformação, a função é aplicada aos dados de interesse e o resultado é armazenado em uma nova variável. A função \texttt{View()} é usada para visualizar o resultado de cada transformação na forma de uma tabela. Os códigos para as demais principais transformações estão \protect\hyperlink{transformauxe7uxf5es}{AQUI}

\begin{Shaded}
\begin{Highlighting}[]
\NormalTok{m\_relcol\_asrq }\OtherTok{\textless{}{-}} \FunctionTok{asin}\NormalTok{(}\FunctionTok{sqrt}\NormalTok{(m\_relcol))}
\CommentTok{\#View(m\_relcol\_asrq)}
\NormalTok{m\_relcol\_asrqpi }\OtherTok{\textless{}{-}} \DecValTok{2}\SpecialCharTok{/}\NormalTok{pi}\SpecialCharTok{*}\NormalTok{(}\FunctionTok{asin}\NormalTok{(}\FunctionTok{sqrt}\NormalTok{(m\_relcol)))}
\CommentTok{\#View(m\_relcol\_asrqpi)}
\CommentTok{\#colSums(m\_relcol\_asrqpi)}
\CommentTok{\#range(m\_relcol\_asrqpi)}
\end{Highlighting}
\end{Shaded}

\hypertarget{reset-point-2-1}{%
\section{Reset point 2}\label{reset-point-2-1}}

Aqui estabelecemos a matriz transformada depois de ter sido relativisada como \texttt{m\_trns}.

\begin{Shaded}
\begin{Highlighting}[]
\NormalTok{m\_trns }\OtherTok{\textless{}{-}}\NormalTok{ m\_relcol\_asrqpi}
\CommentTok{\#m\_dists \textless{}{-} dist(m\_trns, method = "euclidian", diag = TRUE, upper = FALSE)}
\end{Highlighting}
\end{Shaded}

Agora é com você\ldots Refaça toda a análise com as relativizações e transformações adequadas para a matriz de dados fornecida.

\hypertarget{apuxeandices-11}{%
\section*{Apêndices}\label{apuxeandices-11}}
\addcontentsline{toc}{section}{Apêndices}

\hypertarget{relativizauxe7uxf5es}{%
\subsection*{Relativizações}\label{relativizauxe7uxf5es}}
\addcontentsline{toc}{subsection}{Relativizações}

\hypertarget{total-das-colunas}{%
\subsubsection*{Total das colunas}\label{total-das-colunas}}
\addcontentsline{toc}{subsubsection}{Total das colunas}

\begin{Shaded}
\begin{Highlighting}[]
\NormalTok{m\_relcol }\OtherTok{\textless{}{-}} \FunctionTok{decostand}\NormalTok{(m\_bruta,}
                         \AttributeTok{method=}\StringTok{"total"}\NormalTok{,}
                         \AttributeTok{MARGIN =} \DecValTok{2}\NormalTok{)}
\FunctionTok{View}\NormalTok{(m\_relcol)}
\NormalTok{m\_relcol}
\end{Highlighting}
\end{Shaded}

A função \texttt{decostand()} é usada para relativizar os dados armazenados no vetor \texttt{m\_bruta}. O parâmetro \texttt{method} especifica o método de relativização, neste caso ``total'' (relativização pelo total de cada coluna). Outros métodos disponíveis são \texttt{max}, \texttt{normalize}, \texttt{range} e \texttt{rankm} (Figura \ref{tab:5rels}). O parâmetro \texttt{MARGIN} especifica em qual dimensão os cálculos devem ser feitos (1 para linhas, 2 para colunas). Neste caso, \texttt{MARGIN=2} (colunas). A variável \texttt{m\_relcol} armazena o resultado da relativização. Sokal e Rohlf (\protect\hyperlink{ref-RN67}{1995}) tratam dos diversos tipos de relativizações e transformações em dados ecológicos.

\begin{verbatim}
## Warning in table_info$align_vector[column] <-
## unlist(lapply(table_info$align_vector_origin[column], : número de itens para
## para substituir não é um múltiplo do comprimento do substituto
\end{verbatim}

\begin{table}

\caption{\label{tab:5rels}Métodos de relativização da função `decostand`}
\centering
\begin{tabular}[t]{c>{\raggedright\arraybackslash}p{8cm}>{\raggedright\arraybackslash}p{5cm}}
\toprule
Função & Descrição & Uso adequado\\
\midrule
max & Divide cada valor pela maior observação da coluna em que está localizado. Todos os valores resultantes serão menores ou iguais a 1. Adequada para análise em escala relativa & Quando a escala de cada variável é conhecida e a análise deve ser feita em uma escala relativa\\
normalize & Divide cada valor pelo comprimento do vetor. Isso resulta em valores cujo comprimento é sempre igual a 1. Adequada para análise de dados de frequência & Para análise de dados de frequência\\
range & Subtrai o valor mínimo da coluna de cada valor e, em seguida, divide pelo intervalo (ou amplitude) dos valores da coluna. Isso resulta em valores entre 0 e 1. & Quando a amplitude dos valores em cada coluna é importante para a análise\\
rankm & Transforma os valores em suas posições dentro da coluna, em ordem crescente. O menor valor recebe o valor 1 e o maior valor recebe o valor igual ao comprimento da coluna. & Quando a escala absoluta dos valores não é importante, mas a ordem dos valores é significativa.\\
\bottomrule
\end{tabular}
\end{table}

\hypertarget{transformauxe7uxf5es}{%
\subsection*{Transformações}\label{transformauxe7uxf5es}}
\addcontentsline{toc}{subsection}{Transformações}

Essa seção mostra as diversas transformações nos dados armazenados no vetor \texttt{m\_bruta}. Para cada transformação, a função é aplicada aos dados da matriz e o resultado é armazenado em uma nova variável. A função \texttt{View()} pode ser usada para visualizar o resultado de cada transformação na forma de uma tabela. Ela mostra o código que realiza diversas transformações na matriz de dados. Para cada transformação, a função é aplicada aos dados e o resultado é armazenado em uma nova variável.

\hypertarget{log10}{%
\subsubsection*{Log10}\label{log10}}
\addcontentsline{toc}{subsubsection}{Log10}

Não aceita zeros, porque o log de zero é indeterminado. Por isso devemos substituir os valores com erro (infinito negativo) por zero, usando a função \texttt{replace()}.

\begin{Shaded}
\begin{Highlighting}[]
\NormalTok{m\_lg10 }\OtherTok{\textless{}{-}} \FunctionTok{log10}\NormalTok{(m\_bruta)}
\CommentTok{\#View(m\_lg10)}
\NormalTok{m\_lg10 }\OtherTok{\textless{}{-}} \FunctionTok{replace}\NormalTok{(m\_lg10, }\FunctionTok{is.infinite}\NormalTok{(m\_lg10), }\DecValTok{0}\NormalTok{)}
\FunctionTok{head}\NormalTok{(m\_lg10)}
\end{Highlighting}
\end{Shaded}

\hypertarget{logx1}{%
\subsubsection*{Log(x+1)}\label{logx1}}
\addcontentsline{toc}{subsubsection}{Log(x+1)}

Desaconselhavel para valores que variam entre 0-1.

\begin{Shaded}
\begin{Highlighting}[]
\NormalTok{m\_lgx1 }\OtherTok{\textless{}{-}} \FunctionTok{log}\NormalTok{(m\_bruta}\SpecialCharTok{+}\DecValTok{1}\NormalTok{)}
\CommentTok{\#View(m\_lgx1)}
\end{Highlighting}
\end{Shaded}

\hypertarget{exponenciais}{%
\subsubsection*{Exponenciais}\label{exponenciais}}
\addcontentsline{toc}{subsubsection}{Exponenciais}

``Power transformations'', onde o valor de \texttt{p} estabelece a compressão dos dados.

\begin{Shaded}
\begin{Highlighting}[]
\NormalTok{m\_bruta\_P1 }\OtherTok{\textless{}{-}}\NormalTok{ m\_bruta}\SpecialCharTok{\^{}}\DecValTok{1} \CommentTok{\#p=1}
\CommentTok{\#View(m\_bruta\_P1)}
\NormalTok{m\_bruta\_P05 }\OtherTok{\textless{}{-}}\NormalTok{ m\_bruta}\SpecialCharTok{\^{}}\FloatTok{0.5} \CommentTok{\#p=0,5}
\CommentTok{\#View(m\_bruta\_P05)}
\NormalTok{m\_bruta\_P01 }\OtherTok{\textless{}{-}}\NormalTok{ m\_bruta}\SpecialCharTok{\^{}}\FloatTok{0.1} \CommentTok{\#p=0,1}
\CommentTok{\#View(m\_bruta\_P01)}
\end{Highlighting}
\end{Shaded}

\hypertarget{presenuxe7aausuxeancia}{%
\subsubsection*{Presença/Ausência}\label{presenuxe7aausuxeancia}}
\addcontentsline{toc}{subsubsection}{Presença/Ausência}

\begin{Shaded}
\begin{Highlighting}[]
\NormalTok{m\_pa }\OtherTok{\textless{}{-}}\NormalTok{ (m\_bruta}\SpecialCharTok{\textgreater{}}\DecValTok{0}\NormalTok{)}\SpecialCharTok{*}\NormalTok{1L}
\CommentTok{\#View(m\_pa)}
\end{Highlighting}
\end{Shaded}

\hypertarget{raiz-quadrada}{%
\subsubsection*{Raiz quadrada}\label{raiz-quadrada}}
\addcontentsline{toc}{subsubsection}{Raiz quadrada}

\begin{Shaded}
\begin{Highlighting}[]
\NormalTok{m\_rq }\OtherTok{\textless{}{-}} \FunctionTok{sqrt}\NormalTok{(m\_bruta)}
\CommentTok{\#View(m\_rq)}
\end{Highlighting}
\end{Shaded}

\hypertarget{arcoseno-da-raiz-quadrada}{%
\subsubsection*{Arcoseno da raiz quadrada}\label{arcoseno-da-raiz-quadrada}}
\addcontentsline{toc}{subsubsection}{Arcoseno da raiz quadrada}

Os valores de entrada tem que variar entre 0 e 1. Ideal para matrizes relativizadas. Duas formulações estão disponíveis, com e sem a multiplicação por \texttt{2/pi}. A segunda opção costuma ter melhor efeito na normalização dos dados.

\begin{Shaded}
\begin{Highlighting}[]
\NormalTok{m\_asrq }\OtherTok{\textless{}{-}} \FunctionTok{asin}\NormalTok{(}\FunctionTok{sqrt}\NormalTok{(m\_relcol))}
\CommentTok{\#View(m\_asrq)}
\NormalTok{m\_asrqpi }\OtherTok{\textless{}{-}} \DecValTok{2}\SpecialCharTok{/}\NormalTok{pi}\SpecialCharTok{*}\NormalTok{(}\FunctionTok{asin}\NormalTok{(}\FunctionTok{sqrt}\NormalTok{(m\_relcol)))}
\FunctionTok{View}\NormalTok{(m\_asrqpi)}
\end{Highlighting}
\end{Shaded}

\hypertarget{script-limpo-8}{%
\section*{Script limpo}\label{script-limpo-8}}
\addcontentsline{toc}{section}{Script limpo}

Aqui apresento o scrip na íntegra sem os textos ou outros comentários. Você pode copiar e colar no R para executa-lo. Lembre de remover os {\texttt{\#}} ou {\texttt{\#\#}} caso necessite executar essas linhas.

\begin{Shaded}
\begin{Highlighting}[]
\DocumentationTok{\#\# dev.off() \#apaga os graficos, se houver algum}
\DocumentationTok{\#\# rm(list=ls(all=TRUE)) \#limpa a memória}
\DocumentationTok{\#\# cat("\textbackslash{}014") \#limpa o console}
\DocumentationTok{\#\# install.packages("openxlsx") \#importa arquivos do excel}
\DocumentationTok{\#\# install.packages("moments") \#calcula assimetria e curtose dos dados}
\DocumentationTok{\#\# install.packages("vegan") \#estatisticas para ecologia de comunidades}
\DocumentationTok{\#\# getwd()}
\DocumentationTok{\#\# setwd("C:/Seu/Diretório/De/Trabalho")}
\FunctionTok{library}\NormalTok{(openxlsx)}
\CommentTok{\#dir \textless{}{-} getwd() \#criamos um vetor com o diretório  de trbalho }
\CommentTok{\#shell.exec(dir) \#abre o diretorio de trabalho no Windows Explorer}
\NormalTok{m\_bruta }\OtherTok{\textless{}{-}} \FunctionTok{read.xlsx}\NormalTok{(}\StringTok{"D:/Elvio/OneDrive/Disciplinas/\_EcoNumerica/5.Matrizes/peixes06.xlsx"}\NormalTok{,}
                   \AttributeTok{rowNames =}\NormalTok{ T, }\AttributeTok{colNames =}\NormalTok{ T,}
                   \AttributeTok{sheet =} \StringTok{"Sheet1"}\NormalTok{)}
\FunctionTok{str}\NormalTok{(m\_bruta)}
\CommentTok{\#View(m\_bruta)}
\NormalTok{m\_bruta[}\DecValTok{1}\SpecialCharTok{:}\DecValTok{5}\NormalTok{,}\DecValTok{1}\SpecialCharTok{:}\DecValTok{5}\NormalTok{] }\CommentTok{\#[1:5,1:5] mostra apenas as linhas e colunas de 1 a 5.}
\DocumentationTok{\#\# \#View(m\_bruta)}
\DocumentationTok{\#\# print(m\_bruta[1:8,1:8])}
\DocumentationTok{\#\# m\_bruta[1:10,1:10]}
\DocumentationTok{\#\# str(m\_bruta)}
\DocumentationTok{\#\# mode(m\_bruta)}
\DocumentationTok{\#\# class(m\_bruta)}
\CommentTok{\#View(m\_bruta)}
\FunctionTok{print}\NormalTok{(m\_bruta)}
\FunctionTok{head}\NormalTok{(m\_bruta)}
\FunctionTok{range}\NormalTok{(m\_bruta) }\CommentTok{\#menor e maior valores}
\FunctionTok{length}\NormalTok{(m\_bruta) }\CommentTok{\#no. de colunas}
\FunctionTok{ncol}\NormalTok{(m\_bruta) }\CommentTok{\#no. de N colunas}
\FunctionTok{nrow}\NormalTok{(m\_bruta) }\CommentTok{\#no. de M linhas}
\FunctionTok{sum}\NormalTok{(}\FunctionTok{lengths}\NormalTok{(m\_bruta)) }\CommentTok{\#soma os nos. de colunas}
\FunctionTok{length}\NormalTok{(}\FunctionTok{as.matrix}\NormalTok{(m\_bruta)) }\CommentTok{\#tamanho da matriz m x n}
\FunctionTok{sum}\NormalTok{(m\_bruta }\SpecialCharTok{==} \DecValTok{0}\NormalTok{) }\CommentTok{\# número de observações igual a zero}
\FunctionTok{sum}\NormalTok{(m\_bruta }\SpecialCharTok{\textgreater{}} \DecValTok{0}\NormalTok{) }\CommentTok{\# número de observações maiores que zero}
\NormalTok{zeros }\OtherTok{\textless{}{-}}\NormalTok{ (}\FunctionTok{sum}\NormalTok{(m\_bruta }\SpecialCharTok{==} \DecValTok{0}\NormalTok{)}\SpecialCharTok{/}\FunctionTok{length}\NormalTok{(}\FunctionTok{as.matrix}\NormalTok{(m\_bruta)))}\SpecialCharTok{*}\DecValTok{100} \CommentTok{\# proporção de zeros na matriz}
\NormalTok{zeros}
\NormalTok{tamanho }\OtherTok{\textless{}{-}} \FunctionTok{data.frame}\NormalTok{(}
\NormalTok{  Função }\OtherTok{=} \FunctionTok{c}\NormalTok{(}\StringTok{"range"}\NormalTok{, }\StringTok{"lenght"}\NormalTok{, }\StringTok{"m cols"}\NormalTok{, }\StringTok{"n linhas"}\NormalTok{, }\StringTok{"Tamanho"}\NormalTok{, }\StringTok{"Tamanho"}\NormalTok{,}
             \StringTok{"Zeros"}\NormalTok{, }\StringTok{"Nao zeros"}\NormalTok{, }\StringTok{"\% Zeros"}\NormalTok{),}
  \AttributeTok{Resultado =} \FunctionTok{c}\NormalTok{(}\FunctionTok{paste}\NormalTok{(}\FunctionTok{range}\NormalTok{(m\_bruta), }\AttributeTok{collapse =} \StringTok{" {-} "}\NormalTok{), }\FunctionTok{length}\NormalTok{(m\_bruta), }\FunctionTok{ncol}\NormalTok{(m\_bruta),}
                \FunctionTok{nrow}\NormalTok{(m\_bruta), }\FunctionTok{sum}\NormalTok{(}\FunctionTok{lengths}\NormalTok{(m\_bruta)), }\FunctionTok{length}\NormalTok{(}\FunctionTok{as.matrix}\NormalTok{(m\_bruta)), }\FunctionTok{sum}\NormalTok{(m\_bruta }\SpecialCharTok{==} \DecValTok{0}\NormalTok{),}
                \FunctionTok{sum}\NormalTok{(m\_bruta }\SpecialCharTok{\textgreater{}} \DecValTok{0}\NormalTok{), }\FunctionTok{round}\NormalTok{(zeros, }\DecValTok{1}\NormalTok{)))}
\NormalTok{tamanho}
\NormalTok{m\_trab }\OtherTok{\textless{}{-}}\NormalTok{ m\_bruta}
\NormalTok{m\_dists }\OtherTok{\textless{}{-}} \FunctionTok{dist}\NormalTok{(m\_trab, }\AttributeTok{method =} \StringTok{"euclidian"}\NormalTok{, }\AttributeTok{diag =} \ConstantTok{TRUE}\NormalTok{, }\AttributeTok{upper =} \ConstantTok{FALSE}\NormalTok{)}
\CommentTok{\#m\_dists}
\FunctionTok{str}\NormalTok{(m\_dists)}
\FunctionTok{mode}\NormalTok{(m\_dists)}
\FunctionTok{class}\NormalTok{(m\_dists)}
\FunctionTok{length}\NormalTok{(}\FunctionTok{as.matrix}\NormalTok{(m\_dists))}
\FunctionTok{as.matrix}\NormalTok{(m\_dists)[}\DecValTok{1}\SpecialCharTok{:}\DecValTok{10}\NormalTok{, }\DecValTok{1}\SpecialCharTok{:}\DecValTok{10}\NormalTok{]}
\FunctionTok{range}\NormalTok{(m\_trab)}
\FunctionTok{range}\NormalTok{(m\_dists)}
\FunctionTok{min}\NormalTok{(m\_dists)}
\FunctionTok{max}\NormalTok{(m\_dists)}
\FunctionTok{mean}\NormalTok{(m\_dists) }\CommentTok{\#CENTROIDE!! ou Grand mean}
\FunctionTok{sd}\NormalTok{(m\_dists)  }\CommentTok{\#standard deviation}
\NormalTok{centroide }\OtherTok{\textless{}{-}} \FunctionTok{mean}\NormalTok{(m\_dists)}
\NormalTok{centroide}
\FunctionTok{length}\NormalTok{(m\_dists)}
\NormalTok{m }\OtherTok{\textless{}{-}} \FunctionTok{nrow}\NormalTok{(}\FunctionTok{as.matrix}\NormalTok{(m\_dists))}
\NormalTok{m}
\NormalTok{m}\SpecialCharTok{*}\NormalTok{(m}\DecValTok{{-}1}\NormalTok{)}\SpecialCharTok{/}\DecValTok{2}
\FunctionTok{summary}\NormalTok{(m\_dists)}
\NormalTok{Sumario1 }\OtherTok{\textless{}{-}} \FunctionTok{cbind}\NormalTok{(}\FunctionTok{min}\NormalTok{(m\_dists),}
                  \FunctionTok{max}\NormalTok{(m\_dists),}
                  \FunctionTok{sd}\NormalTok{(m\_dists),}
                  \FunctionTok{mean}\NormalTok{(m\_dists),}
                  \FunctionTok{length}\NormalTok{(m\_dists))}
\FunctionTok{colnames}\NormalTok{(Sumario1) }\OtherTok{\textless{}{-}} \FunctionTok{c}\NormalTok{(}\StringTok{"Minimo"}\NormalTok{, }\StringTok{"Maximo"}\NormalTok{, }\StringTok{"Desv.Padr"}\NormalTok{, }\StringTok{"Media"}\NormalTok{, }\StringTok{"m(m{-}1)/2"}\NormalTok{)}
\FunctionTok{rownames}\NormalTok{(Sumario1) }\OtherTok{\textless{}{-}}\NormalTok{ (}\StringTok{"Valores"}\NormalTok{)}
\NormalTok{Sumario1}
\NormalTok{dist }\OtherTok{\textless{}{-}}\NormalTok{ (}\FunctionTok{as.matrix}\NormalTok{(m\_dists))}
\CommentTok{\#View(dist)}
\FunctionTok{str}\NormalTok{(dist)}
\FunctionTok{mode}\NormalTok{(dist)}
\FunctionTok{class}\NormalTok{(dist)}
\CommentTok{\#head(dist)}
\NormalTok{dist[}\DecValTok{1}\SpecialCharTok{:}\DecValTok{5}\NormalTok{, }\DecValTok{1}\SpecialCharTok{:}\DecValTok{5}\NormalTok{] }\CommentTok{\#mostra as 5 primeiras linhas e colunas da matriz}
\FunctionTok{range}\NormalTok{(m\_dists) }\CommentTok{\# intervalo dos valores da matriz}
\FunctionTok{mean}\NormalTok{(m\_dists) }\CommentTok{\# média dos valores da matriz}
\FunctionTok{sd}\NormalTok{(m\_dists) }\CommentTok{\# desvio padrão dos valores da matriz}
\FunctionTok{library}\NormalTok{(moments)}
\NormalTok{sk }\OtherTok{\textless{}{-}} \FunctionTok{skewness}\NormalTok{(}\FunctionTok{as.matrix}\NormalTok{(m\_dists)) }\CommentTok{\#calcula a assimetria da matriz}
\NormalTok{ku }\OtherTok{\textless{}{-}} \FunctionTok{kurtosis}\NormalTok{(}\FunctionTok{as.matrix}\NormalTok{(m\_dists)) }\CommentTok{\#calcula a curtose da matriz}
\NormalTok{sku }\OtherTok{\textless{}{-}} \FunctionTok{cbind}\NormalTok{(sk,ku) }\CommentTok{\# junta os dois valores em um vetor}
\FunctionTok{colnames}\NormalTok{(sku) }\OtherTok{\textless{}{-}} \FunctionTok{c}\NormalTok{(}\StringTok{"assimetria"}\NormalTok{, }\StringTok{"curtose"}\NormalTok{) }\CommentTok{\# nomeia as colunas do vetor}
\NormalTok{sku[}\DecValTok{1}\SpecialCharTok{:}\DecValTok{10}\NormalTok{, ] }\CommentTok{\# mostra as primeiras 10 linhas desse vetor}
\FunctionTok{summary}\NormalTok{(sku) }\CommentTok{\# mostra um resumo estatístico do vetor}
\FunctionTok{range}\NormalTok{(m\_dists)}
\FunctionTok{par}\NormalTok{(}\AttributeTok{mfrow=}\FunctionTok{c}\NormalTok{(}\DecValTok{2}\NormalTok{,}\DecValTok{1}\NormalTok{)) }\CommentTok{\# esse comando faz os gráficos aparecerem um acima do outro}
\FunctionTok{hist}\NormalTok{(m\_dists,}
     \AttributeTok{breaks =} \DecValTok{20}\NormalTok{, }\CommentTok{\# determina o número de colunas do histograma}
     \AttributeTok{xlim =} \FunctionTok{range}\NormalTok{(}\FunctionTok{floor}\NormalTok{(}\FunctionTok{min}\NormalTok{(m\_dists)), }\FunctionTok{ceiling}\NormalTok{(}\FunctionTok{max}\NormalTok{(m\_dists))), }\CommentTok{\# limites dos eixos}
     \AttributeTok{xlab =} \StringTok{"Distr. de Frequências"}\NormalTok{,}
     \AttributeTok{freq =} \ConstantTok{FALSE}\NormalTok{, }\CommentTok{\# mostra as frequências relativas em vez das absolutas}
     \AttributeTok{main =} \StringTok{"Histograma e boxplot da matriz de distâncias"}\NormalTok{)}
\FunctionTok{curve}\NormalTok{(}\FunctionTok{dnorm}\NormalTok{(x, }\AttributeTok{mean=}\FunctionTok{mean}\NormalTok{(m\_dists), }\AttributeTok{sd=}\FunctionTok{sd}\NormalTok{(m\_dists)), }\AttributeTok{add=}\ConstantTok{TRUE}\NormalTok{) }\CommentTok{\# sobrepõe a curva de normalidade}
\FunctionTok{boxplot.default}\NormalTok{(m\_dists, }\AttributeTok{horizontal =} \ConstantTok{TRUE}\NormalTok{, }\AttributeTok{frame =} \ConstantTok{FALSE}\NormalTok{,}
                \AttributeTok{xlab=}\StringTok{"Distr. de Frequências"}\NormalTok{,}
                \AttributeTok{ylim=}\FunctionTok{c}\NormalTok{(}\FunctionTok{floor}\NormalTok{(}\FunctionTok{min}\NormalTok{(m\_dists)), }\FunctionTok{ceiling}\NormalTok{(}\FunctionTok{max}\NormalTok{(m\_dists)))) }\CommentTok{\# limites dos eixos}
\DocumentationTok{\#\# dev.off()}
\FunctionTok{library}\NormalTok{(vegan)}
\NormalTok{m\_relcol }\OtherTok{\textless{}{-}} \FunctionTok{decostand}\NormalTok{(m\_trab,}
                         \AttributeTok{method=}\StringTok{"total"}\NormalTok{,}
                         \AttributeTok{MARGIN =} \DecValTok{2}\NormalTok{)}
\CommentTok{\#View(m\_relcol)}
\NormalTok{m\_relcol}
\CommentTok{\#colSums(m\_relcol)}
\CommentTok{\#range(m\_relcol)}
\NormalTok{m\_relcol\_asrq }\OtherTok{\textless{}{-}} \FunctionTok{asin}\NormalTok{(}\FunctionTok{sqrt}\NormalTok{(m\_relcol))}
\CommentTok{\#View(m\_relcol\_asrq)}
\NormalTok{m\_relcol\_asrqpi }\OtherTok{\textless{}{-}} \DecValTok{2}\SpecialCharTok{/}\NormalTok{pi}\SpecialCharTok{*}\NormalTok{(}\FunctionTok{asin}\NormalTok{(}\FunctionTok{sqrt}\NormalTok{(m\_relcol)))}
\CommentTok{\#View(m\_relcol\_asrqpi)}
\CommentTok{\#colSums(m\_relcol\_asrqpi)}
\CommentTok{\#range(m\_relcol\_asrqpi)}
\NormalTok{m\_trns }\OtherTok{\textless{}{-}}\NormalTok{ m\_relcol\_asrqpi}
\CommentTok{\#m\_dists \textless{}{-} dist(m\_trns, method = "euclidian", diag = TRUE, upper = FALSE)}
\DocumentationTok{\#\# m\_relcol \textless{}{-} decostand(m\_bruta,}
\DocumentationTok{\#\#                          method="total",}
\DocumentationTok{\#\#                          MARGIN = 2)}
\DocumentationTok{\#\# View(m\_relcol)}
\DocumentationTok{\#\# m\_relcol}
\DocumentationTok{\#\# m\_lg10 \textless{}{-} log10(m\_bruta)}
\DocumentationTok{\#\# \#View(m\_lg10)}
\DocumentationTok{\#\# m\_lg10 \textless{}{-} replace(m\_lg10, is.infinite(m\_lg10), 0)}
\DocumentationTok{\#\# head(m\_lg10)}
\DocumentationTok{\#\# m\_lgx1 \textless{}{-} log(m\_bruta+1)}
\DocumentationTok{\#\# \#View(m\_lgx1)}
\DocumentationTok{\#\# m\_bruta\_P1 \textless{}{-} m\_bruta\^{}1 \#p=1}
\DocumentationTok{\#\# \#View(m\_bruta\_P1)}
\DocumentationTok{\#\# m\_bruta\_P05 \textless{}{-} m\_bruta\^{}0.5 \#p=0,5}
\DocumentationTok{\#\# \#View(m\_bruta\_P05)}
\DocumentationTok{\#\# m\_bruta\_P01 \textless{}{-} m\_bruta\^{}0.1 \#p=0,1}
\DocumentationTok{\#\# \#View(m\_bruta\_P01)}
\DocumentationTok{\#\# m\_pa \textless{}{-} (m\_bruta\textgreater{}0)*1L}
\DocumentationTok{\#\# \#View(m\_pa)}
\DocumentationTok{\#\# m\_rq \textless{}{-} sqrt(m\_bruta)}
\DocumentationTok{\#\# \#View(m\_rq)}
\DocumentationTok{\#\# m\_asrq \textless{}{-} asin(sqrt(m\_relcol))}
\DocumentationTok{\#\# \#View(m\_asrq)}
\DocumentationTok{\#\# m\_asrqpi \textless{}{-} 2/pi*(asin(sqrt(m\_relcol)))}
\DocumentationTok{\#\# View(m\_asrqpi)}
\end{Highlighting}
\end{Shaded}

\hypertarget{referuxeancias-12}{%
\section*{Referências}\label{referuxeancias-12}}
\addcontentsline{toc}{section}{Referências}

\hypertarget{outcentr}{%
\chapter{R Módulo 6 - Análise de outliers baseada no desvio padrão do centróide}\label{outcentr}}

\hypertarget{apresentauxe7uxe3o-12}{%
\subsection*{Apresentação}\label{apresentauxe7uxe3o-12}}
\addcontentsline{toc}{subsection}{Apresentação}

A análise de outliers baseada no desvio padrão do centróide em análise multivariada é uma técnica utilizada para identificar valores discrepantes em uma matriz de dados com várias variáveis. Essa técnica consiste em calcular o centróide dos dados (ou seja, a média de cada variável), e então calcular o desvio padrão de cada observação em relação ao centróide. Os valores que estão a uma distância maior que um certo número de desvios padrão do centróide são considerados outliers. Essa técnica é útil para identificar observações que podem estar afetando a análise de dados multivariados, como análise de componentes principais, e que podem precisar ser tratadas de forma especial. No entanto, é importante lembrar que a identificação de outliers é uma questão subjetiva e depende do contexto da análise e do objetivo do estudo.

\hypertarget{sobre-os-dados-do-ppbio-3}{%
\section{Sobre os dados do PPBio}\label{sobre-os-dados-do-ppbio-3}}

Usaremos para esse tutorial dados coletados no Programa de Pesquisa em Biodiversidade - PPBio (Veja \href{https://ppbio.inpa.gov.br/Sobre}{Programa de Pesquisa em Biodiversidade -- PPBio}). Nesta base de dados estão armazenadas informações sobre diversos grupos taxonômicos dstribuidos em diversas unidades amostrais (UA's ou sítios), como peixes, macroinverbrebrados bentônicos, quironomídeos e zooplâncotn, além de dados do habitat, como variáveis físicas e químicas, morfologia do habitat, composição do substrato, estrutura de habitat marginal, entre outros (Figura \ref{fig:6ppbio}).

\begin{figure}
\includegraphics[width=13.21in]{imagens/ppbio} \caption{Parte da planilha de dados brutos do PPBio.}\label{fig:6ppbio}
\end{figure}

Essa é a \textbf{matriz bruta de dados}, porque os valores ainda não foram ajustados para os valores de Captura Por Unidade de Esforço (CPUE), nem foram relativizados ou transformados (Tabela \ref{tab:6arqs}). As matrizes disponíveis para análises, com suas descrições e tipos de dados recomendados são mostradas na Tabela \ref{tab:6arqs}.

\begin{verbatim}
## Warning in table_info$align_vector[column] <-
## unlist(lapply(table_info$align_vector_origin[column], : número de itens para
## para substituir não é um múltiplo do comprimento do substituto
\end{verbatim}

\begin{table}

\caption{\label{tab:6arqs}Exemplos de matrizes disponíveis para análises (PPBio Semiárido), com suas descrições e tipos de dados recomendados.}
\centering
\begin{tabular}[t]{>{\raggedright\arraybackslash}p{2cm}>{\raggedright\arraybackslash}p{2cm}>{\raggedright\arraybackslash}p{7cm}>{\raggedright\arraybackslash}p{5cm}}
\toprule
Arquivo (.xlsx) & Tipo de matriz & Descrição & Tipo de dados\\
\midrule
ppbio06c & Matriz comunitária & O arquivo ppbio06 traz os dados brutos que serão usados nas análises. A matriz de dados brutos contendo 26 localidades em estações do ano diferentes (objetos) x 35 espécies (atributos), antes de qualquer modificação. & Contagens de indivíduos com alta amplitude de variação, sugerido uso de matriz relativizada.\\
ppbio06h & Matriz ambiental & O arquivo ppbio06h traz os dados brutos que serão usados nas análises. A matriz de dados brutos contendo 26 localidades em estações diferentes (objetos) x 35 variáveis ambienteis (atributos) medidas em diferentes escalas espaciais, antes de qualquer modificação. & Unidades de medição diferentes (cm, m, °C, mg/L, etc.), com uma alta amplitude de variação, sugerido uso de matriz transformada e/ou reescalada.\\
ppbio06 & Matriz comunitária & O arquivo ppbio06 traz os dados brutos que serão usados nessa análise. A matriz de dados brutos contendo 26 locais/ocasiões (objetos) x 35 espécies (atributos), antes de qualquer modificação. & Contagens de indivíduos com alta amplitude de variação, sugerido uso de matriz relativizada.\\
ppbio06cpue & Matriz comunitária & O arquivo ppbio06cpue traz os valores depois de terem sidos ajustados pela Captura Por Unidade de Esforço (CPUE), onde o número de indivíduos de cada espécie em uma determinada UA é dividido pelo esforço de captura daquela UA. Isso significa que os dados foram relativizados pela CPUE. A matriz de dados brutos contendo 26 localidades em estações do ano diferentes (objetos) x 35 espécies (atributos), antes de qualquer modificação. & Densidades de indivíduos (no. de indivíduos por Unidade de Esforço de Captura) com alta amplitude de variação, sugerido uso de matriz relativizada.\\
\bottomrule
\end{tabular}
\end{table}

A planilha \texttt{ppbio} contém o delineamento amostral de um dos estudos do Projeto PPBio (Figura \ref{fig:6rm1}. Nas linhas são apresentadas as abreviações dos nomes das unidades amostrais (UA's) e nas colunas são apresentados os nomes abreviados das espécies - temos portando uma matriz comunitária (\ref{tab:6arqs}. No corpo da planilha temos os valores para o tipo de dados amostrado. Quantitativo, semi-quatitativo ou qualitativo.\\
Qual desses tipos de dados você acha que é apresentado na planilha?

\begin{figure}
\includegraphics[width=15.44in]{imagens/rm1.fig3} \caption{Associação entre a planilha de dados brutos do PPBio e o delineamento amostral do estudo.}\label{fig:6rm1}
\end{figure}

Várias das espécies nessa matriz tem grande importância ecológica, como é o caso de \emph{Astyanax bimaculatus} \footnote{A etimologia do gênero \emph{Astyanax} vem da mitologia Grega. Heitor personagem da ``Ilíada'', tinha um filho chamado Astíanax.} (Figura \ref{fig:6asbim}), que é muito comum em rios intermitentes e serve de alimento para predadores maiores como a espécie \emph{Hoplias malabaricus} \footnote{Do Grego, \emph{hoplon}, arma ou armadura, em referência aos dentes caniniformes muito desenvolvidos, e forte estrutura óssea na cabeça.} (Figura \ref{fig:6hmala}).

\begin{figure}
\includegraphics[width=8.89in]{imagens/Asbim_u4} \caption{*Astyanax bimaculatus*, a espécie mais comum da matriz de dados ppbio. Peru, by Eakins, R. Fonte: <https://www.fishbase.se/summary/Astianax-bimaculatus.html>}\label{fig:6asbim}
\end{figure}

\begin{figure}
\includegraphics[width=8.89in]{imagens/Homal_u6} \caption{*Hoplias malabaricus*, espécie que cresce para se tornar um importante predador. Brazil, by Roselet, F.F.G. Fonte: <https://www.fishbase.se/summary/Hoplias-malabaricus.html>}\label{fig:6hmala}
\end{figure}

\hypertarget{organizauxe7uxe3o-buxe1sica-9}{%
\section{Organização básica}\label{organizauxe7uxe3o-buxe1sica-9}}

\begin{Shaded}
\begin{Highlighting}[]
\FunctionTok{dev.off}\NormalTok{() }\CommentTok{\#apaga os graficos, se houver algum}
\FunctionTok{rm}\NormalTok{(}\AttributeTok{list=}\FunctionTok{ls}\NormalTok{(}\AttributeTok{all=}\ConstantTok{TRUE}\NormalTok{)) }\CommentTok{\#limpa a memória}
\FunctionTok{cat}\NormalTok{(}\StringTok{"}\SpecialCharTok{\textbackslash{}014}\StringTok{"}\NormalTok{) }\CommentTok{\#limpa o console}
\end{Highlighting}
\end{Shaded}

Instalando os pacotes necessários para esse módulo

\begin{Shaded}
\begin{Highlighting}[]
\FunctionTok{install.packages}\NormalTok{(}\StringTok{"openxlsx"}\NormalTok{) }\CommentTok{\#importa arquivos do excel}
\FunctionTok{install.packages}\NormalTok{(}\StringTok{"moments"}\NormalTok{) }\CommentTok{\#calcula assimetria e curtose dos dados}
\FunctionTok{install.packages}\NormalTok{(}\StringTok{"matrixStats"}\NormalTok{) }\CommentTok{\#fornece funções rápidas para a estatística de matrizes}
\FunctionTok{install.packages}\NormalTok{(}\StringTok{"gt"}\NormalTok{) }\CommentTok{\#ferramenta para criação de tabelas bonitas e personalizáveis}
\end{Highlighting}
\end{Shaded}

Agora vamos \textbf{definir o diretório de trabalho}. Esse código é usado para obter e definir o diretório de trabalho atual no R. O comando \texttt{getwd()} retorna o caminho do diretório onde o R está lendo e salvando arquivos. O comando \texttt{setwd()} muda esse diretório de trabalho para o caminho especificado entre aspas. No seu caso, você deve ajustar o caminho para o seu próprio diretório de trabalho. \textbf{Lembre de usar a barra ``/'' entre os diretórios. E não a contra-barra ``\textbackslash{}''.}

Definindo o diretório de trabalho e installando os pacotes necessários:

\begin{Shaded}
\begin{Highlighting}[]
\FunctionTok{getwd}\NormalTok{()}
\FunctionTok{setwd}\NormalTok{(}\StringTok{"C:/Seu/Diretório/De/Trabalho"}\NormalTok{)}
\end{Highlighting}
\end{Shaded}

\hypertarget{importando-a-planilha-8}{%
\subsection{Importando a planilha}\label{importando-a-planilha-8}}

Note que o símbolo {\texttt{\#}} em programação R significa que o texto que vem depois dele é um comentário e não será executado pelo programa. Isso é útil para explicar o código ou deixar anotações.\\
- Ajuste a primeira linha do código abaixo para refletir {``C:/Seu/Diretório/De/Trabalho/Planilha.xlsx''}.\\
- Ajuste o parâmetro \texttt{sheet\ =\ "Sheet1"} para refletir a aba correta do arquivo .xlsx a ser importado.

Alternativamente você pode ir na barra de tarefas e escolhes as opções:\\
SESSION -\textgreater{} SET WORKING DIRECTORY -\textgreater{} CHOOSE DIRECTORY

\begin{Shaded}
\begin{Highlighting}[]
\FunctionTok{library}\NormalTok{(openxlsx)}
\NormalTok{m\_bruta }\OtherTok{\textless{}{-}} \FunctionTok{read.xlsx}\NormalTok{(}\StringTok{"D:/Elvio/OneDrive/Disciplinas/\_EcoNumerica/5.Matrizes/ppbio06{-}peixes.xlsx"}\NormalTok{,}
                   \AttributeTok{rowNames =}\NormalTok{ T, }\AttributeTok{colNames =}\NormalTok{ T,}
                   \AttributeTok{sheet =} \StringTok{"Sheet1"}\NormalTok{)}
\FunctionTok{str}\NormalTok{(m\_bruta)}
\NormalTok{m\_bruta\_ma }\OtherTok{\textless{}{-}} \FunctionTok{as.matrix}\NormalTok{(m\_bruta) }\CommentTok{\#lê m\_bruta como uma matriz}
\FunctionTok{str}\NormalTok{(m\_bruta\_ma)}
\CommentTok{\#m\_bruta}
\NormalTok{m\_bruta\_ma[}\DecValTok{1}\SpecialCharTok{:}\DecValTok{5}\NormalTok{,}\DecValTok{1}\SpecialCharTok{:}\DecValTok{5}\NormalTok{] }\CommentTok{\#[1:5,1:5] mostra apenas as linhas e colunas de 1 a 5.}
\end{Highlighting}
\end{Shaded}

\begin{verbatim}
## 'data.frame':    26 obs. of  35 variables:
##  $ ap-davis : num  0 0 0 0 0 0 0 0 0 0 ...
##  $ as-bimac : num  1 99 194 19 23 142 5 46 206 16 ...
##  $ as-fasci : num  0 0 55 0 1 3 1 0 64 0 ...
##  $ ch-bimac : num  0 0 0 0 13 3 0 178 0 0 ...
##  $ ci-ocela : num  0 0 0 0 0 0 40 0 0 13 ...
##  $ ci-orien : num  0 0 5 0 0 69 9 0 25 24 ...
##  $ co-macro : num  0 0 0 0 0 0 0 0 0 0 ...
##  $ co-heter : num  0 0 1 0 0 0 0 0 0 0 ...
##  $ cr-menez : num  0 0 14 0 0 4 0 0 8 0 ...
##  $ cu-lepid : num  0 0 0 0 0 0 0 0 0 0 ...
##  $ cy-gilbe : num  0 0 0 0 0 0 0 0 0 0 ...
##  $ ge-brasi : num  0 0 3 0 0 0 0 0 1 0 ...
##  $ he-margi : num  0 0 0 0 0 1 0 0 0 0 ...
##  $ ho-malab : num  0 0 1 5 0 17 10 2 31 4 ...
##  $ hy-pusar : num  0 0 9 2 0 43 2 0 11 0 ...
##  $ le-melan : num  0 0 0 0 0 0 0 0 0 0 ...
##  $ le-piau  : num  0 0 3 0 0 1 3 0 2 1 ...
##  $ le-taeni : num  0 0 0 0 0 0 0 0 0 0 ...
##  $ mo-costa : num  0 0 0 0 0 0 0 0 0 0 ...
##  $ mo-lepid : num  0 1 39 0 0 1 0 0 0 0 ...
##  $ or-nilot : num  0 2 36 0 0 77 0 0 138 0 ...
##  $ pa-manag : num  0 0 0 0 0 0 0 0 0 0 ...
##  $ pimel-sp : num  0 0 6 0 0 0 0 0 0 0 ...
##  $ po-retic : num  0 0 0 0 0 20 0 0 5 0 ...
##  $ po-vivip : num  0 0 47 15 0 221 32 0 326 10 ...
##  $ pr-brevi : num  9 0 5 0 1 15 5 2 164 0 ...
##  $ ps-rhomb : num  0 0 0 0 0 0 0 0 1 0 ...
##  $ ps-genise: num  0 0 0 0 0 0 0 0 1 0 ...
##  $ se-heter : num  0 0 40 14 4 60 0 0 38 0 ...
##  $ se-piaba : num  0 0 68 0 0 0 0 0 0 0 ...
##  $ se-spilo : num  0 0 0 0 0 0 0 0 1 0 ...
##  $ st-noton : num  0 0 1 0 0 25 0 0 115 0 ...
##  $ sy-marmo : num  0 0 0 0 0 0 1 0 0 0 ...
##  $ te-chalc : num  0 0 0 0 0 0 0 0 0 0 ...
##  $ tr-signa : num  0 0 18 0 0 15 0 0 7 0 ...
##  num [1:26, 1:35] 0 0 0 0 0 0 0 0 0 0 ...
##  - attr(*, "dimnames")=List of 2
##   ..$ : chr [1:26] "S-A-ZA1" "S-R-CC1" "S-R-CT1" "S-R-CP1" ...
##   ..$ : chr [1:35] "ap-davis" "as-bimac" "as-fasci" "ch-bimac" ...
##         ap-davis as-bimac as-fasci ch-bimac ci-ocela
## S-A-ZA1        0        1        0        0        0
## S-R-CC1        0       99        0        0        0
## S-R-CT1        0      194       55        0        0
## S-R-CP1        0       19        0        0        0
## S-A-TA1        0       23        1       13        0
\end{verbatim}

\hypertarget{reset-point-3}{%
\section{Reset point}\label{reset-point-3}}

\leavevmode\vadjust pre{\hypertarget{hello}{}}%
{ATENÇÃO} Aqui substitui-se uma nova matriz de dados, relativizada e/ou transformada, pela matriz de trabalho inicial.

\begin{Shaded}
\begin{Highlighting}[]
\CommentTok{\#m\_bruta \textless{}{-} (m\_bruta)   \# \textless{}1\textgreater{}}
\end{Highlighting}
\end{Shaded}

\begin{enumerate}
\def\labelenumi{\arabic{enumi}.}
\tightlist
\item
  Aqui usaremos as matrizes relativizadas/transformadas/particionadas, etc
\end{enumerate}

Podemos exibir a planilha depois de ter sido importada para o ambiente R/RStudio usando as funções \texttt{View()}, \texttt{print()} ou \texttt{head()}. Note que essas funções são case-sensitive. A função \texttt{ignore.case()} é uma função do pacote stringr que modifica um padrão para que ele não considere o caso das letras nas correspondências. Por exemplo, se você quiser encontrar todas as ocorrências da letra ``a'' em um vetor de caracteres, independente de ser ``A'' ou ``a'', você pode usar essa função.

\begin{Shaded}
\begin{Highlighting}[]
\CommentTok{\#View(m\_bruta)}
\FunctionTok{print}\NormalTok{(m\_bruta)}
\FunctionTok{head}\NormalTok{(m\_bruta)}
\end{Highlighting}
\end{Shaded}

A função \texttt{head()} no RStudio é uma forma de ver as primeiras (n=6) linhas de um objeto, como um vetor, uma matriz, um data frame ou uma lista. Ela é útil para ter uma ideia do conteúdo e da estrutura do objeto.

Também podemos explorar as características da planilha usando as funções \texttt{str()}, \texttt{mode()}, \texttt{class()} e \texttt{length()}. O número de observações ou tamanho do vetor depende do tipo de dados, se eles são uma \texttt{matrix} ou um \texttt{data.frame}.

\begin{Shaded}
\begin{Highlighting}[]
\FunctionTok{str}\NormalTok{(m\_bruta)}
\FunctionTok{mode}\NormalTok{(m\_bruta)}
\FunctionTok{class}\NormalTok{(m\_bruta)}
\CommentTok{\#?str}
\end{Highlighting}
\end{Shaded}

\begin{verbatim}
## 'data.frame':    26 obs. of  35 variables:
##  $ ap-davis : num  0 0 0 0 0 0 0 0 0 0 ...
##  $ as-bimac : num  1 99 194 19 23 142 5 46 206 16 ...
##  $ as-fasci : num  0 0 55 0 1 3 1 0 64 0 ...
##  $ ch-bimac : num  0 0 0 0 13 3 0 178 0 0 ...
##  $ ci-ocela : num  0 0 0 0 0 0 40 0 0 13 ...
##  $ ci-orien : num  0 0 5 0 0 69 9 0 25 24 ...
##  $ co-macro : num  0 0 0 0 0 0 0 0 0 0 ...
##  $ co-heter : num  0 0 1 0 0 0 0 0 0 0 ...
##  $ cr-menez : num  0 0 14 0 0 4 0 0 8 0 ...
##  $ cu-lepid : num  0 0 0 0 0 0 0 0 0 0 ...
##  $ cy-gilbe : num  0 0 0 0 0 0 0 0 0 0 ...
##  $ ge-brasi : num  0 0 3 0 0 0 0 0 1 0 ...
##  $ he-margi : num  0 0 0 0 0 1 0 0 0 0 ...
##  $ ho-malab : num  0 0 1 5 0 17 10 2 31 4 ...
##  $ hy-pusar : num  0 0 9 2 0 43 2 0 11 0 ...
##  $ le-melan : num  0 0 0 0 0 0 0 0 0 0 ...
##  $ le-piau  : num  0 0 3 0 0 1 3 0 2 1 ...
##  $ le-taeni : num  0 0 0 0 0 0 0 0 0 0 ...
##  $ mo-costa : num  0 0 0 0 0 0 0 0 0 0 ...
##  $ mo-lepid : num  0 1 39 0 0 1 0 0 0 0 ...
##  $ or-nilot : num  0 2 36 0 0 77 0 0 138 0 ...
##  $ pa-manag : num  0 0 0 0 0 0 0 0 0 0 ...
##  $ pimel-sp : num  0 0 6 0 0 0 0 0 0 0 ...
##  $ po-retic : num  0 0 0 0 0 20 0 0 5 0 ...
##  $ po-vivip : num  0 0 47 15 0 221 32 0 326 10 ...
##  $ pr-brevi : num  9 0 5 0 1 15 5 2 164 0 ...
##  $ ps-rhomb : num  0 0 0 0 0 0 0 0 1 0 ...
##  $ ps-genise: num  0 0 0 0 0 0 0 0 1 0 ...
##  $ se-heter : num  0 0 40 14 4 60 0 0 38 0 ...
##  $ se-piaba : num  0 0 68 0 0 0 0 0 0 0 ...
##  $ se-spilo : num  0 0 0 0 0 0 0 0 1 0 ...
##  $ st-noton : num  0 0 1 0 0 25 0 0 115 0 ...
##  $ sy-marmo : num  0 0 0 0 0 0 1 0 0 0 ...
##  $ te-chalc : num  0 0 0 0 0 0 0 0 0 0 ...
##  $ tr-signa : num  0 0 18 0 0 15 0 0 7 0 ...
## [1] "list"
## [1] "data.frame"
\end{verbatim}

O símbolo \texttt{?} é usado para acessar a documentação de uma função ou um pacote no R. Como mostrado acima você pode saber mais sobre a função \texttt{str()}, usando o comando \texttt{?str}. Isso vai abrir uma página no menu de ajuda com a descrição, os argumentos, os valores de retorno e os exemplos da função \texttt{str()}. Você também pode usar o símbolo \texttt{?} para obter informações sobre um pacote inteiro. Por exemplo, se você quiser saber mais sobre o pacote \texttt{openxlsx}, você pode digitar \texttt{?openxlsx}. Isso vai abrir uma página com a visão geral, a instalação, os recursos e as referências do pacote solicitado.

Podemos agora calcular o número e a proporção de zeros na matriz usando as funções \texttt{sum()} e \texttt{length()} (Você pode pesquisar o que faz a função \texttt{length()} usando o comando \texttt{?length}).

\hypertarget{tamanho-da-matriz-1}{%
\subsection{Tamanho da matriz}\label{tamanho-da-matriz-1}}

\begin{Shaded}
\begin{Highlighting}[]
\FunctionTok{range}\NormalTok{(m\_bruta) }\CommentTok{\#menor e maior valores}
\FunctionTok{length}\NormalTok{(m\_bruta) }\CommentTok{\#no. de colunas}
\FunctionTok{ncol}\NormalTok{(m\_bruta) }\CommentTok{\#no. de N colunas}
\FunctionTok{nrow}\NormalTok{(m\_bruta) }\CommentTok{\#no. de M linhas}
\FunctionTok{sum}\NormalTok{(}\FunctionTok{lengths}\NormalTok{(m\_bruta)) }\CommentTok{\#soma os nos. de colunas}
\FunctionTok{length}\NormalTok{(}\FunctionTok{as.matrix}\NormalTok{(m\_bruta)) }\CommentTok{\#tamanho da matriz m x n}
\FunctionTok{sum}\NormalTok{(m\_bruta }\SpecialCharTok{==} \DecValTok{0}\NormalTok{) }\CommentTok{\# número de observações igual a zero}
\FunctionTok{sum}\NormalTok{(m\_bruta }\SpecialCharTok{\textgreater{}} \DecValTok{0}\NormalTok{) }\CommentTok{\# número de observações maiores que zero}
\NormalTok{zeros }\OtherTok{\textless{}{-}}\NormalTok{ (}\FunctionTok{sum}\NormalTok{(m\_bruta }\SpecialCharTok{==} \DecValTok{0}\NormalTok{)}\SpecialCharTok{/}\FunctionTok{length}\NormalTok{(}\FunctionTok{as.matrix}\NormalTok{(m\_bruta)))}\SpecialCharTok{*}\DecValTok{100} \CommentTok{\# proporção de zeros na matriz}
\NormalTok{zeros}
\end{Highlighting}
\end{Shaded}

\begin{verbatim}
## [1]   0 511
## [1] 35
## [1] 35
## [1] 26
## [1] 910
## [1] 910
## [1] 716
## [1] 194
## [1] 78.68132
\end{verbatim}

A matriz de dados apresenta um total de 910 valores que variam entre 0, 511 (menor e maior valores). A matriz m x n tem 26 linhas e 35 colunas. Existem 716 observações iguais a zero e 194 observações maiores que zero, representando um percentual de 78.7\% dos valores sendo zeros.

Essas informações podem ser resumidas na Tabela \ref{tab:6tbl-tam} que será gerada abaixo.

\begin{Shaded}
\begin{Highlighting}[]
\NormalTok{tamanho }\OtherTok{\textless{}{-}} \FunctionTok{data.frame}\NormalTok{(}
  \AttributeTok{Comando =} \FunctionTok{c}\NormalTok{(}\StringTok{"range"}\NormalTok{, }\StringTok{"lenght"}\NormalTok{, }\StringTok{"m cols"}\NormalTok{, }\StringTok{"n linhas"}\NormalTok{, }\StringTok{"Tamanho"}\NormalTok{, }\StringTok{"Tamanho"}\NormalTok{,}
             \StringTok{"Zeros"}\NormalTok{, }\StringTok{"Nao zeros"}\NormalTok{, }\StringTok{"\% Zeros"}\NormalTok{),}
  \AttributeTok{Resultado =} \FunctionTok{c}\NormalTok{(}\FunctionTok{paste}\NormalTok{(}\FunctionTok{range}\NormalTok{(m\_bruta), }\AttributeTok{collapse =} \StringTok{" {-} "}\NormalTok{), }\FunctionTok{length}\NormalTok{(m\_bruta), }\FunctionTok{ncol}\NormalTok{(m\_bruta),}
                \FunctionTok{nrow}\NormalTok{(m\_bruta), }\FunctionTok{sum}\NormalTok{(}\FunctionTok{lengths}\NormalTok{(m\_bruta)), }\FunctionTok{length}\NormalTok{(}\FunctionTok{as.matrix}\NormalTok{(m\_bruta)), }\FunctionTok{sum}\NormalTok{(m\_bruta }\SpecialCharTok{==} \DecValTok{0}\NormalTok{),}
                \FunctionTok{sum}\NormalTok{(m\_bruta }\SpecialCharTok{\textgreater{}} \DecValTok{0}\NormalTok{), }\FunctionTok{round}\NormalTok{(zeros, }\DecValTok{1}\NormalTok{))}
\NormalTok{)}
\NormalTok{tamanho}
\end{Highlighting}
\end{Shaded}

\begin{verbatim}
##     Comando Resultado
## 1     range   0 - 511
## 2    lenght        35
## 3    m cols        35
## 4  n linhas        26
## 5   Tamanho       910
## 6   Tamanho       910
## 7     Zeros       716
## 8 Nao zeros       194
## 9   % Zeros      78.7
\end{verbatim}

\begin{longtable}[]{@{}ll@{}}
\caption{\label{tab:6tbl-tam}Resumo das informações sobre o tamanho da matriz}\tabularnewline
\toprule()
Comando & Resultado \\
\midrule()
\endfirsthead
\toprule()
Comando & Resultado \\
\midrule()
\endhead
range & 0 - 511 \\
lenght & 35 \\
m cols & 35 \\
n linhas & 26 \\
Tamanho & 910 \\
Tamanho & 910 \\
Zeros & 716 \\
Nao zeros & 194 \\
\% Zeros & 78.7 \\
\bottomrule()
\end{longtable}

Ou seja, temos uma matriz de tamanho \texttt{m\ x\ n} igual a 26 objetos por 35 atributos, onde 78.68\% dos valores da matriz são iguais a zero!

\hypertarget{cuxe1lculo-de-matriz-de-distuxe2ncias-euclidiana-euclid}{%
\section{\texorpdfstring{Cálculo de matriz de distâncias Euclidiana (\texttt{euclid})}{Cálculo de matriz de distâncias Euclidiana (euclid)}}\label{cuxe1lculo-de-matriz-de-distuxe2ncias-euclidiana-euclid}}

Agora vamos calcular a matriz de \textbf{distâncias euclidianas} (\protect\hyperlink{ref-RN1336}{BORCARD; GILLET; LEGENDRE, 2018}; \protect\hyperlink{ref-RN1278}{LEGENDRE; LEGENDRE, 1998}) entre as UA´s usando a função \texttt{dist()} (pesquise o que faz essa função usando o comando \texttt{?dist}).

\begin{Shaded}
\begin{Highlighting}[]
\NormalTok{euclid }\OtherTok{\textless{}{-}} \FunctionTok{dist}\NormalTok{(m\_bruta, }\AttributeTok{method =} \StringTok{"euclidian"}\NormalTok{, }\AttributeTok{diag =} \ConstantTok{TRUE}\NormalTok{, }\AttributeTok{upper =} \ConstantTok{FALSE}\NormalTok{)}
\CommentTok{\#?dist}
\end{Highlighting}
\end{Shaded}

Pronto, calculamos a matriz de distâncias (Euclidiana). Agora podemos visualizar a matriz.

\begin{Shaded}
\begin{Highlighting}[]
\CommentTok{\#euclid}
\FunctionTok{str}\NormalTok{(euclid)}
\FunctionTok{mode}\NormalTok{(euclid)}
\FunctionTok{class}\NormalTok{(euclid)}
\FunctionTok{length}\NormalTok{(}\FunctionTok{as.matrix}\NormalTok{(euclid))}
\FunctionTok{as.matrix}\NormalTok{(euclid)[}\DecValTok{1}\SpecialCharTok{:}\DecValTok{6}\NormalTok{, }\DecValTok{1}\SpecialCharTok{:}\DecValTok{6}\NormalTok{] }\CommentTok{\#mostra as 5 primeiras linhas e colunas da matriz}
\end{Highlighting}
\end{Shaded}

\begin{verbatim}
##  'dist' num [1:325] 98.4 228.5 29.2 27.1 294.1 ...
##  - attr(*, "Size")= int 26
##  - attr(*, "Labels")= chr [1:26] "S-A-ZA1" "S-R-CC1" "S-R-CT1" "S-R-CP1" ...
##  - attr(*, "Diag")= logi TRUE
##  - attr(*, "Upper")= logi FALSE
##  - attr(*, "method")= chr "euclidean"
##  - attr(*, "call")= language dist(x = m_bruta, method = "euclidian", diag = TRUE, upper = FALSE)
## [1] "numeric"
## [1] "dist"
## [1] 676
##           S-A-ZA1   S-R-CC1  S-R-CT1   S-R-CP1   S-A-TA1  S-R-CT2
## S-A-ZA1   0.00000  98.43780 228.5235  29.24038  27.09243 294.0629
## S-R-CC1  98.43780   0.00000 154.2433  82.79493  77.25283 261.3905
## S-R-CT1 228.52352 154.24331   0.0000 208.52338 209.69263 224.9133
## S-R-CP1  29.24038  82.79493 208.5234   0.00000  23.25941 271.4922
## S-A-TA1  27.09243  77.25283 209.6926  23.25941   0.00000 283.7869
## S-R-CT2 294.06292 261.39051 224.9133 271.49217 283.78689   0.0000
\end{verbatim}

Note que a matriz \texttt{euclid} está como uma classe \texttt{dist}. Uma matriz do tipo \texttt{dist} no R é um objeto que armazena as distâncias entre as linhas de uma matriz ou um data frame. Ela é criada pela função \texttt{dist()}, que calcula as distâncias usando diferentes medidas, como ``euclidean'', ``manhattan'', ``canberra'', ``binary'' ou ``minkowski'' (\protect\hyperlink{ref-RN419}{HORTON; KLEINMAN, 2015}) (Figura \ref{tab:6tbl-dists}).

Visualizando a matriz de distâncias, observamos que ela:

\begin{itemize}
\item
  é uma matriz quadrada que contém as distâncias entre cada par de elementos do conjuto de dados,
\item
  terá a dimensão m x m (m=linhas, n=colunas) e,
\item
  e cada elemento da matriz será a distância entre cada par de observações ou objetos.
\end{itemize}

A \textbf{matriz de distâncias é simétrica}, pois a distância entre i e j é igual à distância entre j e i. A matriz de distâncias também tem \textbf{diagonal zero}, pois a distância entre uma observação e ela mesma é zero.

\begin{verbatim}
## Warning in table_info$align_vector[column] <-
## unlist(lapply(table_info$align_vector_origin[column], : número de itens para
## para substituir não é um múltiplo do comprimento do substituto
\end{verbatim}

\begin{table}[!h]

\caption{\label{tab:6tbl-dists}Medidas de distância disponíves na função `dist()` para calcular a distância entre dois pontos ou objetos em um espaço multidimensionl.}
\centering
\begin{tabular}[t]{>{\centering\arraybackslash}p{4cm}>{\centering\arraybackslash}p{5cm}}
\toprule
Distância & Descrição\\
\midrule
Euclidiana & Mede a distância entre dois pontos no espaço, seguindo o teorema de Pitágoras. É uma medida de distância direta entre dois pontos.\\
Manhattan & Também conhecida como distância da cidade, mede a distância como se estivesse dirigindo em uma cidade. A distância é a soma das diferenças absolutas de cada coordenada.\\
Canberra & Leva em consideração a proporção de diferenças em relação ao tamanho total das variáveis.\\
Binária & Mede a semelhança entre dois objetos. É 0 quando dois valores são iguais e 1 caso contrário.\\
Minkowski & É uma generalização da distância euclidiana e da distância de Manhattan. A distância é controlada por um parâmetro P, que determina a ordem da distância.\\
\bottomrule
\end{tabular}
\end{table}

A matriz \texttt{euclid} está como uma classe \texttt{dist}, mas podemos transformá-la em uma classe \texttt{matrix} usando a função \texttt{as.matrix()}.

\begin{Shaded}
\begin{Highlighting}[]
\NormalTok{euclid\_ma }\OtherTok{\textless{}{-}}\NormalTok{ (}\FunctionTok{as.matrix}\NormalTok{(euclid))}
\CommentTok{\#View(euclid\_ma)}
\FunctionTok{str}\NormalTok{(euclid\_ma)}
\FunctionTok{mode}\NormalTok{(euclid\_ma)}
\FunctionTok{class}\NormalTok{(euclid\_ma)}
\NormalTok{euclid\_ma[}\DecValTok{1}\SpecialCharTok{:}\DecValTok{5}\NormalTok{, }\DecValTok{1}\SpecialCharTok{:}\DecValTok{5}\NormalTok{] }\CommentTok{\#mostra as 5 primeiras linhas e colunas da matriz}
\end{Highlighting}
\end{Shaded}

\begin{verbatim}
##  num [1:26, 1:26] 0 98.4 228.5 29.2 27.1 ...
##  - attr(*, "dimnames")=List of 2
##   ..$ : chr [1:26] "S-A-ZA1" "S-R-CC1" "S-R-CT1" "S-R-CP1" ...
##   ..$ : chr [1:26] "S-A-ZA1" "S-R-CC1" "S-R-CT1" "S-R-CP1" ...
## [1] "numeric"
## [1] "matrix" "array" 
##           S-A-ZA1   S-R-CC1  S-R-CT1   S-R-CP1   S-A-TA1
## S-A-ZA1   0.00000  98.43780 228.5235  29.24038  27.09243
## S-R-CC1  98.43780   0.00000 154.2433  82.79493  77.25283
## S-R-CT1 228.52352 154.24331   0.0000 208.52338 209.69263
## S-R-CP1  29.24038  82.79493 208.5234   0.00000  23.25941
## S-A-TA1  27.09243  77.25283 209.6926  23.25941   0.00000
\end{verbatim}

\hypertarget{calculando-o-centruxf3ide-da-matriz-de-distuxe2ncias-euclid}{%
\section{\texorpdfstring{Calculando o centróide da matriz de distâncias \texttt{euclid}}{Calculando o centróide da matriz de distâncias euclid}}\label{calculando-o-centruxf3ide-da-matriz-de-distuxe2ncias-euclid}}

Primeiro algumas informações básicas da matriz

\begin{Shaded}
\begin{Highlighting}[]
\FunctionTok{range}\NormalTok{(m\_bruta)}
\FunctionTok{range}\NormalTok{(euclid)}
\FunctionTok{min}\NormalTok{(euclid)}
\FunctionTok{max}\NormalTok{(euclid)}
\FunctionTok{mean}\NormalTok{(euclid) }\CommentTok{\#CENTROIDE!! OU Grand mean}
\FunctionTok{sd}\NormalTok{(euclid)  }\CommentTok{\#Standard deviation}
\NormalTok{centroide }\OtherTok{\textless{}{-}} \FunctionTok{mean}\NormalTok{(euclid)}
\NormalTok{centroide}
\end{Highlighting}
\end{Shaded}

\begin{verbatim}
## [1]   0 511
## [1]  16.27882 707.32666
## [1] 16.27882
## [1] 707.3267
## [1] 285.9043
## [1] 172.9708
## [1] 285.9043
\end{verbatim}

A função \texttt{mean()} calcula a média de todos os valores da matriz de distâncias, ou seja, a média multivariada, que é o centróide. Nesse caso o \textbf{centroide assume o valor de 285.9}.

Usamos agora a fórmula \texttt{m*(m-1)/2}, onde \texttt{m} é o no. de objetos sendo comparados, para calcular quantas distâcias temos na nossa matriz.

\begin{Shaded}
\begin{Highlighting}[]
\FunctionTok{length}\NormalTok{(euclid)}
\NormalTok{m }\OtherTok{\textless{}{-}} \FunctionTok{nrow}\NormalTok{(m\_bruta)}
\NormalTok{m}
\NormalTok{m}\SpecialCharTok{*}\NormalTok{(m}\DecValTok{{-}1}\NormalTok{)}\SpecialCharTok{/}\DecValTok{2}
\FunctionTok{summary}\NormalTok{(euclid)}
\end{Highlighting}
\end{Shaded}

\begin{verbatim}
## [1] 325
## [1] 26
## [1] 325
##    Min. 1st Qu.  Median    Mean 3rd Qu.    Max. 
##   16.28  142.70  242.82  285.90  423.64  707.33
\end{verbatim}

Temos então que m é \textbf{26 objetos} (ou linhas), e portanto, a matriz de distâncias tem \textbf{325 valores}.

Fazemos agora um breve sumário do que foi calculado até agora com base na matriz de distâncias \texttt{euclid}.

\begin{Shaded}
\begin{Highlighting}[]
\NormalTok{Sumario1 }\OtherTok{\textless{}{-}} \FunctionTok{cbind}\NormalTok{(}\FunctionTok{min}\NormalTok{(euclid),}
                  \FunctionTok{max}\NormalTok{(euclid),}
                  \FunctionTok{sd}\NormalTok{(euclid),}
                  \FunctionTok{mean}\NormalTok{(euclid),}
                  \FunctionTok{length}\NormalTok{(euclid))}
\FunctionTok{colnames}\NormalTok{(Sumario1) }\OtherTok{\textless{}{-}} \FunctionTok{c}\NormalTok{(}\StringTok{"Minimo"}\NormalTok{, }\StringTok{"Maximo"}\NormalTok{, }\StringTok{"Desv.Padr"}\NormalTok{, }\StringTok{"Media"}\NormalTok{, }\StringTok{"m(m{-}1)/2"}\NormalTok{)}
\FunctionTok{rownames}\NormalTok{(Sumario1) }\OtherTok{\textless{}{-}}\NormalTok{ (}\StringTok{"Valores"}\NormalTok{)}
\NormalTok{Sumario1}
\end{Highlighting}
\end{Shaded}

\begin{verbatim}
##           Minimo   Maximo Desv.Padr    Media m(m-1)/2
## Valores 16.27882 707.3267  172.9708 285.9043      325
\end{verbatim}

\hypertarget{distribuiuxe7uxe3o-de-frequuxeancias-da-matriz-de-distuxe2ncias-euclid}{%
\section{\texorpdfstring{Distribuição de frequências da matriz de distâncias \texttt{euclid}}{Distribuição de frequências da matriz de distâncias euclid}}\label{distribuiuxe7uxe3o-de-frequuxeancias-da-matriz-de-distuxe2ncias-euclid}}

O código abaixo vai plotar um histograma e um boxplot da distribuição de dados armazenada em \texttt{euclid}. Também é adicionada uma curva normal teórica ao histograma utilizando a média e o desvio padrão dos dados (Figura \ref{fig:6euclid}). Atente que o comando \texttt{par(mfrow=c(2,1))} define o layout dos gráficos, especificando que serão plotados 2 gráficos em uma coluna. As funções \texttt{floor(min(euclid)),\ ceiling(max(euclid))} definem os limites dos graficos pelo valores mínimo e máximo do objeto (\texttt{euclid}).

\begin{Shaded}
\begin{Highlighting}[]
\FunctionTok{range}\NormalTok{(euclid)}
\end{Highlighting}
\end{Shaded}

\begin{verbatim}
## [1]  16.27882 707.32666
\end{verbatim}

\begin{Shaded}
\begin{Highlighting}[]
\FunctionTok{par}\NormalTok{(}\AttributeTok{mfrow=}\FunctionTok{c}\NormalTok{(}\DecValTok{2}\NormalTok{,}\DecValTok{1}\NormalTok{))}
\FunctionTok{hist}\NormalTok{(euclid,}
     \AttributeTok{breaks =} \DecValTok{15}\NormalTok{, }\CommentTok{\#determina o no. de colunas do histograma}
     \AttributeTok{xlim =} \FunctionTok{range}\NormalTok{(}\FunctionTok{floor}\NormalTok{(}\FunctionTok{min}\NormalTok{(euclid)), }\FunctionTok{ceiling}\NormalTok{(}\FunctionTok{max}\NormalTok{(euclid))), }
     \AttributeTok{xlab =} \StringTok{"Distr. de Frequências"}\NormalTok{,}
     \AttributeTok{freq =} \ConstantTok{FALSE}\NormalTok{)}
\FunctionTok{curve}\NormalTok{(}\FunctionTok{dnorm}\NormalTok{(x, }\AttributeTok{mean=}\FunctionTok{mean}\NormalTok{(euclid), }\AttributeTok{sd=}\FunctionTok{sd}\NormalTok{(euclid)), }\AttributeTok{add=}\ConstantTok{TRUE}\NormalTok{)}
\FunctionTok{boxplot.default}\NormalTok{(euclid, }\AttributeTok{horizontal =} \ConstantTok{TRUE}\NormalTok{, }\AttributeTok{frame =} \ConstantTok{FALSE}\NormalTok{,}
                \AttributeTok{xlab=}\StringTok{"Distr. de Frequências"}\NormalTok{,}
                \AttributeTok{ylim=}\FunctionTok{c}\NormalTok{(}\FunctionTok{floor}\NormalTok{(}\FunctionTok{min}\NormalTok{(euclid)), }\FunctionTok{ceiling}\NormalTok{(}\FunctionTok{max}\NormalTok{(euclid)))) }\CommentTok{\#Limites do eixo Y}
\end{Highlighting}
\end{Shaded}

\begin{figure}
\centering
\includegraphics{fundamentos_de_ecologia_numerica_files/figure-latex/6euclid-1.pdf}
\caption{\label{fig:6euclid}Distribuições de frequências da matriz de distâncias \texttt{euclid}}
\end{figure}

O comando abaixo apaga os gráficos, use-o apenas se necessário.

\begin{Shaded}
\begin{Highlighting}[]
\FunctionTok{dev.off}\NormalTok{()}
\end{Highlighting}
\end{Shaded}

\hypertarget{procurando-outliers}{%
\section{Procurando outliers}\label{procurando-outliers}}

\hypertarget{listas-de-distuxe2ncias}{%
\subsection{Listas de distâncias}\label{listas-de-distuxe2ncias}}

Vamos agora criar uma lista com as distâncias em desvio padrão entre cada objeto e o \textbf{centróide}, baseada na matriz de distâncias \texttt{euclid\_ma} (ou outra matriz de distâncias criada a partir de uma matriz relativizada/transformada ou bruta, dependendo do arquivo escolhido). As distâncias médias do centroide, expressas em desvios padrão, também podem ser chamadas de \textbf{z-scores}.

Para esse cálculo é necessário que a matriz de distâncias (do tipo \texttt{dist}) seja do tipo \texttt{matrix}. Por isso usamos o comando \texttt{as.matrix} e removemos a diagonal da nova matriz \texttt{euclid\_ma}. Códigos a seguir.

\hypertarget{removendo-os-zeros-da-matriz-de-distuxe2ncias}{%
\subsubsection{Removendo os zeros da matriz de distâncias}\label{removendo-os-zeros-da-matriz-de-distuxe2ncias}}

\begin{Shaded}
\begin{Highlighting}[]
\FunctionTok{library}\NormalTok{(matrixStats)}
\end{Highlighting}
\end{Shaded}

\begin{verbatim}
## Warning: package 'matrixStats' was built under R version 4.3.2
\end{verbatim}

\begin{verbatim}
## 
## Attaching package: 'matrixStats'
\end{verbatim}

\begin{verbatim}
## The following object is masked from 'package:dplyr':
## 
##     count
\end{verbatim}

\begin{Shaded}
\begin{Highlighting}[]
\FunctionTok{library}\NormalTok{(moments)}
\NormalTok{euclid\_ma }\OtherTok{\textless{}{-}} \FunctionTok{as.matrix}\NormalTok{(euclid)}
\NormalTok{euclid\_ma}
\FunctionTok{range}\NormalTok{(euclid\_ma) }\CommentTok{\#valor errado}
\FunctionTok{mean}\NormalTok{(euclid\_ma) }\CommentTok{\#valor errado}
\FunctionTok{sd}\NormalTok{(euclid\_ma) }\CommentTok{\#valor errado}
\FunctionTok{is.na}\NormalTok{(euclid\_ma) }\OtherTok{\textless{}{-}}\NormalTok{ euclid\_ma}\SpecialCharTok{==}\DecValTok{0} \CommentTok{\#atribui n.a. aos valores = 0}
\FunctionTok{mean}\NormalTok{(euclid\_ma, }\AttributeTok{na.rm=}\NormalTok{T)  }\CommentTok{\#valor correto, omite valores n.a. do calculo}
\FunctionTok{mean}\NormalTok{(euclid\_ma, }\AttributeTok{na.rm=}\NormalTok{T) }\CommentTok{\#valor correto, omite valores n.a. do calculo}
\FunctionTok{sd}\NormalTok{(euclid\_ma, }\AttributeTok{na.rm=}\NormalTok{T) }\CommentTok{\#valor correto, omite valores n.a. do calculo}
\CommentTok{\#colMeans(euclid\_ma, na.rm=T) \#omite valores n.a. do calculo}
\CommentTok{\#rowMeans(euclid\_ma, na.rm=T) \#omite valores n.a. do calculo}
\end{Highlighting}
\end{Shaded}

\begin{verbatim}
##           S-A-ZA1   S-R-CC1  S-R-CT1   S-R-CP1   S-A-TA1  S-R-CT2   S-R-CP2
## S-A-ZA1   0.00000  98.43780 228.5235  29.24038  27.09243 294.0629  53.40412
## S-R-CC1  98.43780   0.00000 154.2433  82.79493  77.25283 261.3905 108.10180
## S-R-CT1 228.52352 154.24331   0.0000 208.52338 209.69263 224.9133 224.09150
## S-R-CP1  29.24038  82.79493 208.5234   0.00000  23.25941 271.4922  49.22398
## S-A-TA1  27.09243  77.25283 209.6926  23.25941   0.00000 283.7869  57.82733
## S-R-CT2 294.06292 261.39051 224.9133 271.49217 283.78689   0.0000 268.95167
## S-R-CP2  53.40412 108.10180 224.0915  49.22398  57.82733 268.9517   0.00000
## S-A-TA2 183.74439 185.75791 261.8301 181.24845 166.66133 326.3327 190.15257
## S-R-CT3 460.42155 428.64437 366.6429 443.90877 453.59674 238.3799 438.54532
## S-R-CP3  34.17601  88.06816 215.2557  31.32092  33.13608 275.0436  40.37326
## S-A-TA3 333.86674 274.45765 269.9444 321.82604 309.19735 360.8213 334.61769
## S-R-CT4 187.51533 213.67265 273.0769 189.67604 191.60898 298.0084 190.90835
## S-R-CP4  81.58431 127.98437 226.6208  69.86415  85.24084 238.5351  57.30620
## S-A-TA4 478.64601 402.05970 359.7819 464.33932 453.15560 450.2344 478.08995
## B-A-MU1 190.62529 209.01196 287.5257 191.40272 190.90312 344.8086 197.56012
## B-R-ET1  36.13862 102.02941 228.2740  43.89761  42.52058 291.0859  63.79655
## B-A-GU1  16.27882  97.90301 227.0507  30.36445  28.40775 293.4110  55.07268
## B-R-PC2  91.52595  98.07650 204.8560  86.04650  84.49260 282.9558 103.56158
## B-A-MU2 119.74974  68.23489 165.6351 106.08958 103.04368 262.9848 125.71396
## B-A-GU2 110.95495 147.95607 249.2348 114.19282 113.77610 305.9902 122.87799
## B-R-PC3 151.57506 133.59266 183.3085 135.95588 140.16776 259.3260 157.58490
## B-A-MU3 534.53718 441.89592 367.2261 516.37293 513.70809 451.3635 530.43944
## B-A-GU3 353.04249 363.59180 396.0858 353.76829 353.56612 414.4539 356.90195
## B-R-PC4  89.73294 127.93748 229.5801  87.46999  89.97778 295.9341 103.55675
## B-A-MU4 641.28309 612.29405 595.5023 631.36598 633.67026 590.7597 633.81385
## B-A-GU4 230.36059 244.38903 302.0546 230.92640 230.49512 341.1700 236.11438
##          S-A-TA2  S-R-CT3   S-R-CP3  S-A-TA3  S-R-CT4   S-R-CP4  S-A-TA4
## S-A-ZA1 183.7444 460.4215  34.17601 333.8667 187.5153  81.58431 478.6460
## S-R-CC1 185.7579 428.6444  88.06816 274.4576 213.6726 127.98437 402.0597
## S-R-CT1 261.8301 366.6429 215.25566 269.9444 273.0769 226.62083 359.7819
## S-R-CP1 181.2484 443.9088  31.32092 321.8260 189.6760  69.86415 464.3393
## S-A-TA1 166.6613 453.5967  33.13608 309.1973 191.6090  85.24084 453.1556
## S-R-CT2 326.3327 238.3799 275.04363 360.8213 298.0084 238.53511 450.2344
## S-R-CP2 190.1526 438.5453  40.37326 334.6177 190.9084  57.30620 478.0899
## S-A-TA2   0.0000 478.9008 182.86060 198.3053 264.0038 200.93282 360.8130
## S-R-CT3 478.9008   0.0000 449.24826 478.0994 419.5474 411.12529 530.0566
## S-R-CP3 182.8606 449.2483   0.00000 324.5351 190.8612  74.14850 467.2216
## S-A-TA3 198.3053 478.0994 324.53505   0.0000 384.4750 344.01599 164.3624
## S-R-CT4 264.0038 419.5474 190.86121 384.4750   0.0000 195.29977 515.1932
## S-R-CP4 200.9328 411.1253  74.14850 344.0160 195.2998   0.00000 486.2016
## S-A-TA4 360.8130 530.0566 467.22157 164.3624 515.1932 486.20160   0.0000
## B-A-MU1 262.6481 494.5402 192.39023 377.5142 265.3413 207.01208 506.2549
## B-R-ET1 186.4457 461.2114  47.45524 334.1841 190.2735  88.34025 478.2384
## B-A-GU1 183.8777 462.0985  35.22783 333.2807 189.3278  82.22530 477.9676
## B-R-PC2 195.6246 447.5008  90.57041 315.9209 190.5282 122.53571 451.2815
## B-A-MU2 197.8686 427.9603 110.62549 282.7313 222.6971 140.52046 407.6175
## B-A-GU2 214.5623 466.3872 115.52922 352.2528 207.3089 137.12403 491.9075
## B-R-PC3 223.6672 423.6402 147.47542 315.7942 220.0114 170.50220 440.0898
## B-A-MU3 522.9532 505.9802 520.09038 399.2405 563.5140 534.39873 336.9926
## B-A-GU3 397.2719 518.7784 354.25273 483.3363 352.7846 362.13671 590.8849
## B-R-PC4 202.8398 461.7597  94.17006 341.0029 182.5541 119.84991 481.9544
## B-A-MU4 651.0146 647.9599 634.01104 643.0435 659.8833 626.19965 674.7548
## B-A-GU4 292.6124 476.7599 231.75849 398.4708 265.0623 244.09834 521.8716
##          B-A-MU1   B-R-ET1   B-A-GU1   B-R-PC2   B-A-MU2   B-A-GU2  B-R-PC3
## S-A-ZA1 190.6253  36.13862  16.27882  91.52595 119.74974 110.95495 151.5751
## S-R-CC1 209.0120 102.02941  97.90301  98.07650  68.23489 147.95607 133.5927
## S-R-CT1 287.5257 228.27396 227.05066 204.85605 165.63514 249.23483 183.3085
## S-R-CP1 191.4027  43.89761  30.36445  86.04650 106.08958 114.19282 135.9559
## S-A-TA1 190.9031  42.52058  28.40775  84.49260 103.04368 113.77610 140.1678
## S-R-CT2 344.8086 291.08590 293.41097 282.95583 262.98479 305.99020 259.3260
## S-R-CP2 197.5601  63.79655  55.07268 103.56158 125.71396 122.87799 157.5849
## S-A-TA2 262.6481 186.44570 183.87768 195.62464 197.86864 214.56235 223.6672
## S-R-CT3 494.5402 461.21145 462.09847 447.50084 427.96028 466.38718 423.6402
## S-R-CP3 192.3902  47.45524  35.22783  90.57041 110.62549 115.52922 147.4754
## S-A-TA3 377.5142 334.18408 333.28066 315.92088 282.73132 352.25275 315.7942
## S-R-CT4 265.3413 190.27349 189.32776 190.52821 222.69710 207.30895 220.0114
## S-R-CP4 207.0121  88.34025  82.22530 122.53571 140.52046 137.12403 170.5022
## S-A-TA4 506.2549 478.23843 477.96757 451.28151 407.61747 491.90751 440.0898
## B-A-MU1   0.0000 193.24078 183.63823 201.66556 151.28781 198.33557 225.3242
## B-R-ET1 193.2408   0.00000  36.53765  97.11334 119.98333 112.60107 151.2977
## B-A-GU1 183.6382  36.53765   0.00000  91.72786 115.33863  98.15294 149.6529
## B-R-PC2 201.6656  97.11334  91.72786   0.00000 114.92171 141.58390 114.7606
## B-A-MU2 151.2878 119.98333 115.33863 114.92171   0.00000 153.39817 143.1049
## B-A-GU2 198.3356 112.60107  98.15294 141.58390 153.39817   0.00000 173.1011
## B-R-PC3 225.3242 151.29772 149.65293 114.76062 143.10486 173.10113   0.0000
## B-A-MU3 505.4424 530.61097 531.72831 497.81221 422.87823 539.86572 476.5574
## B-A-GU3 381.5508 348.29729 342.49380 360.00556 364.76705 257.95736 337.1735
## B-R-PC4 205.1536  95.01579  89.16838 122.22520 142.70249 139.05035 150.8211
## B-A-MU4 498.8387 627.17462 635.41089 625.57414 550.83754 632.75193 618.9628
## B-A-GU4 289.1885 227.41592 219.20538 242.81886 251.58299 129.34837 239.0000
##          B-A-MU3  B-A-GU3   B-R-PC4  B-A-MU4  B-A-GU4
## S-A-ZA1 534.5372 353.0425  89.73294 641.2831 230.3606
## S-R-CC1 441.8959 363.5918 127.93748 612.2940 244.3890
## S-R-CT1 367.2261 396.0858 229.58005 595.5023 302.0546
## S-R-CP1 516.3729 353.7683  87.46999 631.3660 230.9264
## S-A-TA1 513.7081 353.5661  89.97778 633.6703 230.4951
## S-R-CT2 451.3635 414.4539 295.93411 590.7597 341.1700
## S-R-CP2 530.4394 356.9019 103.55675 633.8139 236.1144
## S-A-TA2 522.9532 397.2719 202.83984 651.0146 292.6124
## S-R-CT3 505.9802 518.7784 461.75968 647.9599 476.7599
## S-R-CP3 520.0904 354.2527  94.17006 634.0110 231.7585
## S-A-TA3 399.2405 483.3363 341.00293 643.0435 398.4708
## S-R-CT4 563.5140 352.7846 182.55410 659.8833 265.0623
## S-R-CP4 534.3987 362.1367 119.84991 626.1996 244.0983
## S-A-TA4 336.9926 590.8849 481.95435 674.7548 521.8716
## B-A-MU1 505.4424 381.5508 205.15360 498.8387 289.1885
## B-R-ET1 530.6110 348.2973  95.01579 627.1746 227.4159
## B-A-GU1 531.7283 342.4938  89.16838 635.4109 219.2054
## B-R-PC2 497.8122 360.0056 122.22520 625.5741 242.8189
## B-A-MU2 422.8782 364.7670 142.70249 550.8375 251.5830
## B-A-GU2 539.8657 257.9574 139.05035 632.7519 129.3484
## B-R-PC3 476.5574 337.1735 150.82109 618.9628 239.0000
## B-A-MU3   0.0000 624.8288 534.66812 519.9981 566.2808
## B-A-GU3 624.8288   0.0000 357.35557 707.3267 134.3763
## B-R-PC4 534.6681 357.3556   0.00000 641.2628 241.6402
## B-A-MU4 519.9981 707.3267 641.26282   0.0000 669.3982
## B-A-GU4 566.2808 134.3763 241.64023 669.3982   0.0000
## [1]   0.0000 707.3267
## [1] 274.908
## [1] 178.1842
## [1] 285.9043
## [1] 285.9043
## [1] 172.8375
\end{verbatim}

\hypertarget{criando-uma-lista}{%
\subsubsection{Criando uma Lista}\label{criando-uma-lista}}

\begin{Shaded}
\begin{Highlighting}[]
\NormalTok{centroide\_ma }\OtherTok{\textless{}{-}} \FunctionTok{mean}\NormalTok{(euclid\_ma, }\AttributeTok{na.rm=}\NormalTok{T)}
\NormalTok{av.dist }\OtherTok{\textless{}{-}}\NormalTok{ (}\FunctionTok{as.matrix}\NormalTok{(}\FunctionTok{colMeans}\NormalTok{(euclid\_ma, }\AttributeTok{na.rm=}\NormalTok{T)))}
\NormalTok{av.desvpad }\OtherTok{\textless{}{-}}\NormalTok{ (}\FunctionTok{as.matrix}\NormalTok{(}\FunctionTok{colSds}\NormalTok{(euclid\_ma, }\AttributeTok{na.rm=}\NormalTok{T)))}
\NormalTok{dp.centroide\_ma }\OtherTok{\textless{}{-}}\NormalTok{ (av.dist}\SpecialCharTok{{-}}\NormalTok{centroide\_ma)}\SpecialCharTok{/}\NormalTok{(}\FunctionTok{colSds}\NormalTok{(av.dist)) }\CommentTok{\#ou z{-}scores}
\NormalTok{list }\OtherTok{\textless{}{-}} \FunctionTok{as.matrix}\NormalTok{(}\FunctionTok{cbind}\NormalTok{(av.dist, av.desvpad, dp.centroide\_ma))}
\NormalTok{list}
\end{Highlighting}
\end{Shaded}

\begin{verbatim}
##             [,1]      [,2]        [,3]
## S-A-ZA1 202.2608 175.36143 -0.74136373
## S-R-CC1 205.9912 142.40233 -0.70830031
## S-R-CT1 266.3790  93.46194 -0.17306037
## S-R-CP1 195.7828 170.97514 -0.79878039
## S-A-TA1 195.8495 170.39163 -0.79818963
## S-R-CT2 317.0479  82.87727  0.27603716
## S-R-CP2 207.1435 167.09006 -0.69808647
## S-A-TA2 268.5172 124.64348 -0.15410841
## S-R-CT3 455.6678  68.84928  1.50467759
## S-R-CP3 199.1922 170.79087 -0.76856181
## S-A-TA3 347.0119  92.95522  0.54161944
## S-R-CT4 280.9253 131.44373 -0.04413038
## S-R-CP4 217.3521 157.62618 -0.60760370
## S-A-TA4 458.7988  93.50892  1.53242940
## B-A-MU1 278.0482 116.17286 -0.06963167
## B-R-ET1 205.3272 169.74682 -0.71418525
## B-A-GU1 200.2219 174.32564 -0.75943496
## B-R-PC2 215.0274 152.54520 -0.62820821
## B-A-MU2 210.8910 128.79692 -0.66487097
## B-A-GU2 229.0481 151.90782 -0.50393825
## B-R-PC3 235.2979 130.07777 -0.44854354
## B-A-MU3 499.5754  64.70789  1.89384716
## B-A-GU3 394.6795 116.02951  0.96411611
## B-R-PC4 225.0953 158.58714 -0.53897264
## B-A-MU4 624.1237  45.46670  2.99776610
## B-A-GU4 298.2560 130.86907  0.10947773
\end{verbatim}

Ordenamos as distâncias da maior para a menor.

\begin{Shaded}
\begin{Highlighting}[]
\FunctionTok{colnames}\NormalTok{(list, }\AttributeTok{do.NULL =} \ConstantTok{FALSE}\NormalTok{)}
\FunctionTok{colnames}\NormalTok{(list) }\OtherTok{\textless{}{-}} \FunctionTok{c}\NormalTok{(}\StringTok{"Av.Dist"}\NormalTok{, }\StringTok{"Av.StDev"}\NormalTok{, }\StringTok{"DP.Centroide"}\NormalTok{)}
\NormalTok{list2 }\OtherTok{\textless{}{-}}\NormalTok{ list[}\FunctionTok{order}\NormalTok{(list[,}\DecValTok{1}\NormalTok{], }\AttributeTok{decreasing =} \ConstantTok{TRUE}\NormalTok{),] }\CommentTok{\#[,1] ou o nome da coluna}
\NormalTok{list2}
\end{Highlighting}
\end{Shaded}

\begin{verbatim}
## [1] "col1" "col2" "col3"
##          Av.Dist  Av.StDev DP.Centroide
## B-A-MU4 624.1237  45.46670   2.99776610
## B-A-MU3 499.5754  64.70789   1.89384716
## S-A-TA4 458.7988  93.50892   1.53242940
## S-R-CT3 455.6678  68.84928   1.50467759
## B-A-GU3 394.6795 116.02951   0.96411611
## S-A-TA3 347.0119  92.95522   0.54161944
## S-R-CT2 317.0479  82.87727   0.27603716
## B-A-GU4 298.2560 130.86907   0.10947773
## S-R-CT4 280.9253 131.44373  -0.04413038
## B-A-MU1 278.0482 116.17286  -0.06963167
## S-A-TA2 268.5172 124.64348  -0.15410841
## S-R-CT1 266.3790  93.46194  -0.17306037
## B-R-PC3 235.2979 130.07777  -0.44854354
## B-A-GU2 229.0481 151.90782  -0.50393825
## B-R-PC4 225.0953 158.58714  -0.53897264
## S-R-CP4 217.3521 157.62618  -0.60760370
## B-R-PC2 215.0274 152.54520  -0.62820821
## B-A-MU2 210.8910 128.79692  -0.66487097
## S-R-CP2 207.1435 167.09006  -0.69808647
## S-R-CC1 205.9912 142.40233  -0.70830031
## B-R-ET1 205.3272 169.74682  -0.71418525
## S-A-ZA1 202.2608 175.36143  -0.74136373
## B-A-GU1 200.2219 174.32564  -0.75943496
## S-R-CP3 199.1922 170.79087  -0.76856181
## S-A-TA1 195.8495 170.39163  -0.79818963
## S-R-CP1 195.7828 170.97514  -0.79878039
\end{verbatim}

\hypertarget{distribuiuxe7uxe3o-de-frequuxeancias-das-distuxe2ncias-muxe9dias-para-o-centruxf3ide-av.dist}{%
\subsection{\texorpdfstring{Distribuição de frequências das distâncias médias para o centróide \texttt{av.dist}}{Distribuição de frequências das distâncias médias para o centróide av.dist}}\label{distribuiuxe7uxe3o-de-frequuxeancias-das-distuxe2ncias-muxe9dias-para-o-centruxf3ide-av.dist}}

Observe o valor maximo e minimo gerados pela função \texttt{range()} e substitua nas linhas de código assinaladas com \texttt{\#}. Aqui o menor valor foi 195.7828431 e o maior valor foi 624.1236964. Use valores maiores para facilitar a visualização no gráfico (Figura \ref{fig:6avdist}).

\begin{Shaded}
\begin{Highlighting}[]
\FunctionTok{par}\NormalTok{(}\AttributeTok{mfrow=}\FunctionTok{c}\NormalTok{(}\DecValTok{3}\NormalTok{,}\DecValTok{1}\NormalTok{))}
\FunctionTok{hist}\NormalTok{(list2[, }\StringTok{"Av.Dist"}\NormalTok{],}
     \AttributeTok{breaks =} \DecValTok{15}\NormalTok{, }\CommentTok{\#determina o no. de colunas do histograma}
     \AttributeTok{xlab =} \StringTok{"Distr. de Frequências das Distâncias (em dp) para o centroide"}\NormalTok{,}
     \AttributeTok{main =} \StringTok{"Distribuição de Frequência da distância média para o centroide"}\NormalTok{,}
     \AttributeTok{xlim =} \FunctionTok{range}\NormalTok{(}\FunctionTok{floor}\NormalTok{(}\FunctionTok{min}\NormalTok{(av.dist)), }\FunctionTok{ceiling}\NormalTok{(}\FunctionTok{max}\NormalTok{(av.dist))), }\CommentTok{\#substitua aqui o menor e maior valor do \textasciigrave{}range()\textasciigrave{}}
     \AttributeTok{freq =}\NormalTok{ T)}
\FunctionTok{hist}\NormalTok{(list2[, }\StringTok{"Av.Dist"}\NormalTok{],}
     \AttributeTok{breaks =} \DecValTok{15}\NormalTok{, }\CommentTok{\#determina o no. de colunas do histograma}
     \AttributeTok{xlab =} \StringTok{"Distr. de Frequências das Distâncias (em dp) para o centroide"}\NormalTok{,}
     \AttributeTok{main =} \StringTok{"Curva de normalidade ajustada para a Distribuição de Frequência"}\NormalTok{,}
     \AttributeTok{xlim =} \FunctionTok{range}\NormalTok{(}\FunctionTok{floor}\NormalTok{(}\FunctionTok{min}\NormalTok{(av.dist)), }\FunctionTok{ceiling}\NormalTok{(}\FunctionTok{max}\NormalTok{(av.dist))), }\CommentTok{\#substitua aqui o menor e maior valor do \textasciigrave{}range()\textasciigrave{}}
     \AttributeTok{freq =}\NormalTok{ F)}
\FunctionTok{curve}\NormalTok{(}\FunctionTok{dnorm}\NormalTok{(x, }\AttributeTok{mean=}\FunctionTok{mean}\NormalTok{(list2[, }\StringTok{"Av.Dist"}\NormalTok{]), }\AttributeTok{sd=}\FunctionTok{sd}\NormalTok{(list2[, }\StringTok{"Av.Dist"}\NormalTok{])), }\AttributeTok{add=}\ConstantTok{TRUE}\NormalTok{)}
\FunctionTok{boxplot.default}\NormalTok{(list2[, }\StringTok{"Av.Dist"}\NormalTok{], }\AttributeTok{horizontal =} \ConstantTok{TRUE}\NormalTok{, }\AttributeTok{frame =} \ConstantTok{FALSE}\NormalTok{,}
                \AttributeTok{xlab=}\StringTok{"Distr. de Frequências"}\NormalTok{,}
                \AttributeTok{ylim=}\FunctionTok{c}\NormalTok{(}\FunctionTok{floor}\NormalTok{(}\FunctionTok{min}\NormalTok{(av.dist)), }\FunctionTok{ceiling}\NormalTok{(}\FunctionTok{max}\NormalTok{(av.dist)))) }\CommentTok{\#substitua aqui o menor e maior valor do \textasciigrave{}range()\textasciigrave{}}
\end{Highlighting}
\end{Shaded}

\begin{figure}

{\centering \includegraphics{fundamentos_de_ecologia_numerica_files/figure-latex/6avdist-1} 

}

\caption{Distribuição de frequências das distâncias médias para o centroide.}\label{fig:6avdist}
\end{figure}

\begin{Shaded}
\begin{Highlighting}[]
\FunctionTok{par}\NormalTok{(}\AttributeTok{mfrow=}\FunctionTok{c}\NormalTok{(}\DecValTok{1}\NormalTok{,}\DecValTok{1}\NormalTok{))}
\end{Highlighting}
\end{Shaded}

\hypertarget{distribuiuxe7uxe3o-de-frequuxeancias-dos-desvios-paduxf5es-das-distuxe2ncias-muxe9dias-para-o-centruxf3ide-dp.centroide}{%
\subsubsection{\texorpdfstring{Distribuição de frequências dos desvios padões das distâncias médias para o centróide \texttt{DP.Centroide}}{Distribuição de frequências dos desvios padões das distâncias médias para o centróide DP.Centroide}}\label{distribuiuxe7uxe3o-de-frequuxeancias-dos-desvios-paduxf5es-das-distuxe2ncias-muxe9dias-para-o-centruxf3ide-dp.centroide}}

Observe o valor máximo e mínimo gerados pela função \texttt{range()} e substitua nas linhas de código assinaladas com \texttt{\#}. Aqui o menor valor foi -0.7987804 e o maior valor foi 2.9977661. Use valores maiores para facilitar a visualização no gráfico (Figura \ref{fig:4centro}).

\begin{Shaded}
\begin{Highlighting}[]
\FunctionTok{range}\NormalTok{(dp.centroide\_ma)}
\FunctionTok{par}\NormalTok{(}\AttributeTok{mfrow=}\FunctionTok{c}\NormalTok{(}\DecValTok{3}\NormalTok{,}\DecValTok{1}\NormalTok{))}
\FunctionTok{hist}\NormalTok{(list2[, }\StringTok{"DP.Centroide"}\NormalTok{],}
     \AttributeTok{breaks =} \DecValTok{15}\NormalTok{, }\CommentTok{\#determina o no. de colunas do histograma}
     \AttributeTok{xlab =} \StringTok{"Distr. de Frequências das Distâncias dos desvios padões para o centroide"}\NormalTok{,}
     \AttributeTok{main =} \StringTok{"Distribuição de Frequência dos desvio padões das distâncias médias para o centroide"}\NormalTok{,}
     \AttributeTok{xlim =} \FunctionTok{range}\NormalTok{(}\FunctionTok{floor}\NormalTok{(}\FunctionTok{min}\NormalTok{(dp.centroide\_ma)), }\FunctionTok{ceiling}\NormalTok{(}\FunctionTok{max}\NormalTok{(dp.centroide\_ma))), }\CommentTok{\#substitua aqui o menor e maior valor do \textasciigrave{}range()\textasciigrave{}}
     \AttributeTok{freq =}\NormalTok{ T)}
\FunctionTok{hist}\NormalTok{(list2[, }\StringTok{"DP.Centroide"}\NormalTok{],}
     \AttributeTok{breaks =} \DecValTok{15}\NormalTok{, }\CommentTok{\#determina o no. de colunas do histograma}
     \AttributeTok{xlab =} \StringTok{"Distr. de Frequências das Distâncias dos desvios padrões das distâncias médias para o centroide"}\NormalTok{,}
     \AttributeTok{main =} \StringTok{"Curva de normalidade ajustada para a Distribuição de Frequência"}\NormalTok{,}
     \AttributeTok{xlim =} \FunctionTok{range}\NormalTok{(}\FunctionTok{floor}\NormalTok{(}\FunctionTok{min}\NormalTok{(dp.centroide\_ma)), }\FunctionTok{ceiling}\NormalTok{(}\FunctionTok{max}\NormalTok{(dp.centroide\_ma))), }\CommentTok{\#substitua aqui o menor e maior valor do \textasciigrave{}range()\textasciigrave{}}
     \AttributeTok{freq =}\NormalTok{ F)}
\FunctionTok{curve}\NormalTok{(}\FunctionTok{dnorm}\NormalTok{(x, }\AttributeTok{mean=}\FunctionTok{mean}\NormalTok{(list2[, }\StringTok{"DP.Centroide"}\NormalTok{]), }\AttributeTok{sd=}\FunctionTok{sd}\NormalTok{(list2[, }\StringTok{"DP.Centroide"}\NormalTok{])), }\AttributeTok{add=}\ConstantTok{TRUE}\NormalTok{)}
\FunctionTok{boxplot.default}\NormalTok{(list2[, }\StringTok{"DP.Centroide"}\NormalTok{], }\AttributeTok{horizontal =} \ConstantTok{TRUE}\NormalTok{, }\AttributeTok{frame =} \ConstantTok{FALSE}\NormalTok{,}
                \AttributeTok{xlab=}\StringTok{"Distr. de Frequências"}\NormalTok{,}
                \AttributeTok{ylim=}\FunctionTok{c}\NormalTok{(}\FunctionTok{floor}\NormalTok{(}\FunctionTok{min}\NormalTok{(dp.centroide\_ma)), }\FunctionTok{ceiling}\NormalTok{(}\FunctionTok{max}\NormalTok{(dp.centroide\_ma)))) }\CommentTok{\#substitua aqui o menor e maior valor do \textasciigrave{}range()\textasciigrave{}}
\end{Highlighting}
\end{Shaded}

\begin{figure}

{\centering \includegraphics{fundamentos_de_ecologia_numerica_files/figure-latex/4centro-1} 

}

\caption{Distribuição de frequências dos desvios padões das distâncias médias para o centroide.}\label{fig:4centro}
\end{figure}

\begin{Shaded}
\begin{Highlighting}[]
\FunctionTok{par}\NormalTok{(}\AttributeTok{mfrow=}\FunctionTok{c}\NormalTok{(}\DecValTok{1}\NormalTok{,}\DecValTok{1}\NormalTok{))}
\end{Highlighting}
\end{Shaded}

\begin{verbatim}
## [1] -0.7987804  2.9977661
\end{verbatim}

Se necessário apague os gráficos

\begin{Shaded}
\begin{Highlighting}[]
\FunctionTok{dev.off}\NormalTok{()}
\end{Highlighting}
\end{Shaded}

\hypertarget{lista-final-de-outliers}{%
\section{Lista final de outliers}\label{lista-final-de-outliers}}

Agora fazemos a lista final de distâncias com os outliers baseados em um `cutoff'.

\begin{Shaded}
\begin{Highlighting}[]
\NormalTok{cutoff }\OtherTok{\textless{}{-}} \FloatTok{2.0}
\end{Highlighting}
\end{Shaded}

Note que o `cutoff' foi estabelecido no código acima no \texttt{vetor} `cutoff \textless-'. Ou seja, o `cutoff' definindo quem serão considerados outliers, foi estabelecido como sendo valores de \textbf{+2.0} e \textbf{-2.0} desvios padrões da média multivariada ou centroide. Valores acima ou abaixo do `cutoff' definido recebem o nome ``OUT'', na coluna ``Outliers''. Criada com o código abaixo.

\begin{Shaded}
\begin{Highlighting}[]
\FunctionTok{library}\NormalTok{(gt)}
\FunctionTok{format}\NormalTok{(cutoff, }\AttributeTok{nsmall =} \DecValTok{1}\NormalTok{)}
\NormalTok{listf }\OtherTok{\textless{}{-}} \FunctionTok{as.data.frame}\NormalTok{(list2)}
\NormalTok{listf}\SpecialCharTok{$}\NormalTok{Outliers }\OtherTok{\textless{}{-}} \FunctionTok{ifelse}\NormalTok{(listf}\SpecialCharTok{$}\NormalTok{DP.Centroide}\SpecialCharTok{\textgreater{}{-}}\NormalTok{cutoff }\CommentTok{\#CUTOFF MENOR QUE {-}\textquotesingle{}cutoff}
                         \SpecialCharTok{\&}\NormalTok{ listf}\SpecialCharTok{$}\NormalTok{DP.Centroide}\SpecialCharTok{\textless{}}\NormalTok{cutoff,  }\CommentTok{\#CUTOFF MAIOR QUE \textquotesingle{}cutoff\textquotesingle{}}
                         \StringTok{""}\NormalTok{, }\StringTok{"OUT"}\NormalTok{) }
\NormalTok{listf}
\FunctionTok{gt}\NormalTok{(}\FunctionTok{cbind}\NormalTok{(}\AttributeTok{Sitios=}\FunctionTok{rownames}\NormalTok{(listf),listf))}
\end{Highlighting}
\end{Shaded}

\begin{verbatim}
## [1] "2.0"
##          Av.Dist  Av.StDev DP.Centroide Outliers
## B-A-MU4 624.1237  45.46670   2.99776610      OUT
## B-A-MU3 499.5754  64.70789   1.89384716         
## S-A-TA4 458.7988  93.50892   1.53242940         
## S-R-CT3 455.6678  68.84928   1.50467759         
## B-A-GU3 394.6795 116.02951   0.96411611         
## S-A-TA3 347.0119  92.95522   0.54161944         
## S-R-CT2 317.0479  82.87727   0.27603716         
## B-A-GU4 298.2560 130.86907   0.10947773         
## S-R-CT4 280.9253 131.44373  -0.04413038         
## B-A-MU1 278.0482 116.17286  -0.06963167         
## S-A-TA2 268.5172 124.64348  -0.15410841         
## S-R-CT1 266.3790  93.46194  -0.17306037         
## B-R-PC3 235.2979 130.07777  -0.44854354         
## B-A-GU2 229.0481 151.90782  -0.50393825         
## B-R-PC4 225.0953 158.58714  -0.53897264         
## S-R-CP4 217.3521 157.62618  -0.60760370         
## B-R-PC2 215.0274 152.54520  -0.62820821         
## B-A-MU2 210.8910 128.79692  -0.66487097         
## S-R-CP2 207.1435 167.09006  -0.69808647         
## S-R-CC1 205.9912 142.40233  -0.70830031         
## B-R-ET1 205.3272 169.74682  -0.71418525         
## S-A-ZA1 202.2608 175.36143  -0.74136373         
## B-A-GU1 200.2219 174.32564  -0.75943496         
## S-R-CP3 199.1922 170.79087  -0.76856181         
## S-A-TA1 195.8495 170.39163  -0.79818963         
## S-R-CP1 195.7828 170.97514  -0.79878039
\end{verbatim}

\begin{longtable}{lrrrl}
\toprule
Sitios & Av.Dist & Av.StDev & DP.Centroide & Outliers \\ 
\midrule\addlinespace[2.5pt]
B-A-MU4 & 624.1237 & 45.46670 & 2.99776610 & OUT \\ 
B-A-MU3 & 499.5754 & 64.70789 & 1.89384716 &  \\ 
S-A-TA4 & 458.7988 & 93.50892 & 1.53242940 &  \\ 
S-R-CT3 & 455.6678 & 68.84928 & 1.50467759 &  \\ 
B-A-GU3 & 394.6795 & 116.02951 & 0.96411611 &  \\ 
S-A-TA3 & 347.0119 & 92.95522 & 0.54161944 &  \\ 
S-R-CT2 & 317.0479 & 82.87727 & 0.27603716 &  \\ 
B-A-GU4 & 298.2560 & 130.86907 & 0.10947773 &  \\ 
S-R-CT4 & 280.9253 & 131.44373 & -0.04413038 &  \\ 
B-A-MU1 & 278.0482 & 116.17286 & -0.06963167 &  \\ 
S-A-TA2 & 268.5172 & 124.64348 & -0.15410841 &  \\ 
S-R-CT1 & 266.3790 & 93.46194 & -0.17306037 &  \\ 
B-R-PC3 & 235.2979 & 130.07777 & -0.44854354 &  \\ 
B-A-GU2 & 229.0481 & 151.90782 & -0.50393825 &  \\ 
B-R-PC4 & 225.0953 & 158.58714 & -0.53897264 &  \\ 
S-R-CP4 & 217.3521 & 157.62618 & -0.60760370 &  \\ 
B-R-PC2 & 215.0274 & 152.54520 & -0.62820821 &  \\ 
B-A-MU2 & 210.8910 & 128.79692 & -0.66487097 &  \\ 
S-R-CP2 & 207.1435 & 167.09006 & -0.69808647 &  \\ 
S-R-CC1 & 205.9912 & 142.40233 & -0.70830031 &  \\ 
B-R-ET1 & 205.3272 & 169.74682 & -0.71418525 &  \\ 
S-A-ZA1 & 202.2608 & 175.36143 & -0.74136373 &  \\ 
B-A-GU1 & 200.2219 & 174.32564 & -0.75943496 &  \\ 
S-R-CP3 & 199.1922 & 170.79087 & -0.76856181 &  \\ 
S-A-TA1 & 195.8495 & 170.39163 & -0.79818963 &  \\ 
S-R-CP1 & 195.7828 & 170.97514 & -0.79878039 &  \\ 
\bottomrule
\end{longtable}

\hypertarget{particionando-a-matriz}{%
\section{Particionando a matriz}\label{particionando-a-matriz}}

Talvez seja necessário particionar a matriz bruta para remover os outliers. Isso pode ser feito com os códigos abaixo. Atente para o \texttt{vetor} \textbf{\texttt{part}}. Ele define quais objetos ou UA's serão particionados.

\begin{Shaded}
\begin{Highlighting}[]
\NormalTok{part }\OtherTok{\textless{}{-}} \FunctionTok{c}\NormalTok{(}\StringTok{"B{-}A{-}MU4"}\NormalTok{, }\StringTok{"B{-}A{-}MU3"}\NormalTok{, }\StringTok{"S{-}A{-}TA4"}\NormalTok{, }\StringTok{"S{-}R{-}CT3"}\NormalTok{)}
\NormalTok{part}
\end{Highlighting}
\end{Shaded}

\begin{verbatim}
## [1] "B-A-MU4" "B-A-MU3" "S-A-TA4" "S-R-CT3"
\end{verbatim}

\begin{Shaded}
\begin{Highlighting}[]
\NormalTok{m\_bruta\_part }\OtherTok{\textless{}{-}}\NormalTok{ m\_bruta[}\SpecialCharTok{!}\NormalTok{(}\FunctionTok{row.names}\NormalTok{(m\_bruta) }\SpecialCharTok{\%in\%} \FunctionTok{c}\NormalTok{(part)),]}
\CommentTok{\#m\_bruta\_part}
\end{Highlighting}
\end{Shaded}

Criamos uma nova matriz particionada \texttt{m\_bruta\_part} sem \textbf{B-A-MU4, B-A-MU3, S-A-TA4, S-R-CT3}, que apresentavam valores acima do cutoff para outliers.

\leavevmode\vadjust pre{\hypertarget{hello}{}}%
{ATENÇÃO}\\
Quando deletamos linhas ou colunas da matriz de dados comunitária, precisamos ver se isso não gerou linhas ou colunas vazias. Observe por exemplo a espécie \emph{Apareiodon davisi} \footnote{A etimologia do nome \emph{Apareiodon} vem do Grego, \emph{a}, sem, \emph{pareia}, lateral ou bochecha, e \emph{odous} dentição, em referencia a ausência de tentes laterais no aparato bucal dessa espécie.} (Figura \ref{fig:6apdavis}). Por exemplo, se uma determinada espécie só ocorre em uma UA, e aquela UA foi removida, essa espécie não consta mais em nenhuma UA e por isso a sua soma vai ser igual a zero. Isso aconteceu com a espécie \emph{Pseudancistrus genisetiger} \footnote{A etimologia do nome \emph{Pseudancistrus} vem do Grego, \emph{pseudes}, falso, e \emph{agkistron}, gancho, em referência a falsos ganchos presentes na cabeça em algumas espécies do gênero.} (Figura \ref{fig:6pgen}). Essa espécie precisará ser removida.

\begin{figure}
\includegraphics[width=8.89in]{imagens/Apdav_u0} \caption{*Apareiodon davisi*, importante espécie bentopelágica das bacias dos rios Jaguaribe e Paraíba. Brazil, by Ramos, T.P.A. Fonte: <https://www.fishbase.se/summary/Astianax-bimaculatus.html>}\label{fig:6apdavis}
\end{figure}

\begin{figure}
\includegraphics[width=22.22in]{imagens/genis} \caption{*Pseudancistrus genisetiger*, uma espécie endêmica das bacias hidrográfcas do nordeste. By eplanetcatfish.com. Fonte: <https://www.planetcatfish.com> (Provável identificação errada do espécime da foto)](eplanetcatfish.jpg)}\label{fig:6pgen}
\end{figure}

Use o a função \texttt{View(m\_bruta\_part)} e procure linhas ou colunas na matriz \texttt{m\_bruta\_part} onde todos os valores são zero. Quais espécies você encontrou que deixaram de ocorrer na comunidade depois que as UA's outliers foram removidas?.

Complicado? Os códigos abaixo resolvem isso, e mostram algumas propriedades da nova matriz \texttt{m\_bruta\_part2} depois de particionada pela segunda vez para remover linhas/colunas ``zeradas''.

\begin{Shaded}
\begin{Highlighting}[]
\NormalTok{sum }\OtherTok{\textless{}{-}} \FunctionTok{colSums}\NormalTok{(m\_bruta\_part)}
\NormalTok{sum}
\NormalTok{zero\_sum\_cols }\OtherTok{\textless{}{-}} \FunctionTok{names}\NormalTok{(}\FunctionTok{which}\NormalTok{(}\FunctionTok{colSums}\NormalTok{(m\_bruta\_part) }\SpecialCharTok{==} \DecValTok{0}\NormalTok{))}
\NormalTok{zero\_sum\_cols }\CommentTok{\#nomes das espécies zeradas}
\NormalTok{m\_bruta\_part2 }\OtherTok{\textless{}{-}}\NormalTok{ m\_bruta\_part[(}\FunctionTok{colSums}\NormalTok{(m\_bruta\_part) }\SpecialCharTok{!=} \DecValTok{0}\NormalTok{)] }\CommentTok{\#em != a exclamação inverte o sentido}
\NormalTok{zero\_sum\_cols2 }\OtherTok{\textless{}{-}} \FunctionTok{names}\NormalTok{(}\FunctionTok{which}\NormalTok{(}\FunctionTok{colSums}\NormalTok{(m\_bruta\_part2) }\SpecialCharTok{==} \DecValTok{0}\NormalTok{))}
\NormalTok{zero\_sum\_cols2 }\CommentTok{\#nomes das espécies zeradas}
\NormalTok{sum}\OtherTok{\textless{}{-}}\FunctionTok{colSums}\NormalTok{(m\_bruta\_part2)}
\NormalTok{sum}
\CommentTok{\#m\_bruta\_part2}
\CommentTok{\#m\_bruta\_part2 \textless{}{-} as.matrix(m\_bruta\_part2)}
\FunctionTok{str}\NormalTok{(m\_bruta\_part2)}
\FunctionTok{length}\NormalTok{(}\FunctionTok{as.matrix}\NormalTok{(m\_bruta\_part2))}
\end{Highlighting}
\end{Shaded}

\begin{verbatim}
##  ap-davis  as-bimac  as-fasci  ch-bimac  ci-ocela  ci-orien  co-macro  co-heter 
##        27      1040        94       432        70       118         2         1 
##  cr-menez  cu-lepid  cy-gilbe  ge-brasi  he-margi  ho-malab  hy-pusar  le-melan 
##        19        21       131       364         2        69        60         2 
##   le-piau  le-taeni  mo-costa  mo-lepid  or-nilot  pa-manag  pimel-sp  po-retic 
##        11         1         1        41       697       554         6        64 
##  po-vivip  pr-brevi  ps-rhomb ps-genise  se-heter  se-piaba  se-spilo  st-noton 
##       441       111         0         0       258        68         0        90 
##  sy-marmo  te-chalc  tr-signa 
##         1       134       201 
## [1] "ps-rhomb"  "ps-genise" "se-spilo" 
## character(0)
## ap-davis as-bimac as-fasci ch-bimac ci-ocela ci-orien co-macro co-heter 
##       27     1040       94      432       70      118        2        1 
## cr-menez cu-lepid cy-gilbe ge-brasi he-margi ho-malab hy-pusar le-melan 
##       19       21      131      364        2       69       60        2 
##  le-piau le-taeni mo-costa mo-lepid or-nilot pa-manag pimel-sp po-retic 
##       11        1        1       41      697      554        6       64 
## po-vivip pr-brevi se-heter se-piaba st-noton sy-marmo te-chalc tr-signa 
##      441      111      258       68       90        1      134      201 
## 'data.frame':    22 obs. of  32 variables:
##  $ ap-davis: num  0 0 0 0 0 0 0 0 0 0 ...
##  $ as-bimac: num  1 99 194 19 23 142 5 46 16 234 ...
##  $ as-fasci: num  0 0 55 0 1 3 1 0 0 7 ...
##  $ ch-bimac: num  0 0 0 0 13 3 0 178 0 238 ...
##  $ ci-ocela: num  0 0 0 0 0 0 40 0 13 0 ...
##  $ ci-orien: num  0 0 5 0 0 69 9 0 24 0 ...
##  $ co-macro: num  0 0 0 0 0 0 0 0 0 2 ...
##  $ co-heter: num  0 0 1 0 0 0 0 0 0 0 ...
##  $ cr-menez: num  0 0 14 0 0 4 0 0 0 0 ...
##  $ cu-lepid: num  0 0 0 0 0 0 0 0 0 0 ...
##  $ cy-gilbe: num  0 0 0 0 0 0 0 0 0 0 ...
##  $ ge-brasi: num  0 0 3 0 0 0 0 0 0 0 ...
##  $ he-margi: num  0 0 0 0 0 1 0 0 0 0 ...
##  $ ho-malab: num  0 0 1 5 0 17 10 2 4 20 ...
##  $ hy-pusar: num  0 0 9 2 0 43 2 0 0 0 ...
##  $ le-melan: num  0 0 0 0 0 0 0 0 0 0 ...
##  $ le-piau : num  0 0 3 0 0 1 3 0 1 0 ...
##  $ le-taeni: num  0 0 0 0 0 0 0 0 0 0 ...
##  $ mo-costa: num  0 0 0 0 0 0 0 0 0 0 ...
##  $ mo-lepid: num  0 1 39 0 0 1 0 0 0 0 ...
##  $ or-nilot: num  0 2 36 0 0 77 0 0 0 0 ...
##  $ pa-manag: num  0 0 0 0 0 0 0 0 0 0 ...
##  $ pimel-sp: num  0 0 6 0 0 0 0 0 0 0 ...
##  $ po-retic: num  0 0 0 0 0 20 0 0 0 0 ...
##  $ po-vivip: num  0 0 47 15 0 221 32 0 10 0 ...
##  $ pr-brevi: num  9 0 5 0 1 15 5 2 0 0 ...
##  $ se-heter: num  0 0 40 14 4 60 0 0 0 0 ...
##  $ se-piaba: num  0 0 68 0 0 0 0 0 0 0 ...
##  $ st-noton: num  0 0 1 0 0 25 0 0 0 0 ...
##  $ sy-marmo: num  0 0 0 0 0 0 1 0 0 0 ...
##  $ te-chalc: num  0 0 0 0 0 0 0 0 0 0 ...
##  $ tr-signa: num  0 0 18 0 0 15 0 0 0 0 ...
## [1] 704
\end{verbatim}

\hypertarget{exportando-a-matrix-final-de-trabalho}{%
\section{Exportando a matrix final de trabalho}\label{exportando-a-matrix-final-de-trabalho}}

Agora vamos exportar a matrix de trabalho, nesse casso a matrix \texttt{m\_bruta\_part2}, como um arquivo de valores separados por vírgula (\texttt{m\_bruta\_part2csv.csv}). Na verdade, separado por \texttt{";"}, como definido na funçao \texttt{sep\ =\ ";"}. O arquivo será exportado para o diretório de trabalho. Na mesma sequência de códigos, importamos \texttt{m\_bruta\_part2csv.csv} como nossa matriz de trabalho \texttt{m\_trab}.

\begin{Shaded}
\begin{Highlighting}[]
\NormalTok{df }\OtherTok{\textless{}{-}} \FunctionTok{data.frame}\NormalTok{(}\AttributeTok{Sites =} \FunctionTok{rownames}\NormalTok{(m\_bruta), m\_bruta,}
                 \AttributeTok{row.names =} \ConstantTok{NULL}\NormalTok{,}
                 \AttributeTok{check.names =} \ConstantTok{FALSE}\NormalTok{) }\CommentTok{\#add titulo a primeira coluna}

\FunctionTok{write.table}\NormalTok{(m\_bruta\_part2, }\StringTok{"m\_trabcsv.csv"}\NormalTok{,}
            \AttributeTok{sep =} \StringTok{";"}\NormalTok{, }\AttributeTok{dec =} \StringTok{"."}\NormalTok{, }\CommentTok{\#"\textbackslash{}t",}
            \AttributeTok{row.names =} \ConstantTok{TRUE}\NormalTok{,}
            \AttributeTok{quote =} \ConstantTok{TRUE}\NormalTok{,}
            \AttributeTok{append =} \ConstantTok{FALSE}\NormalTok{)}

\NormalTok{m\_trab }\OtherTok{\textless{}{-}} \FunctionTok{read.csv}\NormalTok{(}\StringTok{"m\_trabcsv.csv"}\NormalTok{,}
                   \AttributeTok{sep =} \StringTok{";"}\NormalTok{, }\AttributeTok{dec =} \StringTok{"."}\NormalTok{,}
                   \AttributeTok{row.names =} \DecValTok{1}\NormalTok{,}
                   \AttributeTok{header =} \ConstantTok{TRUE}\NormalTok{, }
                   \AttributeTok{na.strings =} \ConstantTok{NA}\NormalTok{,}
                   \AttributeTok{check.names =} \ConstantTok{FALSE}\NormalTok{, }\CommentTok{\#impede que o R mude os nomes das colunas}
                   \AttributeTok{col.names =} \FunctionTok{gsub}\NormalTok{(}\StringTok{"(\^{}|\_)([a{-}z])"}\NormalTok{, }\StringTok{"}\SpecialCharTok{\textbackslash{}\textbackslash{}}\StringTok{1}\SpecialCharTok{\textbackslash{}\textbackslash{}}\StringTok{U}\SpecialCharTok{\textbackslash{}\textbackslash{}}\StringTok{2"}\NormalTok{,}
                                    \FunctionTok{names}\NormalTok{(m\_trab), }\AttributeTok{perl =} \ConstantTok{TRUE}\NormalTok{))}
\end{Highlighting}
\end{Shaded}

\begin{itemize}
\tightlist
\item
  Inicial Maiuscula:col.names = gsub(``(\^{}\textbar\_)({[}a-z{]})``,''\textbackslash1\textbackslash U\textbackslash2'', names(m\_trab), perl = TRUE))\\
\item
  Substituição: col.names = gsub(``-'', ``.'', names(m\_trab)))\\
\item
  A função \texttt{header=} especifica se o arquivo de entrada contém uma linha de cabeçalho com os nomes das colunas, \texttt{row.names} especifica qual coluna no arquivo de entrada deve ser usada como os nomes das linhas, e se \texttt{NULL}, os nomes das linhas são gerados como números
\end{itemize}

Agora é com você\ldots Refaça toda a análise com a matriz bruta particionada \texttt{m\_bruta\_part2}

\hypertarget{apuxeandices-12}{%
\section*{Apêndices}\label{apuxeandices-12}}
\addcontentsline{toc}{section}{Apêndices}

\hypertarget{outras-formas-de-fazer-a-partiuxe7uxe3o}{%
\subsection*{Outras formas de fazer a partição}\label{outras-formas-de-fazer-a-partiuxe7uxe3o}}
\addcontentsline{toc}{subsection}{Outras formas de fazer a partição}

m\_bruta\_part2 \textless- m\_bruta\_part\[colSums(abs(m_bruta_part), na.rm = F) \> 0\]
m\_bruta\_part2 \textless- subset(m\_bruta\_part, colSums != 0) m\_bruta\_part2 \textless-
m\_bruta\_part\[, colSums(m_bruta_part != 0) \> 0\]

\hypertarget{cuxf3digo}{%
\subsection*{Código}\label{cuxf3digo}}
\addcontentsline{toc}{subsection}{Código}

range(euclidma) par(mfrow=c(2,1)) hist(euclidma, breaks = 15, \#determina
o no. de colunas do histograma xlim = range(floor(min(euclidma)),
ceiling(max(euclidma))), xlab = ``Distr. de Frequências'', freq = FALSE)
curve(dnorm(x, mean=mean(euclidma), sd=sd(euclidma)), add=TRUE)
boxplot.default(euclidma, horizontal = TRUE, frame = FALSE, xlab=``Distr.
de Frequências'', ylim=c(floor(min(euclidma)), ceiling(max(euclidma))))

\hypertarget{script-limpo-9}{%
\section*{Script limpo}\label{script-limpo-9}}
\addcontentsline{toc}{section}{Script limpo}

Aqui apresento o scrip na íntegra sem os textos ou outros comentários. Você pode copiar e colar no R para executa-lo. Lembre de remover os {\texttt{\#}} ou {\texttt{\#\#}} caso necessite executar essas linhas.

\begin{Shaded}
\begin{Highlighting}[]
\DocumentationTok{\#\# dev.off() \#apaga os graficos, se houver algum}
\DocumentationTok{\#\# rm(list=ls(all=TRUE)) \#limpa a memória}
\DocumentationTok{\#\# cat("\textbackslash{}014") \#limpa o console}
\DocumentationTok{\#\# install.packages("openxlsx") \#importa arquivos do excel}
\DocumentationTok{\#\# install.packages("moments") \#calcula assimetria e curtose dos dados}
\DocumentationTok{\#\# install.packages("matrixStats") \#fornece funções rápidas para a estatística de matrizes}
\DocumentationTok{\#\# install.packages("gt") \#ferramenta para criação de tabelas bonitas e personalizáveis}
\DocumentationTok{\#\# getwd()}
\DocumentationTok{\#\# setwd("C:/Seu/Diretório/De/Trabalho")}
\FunctionTok{library}\NormalTok{(openxlsx)}
\NormalTok{m\_bruta }\OtherTok{\textless{}{-}} \FunctionTok{read.xlsx}\NormalTok{(}\StringTok{"D:/Elvio/OneDrive/Disciplinas/\_EcoNumerica/5.Matrizes/peixes06.xlsx"}\NormalTok{,}
                   \AttributeTok{rowNames =}\NormalTok{ T, }\AttributeTok{colNames =}\NormalTok{ T,}
                   \AttributeTok{sheet =} \StringTok{"Sheet1"}\NormalTok{)}
\FunctionTok{str}\NormalTok{(m\_bruta)}
\NormalTok{m\_bruta\_ma }\OtherTok{\textless{}{-}} \FunctionTok{as.matrix}\NormalTok{(m\_bruta) }\CommentTok{\#lê m\_bruta como uma matriz}
\FunctionTok{str}\NormalTok{(m\_bruta\_ma)}
\CommentTok{\#m\_bruta}
\NormalTok{m\_bruta\_ma[}\DecValTok{1}\SpecialCharTok{:}\DecValTok{5}\NormalTok{,}\DecValTok{1}\SpecialCharTok{:}\DecValTok{5}\NormalTok{] }\CommentTok{\#[1:5,1:5] mostra apenas as linhas e colunas de 1 a 5.}
\CommentTok{\#m\_bruta \textless{}{-} (m\_bruta)   \# \textless{}1\textgreater{}}
\CommentTok{\#View(m\_bruta)}
\FunctionTok{print}\NormalTok{(m\_bruta)}
\FunctionTok{head}\NormalTok{(m\_bruta)}
\FunctionTok{str}\NormalTok{(m\_bruta)}
\FunctionTok{mode}\NormalTok{(m\_bruta)}
\FunctionTok{class}\NormalTok{(m\_bruta)}
\CommentTok{\#?str}
\FunctionTok{range}\NormalTok{(m\_bruta) }\CommentTok{\#menor e maior valores}
\FunctionTok{length}\NormalTok{(m\_bruta) }\CommentTok{\#no. de colunas}
\FunctionTok{ncol}\NormalTok{(m\_bruta) }\CommentTok{\#no. de N colunas}
\FunctionTok{nrow}\NormalTok{(m\_bruta) }\CommentTok{\#no. de M linhas}
\FunctionTok{sum}\NormalTok{(}\FunctionTok{lengths}\NormalTok{(m\_bruta)) }\CommentTok{\#soma os nos. de colunas}
\FunctionTok{length}\NormalTok{(}\FunctionTok{as.matrix}\NormalTok{(m\_bruta)) }\CommentTok{\#tamanho da matriz m x n}
\FunctionTok{sum}\NormalTok{(m\_bruta }\SpecialCharTok{==} \DecValTok{0}\NormalTok{) }\CommentTok{\# número de observações igual a zero}
\FunctionTok{sum}\NormalTok{(m\_bruta }\SpecialCharTok{\textgreater{}} \DecValTok{0}\NormalTok{) }\CommentTok{\# número de observações maiores que zero}
\NormalTok{zeros }\OtherTok{\textless{}{-}}\NormalTok{ (}\FunctionTok{sum}\NormalTok{(m\_bruta }\SpecialCharTok{==} \DecValTok{0}\NormalTok{)}\SpecialCharTok{/}\FunctionTok{length}\NormalTok{(}\FunctionTok{as.matrix}\NormalTok{(m\_bruta)))}\SpecialCharTok{*}\DecValTok{100} \CommentTok{\# proporção de zeros na matriz}
\NormalTok{zeros}
\NormalTok{tamanho }\OtherTok{\textless{}{-}} \FunctionTok{data.frame}\NormalTok{(}
\NormalTok{  Função }\OtherTok{=} \FunctionTok{c}\NormalTok{(}\StringTok{"range"}\NormalTok{, }\StringTok{"lenght"}\NormalTok{, }\StringTok{"m cols"}\NormalTok{, }\StringTok{"n linhas"}\NormalTok{, }\StringTok{"Tamanho"}\NormalTok{, }\StringTok{"Tamanho"}\NormalTok{,}
             \StringTok{"Zeros"}\NormalTok{, }\StringTok{"Nao zeros"}\NormalTok{, }\StringTok{"\% Zeros"}\NormalTok{),}
  \AttributeTok{Resultado =} \FunctionTok{c}\NormalTok{(}\FunctionTok{paste}\NormalTok{(}\FunctionTok{range}\NormalTok{(m\_bruta), }\AttributeTok{collapse =} \StringTok{" {-} "}\NormalTok{), }\FunctionTok{length}\NormalTok{(m\_bruta), }\FunctionTok{ncol}\NormalTok{(m\_bruta),}
                \FunctionTok{nrow}\NormalTok{(m\_bruta), }\FunctionTok{sum}\NormalTok{(}\FunctionTok{lengths}\NormalTok{(m\_bruta)), }\FunctionTok{length}\NormalTok{(}\FunctionTok{as.matrix}\NormalTok{(m\_bruta)), }\FunctionTok{sum}\NormalTok{(m\_bruta }\SpecialCharTok{==} \DecValTok{0}\NormalTok{),}
                \FunctionTok{sum}\NormalTok{(m\_bruta }\SpecialCharTok{\textgreater{}} \DecValTok{0}\NormalTok{), }\FunctionTok{round}\NormalTok{(zeros, }\DecValTok{1}\NormalTok{))}
\NormalTok{)}
\NormalTok{tamanho}
\NormalTok{knitr}\SpecialCharTok{::}\FunctionTok{kable}\NormalTok{(tamanho, }\AttributeTok{format =} \StringTok{"markdown"}\NormalTok{, }\AttributeTok{caption =} \StringTok{"Resumo das informações sobre o tamanho da matriz"}\NormalTok{)}
\NormalTok{euclid }\OtherTok{\textless{}{-}} \FunctionTok{dist}\NormalTok{(m\_bruta, }\AttributeTok{method =} \StringTok{"euclidian"}\NormalTok{, }\AttributeTok{diag =} \ConstantTok{TRUE}\NormalTok{, }\AttributeTok{upper =} \ConstantTok{FALSE}\NormalTok{)}
\CommentTok{\#?dist}
\CommentTok{\#euclid}
\FunctionTok{str}\NormalTok{(euclid)}
\FunctionTok{mode}\NormalTok{(euclid)}
\FunctionTok{class}\NormalTok{(euclid)}
\FunctionTok{length}\NormalTok{(}\FunctionTok{as.matrix}\NormalTok{(euclid))}
\FunctionTok{as.matrix}\NormalTok{(euclid)[}\DecValTok{1}\SpecialCharTok{:}\DecValTok{6}\NormalTok{, }\DecValTok{1}\SpecialCharTok{:}\DecValTok{6}\NormalTok{] }\CommentTok{\#mostra as 5 primeiras linhas e colunas da matriz}
\NormalTok{euclid\_ma }\OtherTok{\textless{}{-}}\NormalTok{ (}\FunctionTok{as.matrix}\NormalTok{(euclid))}
\CommentTok{\#View(euclid\_ma)}
\FunctionTok{str}\NormalTok{(euclid\_ma)}
\FunctionTok{mode}\NormalTok{(euclid\_ma)}
\FunctionTok{class}\NormalTok{(euclid\_ma)}
\NormalTok{euclid\_ma[}\DecValTok{1}\SpecialCharTok{:}\DecValTok{5}\NormalTok{, }\DecValTok{1}\SpecialCharTok{:}\DecValTok{5}\NormalTok{] }\CommentTok{\#mostra as 5 primeiras linhas e colunas da matriz}
\FunctionTok{range}\NormalTok{(m\_bruta)}
\FunctionTok{range}\NormalTok{(euclid)}
\FunctionTok{min}\NormalTok{(euclid)}
\FunctionTok{max}\NormalTok{(euclid)}
\FunctionTok{mean}\NormalTok{(euclid) }\CommentTok{\#CENTROIDE!! OU Grand mean}
\FunctionTok{sd}\NormalTok{(euclid)  }\CommentTok{\#Standard deviation}
\NormalTok{centroide }\OtherTok{\textless{}{-}} \FunctionTok{mean}\NormalTok{(euclid)}
\NormalTok{centroide}
\FunctionTok{length}\NormalTok{(euclid)}
\NormalTok{m }\OtherTok{\textless{}{-}} \FunctionTok{nrow}\NormalTok{(m\_bruta)}
\NormalTok{m}
\NormalTok{m}\SpecialCharTok{*}\NormalTok{(m}\DecValTok{{-}1}\NormalTok{)}\SpecialCharTok{/}\DecValTok{2}
\FunctionTok{summary}\NormalTok{(euclid)}
\NormalTok{Sumario1 }\OtherTok{\textless{}{-}} \FunctionTok{cbind}\NormalTok{(}\FunctionTok{min}\NormalTok{(euclid),}
                  \FunctionTok{max}\NormalTok{(euclid),}
                  \FunctionTok{sd}\NormalTok{(euclid),}
                  \FunctionTok{mean}\NormalTok{(euclid),}
                  \FunctionTok{length}\NormalTok{(euclid))}
\FunctionTok{colnames}\NormalTok{(Sumario1) }\OtherTok{\textless{}{-}} \FunctionTok{c}\NormalTok{(}\StringTok{"Minimo"}\NormalTok{, }\StringTok{"Maximo"}\NormalTok{, }\StringTok{"Desv.Padr"}\NormalTok{, }\StringTok{"Media"}\NormalTok{, }\StringTok{"m(m{-}1)/2"}\NormalTok{)}
\FunctionTok{rownames}\NormalTok{(Sumario1) }\OtherTok{\textless{}{-}}\NormalTok{ (}\StringTok{"Valores"}\NormalTok{)}
\NormalTok{Sumario1}
\FunctionTok{range}\NormalTok{(euclid)}
\FunctionTok{par}\NormalTok{(}\AttributeTok{mfrow=}\FunctionTok{c}\NormalTok{(}\DecValTok{2}\NormalTok{,}\DecValTok{1}\NormalTok{))}
\FunctionTok{hist}\NormalTok{(euclid,}
     \AttributeTok{breaks =} \DecValTok{15}\NormalTok{, }\CommentTok{\#determina o no. de colunas do histograma}
     \AttributeTok{xlim =} \FunctionTok{range}\NormalTok{(}\FunctionTok{floor}\NormalTok{(}\FunctionTok{min}\NormalTok{(euclid)), }\FunctionTok{ceiling}\NormalTok{(}\FunctionTok{max}\NormalTok{(euclid))), }
     \AttributeTok{xlab =} \StringTok{"Distr. de Frequências"}\NormalTok{,}
     \AttributeTok{freq =} \ConstantTok{FALSE}\NormalTok{)}
\FunctionTok{curve}\NormalTok{(}\FunctionTok{dnorm}\NormalTok{(x, }\AttributeTok{mean=}\FunctionTok{mean}\NormalTok{(euclid), }\AttributeTok{sd=}\FunctionTok{sd}\NormalTok{(euclid)), }\AttributeTok{add=}\ConstantTok{TRUE}\NormalTok{)}
\FunctionTok{boxplot.default}\NormalTok{(euclid, }\AttributeTok{horizontal =} \ConstantTok{TRUE}\NormalTok{, }\AttributeTok{frame =} \ConstantTok{FALSE}\NormalTok{,}
                \AttributeTok{xlab=}\StringTok{"Distr. de Frequências"}\NormalTok{,}
                \AttributeTok{ylim=}\FunctionTok{c}\NormalTok{(}\FunctionTok{floor}\NormalTok{(}\FunctionTok{min}\NormalTok{(euclid)), }\FunctionTok{ceiling}\NormalTok{(}\FunctionTok{max}\NormalTok{(euclid)))) }\CommentTok{\#Limites do eixo Y}
\DocumentationTok{\#\# dev.off()}
\FunctionTok{library}\NormalTok{(matrixStats)}
\FunctionTok{library}\NormalTok{(moments)}
\NormalTok{euclid\_ma }\OtherTok{\textless{}{-}} \FunctionTok{as.matrix}\NormalTok{(euclid)}
\NormalTok{euclid\_ma}
\FunctionTok{range}\NormalTok{(euclid\_ma) }\CommentTok{\#valor errado}
\FunctionTok{mean}\NormalTok{(euclid\_ma) }\CommentTok{\#valor errado}
\FunctionTok{sd}\NormalTok{(euclid\_ma) }\CommentTok{\#valor errado}
\FunctionTok{is.na}\NormalTok{(euclid\_ma) }\OtherTok{\textless{}{-}}\NormalTok{ euclid\_ma}\SpecialCharTok{==}\DecValTok{0} \CommentTok{\#atribui n.a. aos valores = 0}
\FunctionTok{mean}\NormalTok{(euclid\_ma, }\AttributeTok{na.rm=}\NormalTok{T)  }\CommentTok{\#valor correto, omite valores n.a. do calculo}
\FunctionTok{mean}\NormalTok{(euclid\_ma, }\AttributeTok{na.rm=}\NormalTok{T) }\CommentTok{\#valor correto, omite valores n.a. do calculo}
\FunctionTok{sd}\NormalTok{(euclid\_ma, }\AttributeTok{na.rm=}\NormalTok{T) }\CommentTok{\#valor correto, omite valores n.a. do calculo}
\CommentTok{\#colMeans(euclid\_ma, na.rm=T) \#omite valores n.a. do calculo}
\CommentTok{\#rowMeans(euclid\_ma, na.rm=T) \#omite valores n.a. do calculo}
\NormalTok{centroide\_ma }\OtherTok{\textless{}{-}} \FunctionTok{mean}\NormalTok{(euclid\_ma, }\AttributeTok{na.rm=}\NormalTok{T)}
\NormalTok{av.dist }\OtherTok{\textless{}{-}}\NormalTok{ (}\FunctionTok{as.matrix}\NormalTok{(}\FunctionTok{colMeans}\NormalTok{(euclid\_ma, }\AttributeTok{na.rm=}\NormalTok{T)))}
\NormalTok{av.desvpad }\OtherTok{\textless{}{-}}\NormalTok{ (}\FunctionTok{as.matrix}\NormalTok{(}\FunctionTok{colSds}\NormalTok{(euclid\_ma, }\AttributeTok{na.rm=}\NormalTok{T)))}
\NormalTok{dp.centroide\_ma }\OtherTok{\textless{}{-}}\NormalTok{ (av.dist}\SpecialCharTok{{-}}\NormalTok{centroide\_ma)}\SpecialCharTok{/}\NormalTok{(}\FunctionTok{colSds}\NormalTok{(av.dist)) }\CommentTok{\#ou z{-}scores}
\NormalTok{list }\OtherTok{\textless{}{-}} \FunctionTok{as.matrix}\NormalTok{(}\FunctionTok{cbind}\NormalTok{(av.dist, av.desvpad, dp.centroide\_ma))}
\NormalTok{list}
\FunctionTok{colnames}\NormalTok{(list, }\AttributeTok{do.NULL =} \ConstantTok{FALSE}\NormalTok{)}
\FunctionTok{colnames}\NormalTok{(list) }\OtherTok{\textless{}{-}} \FunctionTok{c}\NormalTok{(}\StringTok{"Av.Dist"}\NormalTok{, }\StringTok{"Av.StDev"}\NormalTok{, }\StringTok{"DP.Centroide"}\NormalTok{)}
\NormalTok{list2 }\OtherTok{\textless{}{-}}\NormalTok{ list[}\FunctionTok{order}\NormalTok{(list[,}\DecValTok{1}\NormalTok{], }\AttributeTok{decreasing =} \ConstantTok{TRUE}\NormalTok{),] }\CommentTok{\#[,1] ou o nome da coluna}
\NormalTok{list2}
\FunctionTok{par}\NormalTok{(}\AttributeTok{mfrow=}\FunctionTok{c}\NormalTok{(}\DecValTok{3}\NormalTok{,}\DecValTok{1}\NormalTok{))}
\FunctionTok{hist}\NormalTok{(list2[, }\StringTok{"Av.Dist"}\NormalTok{],}
     \AttributeTok{breaks =} \DecValTok{15}\NormalTok{, }\CommentTok{\#determina o no. de colunas do histograma}
     \AttributeTok{xlab =} \StringTok{"Distr. de Frequências das Distâncias (em dp) para o centroide"}\NormalTok{,}
     \AttributeTok{main =} \StringTok{"Distribuição de Frequência da distância média para o centroide"}\NormalTok{,}
     \AttributeTok{xlim =} \FunctionTok{range}\NormalTok{(}\FunctionTok{floor}\NormalTok{(}\FunctionTok{min}\NormalTok{(av.dist)), }\FunctionTok{ceiling}\NormalTok{(}\FunctionTok{max}\NormalTok{(av.dist))), }\CommentTok{\#substitua aqui o menor e maior valor do \textasciigrave{}range()\textasciigrave{}}
     \AttributeTok{freq =}\NormalTok{ T)}
\FunctionTok{hist}\NormalTok{(list2[, }\StringTok{"Av.Dist"}\NormalTok{],}
     \AttributeTok{breaks =} \DecValTok{15}\NormalTok{, }\CommentTok{\#determina o no. de colunas do histograma}
     \AttributeTok{xlab =} \StringTok{"Distr. de Frequências das Distâncias (em dp) para o centroide"}\NormalTok{,}
     \AttributeTok{main =} \StringTok{"Curva de normalidade ajustada para a Distribuição de Frequência"}\NormalTok{,}
     \AttributeTok{xlim =} \FunctionTok{range}\NormalTok{(}\FunctionTok{floor}\NormalTok{(}\FunctionTok{min}\NormalTok{(av.dist)), }\FunctionTok{ceiling}\NormalTok{(}\FunctionTok{max}\NormalTok{(av.dist))), }\CommentTok{\#substitua aqui o menor e maior valor do \textasciigrave{}range()\textasciigrave{}}
     \AttributeTok{freq =}\NormalTok{ F)}
\FunctionTok{curve}\NormalTok{(}\FunctionTok{dnorm}\NormalTok{(x, }\AttributeTok{mean=}\FunctionTok{mean}\NormalTok{(list2[, }\StringTok{"Av.Dist"}\NormalTok{]), }\AttributeTok{sd=}\FunctionTok{sd}\NormalTok{(list2[, }\StringTok{"Av.Dist"}\NormalTok{])), }\AttributeTok{add=}\ConstantTok{TRUE}\NormalTok{)}
\FunctionTok{boxplot.default}\NormalTok{(list2[, }\StringTok{"Av.Dist"}\NormalTok{], }\AttributeTok{horizontal =} \ConstantTok{TRUE}\NormalTok{, }\AttributeTok{frame =} \ConstantTok{FALSE}\NormalTok{,}
                \AttributeTok{xlab=}\StringTok{"Distr. de Frequências"}\NormalTok{,}
                \AttributeTok{ylim=}\FunctionTok{c}\NormalTok{(}\FunctionTok{floor}\NormalTok{(}\FunctionTok{min}\NormalTok{(av.dist)), }\FunctionTok{ceiling}\NormalTok{(}\FunctionTok{max}\NormalTok{(av.dist)))) }\CommentTok{\#substitua aqui o menor e maior valor do \textasciigrave{}range()\textasciigrave{}}
\FunctionTok{par}\NormalTok{(}\AttributeTok{mfrow=}\FunctionTok{c}\NormalTok{(}\DecValTok{1}\NormalTok{,}\DecValTok{1}\NormalTok{))}
\FunctionTok{range}\NormalTok{(dp.centroide\_ma)}
\FunctionTok{par}\NormalTok{(}\AttributeTok{mfrow=}\FunctionTok{c}\NormalTok{(}\DecValTok{3}\NormalTok{,}\DecValTok{1}\NormalTok{))}
\FunctionTok{hist}\NormalTok{(list2[, }\StringTok{"DP.Centroide"}\NormalTok{],}
     \AttributeTok{breaks =} \DecValTok{15}\NormalTok{, }\CommentTok{\#determina o no. de colunas do histograma}
     \AttributeTok{xlab =} \StringTok{"Distr. de Frequências das Distâncias dos desvios padões para o centroide"}\NormalTok{,}
     \AttributeTok{main =} \StringTok{"Distribuição de Frequência dos desvio padões das distâncias médias para o centroide"}\NormalTok{,}
     \AttributeTok{xlim =} \FunctionTok{range}\NormalTok{(}\FunctionTok{floor}\NormalTok{(}\FunctionTok{min}\NormalTok{(dp.centroide\_ma)), }\FunctionTok{ceiling}\NormalTok{(}\FunctionTok{max}\NormalTok{(dp.centroide\_ma))), }\CommentTok{\#substitua aqui o menor e maior valor do \textasciigrave{}range()\textasciigrave{}}
     \AttributeTok{freq =}\NormalTok{ T)}
\FunctionTok{hist}\NormalTok{(list2[, }\StringTok{"DP.Centroide"}\NormalTok{],}
     \AttributeTok{breaks =} \DecValTok{15}\NormalTok{, }\CommentTok{\#determina o no. de colunas do histograma}
     \AttributeTok{xlab =} \StringTok{"Distr. de Frequências das Distâncias dos desvios padrões das distâncias médias para o centroide"}\NormalTok{,}
     \AttributeTok{main =} \StringTok{"Curva de normalidade ajustada para a Distribuição de Frequência"}\NormalTok{,}
     \AttributeTok{xlim =} \FunctionTok{range}\NormalTok{(}\FunctionTok{floor}\NormalTok{(}\FunctionTok{min}\NormalTok{(dp.centroide\_ma)), }\FunctionTok{ceiling}\NormalTok{(}\FunctionTok{max}\NormalTok{(dp.centroide\_ma))), }\CommentTok{\#substitua aqui o menor e maior valor do \textasciigrave{}range()\textasciigrave{}}
     \AttributeTok{freq =}\NormalTok{ F)}
\FunctionTok{curve}\NormalTok{(}\FunctionTok{dnorm}\NormalTok{(x, }\AttributeTok{mean=}\FunctionTok{mean}\NormalTok{(list2[, }\StringTok{"DP.Centroide"}\NormalTok{]), }\AttributeTok{sd=}\FunctionTok{sd}\NormalTok{(list2[, }\StringTok{"DP.Centroide"}\NormalTok{])), }\AttributeTok{add=}\ConstantTok{TRUE}\NormalTok{)}
\FunctionTok{boxplot.default}\NormalTok{(list2[, }\StringTok{"DP.Centroide"}\NormalTok{], }\AttributeTok{horizontal =} \ConstantTok{TRUE}\NormalTok{, }\AttributeTok{frame =} \ConstantTok{FALSE}\NormalTok{,}
                \AttributeTok{xlab=}\StringTok{"Distr. de Frequências"}\NormalTok{,}
                \AttributeTok{ylim=}\FunctionTok{c}\NormalTok{(}\FunctionTok{floor}\NormalTok{(}\FunctionTok{min}\NormalTok{(dp.centroide\_ma)), }\FunctionTok{ceiling}\NormalTok{(}\FunctionTok{max}\NormalTok{(dp.centroide\_ma)))) }\CommentTok{\#substitua aqui o menor e maior valor do \textasciigrave{}range()\textasciigrave{}}
\FunctionTok{par}\NormalTok{(}\AttributeTok{mfrow=}\FunctionTok{c}\NormalTok{(}\DecValTok{1}\NormalTok{,}\DecValTok{1}\NormalTok{))}
\DocumentationTok{\#\# dev.off()}
\NormalTok{cutoff }\OtherTok{\textless{}{-}} \FloatTok{2.0}
\FunctionTok{library}\NormalTok{(gt)}
\FunctionTok{format}\NormalTok{(cutoff, }\AttributeTok{nsmall =} \DecValTok{1}\NormalTok{)}
\NormalTok{listf }\OtherTok{\textless{}{-}} \FunctionTok{as.data.frame}\NormalTok{(list2)}
\NormalTok{listf}\SpecialCharTok{$}\NormalTok{Outliers }\OtherTok{\textless{}{-}} \FunctionTok{ifelse}\NormalTok{(listf}\SpecialCharTok{$}\NormalTok{DP.Centroide}\SpecialCharTok{\textgreater{}{-}}\NormalTok{cutoff }\CommentTok{\#CUTOFF MENOR QUE {-}\textquotesingle{}cutoff}
                         \SpecialCharTok{\&}\NormalTok{ listf}\SpecialCharTok{$}\NormalTok{DP.Centroide}\SpecialCharTok{\textless{}}\NormalTok{cutoff,  }\CommentTok{\#CUTOFF MAIOR QUE \textquotesingle{}cutoff\textquotesingle{}}
                         \StringTok{""}\NormalTok{, }\StringTok{"OUT"}\NormalTok{) }
\NormalTok{listf}
\FunctionTok{gt}\NormalTok{(}\FunctionTok{cbind}\NormalTok{(}\AttributeTok{Sitios=}\FunctionTok{rownames}\NormalTok{(listf),listf))}
\NormalTok{part }\OtherTok{\textless{}{-}} \FunctionTok{c}\NormalTok{(}\StringTok{"B{-}A{-}MU4"}\NormalTok{, }\StringTok{"B{-}A{-}MU3"}\NormalTok{, }\StringTok{"S{-}A{-}TA4"}\NormalTok{, }\StringTok{"S{-}R{-}CT3"}\NormalTok{)}
\NormalTok{part}
\NormalTok{m\_bruta\_part }\OtherTok{\textless{}{-}}\NormalTok{ m\_bruta[}\SpecialCharTok{!}\NormalTok{(}\FunctionTok{row.names}\NormalTok{(m\_bruta) }\SpecialCharTok{\%in\%} \FunctionTok{c}\NormalTok{(part)),]}
\CommentTok{\#m\_bruta\_part}
\NormalTok{sum }\OtherTok{\textless{}{-}} \FunctionTok{colSums}\NormalTok{(m\_bruta\_part)}
\NormalTok{sum}
\NormalTok{zero\_sum\_cols }\OtherTok{\textless{}{-}} \FunctionTok{names}\NormalTok{(}\FunctionTok{which}\NormalTok{(}\FunctionTok{colSums}\NormalTok{(m\_bruta\_part) }\SpecialCharTok{==} \DecValTok{0}\NormalTok{))}
\NormalTok{zero\_sum\_cols }\CommentTok{\#nomes das espécies zeradas}
\NormalTok{m\_bruta\_part2 }\OtherTok{\textless{}{-}}\NormalTok{ m\_bruta\_part[(}\FunctionTok{colSums}\NormalTok{(m\_bruta\_part) }\SpecialCharTok{!=} \DecValTok{0}\NormalTok{)] }\CommentTok{\#em != a exclamação inverte o sentido}
\NormalTok{zero\_sum\_cols2 }\OtherTok{\textless{}{-}} \FunctionTok{names}\NormalTok{(}\FunctionTok{which}\NormalTok{(}\FunctionTok{colSums}\NormalTok{(m\_bruta\_part2) }\SpecialCharTok{==} \DecValTok{0}\NormalTok{))}
\NormalTok{zero\_sum\_cols2 }\CommentTok{\#nomes das espécies zeradas}
\NormalTok{sum}\OtherTok{\textless{}{-}}\FunctionTok{colSums}\NormalTok{(m\_bruta\_part2)}
\NormalTok{sum}
\CommentTok{\#m\_bruta\_part2}
\CommentTok{\#m\_bruta\_part2 \textless{}{-} as.matrix(m\_bruta\_part2)}
\FunctionTok{str}\NormalTok{(m\_bruta\_part2)}
\FunctionTok{length}\NormalTok{(}\FunctionTok{as.matrix}\NormalTok{(m\_bruta\_part2))}
\DocumentationTok{\#\# df \textless{}{-} data.frame(Sites = rownames(m\_bruta), m\_bruta,}
\DocumentationTok{\#\#                  row.names = NULL,}
\DocumentationTok{\#\#                  check.names = FALSE) \#add titulo a primeira coluna}
\DocumentationTok{\#\# }
\DocumentationTok{\#\# write.table(m\_bruta\_part2, "m\_trabcsv.csv",}
\DocumentationTok{\#\#             sep = ";", dec = ".", \#"\textbackslash{}t",}
\DocumentationTok{\#\#             row.names = TRUE,}
\DocumentationTok{\#\#             quote = TRUE,}
\DocumentationTok{\#\#             append = FALSE)}
\DocumentationTok{\#\# }
\DocumentationTok{\#\# m\_trab \textless{}{-} read.csv("m\_trabcsv.csv",}
\DocumentationTok{\#\#                    sep = ";", dec = ".",}
\DocumentationTok{\#\#                    row.names = 1,}
\DocumentationTok{\#\#                    header = TRUE,}
\DocumentationTok{\#\#                    na.strings = NA,}
\DocumentationTok{\#\#                    check.names = FALSE, \#impede que o R mude os nomes das colunas}
\DocumentationTok{\#\#                    col.names = gsub("(\^{}|\_)([a{-}z])", "\textbackslash{}\textbackslash{}1\textbackslash{}\textbackslash{}U\textbackslash{}\textbackslash{}2",}
\DocumentationTok{\#\#                                     names(m\_trab), perl = TRUE))}
\end{Highlighting}
\end{Shaded}

\hypertarget{referuxeancias-13}{%
\section*{Referências}\label{referuxeancias-13}}
\addcontentsline{toc}{section}{Referências}

\hypertarget{part-anuxe1lises-multivariadas}{%
\part{ANÁLISES MULTIVARIADAS}\label{part-anuxe1lises-multivariadas}}

\hypertarget{upgma}{%
\chapter{R Módulo 7 - Análise de Classificação Hierárquica - SAHN}\label{upgma}}

\hypertarget{resumo-12}{%
\subsection*{RESUMO}\label{resumo-12}}
\addcontentsline{toc}{subsection}{RESUMO}

Análise de Classificação Hierárquica (SAHN) é um método de análise de dados utilizado para identificar grupos ou clusters de objetos com base em sua similaridade. É amplamente utilizada em várias áreas, incluindo bioinformática, análise de dados espaciais, agronomia, entre outros. A análise de classificação hierárquica é uma técnica exploratória importante que pode ajudar a identificar padrões e relações em dados não estruturados. No entanto, é importante lembrar que a interpretação dos resultados pode ser subjetiva e depender do contexto da aplicação.

\hypertarget{apresentauxe7uxe3o-13}{%
\subsection*{Apresentação}\label{apresentauxe7uxe3o-13}}
\addcontentsline{toc}{subsection}{Apresentação}

Análise de Classificação Hierárquica (do inglês, ``Hierarchical Agglomerative Clustering Analysis'', ou simplesmente ``SAHN'') é um método de análise de dados utilizado para identificar grupos ou ``clusters'' de objetos com base em sua similaridade. Nessa análise, os objetos são inicialmente considerados como clusters individuais e, em seguida, os objetos mais similares são agrupados em um cluster maior, e assim por diante, até que todos os objetos estejam em um único cluster. Isso resulta em uma árvore hierárquica, conhecida como dendrograma, que mostra a relação de similaridade entre os objetos.

Existem dois tipos principais de classificação hierárquica: aglomerativa e divisiva. O método aglomerativo é mais comum e começa com cada objeto em seu próprio cluster, e sucessivamente agrupa os objetos mais próximos, enquanto o método divisivo começa com todos os objetos em um único cluster e sucessivamente divide-os em grupos menores.

Análises de classificação são amplamente utilizadas em várias áreas, incluindo bioinformática, análise de dados espaciais, agronomia, entre outros. Além disso, pode ser utilizada com diferentes medidas de distância e diferentes métodos de agrupamento, como, por exemplo, o método ``complete-linkage'', ``single-linkage'', ``average-linkage'', entre outros.

A Análise de Classificação Hierárquica é uma técnica exploratória importante que pode ajudar a identificar padrões e relações em dados não estruturados. No entanto, é importante lembrar que a interpretação dos resultados pode ser subjetiva e depender do contexto da aplicação.

\hypertarget{organizauxe7uxe3o-buxe1sica-10}{%
\section{Organização básica}\label{organizauxe7uxe3o-buxe1sica-10}}

\begin{Shaded}
\begin{Highlighting}[]
\FunctionTok{dev.off}\NormalTok{() }\CommentTok{\#apaga os graficos, se houver algum}
\FunctionTok{rm}\NormalTok{(}\AttributeTok{list=}\FunctionTok{ls}\NormalTok{(}\AttributeTok{all=}\ConstantTok{TRUE}\NormalTok{)) }\DocumentationTok{\#\#LIMPA A MEMORIA}
\FunctionTok{cat}\NormalTok{(}\StringTok{"}\SpecialCharTok{\textbackslash{}014}\StringTok{"}\NormalTok{) }\CommentTok{\#limpa o console }
\CommentTok{\#rm(list=ls(all=TRUE)); cat("\textbackslash{}014"); dev.off() \# tudo na mesma linha}
\end{Highlighting}
\end{Shaded}

\hypertarget{pacotes}{%
\subsection{Pacotes}\label{pacotes}}

Instalando os pacotes necessários para esse módulo

\begin{Shaded}
\begin{Highlighting}[]
\FunctionTok{install.packages}\NormalTok{(}\StringTok{"tidyverse"}\NormalTok{)}
\FunctionTok{install.packages}\NormalTok{(}\StringTok{"openxlsx"}\NormalTok{)}
\FunctionTok{install.packages}\NormalTok{(}\StringTok{"vegan"}\NormalTok{)}
\FunctionTok{install.packages}\NormalTok{(}\StringTok{"dplyr"}\NormalTok{)}
\FunctionTok{install.packages}\NormalTok{(}\StringTok{"RColorBrewer"}\NormalTok{)}
\FunctionTok{install.packages}\NormalTok{(}\StringTok{"gplots"}\NormalTok{)}
\end{Highlighting}
\end{Shaded}

\begin{Shaded}
\begin{Highlighting}[]
\FunctionTok{library}\NormalTok{(tidyverse)}
\end{Highlighting}
\end{Shaded}

Os códigos acima, são usados para instalar os pacotes necessários para este módulo. Esses códigos são comandos para instalar pacotes no R. Um pacote é uma coleção de funções, dados e documentação que ampliam as capacidades do R (\href{https://cran.r-project.org/}{R CRAN} e \href{https://posit.co/download/rstudio-desktop/}{RStudio}). Depois de instalar um pacote, você precisa carregá-lo na sua sessão R com a função \texttt{library()}. Por exemplo, no código acima, carregamos o pacote \texttt{tidyverse}, usando a função \texttt{library(openxlsx)}. Isso irá permitir que você use as funções do pacote na sua sessão R. Você precisa carregar um pacote toda vez que iniciar uma nova sessão R e quiser usar um pacote instalado. Os demais pacotes instalados serão carregados ao longo desse tutorial a medida que cada pacote for sendo necessário.

Agora vamos \textbf{definir o diretório de trabalho}. Esse código é usado para obter e definir o diretório de trabalho atual no R. O comando \texttt{getwd()} retorna o caminho do diretório onde o R está lendo e salvando arquivos. O comando \texttt{setwd()} muda esse diretório de trabalho para o caminho especificado entre aspas. No seu caso, você deve ajustar o caminho para o seu próprio diretório de trabalho. \textbf{Lembre de usar a barra ``/'' entre os diretórios. E não a contra-barra ``\textbackslash{}''.}

Definindo o diretório de trabalho e instalando os pacotes necessários:

\begin{Shaded}
\begin{Highlighting}[]
\FunctionTok{getwd}\NormalTok{()}
\FunctionTok{setwd}\NormalTok{(}\StringTok{"C:/Seu/Diretório/De/Trabalho"}\NormalTok{)}
\end{Highlighting}
\end{Shaded}

Alternativamente você pode ir na barra de tarefas e escolhes as opções:\\
SESSION -\textgreater{} SET WORKING DIRECTORY -\textgreater{} CHOOSE DIRECTORY

\hypertarget{sobre-os-dados-do-ppbio-4}{%
\subsection{Sobre os dados do PPBio}\label{sobre-os-dados-do-ppbio-4}}

A planilha \texttt{ppbio} contém os dados de abundância de espécies em diferentes unidades amostrais (UA's) (Figura \ref{fig:27ppbio} (Veja \href{https://ppbio.inpa.gov.br/Sobre}{Programa de Pesquisa em Biodiversidade -- PPBio}). Essa é a \textbf{matriz bruta de dados}, porque os valores ainda não foram ajustados para os valores de Captura Por Unidade de Esforço (CPUE), nem foram relativizados ou transformados.

\begin{figure}
\includegraphics[width=13.21in]{imagens/ppbio} \caption{Parte da planilha de dados brutos do PPBio.}\label{fig:27ppbio}
\end{figure}

\hypertarget{importando-a-planilha-de-trabalho-1}{%
\subsection{Importando a planilha de trabalho}\label{importando-a-planilha-de-trabalho-1}}

Note que o sómbolo \texttt{\#} em programação R significa que o texto que vem depois dele é um comentário e não será executado pelo programa. Isso é útil para explicar o código ou deixar anotações. Ajuste a segunda linha do código abaixo para refletir {``C:/Seu/Diretório/De/Trabalho/Planilha.xlsx''}.

\begin{Shaded}
\begin{Highlighting}[]
\FunctionTok{library}\NormalTok{(openxlsx)}
\NormalTok{ppbio }\OtherTok{\textless{}{-}} \FunctionTok{read.xlsx}\NormalTok{(}\StringTok{"D:/Elvio/OneDrive/Disciplinas/\_EcoNumerica/5.Matrizes/ppbio06p{-}peixes.xlsx"}\NormalTok{,}
                   \AttributeTok{rowNames =}\NormalTok{ T,}
                   \AttributeTok{colNames =}\NormalTok{ T,}
                   \AttributeTok{sheet =} \StringTok{"Sheet1"}\NormalTok{)}
\FunctionTok{str}\NormalTok{(ppbio)}
\NormalTok{ppbio\_ma }\OtherTok{\textless{}{-}} \FunctionTok{as.matrix}\NormalTok{(ppbio) }\CommentTok{\#lê ppbio como uma matriz}
\FunctionTok{str}\NormalTok{(ppbio\_ma)}
\CommentTok{\#ppbio}
\CommentTok{\#ppbio\_ma}
\end{Highlighting}
\end{Shaded}

\hypertarget{reset-point-4}{%
\section{Reset point}\label{reset-point-4}}

\leavevmode\vadjust pre{\hypertarget{hello}{}}%
{ATENÇÃO}
Aqui substitui-se uma nova matriz de dados, relativizada e/ou transformada, pela matriz de trabalho inicial.

\begin{Shaded}
\begin{Highlighting}[]
\NormalTok{m\_trab }\OtherTok{\textless{}{-}}\NormalTok{ (ppbio)   }\CommentTok{\# \textless{}1\textgreater{}}
\end{Highlighting}
\end{Shaded}

\begin{enumerate}
\def\labelenumi{\arabic{enumi}.}
\tightlist
\item
  Aqui usaremos as matrizes transposta/relativizada/transformada/particionada
\end{enumerate}

\hypertarget{outra-forma-de-achar-e-importar-uma-planilha-1}{%
\subsubsection{Outra forma de achar e importar uma planilha}\label{outra-forma-de-achar-e-importar-uma-planilha-1}}

\begin{Shaded}
\begin{Highlighting}[]
\FunctionTok{getwd}\NormalTok{()}
\NormalTok{ppbio }\OtherTok{\textless{}{-}} \FunctionTok{read.xlsx}\NormalTok{(}\FunctionTok{file.choose}\NormalTok{(),}
                   \AttributeTok{rowNames =}\NormalTok{ T, }\AttributeTok{colNames =}\NormalTok{ T,}
                   \AttributeTok{sheet =} \StringTok{"Sheet1"}\NormalTok{)}
\end{Highlighting}
\end{Shaded}

\hypertarget{nomenclatura-das-matrizes-em-amd}{%
\section{Nomenclatura das matrizes em AMD}\label{nomenclatura-das-matrizes-em-amd}}

No tocante aos tipos de atributos e seu tratamento pré-análises, uma matriz de de dados pode ser dos tipos apresentados na tabela \ref{tab:tbl-mm}, abaixo.

\begin{verbatim}
## Warning in table_info$align_vector[column] <-
## unlist(lapply(table_info$align_vector_origin[column], : número de itens para
## para substituir não é um múltiplo do comprimento do substituto
\end{verbatim}

\begin{table}

\caption{\label{tab:tbl-mm}Nomenclatura das matrizes em AMD em relação aos atributos das colunas.}
\centering
\begin{tabular}[t]{>{\raggedright\arraybackslash}p{2cm}>{\raggedright\arraybackslash}p{5cm}}
\toprule
Nome & Atributos (colunas)\\
\midrule
Matriz comunitaria & Os atributos são táxons (ex. espécies, gêneros, morfotipos)\\
Matriz ambiental & Os atributos são dados ambientais (ex. pH, condutividade, temperatura)\\
Matriz bruta & Os atributos ainda não receberam nenhum tipo de tratamento estatísco (valores brutos, como coletados)\\
Matriz transposta & Os atributos foram transpostos para as linhas\\
Matriz relativizada & Os atributos foram relativizados por um critério de tamanho ou de variação (ex. dividir os valores de cada coluna pela soma)\\
\addlinespace
Matriz transformada & Foi aplicado um operador matemático a todos os atributos (ex. raiz quadrada, log)\\
\bottomrule
\end{tabular}
\end{table}

\hypertarget{cluster-1-piloto-automuxe1tico}{%
\section{Cluster 1: Piloto automático}\label{cluster-1-piloto-automuxe1tico}}

\leavevmode\vadjust pre{\hypertarget{hello}{}}%
{ATENÇÃO}
NÃO RECOMENDADO

Aqui fazemos uma Análise Cluster no piloto automático. Não decidimos nenhum dos parâmetros importântes para uma Classificação.

\begin{Shaded}
\begin{Highlighting}[]
\NormalTok{cluster1 }\OtherTok{\textless{}{-}} \FunctionTok{hclust}\NormalTok{(}\FunctionTok{dist}\NormalTok{(m\_trab))}
\FunctionTok{plot}\NormalTok{(cluster1, }\AttributeTok{main =} \StringTok{"Cluster Dendrogram {-} Piloto automatico"}\NormalTok{)}
\FunctionTok{rect.hclust}\NormalTok{(cluster1, }\AttributeTok{k=}\DecValTok{3}\NormalTok{, }\AttributeTok{h =} \ConstantTok{NULL}\NormalTok{)}
\CommentTok{\#?dist}
\CommentTok{\#?hclust}
\end{Highlighting}
\end{Shaded}

\includegraphics{fundamentos_de_ecologia_numerica_files/figure-latex/unnamed-chunk-226-1.pdf}

É possivel definir uma caixa mostrado os principais grupos formados. Isso pode ser feito estabelecendo quantos grupos se quer mostrar, usando a função \texttt{k\ =\ no.\ de\ grupos}. Ou pode-se definir os grupos formados até uma determinanda distância, usando a função \texttt{h\ =\ altura\ dos\ grupos} no eixo das distâncias.

\hypertarget{histuxf3rico-das-fusuxf5es}{%
\subsection{Histórico das fusões}\label{histuxf3rico-das-fusuxf5es}}

\begin{Shaded}
\begin{Highlighting}[]
\FunctionTok{library}\NormalTok{(dplyr)}
\NormalTok{cluster1}\SpecialCharTok{$}\NormalTok{merge }\CommentTok{\#mostra o histórico das fusões}
\NormalTok{idrow }\OtherTok{\textless{}{-}} \FunctionTok{mutate}\NormalTok{(m\_trab, }\AttributeTok{id =} \FunctionTok{row\_number}\NormalTok{()) }\CommentTok{\#cria um df com os numeros das linhas }
\NormalTok{idrow }\SpecialCharTok{\%\textgreater{}\%} \FunctionTok{relocate}\NormalTok{(id)}
\CommentTok{\#?relocate}
\end{Highlighting}
\end{Shaded}

\begin{verbatim}
##       [,1] [,2]
##  [1,]   -2   -3
##  [2,]  -14    1
##  [3,]   -8    2
##  [4,]   -5  -11
##  [5,]    3    4
##  [6,]  -15  -18
##  [7,]  -21    5
##  [8,]  -17  -23
##  [9,]  -16    7
## [10,]   -9  -12
## [11,]    6    9
## [12,]  -10   11
## [13,]   -1   -4
## [14,]  -20    8
## [15,]   -6  -13
## [16,]   12   15
## [17,]   13   16
## [18,]  -19   10
## [19,]   14   17
## [20,]   -7   19
## [21,]   18   20
## [22,]  -22   21
##         id ap-davis as-bimac as-fasci ch-bimac ci-ocela ci-orien co-macro
## S-R-CT1  1        0      194       55        0        0        5        0
## S-R-CP1  2        0       19        0        0        0        0        0
## S-A-TA1  3        0       23        1       13        0        0        0
## S-R-CT2  4        0      142        3        3        0       69        0
## S-R-CP2  5        0        5        1        0       40        9        0
## S-A-TA2  6        0       46        0      178        0        0        0
## S-R-CT3  7        0      206       64        0        0       25        0
## S-R-CP3  8        0       16        0        0       13       24        0
## S-A-TA3  9        0      234        7      238        0        0        2
## S-R-CT4 10        0        0        1        0        0        5        0
## S-R-CP4 11        0        0        0        0       11        6        0
## S-A-TA4 12        0      394        0      273        0        0        0
## B-A-MU1 13        0       12        0        0        0        0        0
## B-A-GU1 14        0        2        2        0        0        0        0
## B-R-PC2 15        5       44        0        0        2        0        0
## B-A-MU2 16        0       99        0        0        0        0        0
## B-A-GU2 17        0        0        0        0        0        0        0
## B-R-PC3 18       22       75        7        0        4        0        0
## B-A-MU3 19        0      511        0        0        0        0        0
## B-A-GU3 20        0        6        0        0        0        0        0
## B-R-PC4 21        0        7       17        0        0        0        0
## B-A-MU4 22        0      235        0        0        0        0        0
## B-A-GU4 23        0       13        0        0        0        0        0
##         co-heter cr-menez cu-lepid cy-gilbe ge-brasi he-margi ho-malab hy-pusar
## S-R-CT1        1       14        0        0        3        0        1        9
## S-R-CP1        0        0        0        0        0        0        5        2
## S-A-TA1        0        0        0        0        0        0        0        0
## S-R-CT2        0        4        0        0        0        1       17       43
## S-R-CP2        0        0        0        0        0        0       10        2
## S-A-TA2        0        0        0        0        0        0        2        0
## S-R-CT3        0        8        0        0        1        0       31       11
## S-R-CP3        0        0        0        0        0        0        4        0
## S-A-TA3        0        0        0        0        0        0       20        0
## S-R-CT4        0        1        0       50        3        1        4        3
## S-R-CP4        0        0        0        0        0        0        2        0
## S-A-TA4        0        1        0        0        1        0        9        0
## B-A-MU1        0        0        0        0      190        0        0        0
## B-A-GU1        0        0        0        0        7        0        0        0
## B-R-PC2        0        0        0        0        8        0        0        0
## B-A-MU2        0        0        0        0       67        0        1        0
## B-A-GU2        0        0        0        0       23        0        0        0
## B-R-PC3        0        0       21        0       16        0        2        1
## B-A-MU3        0        0        0        0      145        0        0        0
## B-A-GU3        0        0        0        0       32        0        0        0
## B-R-PC4        0        0        0       81        5        0        1        0
## B-A-MU4        0        0        0        0      509        0        0        0
## B-A-GU4        0        0        0        0       10        0        0        0
##         le-melan le-piau le-taeni mo-costa mo-lepid or-nilot pa-manag pimel-sp
## S-R-CT1        0       3        0        0       39       36        0        6
## S-R-CP1        0       0        0        0        0        0        0        0
## S-A-TA1        0       0        0        0        0        0        0        0
## S-R-CT2        0       1        0        0        1       77        0        0
## S-R-CP2        0       3        0        0        0        0        0        0
## S-A-TA2        0       0        0        0        0        0        0        0
## S-R-CT3        0       2        0        0        0      138        0        0
## S-R-CP3        0       1        0        0        0        0        0        0
## S-A-TA3        0       0        0        0        0        0        0        0
## S-R-CT4        0       0        0        0        0       73        0        0
## S-R-CP4        0       2        0        0        0        0        0        0
## S-A-TA4        0       2        0        0        0        1        0        0
## B-A-MU1        0       0        0        0        0        6        0        0
## B-A-GU1        0       0        0        0        0        3       11        0
## B-R-PC2        2       0        1        0        0        5        0        0
## B-A-MU2        0       0        0        0        0        1        0        0
## B-A-GU2        0       0        0        0        0       36      102        0
## B-R-PC3        0       0        0        1        0       65        0        0
## B-A-MU3        0       0        0        0        0       11        0        0
## B-A-GU3        0       0        0        0        0      247      250        0
## B-R-PC4        0       1        0        0        0        9        0        0
## B-A-MU4        0       0        0        0        0        1        0        0
## B-A-GU4        0       0        0        0        0      129      190        0
##         po-retic po-vivip pr-brevi ps-rhomb ps-genise se-heter se-piaba
## S-R-CT1        0       47        5        0         0       40       68
## S-R-CP1        0       15        0        0         0       14        0
## S-A-TA1        0        0        1        0         0        4        0
## S-R-CT2       20      221       15        0         0       60        0
## S-R-CP2        0       32        5        0         0        0        0
## S-A-TA2        0        0        2        0         0        0        0
## S-R-CT3        5      326      164        1         1       38        0
## S-R-CP3        0       10        0        0         0        0        0
## S-A-TA3        0        0        0        0         0        0        0
## S-R-CT4        0       28       59        0         0        3        0
## S-R-CP4        0       80        0        0         0        3        0
## S-A-TA4        0        0        3        0         0        0        0
## B-A-MU1        0        0        0        0         0        0        0
## B-A-GU1        0        0        0        0         0        0        0
## B-R-PC2        0        0        9        0         0       10        0
## B-A-MU2       10        8        0        0         0        0        0
## B-A-GU2        0        0        0        0         0        0        0
## B-R-PC3        0        0        6        0         0       93        0
## B-A-MU3       46       48        1        0         0        0        0
## B-A-GU3        0        0        0        0         0        0        0
## B-R-PC4        0        0        0        0         0       31        0
## B-A-MU4      266      163        0        0         0        0        0
## B-A-GU4        0        0        0        0         0        0        0
##         se-spilo st-noton sy-marmo te-chalc tr-signa
## S-R-CT1        0        1        0        0       18
## S-R-CP1        0        0        0        0        0
## S-A-TA1        0        0        0        0        0
## S-R-CT2        0       25        0        0       15
## S-R-CP2        0        0        1        0        0
## S-A-TA2        0        0        0        0        0
## S-R-CT3        1      115        0        0        7
## S-R-CP3        0        0        0        0        0
## S-A-TA3        0        0        0        0        0
## S-R-CT4        0       64        0        0      141
## S-R-CP4        0        0        0        0        0
## S-A-TA4        0        0        0        0        0
## B-A-MU1        0        0        0        0        0
## B-A-GU1        0        0        0        0        0
## B-R-PC2        0        0        0       76       23
## B-A-MU2        0        0        0        0        0
## B-A-GU2        0        0        0        0        0
## B-R-PC3        0        0        0       58        0
## B-A-MU3        0        0        0        0        0
## B-A-GU3        0        0        0        0        0
## B-R-PC4        0        0        0        0        4
## B-A-MU4        0        0        0        0        0
## B-A-GU4        0        0        0        0        0
\end{verbatim}

\hypertarget{alguns-ajustes-para-melhorar-a-visualizauxe7uxe3o}{%
\subsubsection{Alguns ajustes para melhorar a visualização}\label{alguns-ajustes-para-melhorar-a-visualizauxe7uxe3o}}

\begin{Shaded}
\begin{Highlighting}[]
\NormalTok{merge }\OtherTok{\textless{}{-}} \FunctionTok{as.data.frame}\NormalTok{(cluster1}\SpecialCharTok{$}\NormalTok{merge)}
\NormalTok{merge[}\FunctionTok{nrow}\NormalTok{(merge)}\SpecialCharTok{+}\DecValTok{1}\NormalTok{,] }\OtherTok{=} \FunctionTok{c}\NormalTok{(}\StringTok{"0"}\NormalTok{,}\StringTok{"0"}\NormalTok{)}
\NormalTok{merge}
\NormalTok{uas }\OtherTok{\textless{}{-}} \FunctionTok{as.data.frame}\NormalTok{(}\FunctionTok{rownames\_to\_column}\NormalTok{(m\_trab,}\AttributeTok{var =} \StringTok{"UAs"}\NormalTok{))}
\NormalTok{uas}
\NormalTok{merges }\OtherTok{\textless{}{-}} \FunctionTok{cbind}\NormalTok{(uas[}\FunctionTok{c}\NormalTok{(}\StringTok{"UAs"}\NormalTok{)], merge)}
\NormalTok{merges}
\end{Highlighting}
\end{Shaded}

\begin{verbatim}
##     V1  V2
## 1   -2  -3
## 2  -14   1
## 3   -8   2
## 4   -5 -11
## 5    3   4
## 6  -15 -18
## 7  -21   5
## 8  -17 -23
## 9  -16   7
## 10  -9 -12
## 11   6   9
## 12 -10  11
## 13  -1  -4
## 14 -20   8
## 15  -6 -13
## 16  12  15
## 17  13  16
## 18 -19  10
## 19  14  17
## 20  -7  19
## 21  18  20
## 22 -22  21
## 23   0   0
##        UAs ap-davis as-bimac as-fasci ch-bimac ci-ocela ci-orien co-macro
## 1  S-R-CT1        0      194       55        0        0        5        0
## 2  S-R-CP1        0       19        0        0        0        0        0
## 3  S-A-TA1        0       23        1       13        0        0        0
## 4  S-R-CT2        0      142        3        3        0       69        0
## 5  S-R-CP2        0        5        1        0       40        9        0
## 6  S-A-TA2        0       46        0      178        0        0        0
## 7  S-R-CT3        0      206       64        0        0       25        0
## 8  S-R-CP3        0       16        0        0       13       24        0
## 9  S-A-TA3        0      234        7      238        0        0        2
## 10 S-R-CT4        0        0        1        0        0        5        0
## 11 S-R-CP4        0        0        0        0       11        6        0
## 12 S-A-TA4        0      394        0      273        0        0        0
## 13 B-A-MU1        0       12        0        0        0        0        0
## 14 B-A-GU1        0        2        2        0        0        0        0
## 15 B-R-PC2        5       44        0        0        2        0        0
## 16 B-A-MU2        0       99        0        0        0        0        0
## 17 B-A-GU2        0        0        0        0        0        0        0
## 18 B-R-PC3       22       75        7        0        4        0        0
## 19 B-A-MU3        0      511        0        0        0        0        0
## 20 B-A-GU3        0        6        0        0        0        0        0
## 21 B-R-PC4        0        7       17        0        0        0        0
## 22 B-A-MU4        0      235        0        0        0        0        0
## 23 B-A-GU4        0       13        0        0        0        0        0
##    co-heter cr-menez cu-lepid cy-gilbe ge-brasi he-margi ho-malab hy-pusar
## 1         1       14        0        0        3        0        1        9
## 2         0        0        0        0        0        0        5        2
## 3         0        0        0        0        0        0        0        0
## 4         0        4        0        0        0        1       17       43
## 5         0        0        0        0        0        0       10        2
## 6         0        0        0        0        0        0        2        0
## 7         0        8        0        0        1        0       31       11
## 8         0        0        0        0        0        0        4        0
## 9         0        0        0        0        0        0       20        0
## 10        0        1        0       50        3        1        4        3
## 11        0        0        0        0        0        0        2        0
## 12        0        1        0        0        1        0        9        0
## 13        0        0        0        0      190        0        0        0
## 14        0        0        0        0        7        0        0        0
## 15        0        0        0        0        8        0        0        0
## 16        0        0        0        0       67        0        1        0
## 17        0        0        0        0       23        0        0        0
## 18        0        0       21        0       16        0        2        1
## 19        0        0        0        0      145        0        0        0
## 20        0        0        0        0       32        0        0        0
## 21        0        0        0       81        5        0        1        0
## 22        0        0        0        0      509        0        0        0
## 23        0        0        0        0       10        0        0        0
##    le-melan le-piau le-taeni mo-costa mo-lepid or-nilot pa-manag pimel-sp
## 1         0       3        0        0       39       36        0        6
## 2         0       0        0        0        0        0        0        0
## 3         0       0        0        0        0        0        0        0
## 4         0       1        0        0        1       77        0        0
## 5         0       3        0        0        0        0        0        0
## 6         0       0        0        0        0        0        0        0
## 7         0       2        0        0        0      138        0        0
## 8         0       1        0        0        0        0        0        0
## 9         0       0        0        0        0        0        0        0
## 10        0       0        0        0        0       73        0        0
## 11        0       2        0        0        0        0        0        0
## 12        0       2        0        0        0        1        0        0
## 13        0       0        0        0        0        6        0        0
## 14        0       0        0        0        0        3       11        0
## 15        2       0        1        0        0        5        0        0
## 16        0       0        0        0        0        1        0        0
## 17        0       0        0        0        0       36      102        0
## 18        0       0        0        1        0       65        0        0
## 19        0       0        0        0        0       11        0        0
## 20        0       0        0        0        0      247      250        0
## 21        0       1        0        0        0        9        0        0
## 22        0       0        0        0        0        1        0        0
## 23        0       0        0        0        0      129      190        0
##    po-retic po-vivip pr-brevi ps-rhomb ps-genise se-heter se-piaba se-spilo
## 1         0       47        5        0         0       40       68        0
## 2         0       15        0        0         0       14        0        0
## 3         0        0        1        0         0        4        0        0
## 4        20      221       15        0         0       60        0        0
## 5         0       32        5        0         0        0        0        0
## 6         0        0        2        0         0        0        0        0
## 7         5      326      164        1         1       38        0        1
## 8         0       10        0        0         0        0        0        0
## 9         0        0        0        0         0        0        0        0
## 10        0       28       59        0         0        3        0        0
## 11        0       80        0        0         0        3        0        0
## 12        0        0        3        0         0        0        0        0
## 13        0        0        0        0         0        0        0        0
## 14        0        0        0        0         0        0        0        0
## 15        0        0        9        0         0       10        0        0
## 16       10        8        0        0         0        0        0        0
## 17        0        0        0        0         0        0        0        0
## 18        0        0        6        0         0       93        0        0
## 19       46       48        1        0         0        0        0        0
## 20        0        0        0        0         0        0        0        0
## 21        0        0        0        0         0       31        0        0
## 22      266      163        0        0         0        0        0        0
## 23        0        0        0        0         0        0        0        0
##    st-noton sy-marmo te-chalc tr-signa
## 1         1        0        0       18
## 2         0        0        0        0
## 3         0        0        0        0
## 4        25        0        0       15
## 5         0        1        0        0
## 6         0        0        0        0
## 7       115        0        0        7
## 8         0        0        0        0
## 9         0        0        0        0
## 10       64        0        0      141
## 11        0        0        0        0
## 12        0        0        0        0
## 13        0        0        0        0
## 14        0        0        0        0
## 15        0        0       76       23
## 16        0        0        0        0
## 17        0        0        0        0
## 18        0        0       58        0
## 19        0        0        0        0
## 20        0        0        0        0
## 21        0        0        0        4
## 22        0        0        0        0
## 23        0        0        0        0
##        UAs  V1  V2
## 1  S-R-CT1  -2  -3
## 2  S-R-CP1 -14   1
## 3  S-A-TA1  -8   2
## 4  S-R-CT2  -5 -11
## 5  S-R-CP2   3   4
## 6  S-A-TA2 -15 -18
## 7  S-R-CT3 -21   5
## 8  S-R-CP3 -17 -23
## 9  S-A-TA3 -16   7
## 10 S-R-CT4  -9 -12
## 11 S-R-CP4   6   9
## 12 S-A-TA4 -10  11
## 13 B-A-MU1  -1  -4
## 14 B-A-GU1 -20   8
## 15 B-R-PC2  -6 -13
## 16 B-A-MU2  12  15
## 17 B-A-GU2  13  16
## 18 B-R-PC3 -19  10
## 19 B-A-MU3  14  17
## 20 B-A-GU3  -7  19
## 21 B-R-PC4  18  20
## 22 B-A-MU4 -22  21
## 23 B-A-GU4   0   0
\end{verbatim}

\hypertarget{cluster-2-criando-a-matriz-de-distuxe2ncias-euclidiana}{%
\section{Cluster 2: Criando a matriz de distâncias Euclidiana}\label{cluster-2-criando-a-matriz-de-distuxe2ncias-euclidiana}}

\begin{Shaded}
\begin{Highlighting}[]
\NormalTok{m\_dists }\OtherTok{\textless{}{-}} \FunctionTok{dist}\NormalTok{(m\_trab, }\AttributeTok{method =} \StringTok{"euclidean"}\NormalTok{,}
                  \AttributeTok{diag =} \ConstantTok{TRUE}\NormalTok{,}
                  \AttributeTok{upper =} \ConstantTok{FALSE}\NormalTok{)}
\CommentTok{\#m\_dists}
\NormalTok{cluster2 }\OtherTok{\textless{}{-}} \FunctionTok{hclust}\NormalTok{(m\_dists, }\AttributeTok{method =} \StringTok{"complete"}\NormalTok{) }\CommentTok{\#método de fusão}
\FunctionTok{plot}\NormalTok{ (cluster2, }\AttributeTok{main =} \StringTok{"Cluster Dendrogram {-} Dist.Euclidiana"}\NormalTok{)}
\FunctionTok{rect.hclust}\NormalTok{(cluster2, }\AttributeTok{k =} \DecValTok{3}\NormalTok{)}
\CommentTok{\#?dist}
\CommentTok{\#?hclust}
\end{Highlighting}
\end{Shaded}

\includegraphics{fundamentos_de_ecologia_numerica_files/figure-latex/unnamed-chunk-229-1.pdf}

O significado do argumento \texttt{method\ =} da função \texttt{dist()} usada no códico acima, depende do contexto em que ela é usada.

\begin{itemize}
\item
  A função \texttt{dist()} no R é usada para calcular as distâncias entre as observações em um conjunto de dados. Essa função tem vários métodos de cálculo de distância, que podem ser especificados pelo argumento \texttt{method\ =}, a exemplo de ``euclidean'', ``manhattan'', ``canberra'', entre outras.
\item
  Quando o argumento \texttt{method\ =} é usado em conjunto com a função \texttt{hclust()}, como no código acima, para realizar uma análise de classificação (ou cluster) hierárquica, então ela passa a representar um método de fusão, a exemplo de UPGMA, ``nearest neighbour'', ``furtherst neighbour'', entre outros.
\end{itemize}

A tabela \ref{tab:27fus} resume os métodos de fusão disponiveis na função \texttt{hclust()}.

\begin{verbatim}
## Warning in table_info$align_vector[column] <-
## unlist(lapply(table_info$align_vector_origin[column], : número de itens para
## para substituir não é um múltiplo do comprimento do substituto
\end{verbatim}

\begin{table}

\caption{\label{tab:27fus}Metodos de fusão disponiveis com o argumento `method=` da função `hclust`. Lembrando que UPGMA significa "Unweighted Pairwise Group Method with Arithmetic Mean"}
\centering
\begin{tabular}[t]{>{\raggedright\arraybackslash}p{2cm}>{\raggedright\arraybackslash}p{5cm}}
\toprule
Method= & Método de fusão\\
\midrule
ward.D & Método de Ward com variância mínima\\
ward.D2 & Método de Ward com variância mínima ajustada\\
single & Método de ligação simples (vizinho mais próximo)\\
complete & Método de ligação completa (vizinho mais distante)\\
average & Método de ligação baseado na média, equivale ao UPGMA\\
\addlinespace
mcquitty & Método de ligação de McQuitty, equivale ao WPGMA\\
median & Método de ligação baseado na mediana, ou WPGMC\\
centroid & Método de ligação baseado no centróide, ou UPGMC\\
\bottomrule
\end{tabular}
\end{table}

\hypertarget{comparauxe7uxe3o-dos-resultados-obtidos-atuxe9-aqui}{%
\subsection{Comparação dos resultados obtidos até aqui}\label{comparauxe7uxe3o-dos-resultados-obtidos-atuxe9-aqui}}

\begin{Shaded}
\begin{Highlighting}[]
\CommentTok{\#dev.off() \#limpa os gráficos}
\FunctionTok{par}\NormalTok{(}\AttributeTok{mfrow =} \FunctionTok{c}\NormalTok{(}\DecValTok{2}\NormalTok{,}\DecValTok{1}\NormalTok{))}
\FunctionTok{plot}\NormalTok{(cluster1, }\AttributeTok{main =} \StringTok{"Cluster Dendrogram {-} Piloto automático"}\NormalTok{)}
\FunctionTok{plot}\NormalTok{(cluster2, }\AttributeTok{main =} \StringTok{"Cluster Dendrogram {-} Dist. Euclidiana"}\NormalTok{)}
\end{Highlighting}
\end{Shaded}

\includegraphics{fundamentos_de_ecologia_numerica_files/figure-latex/unnamed-chunk-230-1.pdf}

\begin{Shaded}
\begin{Highlighting}[]
\FunctionTok{par}\NormalTok{(}\AttributeTok{mfrow =} \FunctionTok{c}\NormalTok{(}\DecValTok{1}\NormalTok{,}\DecValTok{1}\NormalTok{))}
\CommentTok{\#dev.off()}
\end{Highlighting}
\end{Shaded}

\hypertarget{cluster-3-relativizauxe7uxe3o-pelo-total-da-coluna}{%
\section{Cluster 3: Relativização pelo total da coluna}\label{cluster-3-relativizauxe7uxe3o-pelo-total-da-coluna}}

\begin{Shaded}
\begin{Highlighting}[]
\FunctionTok{library}\NormalTok{(vegan)}
\NormalTok{m\_relcol }\OtherTok{\textless{}{-}} \FunctionTok{decostand}\NormalTok{(m\_trab,}
                         \AttributeTok{method=}\StringTok{"total"}\NormalTok{, }\DocumentationTok{\#\#\# OUTROS METODOS: total, max, normalize, range, rankm}
                         \AttributeTok{MARGIN =} \DecValTok{2}\NormalTok{) }\CommentTok{\#1{-}linha, 2{-}coluna}
\CommentTok{\#View(m\_relcol)}
\CommentTok{\#m\_relcol}
\end{Highlighting}
\end{Shaded}

\hypertarget{gruxe1fico-de-clustering-relativizado}{%
\subsection{Gráfico de clustering relativizado}\label{gruxe1fico-de-clustering-relativizado}}

Agora fazemos a classificação da matriz bruta após ter sido relativizada pelo total das colunas.

\begin{Shaded}
\begin{Highlighting}[]
\NormalTok{cluster3 }\OtherTok{\textless{}{-}} \FunctionTok{hclust}\NormalTok{(}\FunctionTok{dist}\NormalTok{(m\_relcol))}
\FunctionTok{plot}\NormalTok{(cluster3, }\AttributeTok{main =} \StringTok{"Cluster Dendrogram {-} Rel.Coluna"}\NormalTok{)}
\FunctionTok{rect.hclust}\NormalTok{(cluster3, }\AttributeTok{k =} \DecValTok{3}\NormalTok{)}
\end{Highlighting}
\end{Shaded}

\includegraphics{fundamentos_de_ecologia_numerica_files/figure-latex/unnamed-chunk-232-1.pdf}

\hypertarget{cluster-4-transformauxe7uxe3o-pelo-arcoseno-da-raiz-quadrada}{%
\section{Cluster 4: Transformação pelo arcoseno da raiz quadrada}\label{cluster-4-transformauxe7uxe3o-pelo-arcoseno-da-raiz-quadrada}}

Agora aplicaremos a transformação do arcoseno da raiz quadrada na matriz relativizada por colunas. Lembre que em ambas as matrizes entramos dados que variam entre 0 e 1, e os resutados retornados também variam entre 0 e 1, mas em um nível de compressão de valores extremos mais desejável. Usaremos a matriz relativizada/transformada para fazermos a proxima classificação.

\begin{Shaded}
\begin{Highlighting}[]
\NormalTok{m\_trns }\OtherTok{\textless{}{-}} \FunctionTok{asin}\NormalTok{(}\FunctionTok{sqrt}\NormalTok{(m\_relcol)) }\CommentTok{\#valores de entrada tem que ser entre 0{-}1}
\CommentTok{\#View(m\_trns)}
\CommentTok{\#m\_trns}
\end{Highlighting}
\end{Shaded}

\hypertarget{gruxe1fico-de-clustering-relativizadotransformado}{%
\subsection{Gráfico de clustering relativizado/transformado}\label{gruxe1fico-de-clustering-relativizadotransformado}}

Aqui fazemos uma Análise Cluster no piloto automático, para comparar com os resultados da primeira classificação, que não havia sido relativizada/transformada.

\begin{Shaded}
\begin{Highlighting}[]
\NormalTok{cluster4 }\OtherTok{\textless{}{-}} \FunctionTok{hclust}\NormalTok{(}\FunctionTok{dist}\NormalTok{(m\_trns))}
\FunctionTok{plot}\NormalTok{(cluster4,  }\AttributeTok{main =} \StringTok{"Cluster Dendrogram {-} ArcSin.SQRT"}\NormalTok{)}
\FunctionTok{rect.hclust}\NormalTok{(cluster4, }\AttributeTok{k =} \DecValTok{3}\NormalTok{)}
\end{Highlighting}
\end{Shaded}

\includegraphics{fundamentos_de_ecologia_numerica_files/figure-latex/unnamed-chunk-234-1.pdf}

\hypertarget{cluster-5-controle-total-dos-paruxe2metros}{%
\section{Cluster 5: Controle total dos parâmetros}\label{cluster-5-controle-total-dos-paruxe2metros}}

\begin{Shaded}
\begin{Highlighting}[]
\NormalTok{vegdist }\OtherTok{\textless{}{-}} \FunctionTok{vegdist}\NormalTok{(m\_trns, }\AttributeTok{method =} \StringTok{"bray"}\NormalTok{,}
                   \AttributeTok{diag =} \ConstantTok{TRUE}\NormalTok{,}
                   \AttributeTok{upper =} \ConstantTok{FALSE}\NormalTok{)}
\FunctionTok{as.matrix}\NormalTok{(vegdist)[}\DecValTok{1}\SpecialCharTok{:}\DecValTok{7}\NormalTok{, }\DecValTok{1}\SpecialCharTok{:}\DecValTok{7}\NormalTok{]}
\NormalTok{cluster5 }\OtherTok{\textless{}{-}} \FunctionTok{hclust}\NormalTok{(vegdist, }\AttributeTok{method =} \StringTok{"average"}\NormalTok{)}
\FunctionTok{plot}\NormalTok{(cluster5, }\AttributeTok{main =} \StringTok{"Cluster Dendrogram {-} Bray{-}Curtis"}\NormalTok{)}
\FunctionTok{rect.hclust}\NormalTok{(cluster5, }\AttributeTok{k =} \DecValTok{3}\NormalTok{, }\AttributeTok{h =} \ConstantTok{NULL}\NormalTok{) }
\end{Highlighting}
\end{Shaded}

\includegraphics{fundamentos_de_ecologia_numerica_files/figure-latex/unnamed-chunk-235-1.pdf}

\begin{Shaded}
\begin{Highlighting}[]
\CommentTok{\#h = 0.8 fornece os grupos formados na altura h}
\CommentTok{\#vegdist}
\NormalTok{cluster5}\SpecialCharTok{$}\NormalTok{merge}
\NormalTok{cluster5}\SpecialCharTok{$}\NormalTok{height}
\NormalTok{idrow }\OtherTok{\textless{}{-}} \FunctionTok{mutate}\NormalTok{(m\_trns, }\AttributeTok{id =} \FunctionTok{row\_number}\NormalTok{()) }
\NormalTok{idrow }\SpecialCharTok{\%\textgreater{}\%} \FunctionTok{relocate}\NormalTok{(id)}
\end{Highlighting}
\end{Shaded}

\begin{verbatim}
##           S-R-CT1   S-R-CP1   S-A-TA1   S-R-CT2   S-R-CP2   S-A-TA2   S-R-CT3
## S-R-CT1 0.0000000 0.8743721 0.9338269 0.6274997 0.8106894 0.9420728 0.6582219
## S-R-CP1 0.8743721 0.0000000 0.6833816 0.7759468 0.7726098 0.7342613 0.8672111
## S-A-TA1 0.9338269 0.6833816 0.0000000 0.8789631 0.9178304 0.5700984 0.9404687
## S-R-CT2 0.6274997 0.7759468 0.8789631 0.0000000 0.7280378 0.8836068 0.5169933
## S-R-CP2 0.8106894 0.7726098 0.9178304 0.7280378 0.0000000 0.8915271 0.8012846
## S-A-TA2 0.9420728 0.7342613 0.5700984 0.8836068 0.8915271 0.0000000 0.9412636
## S-R-CT3 0.6582219 0.8672111 0.9404687 0.5169933 0.8012846 0.9412636 0.0000000
##       [,1] [,2]
##  [1,]  -20  -23
##  [2,]   -8  -11
##  [3,]  -17    1
##  [4,]  -19  -22
##  [5,]   -6  -12
##  [6,]   -4  -10
##  [7,]  -16    4
##  [8,]   -5    2
##  [9,]  -13    7
## [10,]   -9    5
## [11,]   -7    6
## [12,]  -14    3
## [13,]   -2   -3
## [14,]   -1   11
## [15,]  -15  -18
## [16,]  -21   14
## [17,]   10   13
## [18,]    9   12
## [19,]    8   16
## [20,]   17   19
## [21,]   15   20
## [22,]   18   21
##  [1] 0.1402400 0.2637008 0.2827470 0.3726150 0.4095558 0.4631145 0.4766705
##  [8] 0.5326053 0.5584043 0.5683644 0.5934519 0.6112108 0.6833816 0.6834576
## [15] 0.6881517 0.7475412 0.7570089 0.7899914 0.8037471 0.8465927 0.8855551
## [22] 0.9099458
##         id ap-davis   as-bimac   as-fasci   ch-bimac  ci-ocela  ci-orien
## S-R-CT1  1 0.000000 0.29580113 0.63106017 0.00000000 0.0000000 0.1880966
## S-R-CP1  2 0.000000 0.09135411 0.00000000 0.00000000 0.0000000 0.0000000
## S-A-TA1  3 0.000000 0.10054093 0.07963989 0.13621375 0.0000000 0.0000000
## S-R-CT2  4 0.000000 0.25205752 0.13823439 0.06527916 0.0000000 0.7679121
## S-R-CP2  5 0.000000 0.04681561 0.07963989 0.00000000 0.8570719 0.2535816
## S-A-TA2  6 0.000000 0.14242797 0.00000000 0.52646034 0.0000000 0.0000000
## S-R-CT3  7 0.000000 0.30509802 0.68988155 0.00000000 0.0000000 0.4313758
## S-R-CP3  8 0.000000 0.08381382 0.00000000 0.00000000 0.4455406 0.4220958
## S-A-TA3  9 0.000000 0.32588896 0.21207071 0.61998604 0.0000000 0.0000000
## S-R-CT4 10 0.000000 0.00000000 0.07963989 0.00000000 0.0000000 0.1880966
## S-R-CP4 11 0.000000 0.00000000 0.00000000 0.00000000 0.4076059 0.2062968
## S-A-TA4 12 0.000000 0.42841263 0.00000000 0.67165367 0.0000000 0.0000000
## B-A-MU1 13 0.000000 0.07256361 0.00000000 0.00000000 0.0000000 0.0000000
## B-A-GU1 14 0.000000 0.02960230 0.11274751 0.00000000 0.0000000 0.0000000
## B-R-PC2 15 0.444860 0.13927667 0.00000000 0.00000000 0.1698463 0.0000000
## B-A-MU2 16 0.000000 0.20977540 0.00000000 0.00000000 0.0000000 0.0000000
## B-A-GU2 17 0.000000 0.00000000 0.00000000 0.00000000 0.0000000 0.0000000
## B-R-PC3 18 1.125936 0.18225721 0.21207071 0.00000000 0.2413830 0.0000000
## B-A-MU3 19 0.000000 0.49281178 0.00000000 0.00000000 0.0000000 0.0000000
## B-A-GU3 20 0.000000 0.05128768 0.00000000 0.00000000 0.0000000 0.0000000
## B-R-PC4 21 0.000000 0.05540108 0.33420332 0.00000000 0.0000000 0.0000000
## B-A-MU4 22 0.000000 0.32661037 0.00000000 0.00000000 0.0000000 0.0000000
## B-A-GU4 23 0.000000 0.07553213 0.00000000 0.00000000 0.0000000 0.0000000
##         co-macro co-heter  cr-menez cu-lepid  cy-gilbe   ge-brasi  he-margi
## S-R-CT1 0.000000 1.570796 0.7853982 0.000000 0.0000000 0.05425923 0.0000000
## S-R-CP1 0.000000 0.000000 0.0000000 0.000000 0.0000000 0.00000000 0.0000000
## S-A-TA1 0.000000 0.000000 0.0000000 0.000000 0.0000000 0.00000000 0.0000000
## S-R-CT2 0.000000 0.000000 0.3875967 0.000000 0.0000000 0.00000000 0.7853982
## S-R-CP2 0.000000 0.000000 0.0000000 0.000000 0.0000000 0.00000000 0.0000000
## S-A-TA2 0.000000 0.000000 0.0000000 0.000000 0.0000000 0.00000000 0.0000000
## S-R-CT3 0.000000 0.000000 0.5639426 0.000000 0.0000000 0.03131633 0.0000000
## S-R-CP3 0.000000 0.000000 0.0000000 0.000000 0.0000000 0.00000000 0.0000000
## S-A-TA3 1.570796 0.000000 0.0000000 0.000000 0.0000000 0.00000000 0.0000000
## S-R-CT4 0.000000 0.000000 0.1901256 0.000000 0.6659445 0.05425923 0.7853982
## S-R-CP4 0.000000 0.000000 0.0000000 0.000000 0.0000000 0.00000000 0.0000000
## S-A-TA4 0.000000 0.000000 0.1901256 0.000000 0.0000000 0.03131633 0.0000000
## B-A-MU1 0.000000 0.000000 0.0000000 0.000000 0.0000000 0.44626053 0.0000000
## B-A-GU1 0.000000 0.000000 0.0000000 0.000000 0.0000000 0.08293673 0.0000000
## B-R-PC2 0.000000 0.000000 0.0000000 0.000000 0.0000000 0.08867767 0.0000000
## B-A-MU2 0.000000 0.000000 0.0000000 0.000000 0.0000000 0.25918549 0.0000000
## B-A-GU2 0.000000 0.000000 0.0000000 0.000000 0.0000000 0.15073345 0.0000000
## B-R-PC3 0.000000 0.000000 0.0000000 1.570796 0.0000000 0.12557463 0.0000000
## B-A-MU3 0.000000 0.000000 0.0000000 0.000000 0.0000000 0.38659505 0.0000000
## B-A-GU3 0.000000 0.000000 0.0000000 0.000000 0.0000000 0.17806243 0.0000000
## B-R-PC4 0.000000 0.000000 0.0000000 0.000000 0.9048519 0.07007133 0.0000000
## B-A-MU4 0.000000 0.000000 0.0000000 0.000000 0.0000000 0.78441777 0.0000000
## B-A-GU4 0.000000 0.000000 0.0000000 0.000000 0.0000000 0.09917726 0.0000000
##           ho-malab  hy-pusar le-melan   le-piau le-taeni mo-costa  mo-lepid
## S-R-CT1 0.09592969 0.3640209 0.000000 0.4636476 0.000000 0.000000 1.4120161
## S-R-CP1 0.21584866 0.1686344 0.000000 0.0000000 0.000000 0.000000 0.0000000
## S-A-TA1 0.00000000 0.0000000 0.000000 0.0000000 0.000000 0.000000 0.0000000
## S-R-CT2 0.40598283 0.8918340 0.000000 0.2611574 0.000000 0.000000 0.1587802
## S-R-CP2 0.30772498 0.1686344 0.000000 0.4636476 0.000000 0.000000 0.0000000
## S-A-TA2 0.13587479 0.0000000 0.000000 0.0000000 0.000000 0.000000 0.0000000
## S-R-CT3 0.56249105 0.4045563 0.000000 0.3737922 0.000000 0.000000 0.0000000
## S-R-CP3 0.19275669 0.0000000 0.000000 0.2611574 0.000000 0.000000 0.0000000
## S-A-TA3 0.44266923 0.0000000 0.000000 0.0000000 0.000000 0.000000 0.0000000
## S-R-CT4 0.19275669 0.2070324 0.000000 0.0000000 0.000000 0.000000 0.0000000
## S-R-CP4 0.13587479 0.0000000 0.000000 0.3737922 0.000000 0.000000 0.0000000
## S-A-TA4 0.29145679 0.0000000 0.000000 0.3737922 0.000000 0.000000 0.0000000
## B-A-MU1 0.00000000 0.0000000 0.000000 0.0000000 0.000000 0.000000 0.0000000
## B-A-GU1 0.00000000 0.0000000 0.000000 0.0000000 0.000000 0.000000 0.0000000
## B-R-PC2 0.00000000 0.0000000 1.570796 0.0000000 1.570796 0.000000 0.0000000
## B-A-MU2 0.09592969 0.0000000 0.000000 0.0000000 0.000000 0.000000 0.0000000
## B-A-GU2 0.00000000 0.0000000 0.000000 0.0000000 0.000000 0.000000 0.0000000
## B-R-PC3 0.13587479 0.1189585 0.000000 0.0000000 0.000000 1.570796 0.0000000
## B-A-MU3 0.00000000 0.0000000 0.000000 0.0000000 0.000000 0.000000 0.0000000
## B-A-GU3 0.00000000 0.0000000 0.000000 0.0000000 0.000000 0.000000 0.0000000
## B-R-PC4 0.09592969 0.0000000 0.000000 0.2611574 0.000000 0.000000 0.0000000
## B-A-MU4 0.00000000 0.0000000 0.000000 0.0000000 0.000000 0.000000 0.0000000
## B-A-GU4 0.00000000 0.0000000 0.000000 0.0000000 0.000000 0.000000 0.0000000
##           or-nilot  pa-manag pimel-sp  po-retic   po-vivip   pr-brevi ps-rhomb
## S-R-CT1 0.20878001 0.0000000 1.570796 0.0000000 0.22101460 0.13650631 0.000000
## S-R-CP1 0.00000000 0.0000000 0.000000 0.0000000 0.12416313 0.00000000 0.000000
## S-A-TA1 0.00000000 0.0000000 0.000000 0.0000000 0.00000000 0.06089569 0.000000
## S-R-CT2 0.30797141 0.0000000 0.000000 0.2424450 0.49537824 0.23794112 0.000000
## S-R-CP2 0.00000000 0.0000000 0.000000 0.0000000 0.18188749 0.13650631 0.000000
## S-A-TA2 0.00000000 0.0000000 0.000000 0.0000000 0.00000000 0.08617291 0.000000
## S-R-CT3 0.41785958 0.0000000 0.000000 0.1203286 0.61547971 0.89364927 1.570796
## S-R-CP3 0.00000000 0.0000000 0.000000 0.0000000 0.10129161 0.00000000 0.000000
## S-A-TA3 0.00000000 0.0000000 0.000000 0.0000000 0.00000000 0.00000000 0.000000
## S-R-CT4 0.29961011 0.0000000 0.000000 0.0000000 0.17002156 0.48641493 0.000000
## S-R-CP4 0.00000000 0.0000000 0.000000 0.0000000 0.29005641 0.00000000 0.000000
## S-A-TA4 0.03455130 0.0000000 0.000000 0.0000000 0.00000000 0.10560544 0.000000
## B-A-MU1 0.08471752 0.0000000 0.000000 0.0000000 0.00000000 0.00000000 0.000000
## B-A-GU1 0.05986846 0.1415090 0.000000 0.0000000 0.00000000 0.00000000 0.000000
## B-R-PC2 0.07732071 0.0000000 0.000000 0.0000000 0.00000000 0.18360401 0.000000
## B-A-MU2 0.03455130 0.0000000 0.000000 0.1705861 0.09056689 0.00000000 0.000000
## B-A-GU2 0.20878001 0.4439109 0.000000 0.0000000 0.00000000 0.00000000 0.000000
## B-R-PC3 0.28223829 0.0000000 0.000000 0.0000000 0.00000000 0.14962891 0.000000
## B-A-MU3 0.11482305 0.0000000 0.000000 0.3726604 0.22339293 0.06089569 0.000000
## B-A-GU3 0.57389635 0.7374041 0.000000 0.0000000 0.00000000 0.00000000 0.000000
## B-R-PC4 0.10381968 0.0000000 0.000000 0.0000000 0.00000000 0.00000000 0.000000
## B-A-MU4 0.03455130 0.0000000 0.000000 1.0665527 0.42053434 0.00000000 0.000000
## B-A-GU4 0.40318427 0.6263078 0.000000 0.0000000 0.00000000 0.00000000 0.000000
##         ps-genise  se-heter se-piaba se-spilo   st-noton sy-marmo  te-chalc
## S-R-CT1  0.000000 0.3764349 1.570796 0.000000 0.06989994 0.000000 0.0000000
## S-R-CP1  0.000000 0.2192313 0.000000 0.000000 0.00000000 0.000000 0.0000000
## S-A-TA1  0.000000 0.1165111 0.000000 0.000000 0.00000000 0.000000 0.0000000
## S-R-CT2  0.000000 0.4670175 0.000000 0.000000 0.35673339 0.000000 0.0000000
## S-R-CP2  0.000000 0.0000000 0.000000 0.000000 0.00000000 1.570796 0.0000000
## S-A-TA2  0.000000 0.0000000 0.000000 0.000000 0.00000000 0.000000 0.0000000
## S-R-CT3  1.570796 0.3664456 0.000000 1.570796 0.84652593 0.000000 0.0000000
## S-R-CP3  0.000000 0.0000000 0.000000 0.000000 0.00000000 0.000000 0.0000000
## S-A-TA3  0.000000 0.0000000 0.000000 0.000000 0.00000000 0.000000 0.0000000
## S-R-CT4  0.000000 0.1008442 0.000000 0.000000 0.59287085 0.000000 0.0000000
## S-R-CP4  0.000000 0.1008442 0.000000 0.000000 0.00000000 0.000000 0.0000000
## S-A-TA4  0.000000 0.0000000 0.000000 0.000000 0.00000000 0.000000 0.0000000
## B-A-MU1  0.000000 0.0000000 0.000000 0.000000 0.00000000 0.000000 0.0000000
## B-A-GU1  0.000000 0.0000000 0.000000 0.000000 0.00000000 0.000000 0.0000000
## B-R-PC2  0.000000 0.1848546 0.000000 0.000000 0.00000000 0.000000 0.8527660
## B-A-MU2  0.000000 0.0000000 0.000000 0.000000 0.00000000 0.000000 0.0000000
## B-A-GU2  0.000000 0.0000000 0.000000 0.000000 0.00000000 0.000000 0.0000000
## B-R-PC3  0.000000 0.5950206 0.000000 0.000000 0.00000000 0.000000 0.7180303
## B-A-MU3  0.000000 0.0000000 0.000000 0.000000 0.00000000 0.000000 0.0000000
## B-A-GU3  0.000000 0.0000000 0.000000 0.000000 0.00000000 0.000000 0.0000000
## B-R-PC4  0.000000 0.3295526 0.000000 0.000000 0.00000000 0.000000 0.0000000
## B-A-MU4  0.000000 0.0000000 0.000000 0.000000 0.00000000 0.000000 0.0000000
## B-A-GU4  0.000000 0.0000000 0.000000 0.000000 0.00000000 0.000000 0.0000000
##          tr-signa
## S-R-CT1 0.2985914
## S-R-CP1 0.0000000
## S-A-TA1 0.0000000
## S-R-CT2 0.2718802
## S-R-CP2 0.0000000
## S-A-TA2 0.0000000
## S-R-CT3 0.1844947
## S-R-CP3 0.0000000
## S-A-TA3 0.0000000
## S-R-CT4 0.9672667
## S-R-CP4 0.0000000
## S-A-TA4 0.0000000
## B-A-MU1 0.0000000
## B-A-GU1 0.0000000
## B-R-PC2 0.3389861
## B-A-MU2 0.0000000
## B-A-GU2 0.0000000
## B-R-PC3 0.0000000
## B-A-MU3 0.0000000
## B-A-GU3 0.0000000
## B-R-PC4 0.1391234
## B-A-MU4 0.0000000
## B-A-GU4 0.0000000
\end{verbatim}

No código acima, \texttt{h\ =\ 0.8} fornece os grupos formados na altura \texttt{h} do eixos das distâncias do dendrograma. Ou seja, no dendrograma, o eixo y (HEIGHT, ``h'') representa o valor da distancia escolhida entre os objetos ou grupos de objetos. Portanto, se dois objetos ou grupos de objetos foram agrupados num dado valor (0.8, por exemplo) no eixo \texttt{height}, isso significa que a distancia entre esses objetos é 0.8.

\hypertarget{construindo-uma-prancha-de-gruxe1ficos-comparativos}{%
\section{Construindo uma prancha de gráficos comparativos}\label{construindo-uma-prancha-de-gruxe1ficos-comparativos}}

\begin{Shaded}
\begin{Highlighting}[]
\CommentTok{\#dev.off()}
\FunctionTok{par}\NormalTok{(}\AttributeTok{mfrow =} \FunctionTok{c}\NormalTok{ (}\DecValTok{2}\NormalTok{,}\DecValTok{2}\NormalTok{))}
\FunctionTok{plot}\NormalTok{(cluster1, }\AttributeTok{main =} \StringTok{"Dist.Euclidiana"}\NormalTok{)}
\FunctionTok{plot}\NormalTok{(cluster3, }\AttributeTok{main =} \StringTok{"Rel.Coluna"}\NormalTok{)}
\FunctionTok{plot}\NormalTok{(cluster4, }\AttributeTok{main =} \StringTok{"ArcSin.SQRT"}\NormalTok{)}
\FunctionTok{plot}\NormalTok{(cluster5, }\AttributeTok{main =} \StringTok{"Bray{-}Curtis"}\NormalTok{)}
\end{Highlighting}
\end{Shaded}

\includegraphics{fundamentos_de_ecologia_numerica_files/figure-latex/unnamed-chunk-236-1.pdf}

\begin{Shaded}
\begin{Highlighting}[]
\FunctionTok{par}\NormalTok{(}\AttributeTok{mfrow=}\FunctionTok{c}\NormalTok{(}\DecValTok{1}\NormalTok{,}\DecValTok{1}\NormalTok{))}
\CommentTok{\#dev.off()}
\end{Highlighting}
\end{Shaded}

\hypertarget{construindo-heatmaps}{%
\section{\texorpdfstring{Construindo \texttt{heatmaps}}{Construindo heatmaps}}\label{construindo-heatmaps}}

Agora que vimos de forma comparativa a importância de se relativizar e transformar nossos dados, e também de escolher de forma consciente a medida de distância e o método de fusão, podemos avançar um pouco mais na interpretação dos nossos dados.

Um ``heatmap'', ou mapa de calor, é uma imagem em cores (basicamente \texttt{image(t(x))}) com um dendrograma adicionado ao lado esquerdo e/ou ao topo. Normalmente, é realizado o reordenamento das linhas e colunas de acordo com algum conjunto de valores (médias de linha ou coluna) dentro das restrições impostas pelo dendrograma.

Primeiro vamos rever o dendrograma final, usando a Distância de Bray-Curtis e o Método de Fusão UPGMA (ou ``average'').

\begin{Shaded}
\begin{Highlighting}[]
\FunctionTok{library}\NormalTok{(}\StringTok{"RColorBrewer"}\NormalTok{)}
\FunctionTok{library}\NormalTok{(}\StringTok{"gplots"}\NormalTok{)}
\NormalTok{vegdist }\OtherTok{\textless{}{-}} \FunctionTok{vegdist}\NormalTok{(m\_trns, }\AttributeTok{method =} \StringTok{"bray"}\NormalTok{,}
                   \AttributeTok{diag =} \ConstantTok{TRUE}\NormalTok{,}
                   \AttributeTok{upper =} \ConstantTok{FALSE}\NormalTok{)}
\NormalTok{cluster5 }\OtherTok{\textless{}{-}} \FunctionTok{hclust}\NormalTok{(vegdist, }\AttributeTok{method =} \StringTok{"average"}\NormalTok{)}
\FunctionTok{plot}\NormalTok{(cluster5, }\AttributeTok{main =} \StringTok{"Cluster Dendrogram {-} Bray{-}Curtis"}\NormalTok{)}

\NormalTok{heatdist }\OtherTok{\textless{}{-}} \FunctionTok{as.matrix}\NormalTok{(vegdist)}
\CommentTok{\#heatdist}
\end{Highlighting}
\end{Shaded}

\includegraphics{fundamentos_de_ecologia_numerica_files/figure-latex/unnamed-chunk-237-1.pdf}

Na sequência fazemos um \texttt{heatmap} \textbf{objetos x objetos}, ou seja a comparação bidimensional das clustereds entre unidades amostrais.

\begin{Shaded}
\begin{Highlighting}[]
\NormalTok{col }\OtherTok{\textless{}{-}} \FunctionTok{rev}\NormalTok{(}\FunctionTok{heat.colors}\NormalTok{(}\DecValTok{999}\NormalTok{)) }\CommentTok{\#rev() reverte as cores do heatmap}
\FunctionTok{heatmap.2}\NormalTok{(}\AttributeTok{x=}\NormalTok{(}\FunctionTok{as.matrix}\NormalTok{(vegdist)), }\CommentTok{\#objetos x objetos}
          \AttributeTok{Rowv =} \FunctionTok{as.dendrogram}\NormalTok{(cluster5),}
          \AttributeTok{Colv =} \FunctionTok{as.dendrogram}\NormalTok{(cluster5),}
          \AttributeTok{key =}\NormalTok{ T, }\AttributeTok{tracecol =} \ConstantTok{NA}\NormalTok{, }\AttributeTok{revC =}\NormalTok{ T,}
          \AttributeTok{col =}\NormalTok{ heat.colors,  }\CommentTok{\#dissimilaridade = 1 {-} similaridade}
          \AttributeTok{density.info =} \StringTok{"none"}\NormalTok{,}
          \AttributeTok{xlab =} \StringTok{"UA´s"}\NormalTok{, }\AttributeTok{ylab =} \StringTok{"UA´s"}\NormalTok{,}
          \AttributeTok{mar =} \FunctionTok{c}\NormalTok{(}\DecValTok{6}\NormalTok{, }\DecValTok{6}\NormalTok{) }\SpecialCharTok{+} \FloatTok{0.2}\NormalTok{)}
\end{Highlighting}
\end{Shaded}

\includegraphics{fundamentos_de_ecologia_numerica_files/figure-latex/unnamed-chunk-238-1.pdf}

Agora vamos criar um novo dendrograma, para as espécies, mostrando sua similaridade em termos de em quais UA´s elas são mais comuns.

\begin{Shaded}
\begin{Highlighting}[]
\NormalTok{cluster6 }\OtherTok{\textless{}{-}} \FunctionTok{hclust}\NormalTok{((}\FunctionTok{vegdist}\NormalTok{(}\FunctionTok{t}\NormalTok{(m\_trns), }\AttributeTok{method =} \StringTok{"bray"}\NormalTok{,}
                            \AttributeTok{diag =} \ConstantTok{TRUE}\NormalTok{,}
                            \AttributeTok{upper =} \ConstantTok{FALSE}\NormalTok{)), }\AttributeTok{method =} \StringTok{"average"}\NormalTok{)}
\FunctionTok{plot}\NormalTok{(cluster6, }\AttributeTok{main =} \StringTok{"Dendrograma dos atributos"}\NormalTok{)}
\end{Highlighting}
\end{Shaded}

\includegraphics{fundamentos_de_ecologia_numerica_files/figure-latex/unnamed-chunk-239-1.pdf}

Na sequencia, fazemos um novo \texttt{heatmap} \textbf{objetos x atributos}, com a comparação bidimensional das clustereds formadas entre unidades amostrais em relação as espécies mais comuns que ocorreram em cada unidade amostral.

\begin{Shaded}
\begin{Highlighting}[]
\FunctionTok{heatmap.2}\NormalTok{(}\FunctionTok{t}\NormalTok{(}\FunctionTok{as.matrix}\NormalTok{(m\_trns)), }\CommentTok{\#objetos x atributos}
          \AttributeTok{Colv =} \FunctionTok{as.dendrogram}\NormalTok{(cluster5),}
          \AttributeTok{Rowv =} \FunctionTok{as.dendrogram}\NormalTok{(cluster6),}
          \AttributeTok{key =}\NormalTok{ T, }\AttributeTok{tracecol =} \ConstantTok{NA}\NormalTok{, }\AttributeTok{revC =}\NormalTok{ T,}
          \AttributeTok{col =}\NormalTok{ col,}
          \AttributeTok{density.info =} \StringTok{"none"}\NormalTok{,}
          \AttributeTok{xlab =} \StringTok{"Unidades amostrais"}\NormalTok{, }\AttributeTok{ylab =} \StringTok{"Espécies"}\NormalTok{,}
          \AttributeTok{mar =} \FunctionTok{c}\NormalTok{(}\DecValTok{6}\NormalTok{, }\DecValTok{6}\NormalTok{) }\SpecialCharTok{+} \FloatTok{0.1}\NormalTok{)  }\CommentTok{\# adjust margin size}
\end{Highlighting}
\end{Shaded}

\includegraphics{fundamentos_de_ecologia_numerica_files/figure-latex/unnamed-chunk-240-1.pdf}

\hypertarget{percentual-de-encadeamento-chaining}{%
\subsection{Percentual de encadeamento (\% chaining)}\label{percentual-de-encadeamento-chaining}}

Dendrogram Chaining (ou encadeamento) é um termo usado em análises de cluster hierárquico para se referir ao processo de adição sequencial de pequenos grupos a um ou poucos grandes grupos. É possível usar a porcentagem de encadeamento para ajudar a avaliar a qualidade do dendrograma. Análises altamente encadeadas provavelmente não valem a pena serem exploradas em maiores detalhes, provavelmente há alguma falha nos passos anteriores da análise. Por outro lado, pequenas diferenças na porcentagem de encadeamento na extremidade inferior da escala são geralmente um critério pobre para selecionar um algoritmo de ligação em relação a outro (\protect\hyperlink{ref-RN1552}{MCCUNE; GRACE, 2002}).

O método para calcular o grau de encadeamento compara o comprimento médio do caminho do dendrograma com o comprimento mínimo possível (sem encadeamento) e o comprimento máximo possível (encadeamento completo). O comprimento do caminho para cada item no dendrograma é o número de nós que o conectam ao nível mais alto do dendrograma. O encadeamento completo ocorre quando cada fusão envolve a adição de um único item a um único grupo, cada vez mais grande (\protect\hyperlink{ref-RN1552}{MCCUNE; GRACE, 2002}).

Com esse código podemos calcular o percentual de encadeamento para cada fusão em \texttt{cluster1} e fazer um gráfico simples para observação.

\begin{Shaded}
\begin{Highlighting}[]
\CommentTok{\# calculate the percent chaining for each merge}
\NormalTok{h }\OtherTok{\textless{}{-}}\NormalTok{ cluster1}\SpecialCharTok{$}\NormalTok{height}
\NormalTok{n }\OtherTok{\textless{}{-}} \FunctionTok{length}\NormalTok{(h)}
\NormalTok{pc }\OtherTok{\textless{}{-}} \FunctionTok{rep}\NormalTok{(}\ConstantTok{NA}\NormalTok{, n}\DecValTok{{-}1}\NormalTok{)}
\ControlFlowTok{for}\NormalTok{ (i }\ControlFlowTok{in} \DecValTok{1}\SpecialCharTok{:}\NormalTok{(n}\DecValTok{{-}1}\NormalTok{)) \{}
\NormalTok{  pc[i] }\OtherTok{\textless{}{-}} \FunctionTok{abs}\NormalTok{(h[i] }\SpecialCharTok{{-}}\NormalTok{ h[i}\SpecialCharTok{+}\DecValTok{1}\NormalTok{]) }\SpecialCharTok{/}\NormalTok{ h[}\DecValTok{1}\NormalTok{] }\SpecialCharTok{*} \DecValTok{100}
\NormalTok{\}}
\CommentTok{\# plot the percent chaining}
\FunctionTok{plot}\NormalTok{(pc, }\AttributeTok{type =} \StringTok{"h"}\NormalTok{, }\AttributeTok{xlab =} \StringTok{"Merge"}\NormalTok{, }\AttributeTok{ylab =} \StringTok{"Percent chaining"}\NormalTok{)}
\end{Highlighting}
\end{Shaded}

\includegraphics{fundamentos_de_ecologia_numerica_files/figure-latex/unnamed-chunk-241-1.pdf}

Para calcular o percentual geral de encadeamento em um dendrograma, é necessário calcular a diferença na altura entre cada nó e seus nós filhos e, em seguida, somar essas diferenças. Essa soma representa o comprimento total do dendrograma. Em seguida, é necessário calcular a soma das diferenças na altura entre cada nó que representa uma fusão e a altura de seus dois nós filhos. Essa soma representa o encadeamento total no dendrograma.

Finalmente, você pode calcular o percentual geral de encadeamento como o encadeamento total dividido pelo comprimento total do dendrograma, multiplicado por 100.

O eixo Y do gráfico mostra a percentagem de encadeamento, que é a diferença na altura entre fusões sucessivas dividida pela altura total do dendrograma, multiplicada por 100. É possível que a percentagem de encadeamento seja superior a 100 se a altura das duas fusões comparadas for maior que a altura total do dendrograma. Nesse caso, a percentagem de encadeamento é truncada em 100.

Isso pode acontecer, por exemplo, se a altura de uma fusão for negativa, o que pode ocorrer se a distância entre dois grupos for menor que zero, ou se o dendrograma for escalado ou truncado de alguma forma.

\hypertarget{funuxe7uxe3o-loop-para-comparar-os-encadeamentos}{%
\subsection{\texorpdfstring{Função \texttt{loop} para comparar os encadeamentos}{Função loop para comparar os encadeamentos}}\label{funuxe7uxe3o-loop-para-comparar-os-encadeamentos}}

\begin{Shaded}
\begin{Highlighting}[]
\CommentTok{\# create a list of the four cluster analyses}
\NormalTok{cluster\_list }\OtherTok{\textless{}{-}} \FunctionTok{list}\NormalTok{(cluster1, cluster2, cluster3, cluster4, cluster5)}

\FunctionTok{par}\NormalTok{(}\AttributeTok{mfrow=}\FunctionTok{c}\NormalTok{(}\DecValTok{4}\NormalTok{,}\DecValTok{2}\NormalTok{)) }\CommentTok{\#posicionamento dos gráficos}

\CommentTok{\# loop through the list and calculate percent chaining for each}
\ControlFlowTok{for}\NormalTok{ (i }\ControlFlowTok{in} \DecValTok{1}\SpecialCharTok{:}\DecValTok{5}\NormalTok{) \{}
\NormalTok{  h }\OtherTok{\textless{}{-}}\NormalTok{ cluster\_list[[i]]}\SpecialCharTok{$}\NormalTok{height}
\NormalTok{  n }\OtherTok{\textless{}{-}} \FunctionTok{length}\NormalTok{(h)}
\NormalTok{  pc }\OtherTok{\textless{}{-}} \FunctionTok{rep}\NormalTok{(}\ConstantTok{NA}\NormalTok{, n}\DecValTok{{-}1}\NormalTok{)}
  \ControlFlowTok{for}\NormalTok{ (j }\ControlFlowTok{in} \DecValTok{1}\SpecialCharTok{:}\NormalTok{(n}\DecValTok{{-}1}\NormalTok{)) \{}
\NormalTok{    pc[j] }\OtherTok{\textless{}{-}} \FunctionTok{abs}\NormalTok{(h[j] }\SpecialCharTok{{-}}\NormalTok{ h[j}\SpecialCharTok{+}\DecValTok{1}\NormalTok{]) }\SpecialCharTok{/}\NormalTok{ h[}\DecValTok{1}\NormalTok{] }\SpecialCharTok{*} \DecValTok{100}
\NormalTok{  \}}
\CommentTok{\# plot the percent chaining for each cluster analysis}
  \FunctionTok{plot}\NormalTok{(pc, }\AttributeTok{type =} \StringTok{"h"}\NormalTok{, }\AttributeTok{xlab =} \StringTok{"Merge"}\NormalTok{, }\AttributeTok{ylab =} \StringTok{"Percent chaining"}\NormalTok{, }\AttributeTok{main =} \FunctionTok{paste0}\NormalTok{(}\StringTok{"Cluster "}\NormalTok{, i))}
\NormalTok{\}}
\end{Highlighting}
\end{Shaded}

\includegraphics{fundamentos_de_ecologia_numerica_files/figure-latex/unnamed-chunk-242-1.pdf}

\hypertarget{apuxeandices-13}{%
\section*{Apêndices}\label{apuxeandices-13}}
\addcontentsline{toc}{section}{Apêndices}

\hypertarget{sites-consultados-9}{%
\section*{Sites consultados}\label{sites-consultados-9}}
\addcontentsline{toc}{section}{Sites consultados}

\url{https://copyprogramming.com/howto/how-to-color-labels-of-dendogram-with-dendextend-and-heatmap-2-using-pre-d\#efined-sample-groups}
\url{https://www.datacamp.com/tutorial/hierarchical-clustering-R}

\hypertarget{outras-formas-de-fazer-a-partiuxe7uxe3o-1}{%
\subsection*{Outras formas de fazer a partição}\label{outras-formas-de-fazer-a-partiuxe7uxe3o-1}}
\addcontentsline{toc}{subsection}{Outras formas de fazer a partição}

m\_bruta\_part2 \textless- m\_bruta\_part\[colSums(abs(m_bruta_part), na.rm = F) \> 0\]
m\_bruta\_part2 \textless- subset(m\_bruta\_part, colSums != 0) m\_bruta\_part2 \textless-
m\_bruta\_part\[, colSums(m_bruta_part != 0) \> 0\]

\hypertarget{cuxf3digo-1}{%
\subsection*{Código}\label{cuxf3digo-1}}
\addcontentsline{toc}{subsection}{Código}

range(euclidma) par(mfrow=c(2,1)) hist(euclidma, breaks = 15, \#determina
o no. de colunas do histograma xlim = range(floor(min(euclidma)),
ceiling(max(euclidma))), xlab = ``Distr. de Frequências'', freq = FALSE)
curve(dnorm(x, mean=mean(euclidma), sd=sd(euclidma)), add=TRUE)
boxplot.default(euclidma, horizontal = TRUE, frame = FALSE, xlab=``Distr.
de Frequências'', ylim=c(floor(min(euclidma)), ceiling(max(euclidma))))

\hypertarget{script-limpo-10}{%
\section*{Script limpo}\label{script-limpo-10}}
\addcontentsline{toc}{section}{Script limpo}

Aqui apresento o scrip na íntegra sem os textos ou outros comentários. Você pode copiar e colar no R para executa-lo. Lembre de remover os {\texttt{\#}} ou {\texttt{\#\#}} caso necessite executar essas linhas.

\begin{Shaded}
\begin{Highlighting}[]
\DocumentationTok{\#\# dev.off() \#apaga os graficos, se houver algum}
\DocumentationTok{\#\# rm(list=ls(all=TRUE)) \#limpa a memória}
\DocumentationTok{\#\# cat("\textbackslash{}014") \#limpa o console}
\DocumentationTok{\#\# install.packages("openxlsx") \#importa arquivos do excel}
\DocumentationTok{\#\# install.packages("moments") \#calcula assimetria e curtose dos dados}
\DocumentationTok{\#\# install.packages("matrixStats") \#fornece funções rápidas para a estatística de matrizes}
\DocumentationTok{\#\# install.packages("gt") \#ferramenta para criação de tabelas bonitas e personalizáveis}
\DocumentationTok{\#\# getwd()}
\DocumentationTok{\#\# setwd("C:/Seu/Diretório/De/Trabalho")}
\FunctionTok{library}\NormalTok{(openxlsx)}
\NormalTok{m\_bruta }\OtherTok{\textless{}{-}} \FunctionTok{read.xlsx}\NormalTok{(}\StringTok{"D:/Elvio/OneDrive/Disciplinas/\_EcoNumerica/5.Matrizes/peixes06.xlsx"}\NormalTok{,}
                   \AttributeTok{rowNames =}\NormalTok{ T, }\AttributeTok{colNames =}\NormalTok{ T,}
                   \AttributeTok{sheet =} \StringTok{"Sheet1"}\NormalTok{)}
\FunctionTok{str}\NormalTok{(m\_bruta)}
\NormalTok{m\_bruta\_ma }\OtherTok{\textless{}{-}} \FunctionTok{as.matrix}\NormalTok{(m\_bruta) }\CommentTok{\#lê m\_bruta como uma matriz}
\FunctionTok{str}\NormalTok{(m\_bruta\_ma)}
\CommentTok{\#m\_bruta}
\NormalTok{m\_bruta\_ma[}\DecValTok{1}\SpecialCharTok{:}\DecValTok{5}\NormalTok{,}\DecValTok{1}\SpecialCharTok{:}\DecValTok{5}\NormalTok{] }\CommentTok{\#[1:5,1:5] mostra apenas as linhas e colunas de 1 a 5.}
\CommentTok{\#m\_bruta \textless{}{-} (m\_bruta)   \# \textless{}1\textgreater{}}
\CommentTok{\#View(m\_bruta)}
\FunctionTok{print}\NormalTok{(m\_bruta)}
\FunctionTok{head}\NormalTok{(m\_bruta)}
\FunctionTok{str}\NormalTok{(m\_bruta)}
\FunctionTok{mode}\NormalTok{(m\_bruta)}
\FunctionTok{class}\NormalTok{(m\_bruta)}
\CommentTok{\#?str}
\FunctionTok{range}\NormalTok{(m\_bruta) }\CommentTok{\#menor e maior valores}
\FunctionTok{length}\NormalTok{(m\_bruta) }\CommentTok{\#no. de colunas}
\FunctionTok{ncol}\NormalTok{(m\_bruta) }\CommentTok{\#no. de N colunas}
\FunctionTok{nrow}\NormalTok{(m\_bruta) }\CommentTok{\#no. de M linhas}
\FunctionTok{sum}\NormalTok{(}\FunctionTok{lengths}\NormalTok{(m\_bruta)) }\CommentTok{\#soma os nos. de colunas}
\FunctionTok{length}\NormalTok{(}\FunctionTok{as.matrix}\NormalTok{(m\_bruta)) }\CommentTok{\#tamanho da matriz m x n}
\FunctionTok{sum}\NormalTok{(m\_bruta }\SpecialCharTok{==} \DecValTok{0}\NormalTok{) }\CommentTok{\# número de observações igual a zero}
\FunctionTok{sum}\NormalTok{(m\_bruta }\SpecialCharTok{\textgreater{}} \DecValTok{0}\NormalTok{) }\CommentTok{\# número de observações maiores que zero}
\NormalTok{zeros }\OtherTok{\textless{}{-}}\NormalTok{ (}\FunctionTok{sum}\NormalTok{(m\_bruta }\SpecialCharTok{==} \DecValTok{0}\NormalTok{)}\SpecialCharTok{/}\FunctionTok{length}\NormalTok{(}\FunctionTok{as.matrix}\NormalTok{(m\_bruta)))}\SpecialCharTok{*}\DecValTok{100} \CommentTok{\# proporção de zeros na matriz}
\NormalTok{zeros}
\NormalTok{tamanho }\OtherTok{\textless{}{-}} \FunctionTok{data.frame}\NormalTok{(}
\NormalTok{  Função }\OtherTok{=} \FunctionTok{c}\NormalTok{(}\StringTok{"range"}\NormalTok{, }\StringTok{"lenght"}\NormalTok{, }\StringTok{"m cols"}\NormalTok{, }\StringTok{"n linhas"}\NormalTok{, }\StringTok{"Tamanho"}\NormalTok{, }\StringTok{"Tamanho"}\NormalTok{,}
             \StringTok{"Zeros"}\NormalTok{, }\StringTok{"Nao zeros"}\NormalTok{, }\StringTok{"\% Zeros"}\NormalTok{),}
  \AttributeTok{Resultado =} \FunctionTok{c}\NormalTok{(}\FunctionTok{paste}\NormalTok{(}\FunctionTok{range}\NormalTok{(m\_bruta), }\AttributeTok{collapse =} \StringTok{" {-} "}\NormalTok{), }\FunctionTok{length}\NormalTok{(m\_bruta), }\FunctionTok{ncol}\NormalTok{(m\_bruta),}
                \FunctionTok{nrow}\NormalTok{(m\_bruta), }\FunctionTok{sum}\NormalTok{(}\FunctionTok{lengths}\NormalTok{(m\_bruta)), }\FunctionTok{length}\NormalTok{(}\FunctionTok{as.matrix}\NormalTok{(m\_bruta)), }\FunctionTok{sum}\NormalTok{(m\_bruta }\SpecialCharTok{==} \DecValTok{0}\NormalTok{),}
                \FunctionTok{sum}\NormalTok{(m\_bruta }\SpecialCharTok{\textgreater{}} \DecValTok{0}\NormalTok{), }\FunctionTok{round}\NormalTok{(zeros, }\DecValTok{1}\NormalTok{))}
\NormalTok{)}
\NormalTok{tamanho}
\NormalTok{knitr}\SpecialCharTok{::}\FunctionTok{kable}\NormalTok{(tamanho, }\AttributeTok{format =} \StringTok{"markdown"}\NormalTok{, }\AttributeTok{caption =} \StringTok{"Resumo das informações sobre o tamanho da matriz"}\NormalTok{)}
\NormalTok{euclid }\OtherTok{\textless{}{-}} \FunctionTok{dist}\NormalTok{(m\_bruta, }\AttributeTok{method =} \StringTok{"euclidian"}\NormalTok{, }\AttributeTok{diag =} \ConstantTok{TRUE}\NormalTok{, }\AttributeTok{upper =} \ConstantTok{FALSE}\NormalTok{)}
\CommentTok{\#?dist}
\CommentTok{\#euclid}
\FunctionTok{str}\NormalTok{(euclid)}
\FunctionTok{mode}\NormalTok{(euclid)}
\FunctionTok{class}\NormalTok{(euclid)}
\FunctionTok{length}\NormalTok{(}\FunctionTok{as.matrix}\NormalTok{(euclid))}
\FunctionTok{as.matrix}\NormalTok{(euclid)[}\DecValTok{1}\SpecialCharTok{:}\DecValTok{6}\NormalTok{, }\DecValTok{1}\SpecialCharTok{:}\DecValTok{6}\NormalTok{] }\CommentTok{\#mostra as 5 primeiras linhas e colunas da matriz}
\NormalTok{euclid\_ma }\OtherTok{\textless{}{-}}\NormalTok{ (}\FunctionTok{as.matrix}\NormalTok{(euclid))}
\CommentTok{\#View(euclid\_ma)}
\FunctionTok{str}\NormalTok{(euclid\_ma)}
\FunctionTok{mode}\NormalTok{(euclid\_ma)}
\FunctionTok{class}\NormalTok{(euclid\_ma)}
\NormalTok{euclid\_ma[}\DecValTok{1}\SpecialCharTok{:}\DecValTok{5}\NormalTok{, }\DecValTok{1}\SpecialCharTok{:}\DecValTok{5}\NormalTok{] }\CommentTok{\#mostra as 5 primeiras linhas e colunas da matriz}
\FunctionTok{range}\NormalTok{(m\_bruta)}
\FunctionTok{range}\NormalTok{(euclid)}
\FunctionTok{min}\NormalTok{(euclid)}
\FunctionTok{max}\NormalTok{(euclid)}
\FunctionTok{mean}\NormalTok{(euclid) }\CommentTok{\#CENTROIDE!! OU Grand mean}
\FunctionTok{sd}\NormalTok{(euclid)  }\CommentTok{\#Standard deviation}
\NormalTok{centroide }\OtherTok{\textless{}{-}} \FunctionTok{mean}\NormalTok{(euclid)}
\NormalTok{centroide}
\FunctionTok{length}\NormalTok{(euclid)}
\NormalTok{m }\OtherTok{\textless{}{-}} \FunctionTok{nrow}\NormalTok{(m\_bruta)}
\NormalTok{m}
\NormalTok{m}\SpecialCharTok{*}\NormalTok{(m}\DecValTok{{-}1}\NormalTok{)}\SpecialCharTok{/}\DecValTok{2}
\FunctionTok{summary}\NormalTok{(euclid)}
\NormalTok{Sumario1 }\OtherTok{\textless{}{-}} \FunctionTok{cbind}\NormalTok{(}\FunctionTok{min}\NormalTok{(euclid),}
                  \FunctionTok{max}\NormalTok{(euclid),}
                  \FunctionTok{sd}\NormalTok{(euclid),}
                  \FunctionTok{mean}\NormalTok{(euclid),}
                  \FunctionTok{length}\NormalTok{(euclid))}
\FunctionTok{colnames}\NormalTok{(Sumario1) }\OtherTok{\textless{}{-}} \FunctionTok{c}\NormalTok{(}\StringTok{"Minimo"}\NormalTok{, }\StringTok{"Maximo"}\NormalTok{, }\StringTok{"Desv.Padr"}\NormalTok{, }\StringTok{"Media"}\NormalTok{, }\StringTok{"m(m{-}1)/2"}\NormalTok{)}
\FunctionTok{rownames}\NormalTok{(Sumario1) }\OtherTok{\textless{}{-}}\NormalTok{ (}\StringTok{"Valores"}\NormalTok{)}
\NormalTok{Sumario1}
\FunctionTok{range}\NormalTok{(euclid)}
\FunctionTok{par}\NormalTok{(}\AttributeTok{mfrow=}\FunctionTok{c}\NormalTok{(}\DecValTok{2}\NormalTok{,}\DecValTok{1}\NormalTok{))}
\FunctionTok{hist}\NormalTok{(euclid,}
     \AttributeTok{breaks =} \DecValTok{15}\NormalTok{, }\CommentTok{\#determina o no. de colunas do histograma}
     \AttributeTok{xlim =} \FunctionTok{range}\NormalTok{(}\FunctionTok{floor}\NormalTok{(}\FunctionTok{min}\NormalTok{(euclid)), }\FunctionTok{ceiling}\NormalTok{(}\FunctionTok{max}\NormalTok{(euclid))), }
     \AttributeTok{xlab =} \StringTok{"Distr. de Frequências"}\NormalTok{,}
     \AttributeTok{freq =} \ConstantTok{FALSE}\NormalTok{)}
\FunctionTok{curve}\NormalTok{(}\FunctionTok{dnorm}\NormalTok{(x, }\AttributeTok{mean=}\FunctionTok{mean}\NormalTok{(euclid), }\AttributeTok{sd=}\FunctionTok{sd}\NormalTok{(euclid)), }\AttributeTok{add=}\ConstantTok{TRUE}\NormalTok{)}
\FunctionTok{boxplot.default}\NormalTok{(euclid, }\AttributeTok{horizontal =} \ConstantTok{TRUE}\NormalTok{, }\AttributeTok{frame =} \ConstantTok{FALSE}\NormalTok{,}
                \AttributeTok{xlab=}\StringTok{"Distr. de Frequências"}\NormalTok{,}
                \AttributeTok{ylim=}\FunctionTok{c}\NormalTok{(}\FunctionTok{floor}\NormalTok{(}\FunctionTok{min}\NormalTok{(euclid)), }\FunctionTok{ceiling}\NormalTok{(}\FunctionTok{max}\NormalTok{(euclid)))) }\CommentTok{\#Limites do eixo Y}
\DocumentationTok{\#\# dev.off()}
\FunctionTok{library}\NormalTok{(matrixStats)}
\FunctionTok{library}\NormalTok{(moments)}
\NormalTok{euclid\_ma }\OtherTok{\textless{}{-}} \FunctionTok{as.matrix}\NormalTok{(euclid)}
\NormalTok{euclid\_ma}
\FunctionTok{range}\NormalTok{(euclid\_ma) }\CommentTok{\#valor errado}
\FunctionTok{mean}\NormalTok{(euclid\_ma) }\CommentTok{\#valor errado}
\FunctionTok{sd}\NormalTok{(euclid\_ma) }\CommentTok{\#valor errado}
\FunctionTok{is.na}\NormalTok{(euclid\_ma) }\OtherTok{\textless{}{-}}\NormalTok{ euclid\_ma}\SpecialCharTok{==}\DecValTok{0} \CommentTok{\#atribui n.a. aos valores = 0}
\FunctionTok{mean}\NormalTok{(euclid\_ma, }\AttributeTok{na.rm=}\NormalTok{T)  }\CommentTok{\#valor correto, omite valores n.a. do calculo}
\FunctionTok{mean}\NormalTok{(euclid\_ma, }\AttributeTok{na.rm=}\NormalTok{T) }\CommentTok{\#valor correto, omite valores n.a. do calculo}
\FunctionTok{sd}\NormalTok{(euclid\_ma, }\AttributeTok{na.rm=}\NormalTok{T) }\CommentTok{\#valor correto, omite valores n.a. do calculo}
\CommentTok{\#colMeans(euclid\_ma, na.rm=T) \#omite valores n.a. do calculo}
\CommentTok{\#rowMeans(euclid\_ma, na.rm=T) \#omite valores n.a. do calculo}
\NormalTok{centroide\_ma }\OtherTok{\textless{}{-}} \FunctionTok{mean}\NormalTok{(euclid\_ma, }\AttributeTok{na.rm=}\NormalTok{T)}
\NormalTok{av.dist }\OtherTok{\textless{}{-}}\NormalTok{ (}\FunctionTok{as.matrix}\NormalTok{(}\FunctionTok{colMeans}\NormalTok{(euclid\_ma, }\AttributeTok{na.rm=}\NormalTok{T)))}
\NormalTok{av.desvpad }\OtherTok{\textless{}{-}}\NormalTok{ (}\FunctionTok{as.matrix}\NormalTok{(}\FunctionTok{colSds}\NormalTok{(euclid\_ma, }\AttributeTok{na.rm=}\NormalTok{T)))}
\NormalTok{dp.centroide\_ma }\OtherTok{\textless{}{-}}\NormalTok{ (av.dist}\SpecialCharTok{{-}}\NormalTok{centroide\_ma)}\SpecialCharTok{/}\NormalTok{(}\FunctionTok{colSds}\NormalTok{(av.dist)) }\CommentTok{\#ou z{-}scores}
\NormalTok{list }\OtherTok{\textless{}{-}} \FunctionTok{as.matrix}\NormalTok{(}\FunctionTok{cbind}\NormalTok{(av.dist, av.desvpad, dp.centroide\_ma))}
\NormalTok{list}
\FunctionTok{colnames}\NormalTok{(list, }\AttributeTok{do.NULL =} \ConstantTok{FALSE}\NormalTok{)}
\FunctionTok{colnames}\NormalTok{(list) }\OtherTok{\textless{}{-}} \FunctionTok{c}\NormalTok{(}\StringTok{"Av.Dist"}\NormalTok{, }\StringTok{"Av.StDev"}\NormalTok{, }\StringTok{"DP.Centroide"}\NormalTok{)}
\NormalTok{list2 }\OtherTok{\textless{}{-}}\NormalTok{ list[}\FunctionTok{order}\NormalTok{(list[,}\DecValTok{1}\NormalTok{], }\AttributeTok{decreasing =} \ConstantTok{TRUE}\NormalTok{),] }\CommentTok{\#[,1] ou o nome da coluna}
\NormalTok{list2}
\FunctionTok{par}\NormalTok{(}\AttributeTok{mfrow=}\FunctionTok{c}\NormalTok{(}\DecValTok{3}\NormalTok{,}\DecValTok{1}\NormalTok{))}
\FunctionTok{hist}\NormalTok{(list2[, }\StringTok{"Av.Dist"}\NormalTok{],}
     \AttributeTok{breaks =} \DecValTok{15}\NormalTok{, }\CommentTok{\#determina o no. de colunas do histograma}
     \AttributeTok{xlab =} \StringTok{"Distr. de Frequências das Distâncias (em dp) para o centroide"}\NormalTok{,}
     \AttributeTok{main =} \StringTok{"Distribuição de Frequência da distância média para o centroide"}\NormalTok{,}
     \AttributeTok{xlim =} \FunctionTok{range}\NormalTok{(}\FunctionTok{floor}\NormalTok{(}\FunctionTok{min}\NormalTok{(av.dist)), }\FunctionTok{ceiling}\NormalTok{(}\FunctionTok{max}\NormalTok{(av.dist))), }\CommentTok{\#substitua aqui o menor e maior valor do \textasciigrave{}range()\textasciigrave{}}
     \AttributeTok{freq =}\NormalTok{ T)}
\FunctionTok{hist}\NormalTok{(list2[, }\StringTok{"Av.Dist"}\NormalTok{],}
     \AttributeTok{breaks =} \DecValTok{15}\NormalTok{, }\CommentTok{\#determina o no. de colunas do histograma}
     \AttributeTok{xlab =} \StringTok{"Distr. de Frequências das Distâncias (em dp) para o centroide"}\NormalTok{,}
     \AttributeTok{main =} \StringTok{"Curva de normalidade ajustada para a Distribuição de Frequência"}\NormalTok{,}
     \AttributeTok{xlim =} \FunctionTok{range}\NormalTok{(}\FunctionTok{floor}\NormalTok{(}\FunctionTok{min}\NormalTok{(av.dist)), }\FunctionTok{ceiling}\NormalTok{(}\FunctionTok{max}\NormalTok{(av.dist))), }\CommentTok{\#substitua aqui o menor e maior valor do \textasciigrave{}range()\textasciigrave{}}
     \AttributeTok{freq =}\NormalTok{ F)}
\FunctionTok{curve}\NormalTok{(}\FunctionTok{dnorm}\NormalTok{(x, }\AttributeTok{mean=}\FunctionTok{mean}\NormalTok{(list2[, }\StringTok{"Av.Dist"}\NormalTok{]), }\AttributeTok{sd=}\FunctionTok{sd}\NormalTok{(list2[, }\StringTok{"Av.Dist"}\NormalTok{])), }\AttributeTok{add=}\ConstantTok{TRUE}\NormalTok{)}
\FunctionTok{boxplot.default}\NormalTok{(list2[, }\StringTok{"Av.Dist"}\NormalTok{], }\AttributeTok{horizontal =} \ConstantTok{TRUE}\NormalTok{, }\AttributeTok{frame =} \ConstantTok{FALSE}\NormalTok{,}
                \AttributeTok{xlab=}\StringTok{"Distr. de Frequências"}\NormalTok{,}
                \AttributeTok{ylim=}\FunctionTok{c}\NormalTok{(}\FunctionTok{floor}\NormalTok{(}\FunctionTok{min}\NormalTok{(av.dist)), }\FunctionTok{ceiling}\NormalTok{(}\FunctionTok{max}\NormalTok{(av.dist)))) }\CommentTok{\#substitua aqui o menor e maior valor do \textasciigrave{}range()\textasciigrave{}}
\FunctionTok{par}\NormalTok{(}\AttributeTok{mfrow=}\FunctionTok{c}\NormalTok{(}\DecValTok{1}\NormalTok{,}\DecValTok{1}\NormalTok{))}
\FunctionTok{range}\NormalTok{(dp.centroide\_ma)}
\FunctionTok{par}\NormalTok{(}\AttributeTok{mfrow=}\FunctionTok{c}\NormalTok{(}\DecValTok{3}\NormalTok{,}\DecValTok{1}\NormalTok{))}
\FunctionTok{hist}\NormalTok{(list2[, }\StringTok{"DP.Centroide"}\NormalTok{],}
     \AttributeTok{breaks =} \DecValTok{15}\NormalTok{, }\CommentTok{\#determina o no. de colunas do histograma}
     \AttributeTok{xlab =} \StringTok{"Distr. de Frequências das Distâncias dos desvios padões para o centroide"}\NormalTok{,}
     \AttributeTok{main =} \StringTok{"Distribuição de Frequência dos desvio padões das distâncias médias para o centroide"}\NormalTok{,}
     \AttributeTok{xlim =} \FunctionTok{range}\NormalTok{(}\FunctionTok{floor}\NormalTok{(}\FunctionTok{min}\NormalTok{(dp.centroide\_ma)), }\FunctionTok{ceiling}\NormalTok{(}\FunctionTok{max}\NormalTok{(dp.centroide\_ma))), }\CommentTok{\#substitua aqui o menor e maior valor do \textasciigrave{}range()\textasciigrave{}}
     \AttributeTok{freq =}\NormalTok{ T)}
\FunctionTok{hist}\NormalTok{(list2[, }\StringTok{"DP.Centroide"}\NormalTok{],}
     \AttributeTok{breaks =} \DecValTok{15}\NormalTok{, }\CommentTok{\#determina o no. de colunas do histograma}
     \AttributeTok{xlab =} \StringTok{"Distr. de Frequências das Distâncias dos desvios padrões das distâncias médias para o centroide"}\NormalTok{,}
     \AttributeTok{main =} \StringTok{"Curva de normalidade ajustada para a Distribuição de Frequência"}\NormalTok{,}
     \AttributeTok{xlim =} \FunctionTok{range}\NormalTok{(}\FunctionTok{floor}\NormalTok{(}\FunctionTok{min}\NormalTok{(dp.centroide\_ma)), }\FunctionTok{ceiling}\NormalTok{(}\FunctionTok{max}\NormalTok{(dp.centroide\_ma))), }\CommentTok{\#substitua aqui o menor e maior valor do \textasciigrave{}range()\textasciigrave{}}
     \AttributeTok{freq =}\NormalTok{ F)}
\FunctionTok{curve}\NormalTok{(}\FunctionTok{dnorm}\NormalTok{(x, }\AttributeTok{mean=}\FunctionTok{mean}\NormalTok{(list2[, }\StringTok{"DP.Centroide"}\NormalTok{]), }\AttributeTok{sd=}\FunctionTok{sd}\NormalTok{(list2[, }\StringTok{"DP.Centroide"}\NormalTok{])), }\AttributeTok{add=}\ConstantTok{TRUE}\NormalTok{)}
\FunctionTok{boxplot.default}\NormalTok{(list2[, }\StringTok{"DP.Centroide"}\NormalTok{], }\AttributeTok{horizontal =} \ConstantTok{TRUE}\NormalTok{, }\AttributeTok{frame =} \ConstantTok{FALSE}\NormalTok{,}
                \AttributeTok{xlab=}\StringTok{"Distr. de Frequências"}\NormalTok{,}
                \AttributeTok{ylim=}\FunctionTok{c}\NormalTok{(}\FunctionTok{floor}\NormalTok{(}\FunctionTok{min}\NormalTok{(dp.centroide\_ma)), }\FunctionTok{ceiling}\NormalTok{(}\FunctionTok{max}\NormalTok{(dp.centroide\_ma)))) }\CommentTok{\#substitua aqui o menor e maior valor do \textasciigrave{}range()\textasciigrave{}}
\FunctionTok{par}\NormalTok{(}\AttributeTok{mfrow=}\FunctionTok{c}\NormalTok{(}\DecValTok{1}\NormalTok{,}\DecValTok{1}\NormalTok{))}
\DocumentationTok{\#\# dev.off()}
\NormalTok{cutoff }\OtherTok{\textless{}{-}} \FloatTok{2.0}
\FunctionTok{library}\NormalTok{(gt)}
\FunctionTok{format}\NormalTok{(cutoff, }\AttributeTok{nsmall =} \DecValTok{1}\NormalTok{)}
\NormalTok{listf }\OtherTok{\textless{}{-}} \FunctionTok{as.data.frame}\NormalTok{(list2)}
\NormalTok{listf}\SpecialCharTok{$}\NormalTok{Outliers }\OtherTok{\textless{}{-}} \FunctionTok{ifelse}\NormalTok{(listf}\SpecialCharTok{$}\NormalTok{DP.Centroide}\SpecialCharTok{\textgreater{}{-}}\NormalTok{cutoff }\CommentTok{\#CUTOFF MENOR QUE {-}\textquotesingle{}cutoff}
                         \SpecialCharTok{\&}\NormalTok{ listf}\SpecialCharTok{$}\NormalTok{DP.Centroide}\SpecialCharTok{\textless{}}\NormalTok{cutoff,  }\CommentTok{\#CUTOFF MAIOR QUE \textquotesingle{}cutoff\textquotesingle{}}
                         \StringTok{""}\NormalTok{, }\StringTok{"OUT"}\NormalTok{) }
\NormalTok{listf}
\FunctionTok{gt}\NormalTok{(}\FunctionTok{cbind}\NormalTok{(}\AttributeTok{Sitios=}\FunctionTok{rownames}\NormalTok{(listf),listf))}
\NormalTok{part }\OtherTok{\textless{}{-}} \FunctionTok{c}\NormalTok{(}\StringTok{"B{-}A{-}MU4"}\NormalTok{, }\StringTok{"B{-}A{-}MU3"}\NormalTok{, }\StringTok{"S{-}A{-}TA4"}\NormalTok{, }\StringTok{"S{-}R{-}CT3"}\NormalTok{)}
\NormalTok{part}
\NormalTok{m\_bruta\_part }\OtherTok{\textless{}{-}}\NormalTok{ m\_bruta[}\SpecialCharTok{!}\NormalTok{(}\FunctionTok{row.names}\NormalTok{(m\_bruta) }\SpecialCharTok{\%in\%} \FunctionTok{c}\NormalTok{(part)),]}
\CommentTok{\#m\_bruta\_part}
\NormalTok{sum }\OtherTok{\textless{}{-}} \FunctionTok{colSums}\NormalTok{(m\_bruta\_part)}
\NormalTok{sum}
\NormalTok{zero\_sum\_cols }\OtherTok{\textless{}{-}} \FunctionTok{names}\NormalTok{(}\FunctionTok{which}\NormalTok{(}\FunctionTok{colSums}\NormalTok{(m\_bruta\_part) }\SpecialCharTok{==} \DecValTok{0}\NormalTok{))}
\NormalTok{zero\_sum\_cols }\CommentTok{\#nomes das espécies zeradas}
\NormalTok{m\_bruta\_part2 }\OtherTok{\textless{}{-}}\NormalTok{ m\_bruta\_part[(}\FunctionTok{colSums}\NormalTok{(m\_bruta\_part) }\SpecialCharTok{!=} \DecValTok{0}\NormalTok{)] }\CommentTok{\#em != a exclamação inverte o sentido}
\NormalTok{zero\_sum\_cols2 }\OtherTok{\textless{}{-}} \FunctionTok{names}\NormalTok{(}\FunctionTok{which}\NormalTok{(}\FunctionTok{colSums}\NormalTok{(m\_bruta\_part2) }\SpecialCharTok{==} \DecValTok{0}\NormalTok{))}
\NormalTok{zero\_sum\_cols2 }\CommentTok{\#nomes das espécies zeradas}
\NormalTok{sum}\OtherTok{\textless{}{-}}\FunctionTok{colSums}\NormalTok{(m\_bruta\_part2)}
\NormalTok{sum}
\CommentTok{\#m\_bruta\_part2}
\CommentTok{\#m\_bruta\_part2 \textless{}{-} as.matrix(m\_bruta\_part2)}
\FunctionTok{str}\NormalTok{(m\_bruta\_part2)}
\FunctionTok{length}\NormalTok{(}\FunctionTok{as.matrix}\NormalTok{(m\_bruta\_part2))}
\DocumentationTok{\#\# df \textless{}{-} data.frame(Sites = rownames(m\_bruta), m\_bruta,}
\DocumentationTok{\#\#                  row.names = NULL,}
\DocumentationTok{\#\#                  check.names = FALSE) \#add titulo a primeira coluna}
\DocumentationTok{\#\# }
\DocumentationTok{\#\# write.table(m\_bruta\_part2, "m\_trabcsv.csv",}
\DocumentationTok{\#\#             sep = ";", dec = ".", \#"\textbackslash{}t",}
\DocumentationTok{\#\#             row.names = TRUE,}
\DocumentationTok{\#\#             quote = TRUE,}
\DocumentationTok{\#\#             append = FALSE)}
\DocumentationTok{\#\# }
\DocumentationTok{\#\# m\_trab \textless{}{-} read.csv("m\_trabcsv.csv",}
\DocumentationTok{\#\#                    sep = ";", dec = ".",}
\DocumentationTok{\#\#                    row.names = 1,}
\DocumentationTok{\#\#                    header = TRUE,}
\DocumentationTok{\#\#                    na.strings = NA,}
\DocumentationTok{\#\#                    check.names = FALSE, \#impede que o R mude os nomes das colunas}
\DocumentationTok{\#\#                    col.names = gsub("(\^{}|\_)([a{-}z])", "\textbackslash{}\textbackslash{}1\textbackslash{}\textbackslash{}U\textbackslash{}\textbackslash{}2",}
\DocumentationTok{\#\#                                     names(m\_trab), perl = TRUE))}
\end{Highlighting}
\end{Shaded}

\hypertarget{referuxeancias-14}{%
\section*{Referências}\label{referuxeancias-14}}
\addcontentsline{toc}{section}{Referências}

\hypertarget{refs}{}
\begin{CSLReferences}{0}{1}
\leavevmode\vadjust pre{\hypertarget{ref-RN1484}{}}%
BEASLEY, C. R. \textbf{Bioestatística usando R. Apostila de exemplos para o biólogo}. {[}s.l: s.n.{]}. Disponível em: \textless{}\url{http://www.ead.unicamp.br/minicurso/bw/texto/fdl.pt.html}\textgreater.

\leavevmode\vadjust pre{\hypertarget{ref-RN1336}{}}%
BORCARD, D.; GILLET, F.; LEGENDRE, P. \textbf{Numerical Ecology with R}. 2. ed. {[}s.l.{]} : Springer, 2018. p. 435

\leavevmode\vadjust pre{\hypertarget{ref-RN146}{}}%
BRADSTOCK, R. A.; TOZER, M. G.; KEITH, D. A. Effects of high frequency fire on floristic composition and abundance in a fire-prone heathland near Sydney. \textbf{Australian Journal of Botany}, {[}s. l.{]}, v. 45, p. 641--655, 1997.

\leavevmode\vadjust pre{\hypertarget{ref-RN2923}{}}%
DIXON, P. \href{https://doi.org/10.1111/j.1654-1103.2003.tb02228.x}{VEGAN, a package of R functions for community ecology}. \textbf{Journal of Vegetation Science}, {[}s. l.{]}, v. 14, p. 927--930, 2003.

\leavevmode\vadjust pre{\hypertarget{ref-RN2942}{}}%
FERREIRA, E. B.; OLIVEIRA, M. S. de. \textbf{Introdução à Estatística com R}. Alfenas, MG: Editora Universidade Federal de Alfenas, 2020. p. 194 Disponível em: \textless{}\url{https://www.unifal-mg.edu.br/bibliotecas/wp-content/uploads/sites/125/2021/12/32-EBR_Unifal.pdf}\textgreater{}

\leavevmode\vadjust pre{\hypertarget{ref-RN182}{}}%
GOTELLI, N. J.; ELLISON, A. M. \textbf{Princípios de Estatística em Ecologia}. Porto Alegre: Artmed, 2011. p. 527

\leavevmode\vadjust pre{\hypertarget{ref-RN419}{}}%
HORTON, N. J.; KLEINMAN, K. \textbf{Using R and RStudio for Data Management, Statistical Analysis, and Graphics}. 2. ed. {[}s.l.{]} : CRC Press, 2015. p. 253

\leavevmode\vadjust pre{\hypertarget{ref-RN1073}{}}%
HURLBERT, S. H. \href{https://doi.org/10.2307/1934145}{The nonconcept of species diversity: a critique and alternative parameters}. \textbf{Ecology}, {[}s. l.{]}, v. 52, n. 4, p. 577--586, 1971.

\leavevmode\vadjust pre{\hypertarget{ref-RN1278}{}}%
LEGENDRE, P.; LEGENDRE, L. \textbf{Numerical Ecology}. 2nd English ed ed. Amsterdam, The Netherlands ; New York ; Oxford. p. 853

\leavevmode\vadjust pre{\hypertarget{ref-RN136}{}}%
LUDWIG, J. A.; REYNOLDS, J. F. \textbf{Statistical Ecology. A Primer on Methods and Computing}. New York, NY: John Wiley \& Sons, 1988. p. 337

\leavevmode\vadjust pre{\hypertarget{ref-RN357}{}}%
MAGURRAN, A. E. \textbf{Ecological Diversity and Its Measurement}. {[}s.l.{]} : Croom Helm Ltd, 1988. p. 179

\leavevmode\vadjust pre{\hypertarget{ref-RN1552}{}}%
MCCUNE, B.; GRACE, J. B. \textbf{Analysis of Ecological Communities}. Gleneden Beach, Oregon, U.S.A.: MjM Software Design, 2002. p. 300

\leavevmode\vadjust pre{\hypertarget{ref-RN3024}{}}%
MEDEIROS, E. S. F.; ROSA, I. L. Reprodução de tucunaré Cichla ocellaris (Perciformes: Cichlidae) Bloch \& Schneider, 1801 na represa do rio Gramame, Alhandra / PB. \textbf{Resumos. II Encontro de Iniciação Científica da UFPB}, {[}s. l.{]}, p. 82--82, 1994.

\leavevmode\vadjust pre{\hypertarget{ref-RN1965}{}}%
MEDEIROS, E. S. F.; SILVA, M. J. da; RAMOS, T. P. A.; RAMOS, R. T. C. \href{https://doi.org/10.1590/S2179-975X3023}{Environmental variables as predictors of fish community composition in semiarid aquatic systems}. \textbf{Acta Limnologica Brasiliensia}, {[}s. l.{]}, v. 36, n. e4, p. 1--13, 2024.

\leavevmode\vadjust pre{\hypertarget{ref-RN2491}{}}%
MEDEIROS, E. S. F.; SILVA, M. J.; RAMOS, R. T. C. Application of catchment- and local-scale variables for aquatic habitat characterization and assessment in the Brazilian semi-arid region. \textbf{Neotropical Biology and Conservation}, {[}s. l.{]}, v. 3, n. 1, p. 13--20, 2008. Disponível em: \textless{}\url{https://revistas.unisinos.br/index.php/neotropical/article/view/5440}\textgreater{}

\leavevmode\vadjust pre{\hypertarget{ref-RN2785}{}}%
MELO, A. S. \textbf{Conversando com o R usando 57 palavras: Introdução à programação com exemplos em Ecologia}. {[}s.l: s.n.{]}. Disponível em: \textless{}\url{https://ecoevol.ufg.br/adrimelo/prog/Conversando_com_o_R_usando_57_palavras-v13.pdf}\textgreater.

\leavevmode\vadjust pre{\hypertarget{ref-RN2832}{}}%
OKSANEN, J.; BLANCHET, F. G.; FRIENDLY, M.; KINDT, R.; LEGENDRE, P.; MCGLINN, D.; MINCHIN, P. R.; O'HARA, R. B.; SIMPSON, G. L.; SOLYMOS, P.; STEVENS, M. H. H.; SZOECS, E.; WAGNER, H. \textbf{Vegan: Community Ecology Package. R package version 2.5-7}. {[}s.l.{]} : The Comprehensive R Archive Network Archive, 2020. Disponível em: \textless{}\url{https://cran.r-project.org/web/packages/vegan/vegan.pdf}\textgreater.

\leavevmode\vadjust pre{\hypertarget{ref-RN2941}{}}%
OLIVEIRA, P. F. de; GUERRA, S.; MCDONNELL, R. \textbf{Ciência de dados com R: Introdução}. Brasília, DF: Editora IBPAD, 2018. p. 240 Disponível em: \textless{}\url{https://cdr.ibpad.com.br/}\textgreater{}

\leavevmode\vadjust pre{\hypertarget{ref-RN465}{}}%
PIELOU, E. C. \textbf{Ecological Diversity}. New York, NY: John Wiley \& Sons, 1975. p. 165

\leavevmode\vadjust pre{\hypertarget{ref-RN2908}{}}%
SILVA, E. P. da; DUARTE, M. R. N.; MEDEIROS, E. S. F. \href{https://doi.org/10.4013/nbc.2018.131.11}{Length-weight relationship of two fish species from a dryland intermittent river in northeastern Brazil}. \textbf{Neotropical Biology and Conservation}, {[}s. l.{]}, v. 13, n. 1, 2018.

\leavevmode\vadjust pre{\hypertarget{ref-RN2940}{}}%
SILVA, F. R. da; GONÇALVES-SOUZA, T.; PATERNO, G. B.; PROVETE, D. B.; VANCINE, M. H. \textbf{Análises ecológicas no R}. Recife, PE : Bauru, SP: Nupeea, Canal 6, 2022. Disponível em: \textless{}\url{https://analises-ecologicas.netlify.app/}\textgreater{}

\leavevmode\vadjust pre{\hypertarget{ref-RN2724}{}}%
SILVA, M. J.; FIGUEIREDO, B. R. S.; RAMOS, R. T. C.; MEDEIROS, E. S. F. \href{https://doi.org/10.1590/S1679-62252010005000010\%20}{Food resources used by three species of fish in the semi-arid region of Brazil}. \textbf{Neotropical Ichthyology}, {[}s. l.{]}, v. 8, n. 4, p. 825--833, 2010.

\leavevmode\vadjust pre{\hypertarget{ref-RN286}{}}%
SOKAL, R. R.; ROHLF, F. J. \textbf{Introduction to Biostatistics}. 2. ed. Mineola, New York: Dover Publications, 1987. p. 363

\leavevmode\vadjust pre{\hypertarget{ref-RN67}{}}%
SOKAL, R. R.; ROHLF, F. J. \textbf{Biometry: The Principles and Practice of Statistics in Biological Research}. 3. ed. New York: W.H. Freeman; Company, 1995. p. 776

\leavevmode\vadjust pre{\hypertarget{ref-RN1787}{}}%
TEAM, R. D. C. \textbf{An Introduction to R}. {[}s.l: s.n.{]}. Disponível em: \textless{}\url{https://cran.r-project.org/doc/manuals/r-release/R-intro.html}\textgreater.

\leavevmode\vadjust pre{\hypertarget{ref-RN2774}{}}%
TEAM, R. D. C. \textbf{R: A language and environment for statistical computing}. Austria: R Foundation for Statistical Computing, 2017. Disponível em: \textless{}\url{https://www.r-project.org/}\textgreater{}

\leavevmode\vadjust pre{\hypertarget{ref-RN358}{}}%
TEAM, R. S. \textbf{RStudio: Integrated Development Environment for R}. Boston, MA: RStudio, PBC, 2022. Disponível em: \textless{}\url{https://posit.co/products/open-source/rstudio/}\textgreater{}

\leavevmode\vadjust pre{\hypertarget{ref-RN2759}{}}%
VALENTIN, J. L. \textbf{Ecologia Numérica. Uma Introdução à Análise Multivariada de Dados Ecológicos}. 1. ed. Rio de Janeiro: Interciência, 2000. p. 117

\leavevmode\vadjust pre{\hypertarget{ref-RN2092}{}}%
WILLIAMS, D. G. \textbf{Study Guide: Classification and Ordination}. (v1.1), 2002. ed. Canberra, Australia: Flexible Delivery Development Unit, Centre for Enhancement of Learning, Teaching; Scholarship (CELTS), University of Camberra, Australia, 2002.

\leavevmode\vadjust pre{\hypertarget{ref-RN158}{}}%
ZAR, J. H. \textbf{Biostatistical Analysis}. 4th. ed. Upper Saddle River, New Jersey, USA: Prentice Hall, 1999. p. 663

\end{CSLReferences}

\end{document}
