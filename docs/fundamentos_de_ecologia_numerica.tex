% Options for packages loaded elsewhere
\PassOptionsToPackage{unicode}{hyperref}
\PassOptionsToPackage{hyphens}{url}
%
\documentclass[
]{book}
\usepackage{amsmath,amssymb}
\usepackage{lmodern}
\usepackage{iftex}
\ifPDFTeX
  \usepackage[T1]{fontenc}
  \usepackage[utf8]{inputenc}
  \usepackage{textcomp} % provide euro and other symbols
\else % if luatex or xetex
  \usepackage{unicode-math}
  \defaultfontfeatures{Scale=MatchLowercase}
  \defaultfontfeatures[\rmfamily]{Ligatures=TeX,Scale=1}
\fi
% Use upquote if available, for straight quotes in verbatim environments
\IfFileExists{upquote.sty}{\usepackage{upquote}}{}
\IfFileExists{microtype.sty}{% use microtype if available
  \usepackage[]{microtype}
  \UseMicrotypeSet[protrusion]{basicmath} % disable protrusion for tt fonts
}{}
\makeatletter
\@ifundefined{KOMAClassName}{% if non-KOMA class
  \IfFileExists{parskip.sty}{%
    \usepackage{parskip}
  }{% else
    \setlength{\parindent}{0pt}
    \setlength{\parskip}{6pt plus 2pt minus 1pt}}
}{% if KOMA class
  \KOMAoptions{parskip=half}}
\makeatother
\usepackage{xcolor}
\usepackage{longtable,booktabs,array}
\usepackage{calc} % for calculating minipage widths
% Correct order of tables after \paragraph or \subparagraph
\usepackage{etoolbox}
\makeatletter
\patchcmd\longtable{\par}{\if@noskipsec\mbox{}\fi\par}{}{}
\makeatother
% Allow footnotes in longtable head/foot
\IfFileExists{footnotehyper.sty}{\usepackage{footnotehyper}}{\usepackage{footnote}}
\makesavenoteenv{longtable}
\usepackage{graphicx}
\makeatletter
\def\maxwidth{\ifdim\Gin@nat@width>\linewidth\linewidth\else\Gin@nat@width\fi}
\def\maxheight{\ifdim\Gin@nat@height>\textheight\textheight\else\Gin@nat@height\fi}
\makeatother
% Scale images if necessary, so that they will not overflow the page
% margins by default, and it is still possible to overwrite the defaults
% using explicit options in \includegraphics[width, height, ...]{}
\setkeys{Gin}{width=\maxwidth,height=\maxheight,keepaspectratio}
% Set default figure placement to htbp
\makeatletter
\def\fps@figure{htbp}
\makeatother
\setlength{\emergencystretch}{3em} % prevent overfull lines
\providecommand{\tightlist}{%
  \setlength{\itemsep}{0pt}\setlength{\parskip}{0pt}}
\setcounter{secnumdepth}{5}
\usepackage{booktabs}
\usepackage{amsthm}
\makeatletter
\def\thm@space@setup{%
  \thm@preskip=8pt plus 2pt minus 4pt
  \thm@postskip=\thm@preskip
}
\makeatother
\ifLuaTeX
  \usepackage{selnolig}  % disable illegal ligatures
\fi
\usepackage[]{natbib}
\bibliographystyle{D:/Elvio/OneDrive/My EN files/R/abnt.csl}
\IfFileExists{bookmark.sty}{\usepackage{bookmark}}{\usepackage{hyperref}}
\IfFileExists{xurl.sty}{\usepackage{xurl}}{} % add URL line breaks if available
\urlstyle{same} % disable monospaced font for URLs
\hypersetup{
  pdftitle={Fundamentos de Ecologia Numérica},
  pdfauthor={Prof.~Elvio S. F. Medeiros; Laboratório de Ecologia; Universidade Estadual da Paraíba; Campus V, João Pessoa, PB},
  hidelinks,
  pdfcreator={LaTeX via pandoc}}

\title{Fundamentos de Ecologia Numérica}
\usepackage{etoolbox}
\makeatletter
\providecommand{\subtitle}[1]{% add subtitle to \maketitle
  \apptocmd{\@title}{\par {\large #1 \par}}{}{}
}
\makeatother
\subtitle{Disciplina de Ecologia Numérica do Curso de Ciências Biológicas do Campus V da UEPB}
\author{Prof.~Elvio S. F. Medeiros \and Laboratório de Ecologia \and Universidade Estadual da Paraíba \and Campus V, João Pessoa, PB}
\date{2026-01-10}

\begin{document}
\maketitle

{
\setcounter{tocdepth}{1}
\tableofcontents
}
\hypertarget{apresentauxe7uxe3o}{%
\chapter{Apresentação}\label{apresentauxe7uxe3o}}

\hypertarget{motivauxe7uxe3o}{%
\section{Motivação}\label{motivauxe7uxe3o}}

Este é um site experimental da disciplina de graduação de Ecologia Numérica do Curso de Ciências Biológicas da Universidade Estadual da Paraíba.

O site experimental da disciplina de Ecologia Numérica da UEPB é uma plataforma criada com o objetivo de apresentar exemplos didáticos aos alunos, explorando diferentes conceitos e técnicas utilizadas na análise de dados em Ecologia. É importante destacar que os exemplos apresentados no site podem conter imprecisões ou códigos demasiado extensos (aparentemente desnecessários para um ecólogo e programador de R mais experiente), mas isso é proposital, pois a finalidade é oferecer um material didático para os alunos, que possa facilitar a compreensão dos conceitos abordados durante as aulas. Dessa forma, o site é uma ferramenta complementar às aulas presenciais, que busca fornecer exemplos práticos e didáticos para os estudantes de Ecologia Numérica da UEPB.

Como mencionado, a finalidade é didatica e de apresentar o tema seguindo passo-a-passo e contextualizando com dados ecológicos reais. Por isso, além das informações sobre o conteúdo do site experimental da disciplina de ecologia numérica da UEPB, é importante destacar que qualquer erro ou problema relacionado ao site pode ser reportado ao \href{elviomedeiros@servidor.uepb.edu.br}{responsável pelo site}.

É fundamental que os usuários do site compreendam que plágio e outras formas de má conduta acadêmica são inaceitáveis e podem ter sérias consequências. O \href{elviomedeiros@servidor.uepb.edu.br}{responsável pelo site} se compromete a remover qualquer conteúdo ofensivo ou sem citar a devida fonte consultada, que possa ter sido publicado erroneamente ou por engano.

\hypertarget{organizauxe7uxe3o-do-livro}{%
\section{Organização do livro}\label{organizauxe7uxe3o-do-livro}}

Este livro é organizado em \texttt{QUATRO} partes:

Na \texttt{PARTE\ I}, são apresentadas as bases teóricas fundamentais sobre ecologia numérica e análise multivariada.

Na \texttt{PARTE\ II}, é apresentada a fundamentação teórica para o entedimento dos conjuntos de dados que são trabalhados ao longo da \texttt{PARTE\ III} e execução das estatísticas descritas.

Na \texttt{PARTE\ III}, são apresentados os Módulos do RStudio, na forma de tutoriais discutidos, que podem ser executados passo-a-passo.

Por fim, na \texttt{PARTE\ IV}, são apresentadas informações gerais de apoio para o melhor entendimento do conteúdo, como uma lista geral das tabelas usadas na \texttt{PARTE\ II}, um glossário de termos, bibliografia geral, entre outras fontes de informação.

\hypertarget{como-estudar-por-esse-livro}{%
\section{Como estudar por esse livro?}\label{como-estudar-por-esse-livro}}

Simples! Siga a sequência que é apresentada.

\begin{itemize}
\tightlist
\item
  Estude a teoria (\texttt{PARTE\ I}), certifique-se de que entende os conceitos e é capaz de tomar decisões fundamentadas na teoria. Revise/relembre do Ensino Médio a matemática necessária (Conjuntos, Matriz, Álgebra, Teorema de Pitágoras, etc.).
\item
  Leia a fundamentação das bases de dados que usará subsequentemente (\texttt{PARTE\ II}). Mas lembre, você não precisa ser especialista nos grupos taxonômicos apresentados aqui como exemplo. O ecólogo trabalha com dados, e eles falam por si só.
\item
  Parta para a ação (\texttt{PARTE\ III}), começe com os primeiros módulos do R, siga as instruções passo-a-passo, execute linha por linha de código, atente para os resultados.
\item
  Consulte os detalhes e referências apresentados na \texttt{PARTE\ IV}.
\end{itemize}

\hypertarget{para-um-estudo-aprofundado}{%
\subsection{Para um estudo aprofundado}\label{para-um-estudo-aprofundado}}

Teste de Novo

  \bibliography{C:/Github.Repos/Fun\_Ecologia\_Numerica/refs/my\_references-x9.bib,packages.bib}

\end{document}
