% Options for packages loaded elsewhere
\PassOptionsToPackage{unicode}{hyperref}
\PassOptionsToPackage{hyphens}{url}
%
\documentclass[
]{book}
\usepackage{amsmath,amssymb}
\usepackage{lmodern}
\usepackage{iftex}
\ifPDFTeX
  \usepackage[T1]{fontenc}
  \usepackage[utf8]{inputenc}
  \usepackage{textcomp} % provide euro and other symbols
\else % if luatex or xetex
  \usepackage{unicode-math}
  \defaultfontfeatures{Scale=MatchLowercase}
  \defaultfontfeatures[\rmfamily]{Ligatures=TeX,Scale=1}
\fi
% Use upquote if available, for straight quotes in verbatim environments
\IfFileExists{upquote.sty}{\usepackage{upquote}}{}
\IfFileExists{microtype.sty}{% use microtype if available
  \usepackage[]{microtype}
  \UseMicrotypeSet[protrusion]{basicmath} % disable protrusion for tt fonts
}{}
\makeatletter
\@ifundefined{KOMAClassName}{% if non-KOMA class
  \IfFileExists{parskip.sty}{%
    \usepackage{parskip}
  }{% else
    \setlength{\parindent}{0pt}
    \setlength{\parskip}{6pt plus 2pt minus 1pt}}
}{% if KOMA class
  \KOMAoptions{parskip=half}}
\makeatother
\usepackage{xcolor}
\usepackage{color}
\usepackage{fancyvrb}
\newcommand{\VerbBar}{|}
\newcommand{\VERB}{\Verb[commandchars=\\\{\}]}
\DefineVerbatimEnvironment{Highlighting}{Verbatim}{commandchars=\\\{\}}
% Add ',fontsize=\small' for more characters per line
\usepackage{framed}
\definecolor{shadecolor}{RGB}{248,248,248}
\newenvironment{Shaded}{\begin{snugshade}}{\end{snugshade}}
\newcommand{\AlertTok}[1]{\textcolor[rgb]{0.94,0.16,0.16}{#1}}
\newcommand{\AnnotationTok}[1]{\textcolor[rgb]{0.56,0.35,0.01}{\textbf{\textit{#1}}}}
\newcommand{\AttributeTok}[1]{\textcolor[rgb]{0.77,0.63,0.00}{#1}}
\newcommand{\BaseNTok}[1]{\textcolor[rgb]{0.00,0.00,0.81}{#1}}
\newcommand{\BuiltInTok}[1]{#1}
\newcommand{\CharTok}[1]{\textcolor[rgb]{0.31,0.60,0.02}{#1}}
\newcommand{\CommentTok}[1]{\textcolor[rgb]{0.56,0.35,0.01}{\textit{#1}}}
\newcommand{\CommentVarTok}[1]{\textcolor[rgb]{0.56,0.35,0.01}{\textbf{\textit{#1}}}}
\newcommand{\ConstantTok}[1]{\textcolor[rgb]{0.00,0.00,0.00}{#1}}
\newcommand{\ControlFlowTok}[1]{\textcolor[rgb]{0.13,0.29,0.53}{\textbf{#1}}}
\newcommand{\DataTypeTok}[1]{\textcolor[rgb]{0.13,0.29,0.53}{#1}}
\newcommand{\DecValTok}[1]{\textcolor[rgb]{0.00,0.00,0.81}{#1}}
\newcommand{\DocumentationTok}[1]{\textcolor[rgb]{0.56,0.35,0.01}{\textbf{\textit{#1}}}}
\newcommand{\ErrorTok}[1]{\textcolor[rgb]{0.64,0.00,0.00}{\textbf{#1}}}
\newcommand{\ExtensionTok}[1]{#1}
\newcommand{\FloatTok}[1]{\textcolor[rgb]{0.00,0.00,0.81}{#1}}
\newcommand{\FunctionTok}[1]{\textcolor[rgb]{0.00,0.00,0.00}{#1}}
\newcommand{\ImportTok}[1]{#1}
\newcommand{\InformationTok}[1]{\textcolor[rgb]{0.56,0.35,0.01}{\textbf{\textit{#1}}}}
\newcommand{\KeywordTok}[1]{\textcolor[rgb]{0.13,0.29,0.53}{\textbf{#1}}}
\newcommand{\NormalTok}[1]{#1}
\newcommand{\OperatorTok}[1]{\textcolor[rgb]{0.81,0.36,0.00}{\textbf{#1}}}
\newcommand{\OtherTok}[1]{\textcolor[rgb]{0.56,0.35,0.01}{#1}}
\newcommand{\PreprocessorTok}[1]{\textcolor[rgb]{0.56,0.35,0.01}{\textit{#1}}}
\newcommand{\RegionMarkerTok}[1]{#1}
\newcommand{\SpecialCharTok}[1]{\textcolor[rgb]{0.00,0.00,0.00}{#1}}
\newcommand{\SpecialStringTok}[1]{\textcolor[rgb]{0.31,0.60,0.02}{#1}}
\newcommand{\StringTok}[1]{\textcolor[rgb]{0.31,0.60,0.02}{#1}}
\newcommand{\VariableTok}[1]{\textcolor[rgb]{0.00,0.00,0.00}{#1}}
\newcommand{\VerbatimStringTok}[1]{\textcolor[rgb]{0.31,0.60,0.02}{#1}}
\newcommand{\WarningTok}[1]{\textcolor[rgb]{0.56,0.35,0.01}{\textbf{\textit{#1}}}}
\usepackage{longtable,booktabs,array}
\usepackage{calc} % for calculating minipage widths
% Correct order of tables after \paragraph or \subparagraph
\usepackage{etoolbox}
\makeatletter
\patchcmd\longtable{\par}{\if@noskipsec\mbox{}\fi\par}{}{}
\makeatother
% Allow footnotes in longtable head/foot
\IfFileExists{footnotehyper.sty}{\usepackage{footnotehyper}}{\usepackage{footnote}}
\makesavenoteenv{longtable}
\usepackage{graphicx}
\makeatletter
\def\maxwidth{\ifdim\Gin@nat@width>\linewidth\linewidth\else\Gin@nat@width\fi}
\def\maxheight{\ifdim\Gin@nat@height>\textheight\textheight\else\Gin@nat@height\fi}
\makeatother
% Scale images if necessary, so that they will not overflow the page
% margins by default, and it is still possible to overwrite the defaults
% using explicit options in \includegraphics[width, height, ...]{}
\setkeys{Gin}{width=\maxwidth,height=\maxheight,keepaspectratio}
% Set default figure placement to htbp
\makeatletter
\def\fps@figure{htbp}
\makeatother
\setlength{\emergencystretch}{3em} % prevent overfull lines
\providecommand{\tightlist}{%
  \setlength{\itemsep}{0pt}\setlength{\parskip}{0pt}}
\setcounter{secnumdepth}{5}
\usepackage{booktabs}
\usepackage{amsthm}
\makeatletter
\def\thm@space@setup{%
  \thm@preskip=8pt plus 2pt minus 4pt
  \thm@postskip=\thm@preskip
}
\makeatother
\ifLuaTeX
  \usepackage{selnolig}  % disable illegal ligatures
\fi
\usepackage[]{natbib}
\bibliographystyle{D:/Elvio/OneDrive/My EN files/R/abnt.csl}
\IfFileExists{bookmark.sty}{\usepackage{bookmark}}{\usepackage{hyperref}}
\IfFileExists{xurl.sty}{\usepackage{xurl}}{} % add URL line breaks if available
\urlstyle{same} % disable monospaced font for URLs
\hypersetup{
  pdftitle={Resources},
  pdfauthor={Prof.~Elvio S. F. Medeiros; Laboratório de Ecologia; Universidade Estadual da Paraíba; Campus V, João Pessoa, PB},
  hidelinks,
  pdfcreator={LaTeX via pandoc}}

\title{Resources}
\usepackage{etoolbox}
\makeatletter
\providecommand{\subtitle}[1]{% add subtitle to \maketitle
  \apptocmd{\@title}{\par {\large #1 \par}}{}{}
}
\makeatother
\subtitle{Disciplina de Ecologia Numérica do Curso de Ciências Biológicas do Campus V da UEPB}
\author{Prof.~Elvio S. F. Medeiros \and Laboratório de Ecologia \and Universidade Estadual da Paraíba \and Campus V, João Pessoa, PB}
\date{2026-01-10}

\begin{document}
\maketitle

{
\setcounter{tocdepth}{1}
\tableofcontents
}
\hypertarget{apresentauxe7uxe3o}{%
\chapter{Apresentação}\label{apresentauxe7uxe3o}}

\hypertarget{motivauxe7uxe3o}{%
\section{Motivação}\label{motivauxe7uxe3o}}

Este é um site experimental da disciplina de graduação de Ecologia Numérica do Curso de Ciências Biológicas da Universidade Estadual da Paraíba.

O site experimental da disciplina de Ecologia Numérica da UEPB é uma plataforma criada com o objetivo de apresentar exemplos didáticos aos alunos, explorando diferentes conceitos e técnicas utilizadas na análise de dados em Ecologia. É importante destacar que os exemplos apresentados no site podem conter imprecisões ou códigos demasiado extensos (aparentemente desnecessários para um ecólogo e programador de R mais experiente), mas isso é proposital, pois a finalidade é oferecer um material didático para os alunos, que possa facilitar a compreensão dos conceitos abordados durante as aulas. Dessa forma, o site é uma ferramenta complementar às aulas presenciais, que busca fornecer exemplos práticos e didáticos para os estudantes de Ecologia Numérica da UEPB.

Como mencionado, a finalidade é didatica e de apresentar o tema seguindo passo-a-passo e contextualizando com dados ecológicos reais. Por isso, além das informações sobre o conteúdo do site experimental da disciplina de ecologia numérica da UEPB, é importante destacar que qualquer erro ou problema relacionado ao site pode ser reportado ao \href{elviomedeiros@servidor.uepb.edu.br}{responsável pelo site}.

É fundamental que os usuários do site compreendam que plágio e outras formas de má conduta acadêmica são inaceitáveis e podem ter sérias consequências. O \href{elviomedeiros@servidor.uepb.edu.br}{responsável pelo site} se compromete a remover qualquer conteúdo ofensivo ou sem citar a devida fonte consultada, que possa ter sido publicado erroneamente ou por engano.

\hypertarget{organizauxe7uxe3o-do-livro}{%
\section{Organização do livro}\label{organizauxe7uxe3o-do-livro}}

Este livro é organizado em \texttt{QUATRO} partes:

Na \texttt{PARTE\ I}, são apresentadas as bases teóricas fundamentais sobre ecologia numérica e análise multivariada.

Na \texttt{PARTE\ II}, é apresentada a fundamentação teórica para o entedimento dos conjuntos de dados que são trabalhados ao longo da \texttt{PARTE\ III} e execução das estatísticas descritas.

Na \texttt{PARTE\ III}, são apresentados os Módulos do RStudio, na forma de tutoriais discutidos, que podem ser executados passo-a-passo.

Por fim, na \texttt{PARTE\ IV}, são apresentadas informações gerais de apoio para o melhor entendimento do conteúdo, como uma lista geral das tabelas usadas na \texttt{PARTE\ II}, um glossário de termos, bibliografia geral, entre outras fontes de informação.

\hypertarget{como-estudar-por-esse-livro}{%
\section{Como estudar por esse livro?}\label{como-estudar-por-esse-livro}}

Simples! Siga a sequência que é apresentada.

\begin{itemize}
\tightlist
\item
  Estude a teoria (\texttt{PARTE\ I}), certifique-se de que entende os conceitos e é capaz de tomar decisões fundamentadas na teoria. Revise/relembre do Ensino Médio a matemática necessária (Conjuntos, Matriz, Álgebra, Teorema de Pitágoras, etc.).
\item
  Leia a fundamentação das bases de dados que usará subsequentemente (\texttt{PARTE\ II}). Mas lembre, você não precisa ser especialista nos grupos taxonômicos apresentados aqui como exemplo. O ecólogo trabalha com dados, e eles falam por si só.
\item
  Parta para a ação (\texttt{PARTE\ III}), começe com os primeiros módulos do R, siga as instruções passo-a-passo, execute linha por linha de código, atente para os resultados.
\item
  Consulte os detalhes e referências apresentados na \texttt{PARTE\ IV}.
\end{itemize}

\hypertarget{para-um-estudo-aprofundado}{%
\subsection{Para um estudo aprofundado}\label{para-um-estudo-aprofundado}}

IMPORTANT: You can delete everything in here and start fresh. You might want to start by not deleting anything above this line until you know what that stuff is doing.

This is an .Rmd file. It is plain text with special features. Any time you write just like this, it will be compiled to normal text in the website. If you put a \# in front of your text, it will create a top level-header.

\hypertarget{my-first-post}{%
\chapter{My first post}\label{my-first-post}}

2018 \textbar{} 7 \textbar{} 23 Last compiled: 2026-01-10

Notice that whatever you define as a top level header, automatically gets put into the table of contents bar on the left.

\hypertarget{second-level-header}{%
\section{Second level header}\label{second-level-header}}

You can add more headers by adding more hashtags. These won't be put into the table of contents

\hypertarget{third-level-header}{%
\subsection{third level header}\label{third-level-header}}

Here's an even lower level header

\hypertarget{my-second-post-note-the-order}{%
\chapter{My second post (note the order)}\label{my-second-post-note-the-order}}

2018 \textbar{} 7 \textbar{} 23 Last compiled: 2026-01-10

I'm writing this tutorial going from the top down. And, this is how it will be printed. So, notice the second post is second in the list. If you want your most recent post to be at the top, then make a new post starting at the top. If you want the oldest first, do, then keep adding to the bottom

\hypertarget{adding-r-stuff}{%
\chapter{Adding R stuff}\label{adding-r-stuff}}

So far this is just a blog where you can write in plain text and serve your writing to a webpage. One of the main purposes of this lab journal is to record your progress learning R. The reason I am asking you to use this process is because you can both make a website, and a lab journal, and learn R all in R-studio. This makes everything really convenient and in the sam place.

So, let's say you are learning how to make a histogram in R. For example, maybe you want to sample 100 numbers from a normal distribution with mean = 0, and standard deviation =1, and then you want to plot a histogram. You can do this right here by using an r code block, like this:

\begin{Shaded}
\begin{Highlighting}[]
\NormalTok{samples }\OtherTok{\textless{}{-}} \FunctionTok{rnorm}\NormalTok{(}\DecValTok{100}\NormalTok{, }\AttributeTok{mean=}\DecValTok{0}\NormalTok{, }\AttributeTok{sd=}\DecValTok{1}\NormalTok{)}
\FunctionTok{hist}\NormalTok{(samples)}
\end{Highlighting}
\end{Shaded}

\includegraphics{fundamentos_de_ecologia_numerica_files/figure-latex/unnamed-chunk-1-1.pdf}

When you knit this R Markdown document, you will see that the histogram is printed to the page, along with the R code. This document can be set up to hide the R code in the webpage, just delete the comment (hashtag), from the cold folding option in the yaml header up top. For purposes of letting yourself see the code, and me see the code, best to keep it the way that it is. You learn all of these things and more can be customized in each R code block.

\hypertarget{the-big-idea}{%
\chapter{The big idea}\label{the-big-idea}}

Use this lab journal to record what you do in R. This way I will be able to see what you are doing and help you along the way. You will also be creating a repository of all the things you do. You can make posts about everything. Learning specific things in R (project unrelated), and doing things for the project that we will discuss at the beginning of the Fall semester. You can get started now by fiddling around with googling things, and trying stuff out in R. I've placed some helpful starting links in the links page on this website

\hypertarget{what-can-you-do-right-now-by-yourself}{%
\chapter{What can you do right now by yourself?}\label{what-can-you-do-right-now-by-yourself}}

It's hard to learn programming when you don't have specific problems that you are trying to solve. Everything just seems abstract.

I wrote an \href{https://crumplab.github.io/programmingforpsych/}{introductory programming book that introduces R}, and gives some \href{https://crumplab.github.io/programmingforpsych/programming-challenges-i-learning-the-fundamentals.html}{concrete problems for you to solve}.

To get the hang of journaling and solving the problems to learn programming, my suggestion is that you use this .Rmd file to solve the problems. It would look like this:

\hypertarget{problem-1}{%
\chapter{Problem 1}\label{problem-1}}

Do simple math with numbers, addition, subtraction, multiplication, division

\begin{Shaded}
\begin{Highlighting}[]
\DecValTok{1}\SpecialCharTok{+}\DecValTok{2}
\end{Highlighting}
\end{Shaded}

\begin{verbatim}
## [1] 3
\end{verbatim}

\begin{Shaded}
\begin{Highlighting}[]
\DecValTok{2}\SpecialCharTok{*}\DecValTok{5}
\end{Highlighting}
\end{Shaded}

\begin{verbatim}
## [1] 10
\end{verbatim}

\begin{Shaded}
\begin{Highlighting}[]
\DecValTok{5}\SpecialCharTok{/}\DecValTok{3}
\end{Highlighting}
\end{Shaded}

\begin{verbatim}
## [1] 1.666667
\end{verbatim}

\begin{Shaded}
\begin{Highlighting}[]
\NormalTok{(}\DecValTok{1}\SpecialCharTok{+}\DecValTok{6}\SpecialCharTok{+}\DecValTok{4}\NormalTok{)}\SpecialCharTok{/}\DecValTok{5}
\end{Highlighting}
\end{Shaded}

\begin{verbatim}
## [1] 2.2
\end{verbatim}

\hypertarget{problem-2}{%
\chapter{Problem 2}\label{problem-2}}

Put numbers into variables, do simple math on the variables

\begin{Shaded}
\begin{Highlighting}[]
\NormalTok{a}\OtherTok{\textless{}{-}}\DecValTok{1}
\NormalTok{b}\OtherTok{\textless{}{-}}\DecValTok{2}
\NormalTok{a}\SpecialCharTok{+}\NormalTok{b}
\end{Highlighting}
\end{Shaded}

\begin{verbatim}
## [1] 3
\end{verbatim}

\begin{Shaded}
\begin{Highlighting}[]
\NormalTok{d}\OtherTok{\textless{}{-}}\FunctionTok{c}\NormalTok{(}\DecValTok{1}\NormalTok{,}\DecValTok{2}\NormalTok{,}\DecValTok{3}\NormalTok{)}
\NormalTok{e}\OtherTok{\textless{}{-}}\FunctionTok{c}\NormalTok{(}\DecValTok{5}\NormalTok{,}\DecValTok{6}\NormalTok{,}\DecValTok{7}\NormalTok{)}
\NormalTok{d}\SpecialCharTok{+}\NormalTok{e}
\end{Highlighting}
\end{Shaded}

\begin{verbatim}
## [1]  6  8 10
\end{verbatim}

\begin{Shaded}
\begin{Highlighting}[]
\NormalTok{d}\SpecialCharTok{*}\NormalTok{e}
\end{Highlighting}
\end{Shaded}

\begin{verbatim}
## [1]  5 12 21
\end{verbatim}

\begin{Shaded}
\begin{Highlighting}[]
\NormalTok{d}\SpecialCharTok{/}\NormalTok{e}
\end{Highlighting}
\end{Shaded}

\begin{verbatim}
## [1] 0.2000000 0.3333333 0.4285714
\end{verbatim}

\hypertarget{problem-3}{%
\chapter{Problem 3}\label{problem-3}}

Write code that will place the numbers 1 to 100 separately into a variable using for loop. Then, again using the seq function.

\begin{Shaded}
\begin{Highlighting}[]
\CommentTok{\# for loop solution}
\CommentTok{\# i becomes the number 1 to 100 at each step of the loop}


\NormalTok{a }\OtherTok{\textless{}{-}} \FunctionTok{length}\NormalTok{(}\DecValTok{100}\NormalTok{) }\CommentTok{\# make empty variable, set length to 100}
\ControlFlowTok{for}\NormalTok{ (i }\ControlFlowTok{in} \DecValTok{1}\SpecialCharTok{:}\DecValTok{100}\NormalTok{)\{}
\NormalTok{  a[i] }\OtherTok{\textless{}{-}}\NormalTok{i }\CommentTok{\#assigns the number in i, to the ith index of a}
\NormalTok{\}}

\FunctionTok{print}\NormalTok{(a)}
\end{Highlighting}
\end{Shaded}

\begin{verbatim}
##   [1]   1   2   3   4   5   6   7   8   9  10  11  12  13  14  15  16  17  18
##  [19]  19  20  21  22  23  24  25  26  27  28  29  30  31  32  33  34  35  36
##  [37]  37  38  39  40  41  42  43  44  45  46  47  48  49  50  51  52  53  54
##  [55]  55  56  57  58  59  60  61  62  63  64  65  66  67  68  69  70  71  72
##  [73]  73  74  75  76  77  78  79  80  81  82  83  84  85  86  87  88  89  90
##  [91]  91  92  93  94  95  96  97  98  99 100
\end{verbatim}

\begin{Shaded}
\begin{Highlighting}[]
\CommentTok{\# for loop solution \#2}

\NormalTok{a}\OtherTok{\textless{}{-}}\FunctionTok{c}\NormalTok{() }\CommentTok{\#create empty variable using combine command}
\ControlFlowTok{for}\NormalTok{ (i }\ControlFlowTok{in} \DecValTok{1}\SpecialCharTok{:}\DecValTok{100}\NormalTok{)\{}
\NormalTok{  a}\OtherTok{\textless{}{-}}\FunctionTok{c}\NormalTok{(a,i) }\CommentTok{\# keeps combining a with itself and the new number in i}
\NormalTok{\}}
\FunctionTok{print}\NormalTok{(a)}
\end{Highlighting}
\end{Shaded}

\begin{verbatim}
##   [1]   1   2   3   4   5   6   7   8   9  10  11  12  13  14  15  16  17  18
##  [19]  19  20  21  22  23  24  25  26  27  28  29  30  31  32  33  34  35  36
##  [37]  37  38  39  40  41  42  43  44  45  46  47  48  49  50  51  52  53  54
##  [55]  55  56  57  58  59  60  61  62  63  64  65  66  67  68  69  70  71  72
##  [73]  73  74  75  76  77  78  79  80  81  82  83  84  85  86  87  88  89  90
##  [91]  91  92  93  94  95  96  97  98  99 100
\end{verbatim}

\begin{Shaded}
\begin{Highlighting}[]
\CommentTok{\# seq solution}
\NormalTok{a }\OtherTok{\textless{}{-}} \FunctionTok{seq}\NormalTok{(}\DecValTok{1}\NormalTok{,}\DecValTok{100}\NormalTok{,}\DecValTok{1}\NormalTok{) }\CommentTok{\# look up help for seq using ?seq() in console}
\FunctionTok{print}\NormalTok{(a)}
\end{Highlighting}
\end{Shaded}

\begin{verbatim}
##   [1]   1   2   3   4   5   6   7   8   9  10  11  12  13  14  15  16  17  18
##  [19]  19  20  21  22  23  24  25  26  27  28  29  30  31  32  33  34  35  36
##  [37]  37  38  39  40  41  42  43  44  45  46  47  48  49  50  51  52  53  54
##  [55]  55  56  57  58  59  60  61  62  63  64  65  66  67  68  69  70  71  72
##  [73]  73  74  75  76  77  78  79  80  81  82  83  84  85  86  87  88  89  90
##  [91]  91  92  93  94  95  96  97  98  99 100
\end{verbatim}

\hypertarget{replace-this-with-problem-4}{%
\chapter{Replace this with problem 4}\label{replace-this-with-problem-4}}

And keep going. Try to solve the problems with different scripts that provide the same solution. Good luck, Happy coding.

\hypertarget{r-and-r-studio}{%
\subsection{R and R-Studio}\label{r-and-r-studio}}

\href{http://www.r-project.org}{R} is a free open-source programming language that can be used for statistical analysis, data-simulation, graphing, and lots of other stuff. Another free program is \href{http://www.rstudio.com}{R-studio}, that provides a nice graphic interface for R. Download R first, then download R-studio. Both can run on PCs, Macs or Linux. Students will be learning R in the stats labs using the lab manual \href{}{}.

\hypertarget{additional-r-resources}{%
\subsection{Additional R resources}\label{additional-r-resources}}

\begin{enumerate}
\def\labelenumi{\arabic{enumi}.}
\tightlist
\item
  Google is great, Google your problem
\item
  \href{https://stackoverflow.com}{Stackoverflow} is great, google will often take you there because someone has already asked your question, and someone else has answered, usually many people have answered your question many ways.
\item
  Danielle Navarro wrote a \href{https://compcogscisydney.org/learning-statistics-with-r/}{free Psych Stats textbook using R}, it's worth checking out (some of our textbook are based on Danielle's)
\item
  I am currently writing another stats textbook (incorporating some of the above). You can read it while it's being made right here \url{https://crumplab.github.io/statistics/}, also check out the lab manual for more specific things about doing various stats in R (also in draft right now) \url{https://crumplab.github.io/statisticsLab/}
\item
  Daniell Navarro recently made this website for introducing R, it's great, check it out (also made using this R markdown process): \url{http://compcogscisydney.org/psyr/}
\item
  Check out my slightly older programming book that also introduces R \url{https://crumplab.github.io/programmingforpsych/}
\item
  This is the definitive guide for all things R Markdown (you will find this very useful as you get better at this skill): \url{https://bookdown.org/yihui/rmarkdown/}
\end{enumerate}

  \bibliography{D:/Elvio/OneDrive/My EN files/My Libraries/my references-x9.bib,packages.bib}

\end{document}
